\documentclass[12pt]{amsart}
\usepackage{preamble}
\DeclareMathOperator{\stab}{\mathrm{stab}}

\begin{document}
\begin{center}
    \textsc{Math 601. HW 1\\ Ian Jorquera}
\end{center}
\vspace{1em}
\begin{itemize}
%\item[(1)] % 5 points but ill skip
%           Let $G$ be a finite group and $V$ a finite dimensional vector space such that $\rho:G\ra GL(V)$ is a 
%           $G$-representation, meaning $\rho$ is a group homomorphism. Notice that this defines a group action 
%           $G\times V\ra V$ where $g\cdot v= \rho(g)v$. Notice that this is an action because 
%           $g\cdot(h\cdot v)= g\cdot(\rho(h)v)= \rho(g)\rho(h)v= \rho(gh)v$. This action also preserves the 
%           vector space structure, $g\cdot(rv+u)= \rho(g)(rv+u)= r\rho(g)v+ \rho(g)u=r g\cdot v+g\cdot u$. \idjtodo{need to show action by 1 is good}
%           
%           Now notice that a group action $G\times V\ra V$ that preserves the vector space structure is equivalent 
%           to $V$ being a $\C G$ modules. We only need to check compatibility with scalars. 
%           Let $r,k\in\C$ and $rg+kh\in \C G$ and consider the action 
%           %$(rg+kh)\cdot v=r(g\cdot v)+k(h\cdot v)=r\rho(g)v+k\rho(h)v=\rho(g)rv+\rho(h)kv=g\cdot rv+h\cdot kv$
%           \idjtodo{what is this? the formal linear combination work out because of "formality"? IDK here.}
%           Finally we will show that a $\C G$-module $V$ gives a representation, 
%           Let $\phi:G\ra GL(V)$ map $g\mapsto (v\mapsto g\cdot v)$ \idjtodo{show this is hom.}
%
%           \idjtodo{this is so tedeous}

\item[(2)] % 2 points
           Let $B_2$ be the upper triangular matrices in $GL_2(\C)$. First to show that there is not decomposition 
           notice that the matrices $\begin{bmatrix}1 & 2\\ 0 & 1\end{bmatrix}$ and 
           $\begin{bmatrix}2 & 2\\ 0 & 1\end{bmatrix}$ do not share both of their eigenvectors and therefore 
           $V$ can not be decomposed as a G module. 
           
           However notice we can act on the 1-dim subspace spanned by $\begin{bmatrix}1\\ 0 \end{bmatrix}$ 
           and the action is closed on this subspace as this vector is a common eigenvector for all upper triangular matrices
           meaning this repn on 
           $B_2$ is reducible.

\item[(3)] % 3 points
           Notice that $B_n$ acts on $\C^n$. However notice that for $1\leq \ell < n$ the group
           $B_n$ also acts on the $\ell$-dimensional space spanned by the first $\ell$ elementary vectors.
           This follows from $B_n$ being the upper triangular matrices, and so action by 
           $B_n$ only add the rows of the vector upward
           Therefore every space spanned by the first $\ell$ elementary vectors is a subrepresentation.

\item[(5)] % 4 points
            Consider the permutation representation of $S_3\ra GL_3(\C)$.

            Because the image of this representation are the permutation matrices, we know that the all $1$s vector
            is an eigenvector.
            This means with a change of basis matrix $\begin{pmatrix}
                1 & -2 & 1 \\
                -2 & 1 & 1 \\
                1 & 1 & 1
            \end{pmatrix}$ which will isolate the eigenvector in the third component with  
            the other two columns being orthogonal, we can block diagonalize each matrix 
            in our representation to get the following decomposed representation.
            Computations done using a computer

            $()\mapsto\begin{pmatrix}[cc|c]
                1 & 0 & 0 \\
                0 & 1 & 0 \\
                \hline
                0 & 0 & 1
            \end{pmatrix}$ $(1,2)\mapsto\begin{pmatrix}[cc|c]
                0 & 1 & 0 \\
                1 & 0 & 0 \\
                \hline
                0 & 0 & 1
            \end{pmatrix}$

            $(1,3)\mapsto\begin{pmatrix}[cc|c]
                1 & 0 & 0 \\
                -1 & -1 & 0 \\
                \hline
                0 & 0 & 1
            \end{pmatrix}$ $(2,3)\mapsto\begin{pmatrix}[cc|c]
                -1 & -1 & 0 \\
                0 & 1 & 0 \\
                \hline
                0 & 0 & 1
            \end{pmatrix}$

            $(1,2,3)\mapsto\begin{pmatrix}[cc|c]
                -1 & -1 & 0 \\
                1 & 0 & 0 \\
                \hline
                0 & 0 & 1
            \end{pmatrix}$ $(1,3,2)\mapsto\begin{pmatrix}[cc|c]
                0 & 1 & 0 \\
                -1 & -1 & 0 \\
                \hline
                0 & 0 & 1
            \end{pmatrix}$

\item[(7)]
Let $G$ be a Lie Group. Let $N$ be an open neighborhood(or an open set really.) containing $e$. 
Consider any element $g\in G$ which defines a diffeomorphism $g\cdot - :G\ra G$, therefore a homeomorphism. Notice
that this means that for any $n\in\ip N$ we have that $n\cdot N$ is an open set as $N$ was open. 
Notice also that $n\cdot N$ is contained in $\ip{N}$.
This means that for an $n\in \ip N$ there is an open set $n\cdot N\se \ip N$ containing $n$ and so 
$\ip N$ is open. Equivalently this means $\ip N = \bigcup_{n\in\ip N} nN$ and is therefore open.

Now we will show that $\ip N$ is closed or $\ip N^c$ is open. Consider a $p\in\ip N^c$, then $p\cdot N$ is open.
Notice also that it is distinct from $\ip N$ as if there existed an element $x\in p\cdot N\cap \ip N$ then $
x=p\cdot n_0=\prod^k_{i=1}n_i$ in which case $p=\prod^k_{i=1}n_i n_0^{-1}\in\ip N$ which is not the case. 
So every point $p$ is contained in an open set $p\cdot N$ contained in $\ip N^c$ so $\ip N ^c$ is open.

Finally because $\ip N$ is both open and closed, contains $e$ and $G$ is connected we know $\ip N= G$ 


\end{itemize}

\end{document}

