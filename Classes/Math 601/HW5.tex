\documentclass[12pt]{amsart}
\usepackage{preamble}
\DeclareMathOperator{\stab}{\mathrm{stab}}

\begin{document}
\begin{center}
    \textsc{Math 601. HW 3\\ Ian Jorquera}
\end{center}
\vspace{1em}
\begin{itemize} % 14 point completed
    \item[(1)] % 3 points
    Let $w$ be a word of $1$s and $2$s. Let $b$ be the position of some $2$ in the word. 
    And consider the case where the $2$ is not paired with some $1$. This must also mean there 
    are no unpaired $2$s to the left of position $b$. So the removal of $b$ does not unbracket any $1$s.
    Now consider the case that the $2$ at position $b$ was paired with a $1$ at position $c>b$. 
    In which case we must consider a few cases. 

    First assume that the only $2$s to the left of $b$ are also bracketed with a $1$ also to the left of $b$. 
    In this case removing the $2$ at $b$ will result in the $1$ at $c$ being unbracketed but no others, 
    and no $2$ bracketing with the $1$ at $c$.
    
    next assume that there exists an unbracketed $2$ at 
    position $a$ to the left of $b$, such that $a$ is also the first $2$ to the left of $b$ that is not bracketed
    with a $1$ also to the left of $b$. In this case the removal of $b$ will result in the $2$ at $a$ 
    bracketing with the $1$ at $c$. So no new $1$s are unbracketed.
    
    Finally assume that there is a $2$ at position $a<b$ 
    which is paired with a $1$ at position $d>c$ such that $a$ is the first such $2$ to the left of $b$.
    Notice that the removal of $b$ will result in the $2$ at $a$ pairing with the $1$ at $c$. And if 
    there is another $2$ (minimally) to the left of $a$ not paired with a $1$ to the left of $a$ then 
    that $2$ would pair with the $1$ at $d$. If that $2$ was itself paired with a $1$ to the right of $d$ 
    then this process would repeat in an inductive fashion, otherwise it wouldn't be paired with a 
    $1$ at all and there would be no new $1$s
    left unpaired. If there wasn't an unpaired $2$ to the left of $a$ then $d$ would be a new unpaired $1$ 
    but no other $1$s would be unpaired.

    This also shows that at most one new $1$ will be unpaired with the removal of any $2$.\\

    \item[(2)] % 1 point
    For word crystals, $\phi_i$ counts the number of 
    $i+1$s that are not paired with an $i$. And likewise $\epsilon_i$ counts
    the number of $i$s that are not paired with $i+1$.
    This means word crystals are seminormal, because $\phi_i(w)$ counts the number of 
    unpaired $i+1$s which means we can apply the operator $E_i$ that many times. Likewise 
    $F_i$ can be applied for each unpaired $i$ of which there are $\epsilon_i(w)$.\\

    \item[(3)] % 2 points
    Recall first that the $wt:B\ra \Z\ip{L_1,L_2,\dots, L_n}/(L_1+\dots+L_n)$ is the function that maps any 
    word $w\mapsto(\#1s, \#2s,\dots,\#ns)$. Notice that if $x,y$ are two words such that $e_i(x)=y$ and $x=f_i(y)$
    Notice that $x$ is equal to the word $y$ but with the right most $i$ not paired with an $i+1$ 
    changed to an $i+1$. Notice that the weight functions would be as follows, if $wt(x)=(w_1, w_2,\dots,w_n)$ then
    $wt(y)=(w_1, w_2,\dots,w_i+1,w_{i+1}-1,w_n)=w(x)+\alpha_i$ Notice also that because word crystals satisfy axiom S0,
    we also know that this means $\epsilon_i(y)=\epsilon_i(x)-1$ and $\phi_i(y)=\phi_i(x)+1$. This shows that Word crystals satisfy K1.
    
    Now to see word crystals satisfy K2, let $x$ be a word. Recall that $wt(x)=(\#1s, \#2s,\dots,\#ns)$. 
    Notice that because the weights classes formed by the relation $L_1+\dots+L_n=0$, let
    $k$ be the number of paired $i+1$ with an $i$ in which case we have that $wt(x)=(\#1s-k, \#2s-k,\dots-k,\#ns-k)$
    where $wt(x)_i$ represents the number of $i$s that are not paired with a $i+1$, or in other words $\phi_i(x)$
    and $wt(x)_{i+1}$ represents the number of $i+1$s that are not paired with a $i$, or in other words $\epsilon_i(x)$.
    So we have $wt(x)_i-wt(x)_{i+1}=wt(x)_i-wt(x)_{i+1}=(\# is-k) - (\# i+1s-k)$ which is closed under the 
    relation of adding or subtraction a constant to each of the terms in the weight. This proves that word crystal Satisfy K1 and K2.\\

    \item[(4)] % 3 points
    Using the above for $\phi_i$ and $\epsilon_i$ for word crystal of words consisting 
    of $1,\dots,n$ we can consider the length axiom (a).
    let $|i-j|>1$ and assume $y=e_i(x)$ which means $y$ is the word after applying $E_i$ to the word $x$, 
    which changes the left most unpaired $i+1$ to an $i$. Notice that because $|i-j|>1$ this can only have affected 
    the number of paired and unpaired $i-1$s and $i+1$s, and not the number of $j$s. 
    So $\epsilon_j(x)=\epsilon_j(y)$ and $\phi_j(x)=\phi_j(y)$ as desired.
    For axiom (b) consider the case where $|i-j|=1$ and specifically when $j=i+1$ in which case
    $y=e_i(x)$ is the word $x$ with the left most unpaired $i+1=j$ changed to an $i$, 
    considering pairing of $j$s with $i$s. Notice there are two cases if $j$ was paired with some $j+1$ or not.
    If it was then the number of $j$ that are unpaired with a $j+1$ does not change, 
    but this makes it so there are 1 addition $j+1$ that is unpaired, 
    so $\phi_j(x)=\phi_j(y)$ and $\epsilon_j(x)+1=\epsilon_j(y)$. In the other case if $j$ was 
    not paired with some $j+1$, then the number $j$s that are unpaired would decrease by $1$ but the number 
    of $j+1$s that are unpaired would not change as no pairings were broken in the removal of the $j$. 
    So $\phi_j(x)-1=\phi_j(y)$ and $\epsilon_j(x)=\epsilon_j(y)$

    Finally notice that the case where $j=i-1$.
    $y=e_i(x)$ is the word $x$ with the left most unpaired $i+1$ changed to an $i$, 
    considering pairing of $i+1$s with $i$s. Notice there are two cases if 
    the new $i$ pairs with some $j$ or not.
    If it pairs with a $j$ then the number of $j+1=i$s, unpaired with a $j$ does not change, 
    but this makes it so there is 1 addition $j$ that is now paired, so the number of unpaired $j$s decreases by one, 
    so $\phi_j(x)-1=\phi_j(y)$ and $\epsilon_j(x)=\epsilon_j(y)$. In the other case 
    if the new $i$ does not pairs with a $j$ then the number of $j+1=i$s, unpaired with a $j$ increases by 1, 
    but no additional $j$ are now paired or unpaired, so the number of unpaired $j$s does not change, 
    so $\phi_j(x)=\phi_j(y)$ and $\epsilon_j(x)+1=\epsilon_j(y)$.\\

    %\item[(6)] % 3 points 
    %First we will show that Type A Kashiwara crystals are graded posets with the covering 
    %relation given by the $f_i$ arrows. That means $y$ covers $x$ if $f_i(y)=x$. This suggests that the
    %rank function is probably the $\sum \phi_i$. 

    %Lets prove a little lemma.. im not sure what to do here
    
    %From K2 we have that 
    %$\sum_{i=1}^n\phi_i(x)=\sum_{i=1}^n\epsilon_i(x)+\text{wt}(x)_1-\text{wt}(x)_{n}$.
    \item[(7)] 
    \begin{itemize}
        \item[(a)] % 3 points
        Assume that $w$ and $v$ are knuth equivalent, 
        and that they differ by at most one knuth move. 
        We may also assume that both words are made up of distinct characters, 
        as if they were not we can relabel with in the following way 
        $iij\dots\mapsto i(i+1)(j+1)\dots$ where for any consecutive 
        numbers we keep the first the same and add $1$ to all remaining characters.
        Let $a$ be the smallest character in the words $w$ and $v$.
        We will consider the possible knuth moves, considering the removal of 
        this smallest letter $a$.

        First consider the case where the smallest letter $a$ occurs in $w$ and 
        $v$ in the same place, outside of any knuth move. 
        Meaning the knuth move(or possible no move at all) making $w$ and $v$ 
        equivalent is not affected by the removal of $a$. And so with the removal of $a$
        the two words are still equivalent.

        Now assume again $a$ is the smallest letter and for $a<b<c$ that 
        $w=\dots acb\dots$ and $v=\dots cab\dots$ then removal of $a$ would result in
        $w'=\dots cb\dots$ and $v'=\dots cb\dots$, meaning $w$ and $v$ with the removal of 
        $a$ are the same word, and so would be knuth equivalent.

        Now assume for $a$, the smallest letter and for $a<b<c$ that 
        $w=\dots bac\dots$ and $v=\dots bca\dots$ then removal of $a$ would result in
        $w'=\dots bc\dots$ and $v'=\dots bc\dots$, meaning $w$ and $v$ with the removal of 
        $a$ are the same word, and so would be knuth equivalent.\\

        % this is important because knuth equivalence have the same insertion tableau

        \item[(b)] % 2 points 
        In order for the concatenated reading word $\text{rw}(S)\text{rw}(T)$ to be ballot
        every suffix has to be ballot. And because $\text{rw}(T)$ is a suffix, it must be ballot.
        SSYT that have ballot reading words are highest weight tableau.
        
        Now notice that if we had the pair $(T,T)$ of tableau of of shape $\mu$ 
        that are height weight, the RSK bijection would standardize the right tableau to 
        be of the formed of counting from $1$ to $|\mu|$ starting from bottom left 
        reading right then moving up a row. Then we would unbump according to tht tableau, and
        at every step we would have a suffix that is ballot, because taking out a letter $j$ 
        from the unstandardized tableau would require that we first had to 
        unbump a letter from all rows below it.
        The un-standardizing would then gives the unique weakly increasing word 
        of the content $\mu$.\\

        \item[(c)(iii)] % 1 point
        This map is well defined because RSK is a bijection, meaning the two 
        line arrow created in 7b is uniquely determined by the tableau $T$. And then the process of
        inserting the bottom row of the two line array also unique determines the resulting tableau by RSK,
        so then the resulting skew tableau must be well-defined as 
        the letters $a_1,\dots, a_{|\mu|}$ are unique determined. 
    \end{itemize}
\end{itemize}

\end{document}

