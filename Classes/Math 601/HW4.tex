\documentclass[12pt]{amsart}
\usepackage{preamble}
\DeclareMathOperator{\stab}{\mathrm{stab}}

\begin{document}
\begin{center}
    \textsc{Math 601. HW 3\\ Ian Jorquera}
\end{center}
\vspace{1em}
\begin{itemize}
    % 15 points points completed
    \item[(1)] % 5 points 
    For $\mathfrak{sl_2}$ representation, with weight spaces corresponding to words of $1$'s and $2$s, a word 
    to be lowest weight means that the lowering operator $F$ sends the word to $0$ this is precisely the case 
    where the word is anti-ballot, where for all prefixes of the words the number of $2$s is greater than or 
    equal to the number of $1$s. To see that this is in fact lowest weight let $w=w_1w_2\dots w_n$ 
    be a word of $1$s and $2$s that is anti-ballot. 
    Meaning every prefix has at least as many $2$s as $1$s. Notice that this 
    means that for every prefix, every $1$ will be paired with a $2$. 
    This means that there are no unpaired $1$s Applying $F$ would then result in $0$.
    Now assume that $w$ is a word that is not anti-ballot. Meaning there is a prefix 
    $w_1\dots w_i$ that contains more $1$s then $2$s. We may assume that 
    $w_i=1$ and this suffix is of minimal length 
    in which case $w_i$ would be the left most unpaired $1$, as there are less 
    $2$s then $1$s 
    in the prefix. This means that $Fw=w_1w_2\dots w_{i-1}2w_{i+1}\dots w_n$ and 
    so $w$ did not represent a lowest weight vector.


    For $\mathfrak{sl_3}$ representation, with weight spaces corresponding to words of $1$'s $2$s and $3$s, a word 
    to be lowest weight means that the lowering operators $F_1$, and $F_2$ both send the word to $0$ 
    this is precisely the case 
    where the word is anti-ballot in both $1$s and $2$s ignoring $3$s and anti-ballot in $2$s and $3$s ignoring the $1$s,
    where for all prefixes of the words the number of $2$s is greater than or 
    equal to the number of $1$s and the number of $3$s is greater than or 
    equal to the number of $2$s. To see that this is in fact lowest weight let $w=w_1w_2\dots w_n$ 
    be a word of $1$s, $2$s, and $3$s that is anti-ballot. 
    Meaning every prefix has at least as many $2$s as $1$s and $3$s as $2$. Notice that this 
    means that for every prefix, every $1$ will be paired with a $2$ and every $2$ will be paired with a $3$. 
    This means that there are no unpaired $1$s. Applying $F_1$ would then result in $0$.
    This also means that there are no unpaired $2$s. Applying $F_2$ would then result in $0$.
    Now assume that $w$ is a word that is not anti-ballot. Meaning there is a prefix 
    $w_1\dots w_i$ that contains more $1$s then $2$s or more $2$s then $3$s. Applying $F_1$ or $F_2$ 
    would result in the first unpaired $1$ or $2$ being switched to a $2$ or $3$ respectively.
    In either case this would mean that $w$ did not represent a lowest weight vector.\\


    \item[(2)] % 3 points
    For $\mathfrak{sl}_3$ tableau crystals, the possible tableau will have shape 
    $\lambda={\lambda_1,\lambda_2,\lambda_3}$ where the first $\lambda_1$ columns of will 
    have $3$ rows and will be filled with content $3,2,1$ reading down. The next  $\lambda_2-\lambda_1$ columns
    will have $2$ rows and will be filled with content $3,2$ reading down. 
    Finally the remaining columns $\lambda_3-\lambda_2$ will all have $1$ row and be filled with $3$s.
    These are the lowest weight Tableau, as there would be no unpaired $2$s or unpaired $1$s in the reading words.

    This means that irreducible $\mathfrak{sl}_3$ representations have unique lowest weights 
    because any irreducible representation have weight spaces corresponding to a particular shape $\lambda$ 
    which has a unique filling as described above.\\

    \item[(3)] % 2 points 
    First we will order the basis $\mathfrak{sl}_3$ in the following way 
    $E_{12},E_{13},E_{23},E_{21},E_{31},E_{32},H_{12},H_{23}$
    In which case we compute the following commutators:
    $[E_{12},E_{2k}]=E_{1k}$ when $k\neq 1$. We also have that $[E_{12}, E_{31}]=-[E_{31}, E_{12}]=-E_{32}$
    and 
    $[E_{12},E_{\ell k}]=0$ for all other cases $\ell\neq 2$.
    $[E_{12},E_{21}]=H_{12}$
    And $[E_{12},H_{12}]=-2E_{12}$ and $[E_{12},H_{23}]=E_{12}$
    This all gives us the following matrix. 
    missing 

    
    \begin{tikzpicture}
        \matrix [column sep=1mm]
        {
            \node { };        &[2mm] \node{$E_{12}$}; &[-1mm] \node{$E_{13}$}; &[-1mm] \node {$E_{23}$}; &[-1mm] \node {$E_{21}$}; &[-1mm] \node {$E_{31}$}; &[-1mm] \node {$E_{32}$}; &[-1mm] \node {$H_{12}$}; &[-1mm] \node {$H_{23}$}; \\
            \node {$E_{12}$}; &[2mm] \node(a){0}; &              \node {0}; &              \node {0}; &              \node {0}; &              \node {0}; &              \node {0}; &              \node {-2}; &              \node (b) {1}; \\
            \node {$E_{13}$}; &[2mm] \node   {0}; &              \node {0}; &              \node {1}; &              \node {0}; &              \node {0}; &              \node {0}; &              \node {0}; &              \node {0}; \\
            \node {$E_{23}$}; &[2mm] \node   {0}; &              \node {0}; &              \node {0}; &              \node {0}; &              \node {0}; &              \node {0}; &              \node {0}; &              \node {0}; \\
            \node {$E_{21}$}; &[2mm] \node   {0}; &              \node {0}; &              \node {0}; &              \node {0}; &              \node {0}; &              \node {0}; &              \node {0}; &              \node {0}; \\
            \node {$E_{31}$}; &[2mm] \node   {0}; &              \node {0}; &              \node {0}; &              \node {0}; &              \node {0}; &              \node {0}; &              \node {0}; &              \node {0}; \\
            \node {$E_{32}$}; &[2mm] \node   {0}; &              \node {0}; &              \node {0}; &              \node {0}; &              \node {-1}; &              \node {0}; &              \node {0}; &              \node {0}; \\
            \node {$H_{12}$}; &[2mm] \node   {0}; &              \node {0}; &              \node {0}; &              \node {1}; &              \node {0}; &              \node {0}; &              \node {0}; &              \node {0}; \\
            \node {$H_{23}$}; &[2mm] \node(c){0}; &              \node {0}; &              \node {0}; &              \node {0}; &              \node {0}; &              \node {0}; &              \node {0}; &              \node (d){0}; \\
        };
        \draw [thick] (-2.5,2.2) to [square right brace] (-2.5, -2.8);
        \draw [thick] (3.8,2.2) to [square left brace] (3.8, -2.8);
      \end{tikzpicture}\\


    \item[(4)] % 2 points
    Consider the embedding $\iota:\mathfrak{sl}_2\hookrightarrow\mathfrak{sl}_3$
    Notice first that this map is a linear map which is clearly an embedding. idk maybe show but that would be tedious
    So we must only show that the map respects the lie for the basis elements of $\mathfrak{sl}_2$
    Notice that 
    \[[\iota(E),\iota(F)]=\begin{bmatrix}
        0&1&0\\0&0&0\\0&0&0
    \end{bmatrix}\begin{bmatrix}
        0&0&0\\1&0&0\\0&0&0
    \end{bmatrix}-\begin{bmatrix}
        0&0&0\\1&0&0\\0&0&0
    \end{bmatrix}\begin{bmatrix}
        0&1&0\\0&0&0\\0&0&0
    \end{bmatrix} =\begin{bmatrix}
        1&0&0\\0&-1&0\\0&0&0
    \end{bmatrix} =\iota(H) = \iota([\iota(E),\iota(F)])\]

    \[[\iota(H),\iota(F)]=\begin{bmatrix}
        1&0&0\\0&-1&0\\0&0&0
    \end{bmatrix}\begin{bmatrix}
        0&0&0\\1&0&0\\0&0&0
    \end{bmatrix}-\begin{bmatrix}
        0&0&0\\1&0&0\\0&0&0
    \end{bmatrix}\begin{bmatrix}
        1&0&0\\0&-1&0\\0&0&0
    \end{bmatrix}=\begin{bmatrix}
        0&0&0\\-2&0&0\\0&0&0
    \end{bmatrix} =\iota(-2F) = \iota([\iota(H),\iota(F)])\]

    \[[\iota(H),\iota(E)]=\begin{bmatrix}
        1&0&0\\0&-1&0\\0&0&0
    \end{bmatrix}\begin{bmatrix}
        0&1&0\\0&0&0\\0&0&0
    \end{bmatrix}-\begin{bmatrix}
        0&1&0\\0&0&0\\0&0&0
    \end{bmatrix}\begin{bmatrix}
        1&0&0\\0&-1&0\\0&0&0
    \end{bmatrix}=\begin{bmatrix}
        0&2&0\\0&0&0\\0&0&0
    \end{bmatrix} =\iota(2E) = \iota([\iota(H),\iota(E)])\]

    All other products are accounted for by properties of the Lie Bracket, because the lie bracket is bilinear 
    and anti-symmetric, with every element lie bracketed with itself being zero.\\

    \item[(7)] % 3 points
    \begin{enumerate}[label=(\alph*)]
        \item %1 point
        The homogenous symmetric function $h_d(x_1,\dots, x_n)$ corresponds to the schur function 
        $S_{(d)}(x_1,\dots,x_n)$ because the SSYT of shape $(d)$ allows repeats 
        \item %2 points
        Recall that the formal character of the irreducible representation 
        $V^{(a,b)}$ is the schur polynomial on $x_1,x_2,x_3$ for tableau of shape $\lambda$. Meaning
        $V^{(\mu_1,0)}$ has as its character $S_{(\mu_1)}(x_1,x_2,x_3)=h_{\mu_1}(x_1,x_2,x_3)$.
        And because the character of a tensor product is the product of the characters we have that 
        the character of $V^{(\mu_1,0)}\otimes V^{(\mu_2,0)}\otimes \dots \otimes V^{(\mu_k,0)}$
        is $h_{\mu_1}(x_1,x_2,x_3)\cdot h_{\mu_2}(x_1,x_2,x_3)\cdots h_{\mu_k}(x_1,x_2,x_3)= h_\mu(x_1,x_2,x_3)$
    \end{enumerate}
\end{itemize}

\end{document}

