\documentclass[11pt]{article}
\usepackage{fullpage}
\usepackage{amsmath,amsthm, amssymb, amsfonts, amscd}
\usepackage[mathscr]{eucal}
\usepackage{graphicx}
\usepackage{psfrag}
\usepackage[usenames,dvipsnames]{color}
\usepackage{subfigure}

%       Theorem environments

%% \theoremstyle{plain} %% This is the default
\newtheorem{theorem}{Theorem}[section]
\newtheorem{corollary}[theorem]{Corollary}
\newtheorem{lemma}[theorem]{Lemma}
\newtheorem{proposition}[theorem]{Proposition}
\newtheorem{ax}{Axiom}
\newtheorem{conjecture}[theorem]{Conjecture}

\theoremstyle{definition}
\newtheorem{definition}{Definition}[section]

\theoremstyle{definition}
\newtheorem{remark}[theorem]{Remark}



\input epsf

%Math Definitions


\newcommand{\Q}{{\mathbb Q}}
\newcommand{\Z}{{\mathbb Z}}
\newcommand{\R}{{\mathbb R}}
\newcommand{\C}{{\mathbb C}}
\newcommand{\N}{{\mathbb N}}

\newcommand{\RP}{{\mathbb{RP}}}
\newcommand{\CP}{{\mathbb{CP}}}



\newcommand{\sgn}{\operatorname{sgn}}
\newcommand{\dx}{\,\mathrm{d}x}
\newcommand{\dy}{\mathrm{d}y}
\newcommand{\dz}{\mathrm{d}z}
\newcommand{\inv}{\operatorname{inv}}
\newcommand{\I}{\mathbf{i}}
\newcommand{\J}{\mathbf{j}}
\newcommand{\K}{\mathbf{k}}
\newcommand{\SO}{\operatorname{SO}}
\newcommand{\dt}{\operatorname{dt}}
\newcommand{\tw}{\operatorname{tw}}
\newcommand{\ds}{\operatorname{ds}}
\newcommand{\st}{\operatorname{st}}
\newcommand{\sech}{\operatorname{sech}}
\newcommand{\range}{\operatorname{range}}
\newcommand{\nul}{\operatorname{null}}
\newcommand{\spa}{\operatorname{span}}
\newcommand{\quat}{\mathbb{H}}
\renewcommand{\Re}{\operatorname{Re}}
\renewcommand{\Im}{\operatorname{Im}}
\newcommand{\from}{\co\!\!}


\newcommand{\Field}{\mathbb{F}}

\usepackage{xcolor}
\definecolor{linkblue}{HTML}{003d73}
\definecolor{linkgreen}{HTML}{006161}
\definecolor{linkred}{HTML}{a11950}
\usepackage{hyperref}
\hypersetup{
	pdftitle={Math 670 HW \#1},
	pdfauthor={Clayton Shonkwiler},
	pdfsubject={differential geometry},
	pdfkeywords={differential geometry, homework, Math 670},
	colorlinks=true,
	linkcolor=linkblue,
	citecolor=linkgreen,
	urlcolor=linkred
}
% \pagestyle{empty}
\def\co{\colon}

\begin{document}
\begin{center}
{\Large\textbf{Math 670 HW \#1}}\\
Due 11:59 PM Friday, February 21
\end{center}




\begin{enumerate}	
	
	\item A smooth manifold $M$ is called \emph{orientable} if there exists a collection of coordinate charts $\{(U_\alpha, \phi_\alpha)\}$ so that, for every $\alpha, \beta$ such that $\phi_\alpha(U_\alpha) \cap \phi_\beta(U_\beta) = W \neq \emptyset$, the differential of the change of coordinates $\phi_\beta^{-1} \circ \phi_\alpha$ has positive determinant.
	
	\begin{enumerate}
		\item Show that for any $n$, the sphere $S^n$ is orientable.
		\begin{proof}
            
        \end{proof}		
		\item Prove that, if $M$ and $N$ are smooth manifolds and $f: M \to N$ is a local diffeomorphism at all points of $M$, then $N$ being orientable implies that $M$ is orientable. Is the converse true?
        \begin{proof}
            Becasue $N$ is orientable, there is an atlas $\{(V_\beta,\psi_\beta)\}$ for $N$ such that any change of variables has positive determinant. 
            Now we will consider an atlas $\{(U_\alpha,\phi_\alpha)\}$ for $M$.
            Any point $p\in M$, there exists chart $(U,\phi)$ and $(V,\psi)$ where $p\in\phi(U)$ and $f(p)\in\psi(V)$ and $f:\phi(U)\rightarrow\psi(V)$ is a diffeomorphism.
            Now consider a second chart $(U_2,\phi_2)$ containing the point $p$. Now we want to show that $\phi_2^{-1}\circ\phi$ defined on $U\cap U_2$ 
            has positive determinant. Let $(V_2,\psi_2)$ be a chart containing $f(\phi_2(U_2))$. Notice that from chasing diagrams we have that 
            \[\phi_2^{-1}\circ\phi=\phi_2^{-1}\circ f^{-1} \circ \psi_2\circ\psi_2^{-1}\circ\psi\circ\psi^{-1}\circ f \circ\phi\]
            on $U\cap U_2$ in which case 
            \[\det(\phi_2^{-1}\circ\phi)=\det(\phi_2^{-1}\circ f^{-1} \circ \psi_2)\det(\psi_2^{-1}\circ\psi)\det(\psi^{-1}\circ f \circ\phi)\]
            And becasue 




        \end{proof} 
    \end{enumerate}
	
	
	
	\item Supply the details for the proof that, if $F \from \operatorname{Mat}_{d \times d}(\C) \to \EuScript{H}(d)$ is given by $F(U) = UU^*$ (where $U^*$ is the conjugate transpose [a.k.a., Hermitian adjoint] of $U$), then the unitary group
	\[
		U(d) = F^{-1}(I_{d \times d})
	\]
	is a submanifold of $\operatorname{Mat}_{d \times d}(\C)$ of dimension $d^2$. (Hint: it may be helpful to remember that a Hermitian matrix $M$ can always be written as $M = \frac{1}{2}(M + M^*)$.)
	
    \begin{proof}
        
    \end{proof}

	\item Let $M$ be a compact manifold of dimension $n$ and let $f:M \to \R^n$ be a smooth map. Prove that $f$ must have at least one critical point.
	
	
	\item Prove that, if $X, Y$, and $Z$ are smooth vector fields on a smooth manifold $M$ and $a,b \in \R$, $f,g \in C^\infty (M)$, then
	\begin{enumerate}
		\item $[X,Y] = -[Y,X]$ (anticommutivity)
		\item $[aX+bY,Z] = a[X,Z]+b[Y,Z]$ (linearity)
		\item $[[X,Y],Z] + [[Y,Z],X] + [[Z,X],Y] = 0$ (Jacobi identity)
		\item $[fX,gY] = fg[X,Y] + f(Xg)Y - g(Yf)X$.
	\end{enumerate}
	
		 
\end{enumerate}


	
	


\end{document}
