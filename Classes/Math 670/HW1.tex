\documentclass[11pt]{article}
\usepackage{fullpage}
\usepackage{amsmath,amsthm, amssymb, amsfonts, amscd}
\usepackage[mathscr]{eucal}
\usepackage{graphicx}
\usepackage{psfrag}
\usepackage[usenames,dvipsnames]{color}
\usepackage{subfigure}

%       Theorem environments

%% \theoremstyle{plain} %% This is the default
\newtheorem{theorem}{Theorem}[section]
\newtheorem{corollary}[theorem]{Corollary}
\newtheorem{lemma}[theorem]{Lemma}
\newtheorem{proposition}[theorem]{Proposition}
\newtheorem{ax}{Axiom}
\newtheorem{conjecture}[theorem]{Conjecture}

\theoremstyle{definition}
\newtheorem{definition}{Definition}[section]

\theoremstyle{definition}
\newtheorem{remark}[theorem]{Remark}



\input epsf

%Math Definitions


\newcommand{\Q}{{\mathbb Q}}
\newcommand{\Z}{{\mathbb Z}}
\newcommand{\R}{{\mathbb R}}
\newcommand{\C}{{\mathbb C}}
\newcommand{\N}{{\mathbb N}}

\newcommand{\RP}{{\mathbb{RP}}}
\newcommand{\CP}{{\mathbb{CP}}}



\newcommand{\sgn}{\operatorname{sgn}}
\newcommand{\dx}{\,\mathrm{d}x}
\newcommand{\dy}{\mathrm{d}y}
\newcommand{\dz}{\mathrm{d}z}
\newcommand{\inv}{\operatorname{inv}}
\newcommand{\I}{\mathbf{i}}
\newcommand{\J}{\mathbf{j}}
\newcommand{\K}{\mathbf{k}}
\newcommand{\SO}{\operatorname{SO}}
\newcommand{\dt}{\operatorname{dt}}
\newcommand{\tw}{\operatorname{tw}}
\newcommand{\ds}{\operatorname{ds}}
\newcommand{\st}{\operatorname{st}}
\newcommand{\sech}{\operatorname{sech}}
\newcommand{\range}{\operatorname{range}}
\newcommand{\nul}{\operatorname{null}}
\newcommand{\spa}{\operatorname{span}}
\newcommand{\quat}{\mathbb{H}}
\renewcommand{\Re}{\operatorname{Re}}
\renewcommand{\Im}{\operatorname{Im}}
\newcommand{\from}{\co\!\!}


\newcommand{\Field}{\mathbb{F}}

\usepackage{xcolor}
\definecolor{linkblue}{HTML}{003d73}
\definecolor{linkgreen}{HTML}{006161}
\definecolor{linkred}{HTML}{a11950}
\usepackage{hyperref}
\hypersetup{
	pdftitle={Math 670 HW \#1},
	pdfauthor={Clayton Shonkwiler},
	pdfsubject={differential geometry},
	pdfkeywords={differential geometry, homework, Math 670},
	colorlinks=true,
	linkcolor=linkblue,
	citecolor=linkgreen,
	urlcolor=linkred
}
% \pagestyle{empty}
\def\co{\colon}

\begin{document}
\begin{center}
{\Large\textbf{Math 670 HW \#1}}\\
Due 11:59 PM Friday, February 21
\end{center}




\begin{enumerate}	
	
	\item A smooth manifold $M$ is called \emph{orientable} if there exists a collection of coordinate 
    charts $\{(U_\alpha, \phi_\alpha)\}$ so that, for every $\alpha, \beta$ such that 
    $\phi_\alpha(U_\alpha) \cap \phi_\beta(U_\beta) = W \neq \emptyset$, the differential of the change of 
    coordinates $\phi_\beta^{-1} \circ \phi_\alpha$ has positive determinant.
	
	\begin{enumerate}
		\item Show that for any $n$, the sphere $S^n$ is orientable.
		\begin{proof}
            Here we will consider theatlas from the notes, generated by $\{(\R^n,\phi_N),(\R^n,\phi_S)\}$
            Notice that because $\phi_N(\R^2)\cap\phi_S(\R^n)=S^2-\{N,S\}$ we can consider the change of coordinates 
            map which is $(\phi_N\circ \phi_S)(\vec{x})=\frac{1}{||\vec x||^2}\vec x$.
            Notice that becasue this is a map from $\R^n$ to $\R^n$ the differential is the Jacobian 
            which is just $\frac{1}{||\vec x||^2}I$, which has positive determinant.

        \end{proof}		
		\item Prove that, if $M$ and $N$ are smooth manifolds and $f: M \to N$ is a local diffeomorphism at all points of $M$, then $N$ being orientable implies that $M$ is orientable. Is the converse true?
        \begin{proof}
			First we know that any point $p\in M$, there exists open sets $U_p$ and $V_{p}$ where $p\in U_p$ and 
			$f(p)\in V_p$ and $f:U\rightarrow V$ is a diffeomorphism.
            Because $N$ is orientable, there is a collection of charts $\{(U_\alpha,\psi_\alpha)\}$ for $N$ such that any change of variables has differential with 
            positive determinant. 
            Now we will construct a collection of charts considering all $\alpha$ and $p\in M$, $\{(V_\alpha\cap \psi_\alpha^{-1}(V_p),f^{-1}\circ\psi_\alpha|_{V_\alpha\cap\psi_\alpha^{-1}(V_p)})\}$ for $M$.
			Notice that by construction these collections of charts will cover all of $M$ and the change of variables when restricted to $V_\alpha\cap \psi_\alpha^{-1}(V_p)$ will be
			\[(f^{-1}\circ\psi_\alpha)\circ (f^{-1}\circ\psi_\alpha)=\psi_\alpha^{-1}\circ f\circ f^{-1}\circ\psi_\alpha=\psi_\alpha^{-1}\circ \psi_\alpha\] 
			And so the differential must also have positive determinant.

			The converse is not true. Consider two manifolds $M$,$N$ such that $M$ is orientable and $N$ is not, 
			and consider the inclusion map of $M\rightarrow M\times N$, which satisfies the assumptions but $M\times N$ is not orientable.
        \end{proof} 
    \end{enumerate}
	
	
	
	\item Supply the details for the proof that, if $F \from \operatorname{Mat}_{d \times d}(\C) \to \EuScript{H}(d)$ is given by $F(U) = UU^*$ (where $U^*$ is the conjugate transpose [a.k.a., Hermitian adjoint] of $U$), then the unitary group
	\[
		U(d) = F^{-1}(I_{d \times d})
	\]
	is a submanifold of $\operatorname{Mat}_{d \times d}(\C)$ of dimension $d^2$. (Hint: it may be helpful to remember that a Hermitian matrix $M$ can always be written as $M = \frac{1}{2}(M + M^*)$.)
	
    \begin{proof}
        Notice first that $\text{Mat}_{d\times d}(\C)$ is a real manifold with dimension $2d^2$. Likewise $\mathcal H(d)$ is a real manifold with with dimension $d$, 
		this can be computed by directly entries that satisfy $M=M^*$, where the diagonal has to be all real entries. Next recall that 
		$T_I\text{Mat}_{d\times d} \cong \text{Mat}_{d\times d}$. We can also use the a defining equation $M=\frac{1}{2}(M+M^*)$ to determine that any curve 
		$\alpha(t)$ that satisfies this relation would have a derivative at $t=0$ equal to 
		$\frac{d}{dt}|_{t=0}\left[\alpha(t)\right]=\frac{d}{dt}|_{t=0}\left[\frac{1}{2}(\alpha(t)+\alpha^*(t))\right]=\frac{1}{2}(\alpha'(0)+\alpha'(t)^*)$
		and so $T_I\mathcal H(d)\cong \mathcal H(d)$

		Here we want to use the level set theorem. So we will show that $I\in\mathcal H(d)$ is a regular point. Notice that $F(I)=I$. To show $I$ is regular
		we will show that the differential $dF_I:T_I \text{Mat}_{d\times d}\rightarrow T_I\mathcal H(d)$ is surjective.
		Consider a smooth curve $\alpha(t)$ through $I$ with velocity $v$ in $\text{Mat}_{d\times d}$. This would give us the curve $\beta(t)=F\circ\alpha(t)=\alpha(t)\alpha(t)^*$.
		Where the derivative at $t=0$ is $\frac{d}{dt}|_{t=0}\left[\beta(t)\right]=\alpha'(0)\alpha(0)^*+\alpha(0)\alpha'(0)^*=v+v^*$.
		This shows that for any $v\in T_I\mathcal H(d)\cong \mathcal H(d)$ we have that the tangent vector $\frac{1}{2}v$ would map to the tangent vector $v$.
		This shows that $dF_I$ is surjective.

		So from the level set theorem we have that $F^{-1}(I)=\{UU^*=1\}=U(d)$ is a submanifold of dimension $d^2$. 
    \end{proof}

	\item Let $M$ be a compact manifold of dimension $n$ and let $f:M \to \R^n$ be a smooth map. Prove that $f$ must have at least one critical point.
	
    \begin{proof}
        First notice that $f(M)\subseteq \R^n$ is compact because $f$ is continuous and $M$ is compact. This means the image $f(M)$ is closed and bounded.
	Let $q$ be a point of the boundary of $f(M)$, and $p$ and point that maps to $q$. Notice that this means that there is a direction $v$ in the tangent space
	$T_q\R^n$ that would point out of $f(M)$, meaning any curve $\beta(t)$ through $q$ with velocity $v$ would leave $f(M)$. So $v$ is not in the image of the differential
	$df_p$ and so the differential is not surjective. And so $p$ is a critical point.
    \end{proof}
	
	\item Prove that, if $X, Y$, and $Z$ are smooth vector fields on a smooth manifold $M$ and $a,b \in \R$, $f,g \in C^\infty (M)$, then
	\begin{enumerate}
		\item $[X,Y] = -[Y,X]$ (anticommutivity)
		\item \begin{proof}
			$[X,Y]=XY-YX=-(YX-XY)=-[X,Y]$
		\end{proof}
		\item $[aX+bY,Z] = a[X,Z]+b[Y,Z]$ (linearity)
		\item 
		\begin{proof}
		Let $a,b\in\R$ then notice that because $\mathcal X(M)$ is a $\R$-module and $XY$ is well 
		defined(as a distributive product on vector fields that gives back a differential operator) we have that 
		$[aX+bY,Z]=(aX+bY)Z-Z(aX+bY)=aXZ+bYZ-aZX-bZY=a[X,Z]+b[Y,Z]$	
		\end{proof}
		\item $[[X,Y],Z] + [[Y,Z],X] + [[Z,X],Y] = 0$ (Jacobi identity)
		\begin{proof}
			\begin{multline*}
				[[X,Y],Z] + [[Y,Z],X] + [[Z,X],Y] = [XY-YX,Z] + [YZ-ZY,X] + [ZX-XZ,Y]\\
				= (XY-YX)Z-Z(XY-YX) + (YZ-ZY)X-X(YZ-ZY) + (ZX-XZ)Y-Y(ZX-XZ)\\
				= XYZ-YXZ-ZXY-ZYX + YZX-ZYX-XYZ-XZY + ZXY-XZY-YZX-YXZ\\
				= 0
			\end{multline*}
		\end{proof}
		\item $[fX,gY] = fg[X,Y] + f(Xg)Y - g(Yf)X$.
		\begin{proof}
			This follows from the fact that $X$ is a first order differential operator, and so we can utilize the product rule in the case 
			$X(gY(h))=X(g)Y(h)+gX(Y(h))$ where $h\in C^\infty(M)$ and so we can simply write $X(gY)=(Xg)Y+gXY$ which gives us
			\begin{align*}
				[fX,gY] &= fX(gY)-gY(fX)\\
				&= f(Xg)Y+fgXY-g(Yf)X-gfYX\\
				&= fg[X,Y] + f(Xg)Y - g(Yf)X
			\end{align*}
		\end{proof}
	\end{enumerate}
	
\end{enumerate}


	
	


\end{document}
