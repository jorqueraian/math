\documentclass[12pt]{amsart}
% packages
\usepackage{graphicx}
\usepackage{setspace}
\usepackage{amssymb,amsmath,amsthm,amsfonts,amscd}
\usepackage{hyperref}
\usepackage{color}
\usepackage{booktabs}
\usepackage{tabularx}
\usepackage{enumitem}
\usepackage[retainorgcmds]{IEEEtrantools}
\usepackage[notref,notcite,final]{showkeys}
\usepackage[final]{pdfpages}
\usepackage{fancyhdr}
\usepackage{upgreek}
\usepackage{multicol}
\usepackage{fontawesome}
\usepackage{halloweenmath}
\usepackage{ytableau}
% set margin as 0.75in
\usepackage[margin=0.75in]{geometry}

% tikz-related settings
\usepackage{tkz-berge}
\usetikzlibrary{calc,quotes}
\usetikzlibrary{arrows.meta}
\usetikzlibrary{positioning, automata}
\usetikzlibrary{decorations.pathreplacing}

%% For table
\usepackage{tikz}
\usetikzlibrary{tikzmark}

% theorem environments with italic font
\newtheorem{thm}{Theorem}[section]
\newtheorem*{thm*}{Theorem}
\newtheorem{lemma}[thm]{Lemma}
\newtheorem{prop}[thm]{Proposition}
\newtheorem{claim}[thm]{Claim}
\newtheorem{corollary}[thm]{Corollary}
\newtheorem{conjecture}[thm]{Conjecture}
\newtheorem{question}[thm]{Question}
\newtheorem{procedure}[thm]{Procedure}
\newtheorem{assumption}[thm]{Assumption}

% theorem environments with roman font (use lower-case version in body
% of text, e.g., \begin{example} rather than \begin{Example})
\newtheorem{Definition}[thm]{Definition}
\newenvironment{definition}
{\begin{Definition}\rm}{\end{Definition}}
\newtheorem{Example}[thm]{Example}
\newenvironment{example}
{\begin{Example}\rm}{\end{Example}}

\theoremstyle{definition}
\newtheorem{remark}[thm]{\textbf{Remark}}

% special sets
\newcommand{\A}{\mathbb{A}}
\newcommand{\C}{\mathbb{C}}
\newcommand{\F}{\mathbb{F}}
\newcommand{\N}{\mathbb{N}}
\newcommand{\Q}{\mathbb{Q}}
\newcommand{\R}{\mathbb{R}}
\newcommand{\Z}{\mathbb{Z}}
\newcommand{\cals}{\mathcal{S}}
\newcommand{\ZZ}{\mathbb{Z}_{\ge 0}}
\newcommand{\cala}{\mathcal{A}}
\newcommand{\calb}{\mathcal{B}}
\newcommand{\cald}{\mathcal{D}}
\newcommand{\calh}{\mathcal{H}}
\newcommand{\call}{\mathcal{L}}
\newcommand{\calr}{\mathcal{R}}
\newcommand{\la}{\mathbf{a}}
\newcommand{\lgl}{\mathfrak{gl}}
\newcommand{\lsl}{\mathfrak{sl}}
\newcommand{\lieg}{\mathfrak{g}}

% math operators
\DeclareMathOperator{\kernel}{\mathrm{ker}}
\DeclareMathOperator{\image}{\mathrm{im}}
\DeclareMathOperator{\rad}{\mathrm{rad}}
\DeclareMathOperator{\id}{\mathrm{id}}
\DeclareMathOperator{\hum}{[\mathrm{Hum}]}
\DeclareMathOperator{\eh}{[\mathrm{EH}]}
\DeclareMathOperator{\lcm}{\mathrm{lcm}}
\DeclareMathOperator{\Aut}{\mathrm{Aut}}
\DeclareMathOperator{\Inn}{\mathrm{Inn}}
\DeclareMathOperator{\Out}{\mathrm{Out}}
\DeclareMathOperator{\Gal}{\mathrm{Gal}}


% frequently used shorthands
\newcommand{\ra}{\rightarrow}
\newcommand{\se}{\subseteq}
\newcommand{\ip}[1]{\langle#1\rangle}
\newcommand{\dual}{^*}
\newcommand{\inverse}{^{-1}}
\newcommand{\norm}[2]{\|#1\|_{#2}}
\newcommand{\abs}[1]{\lvert #1 \rvert}
\newcommand{\Abs}[1]{\bigg| #1 \bigg|}
\newcommand\bm[1]{\begin{bmatrix}#1\end{bmatrix}}
\newcommand{\op}{\text{op}}

% nicer looking empty set
\let\oldemptyset\emptyset
\let\emptyset\varnothing

%the var phi gang
\let\oldphi\phi
\let\phi\varphi

\setlist[enumerate,1]{topsep=1em,leftmargin=1.8em, itemsep=0.5em, label=\textup{(}\arabic*\textup{)}}
\setlist[enumerate,2]{topsep=0.5em,leftmargin=3em, itemsep=0.3em}

%pagestyle
%\pagestyle{fancy} 

\begin{document}
\begin{center}
    \textsc{Math 502. HW 2\\ Ian Jorquera}
\end{center}
\vspace{1em}
% See http://www.mathematicalgemstones.com/maria/Math501Fall22.php
% for problems

% sage: https://sagecell.sagemath.org/
\begin{itemize}

\item[(1)]
\begin{itemize}
    \item First recall that we can write $s_{(2,1)}=2m_{(1,1,1)}+m_{(2,1)}$. This gives us that $\ip{s_{(2,1)},h_{(1,1,1)}}=\ip{2m_{(1,1,1)}+m_{(2,1)},h_{(1,1,1)}}=2\ip{m_{(1,1,1)},h_{(1,1,1)}}+\ip{m_{(2,1)},h_{(1,1,1)}}=2+0=2$.\\
    \item In this case we have that $\ip{s_{(2,1,1)},s_{(3,2)}}=\delta_{{(2,1,1)},{(3,2)}}=0$.\\
    \item In this case we can rewrite $e_{(2,1)}$ in terms of monomial basis by counting binary matrices and we find that $e_{(2,1)}=3m_{(1,1,1)}+m_{(2,1)}$. This gives us that $\ip{e_{(2,1)},h_{(2,1)}}=\ip{3m_{(1,1,1)}+m_{(2,1)},h_{(2,1)}}=3\ip{m_{(1,1,1)},h_{(2,1)}}+\ip{m_{(2,1)},h_{(2,1)}}=0+1=1$.\\
    \item Notice that $z_{(3,2,2,1)}=(1^1\cdot 1!)(2^2\cdot 2!)(3^1\cdot 1!)=24$ and so $\ip{p_{(3,2,2,1)},p_{(3,2,2,1)}}=24\ip{p_{(3,2,2,1)},\frac{p_{(3,2,2,1)}}{24}}=24$.\\
\end{itemize}

\item[(2)] Recall that the Jacobi-Trudi formula is $s_\lambda=\det(h_{\lambda_i-i+j})_{ij}$ and so with the omega involution we have that $s_\lambda=\omega s_\lambda^t=\omega\det(h_{\lambda^t_i-i+j})_{ij}=\omega\sum_{\pi\in S_{\ell(\lambda)}}\text{sgn}(\pi)\prod_{i=1}^nh_{\lambda^t_i-i+\pi(i)}=\sum_{\pi\in S_{\ell(\lambda)}}\text{sgn}(\pi)\prod_{i=1}^n\omega  h_{\lambda^t_i-i+\pi(i)}=\sum_{\pi\in S_{\ell(\lambda)}}\text{sgn}(\pi)\prod_{i=1}^n e_{\lambda^t_i-i+\pi(i)}=\det(e_{\lambda^t_i-i+j})_{ij}$.\\

\item[(3)] The following permutations correspond the the following 2 line arrays and pairs of young tablau under the RSK bijection

\begin{tabular}{cc}
    $\begin{pmatrix}1 & 2 & 3\\ 1 & 2 & 3\end{pmatrix}$ & \ytableausetup{smalltableaux}
\ytableaushort{
123},\ytableausetup{smalltableaux}
\ytableaushort{
123} \\
  $\begin{pmatrix}1 & 3 & 2\\ 1 & 2 & 3\end{pmatrix}$ & \ytableausetup{smalltableaux}
\ytableaushort{
3,12},\ytableausetup{smalltableaux}
\ytableaushort{
3,12} \\
  $\begin{pmatrix}2 & 1 & 3\\ 1 & 2 & 3\end{pmatrix}$ & \ytableausetup{smalltableaux}
\ytableaushort{
2,13},\ytableausetup{smalltableaux}
\ytableaushort{
2,13} \\
  $\begin{pmatrix}3 & 2 & 1\\ 1 & 2 & 3\end{pmatrix}$ & \ytableausetup{smalltableaux}
\ytableaushort{
3,2,1},\ytableausetup{smalltableaux}
\ytableaushort{
3,2,1} \\
 $\begin{pmatrix}2 & 3 & 1\\ 1 & 2 & 3\end{pmatrix}$ & \ytableausetup{smalltableaux}
\ytableaushort{
2,13},\ytableausetup{smalltableaux}
\ytableaushort{
3,12} \\
 $\begin{pmatrix}3 & 1 & 2\\ 1 & 2 & 3\end{pmatrix}$ & \ytableausetup{smalltableaux}
\ytableaushort{
3,12},\ytableausetup{smalltableaux}
\ytableaushort{
2,13} \\
\end{tabular}

\item[(4)] we have the following words with there corresponding pairs of young tablau.\\
\begin{tabular}{cc}
    $\begin{pmatrix}1 & 1 & 2 & 3\\ 1 & 2 & 3 & 4\end{pmatrix}$ & \ytableausetup{smalltableaux}
\ytableaushort{
1123},\ytableausetup{smalltableaux}
\ytableaushort{
1123}\\
$\begin{pmatrix}1 & 2 & 1 & 3\\ 1 & 2 & 3 & 4\end{pmatrix}$ & \ytableausetup{smalltableaux}
\ytableaushort{
2,113},\ytableausetup{smalltableaux}
\ytableaushort{
3,124}\\
$\begin{pmatrix}1 & 2 & 3 & 1\\ 1 & 2 & 3 & 4\end{pmatrix}$ & \ytableausetup{smalltableaux}
\ytableaushort{
2,113},\ytableausetup{smalltableaux}
\ytableaushort{
4,123}\\
$\begin{pmatrix}1 & 1 & 3 & 2\\ 1 & 2 & 3 & 4\end{pmatrix}$ & \ytableausetup{smalltableaux}
\ytableaushort{
3,112},\ytableausetup{smalltableaux}
\ytableaushort{
4,123}\\
$\begin{pmatrix}2 & 1 & 1 & 3\\ 1 & 2 & 3 & 4\end{pmatrix}$ & \ytableausetup{smalltableaux}
\ytableaushort{
2,113},\ytableausetup{smalltableaux}
\ytableaushort{
2,134}\\
$\begin{pmatrix}2 & 1 & 3 & 1\\ 1 & 2 & 3 & 4\end{pmatrix}$ & \ytableausetup{smalltableaux}
\ytableaushort{
23,11},\ytableausetup{smalltableaux}
\ytableaushort{
24,13}\\
$\begin{pmatrix}2 & 3 & 1 & 1\\ 1 & 2 & 3 & 4\end{pmatrix}$ & \ytableausetup{smalltableaux}
\ytableaushort{
23,11},\ytableausetup{smalltableaux}
\ytableaushort{
34,12}\\
$\begin{pmatrix}3 & 1 & 1 & 2\\ 1 & 2 & 3 & 4\end{pmatrix}$ & \ytableausetup{smalltableaux}
\ytableaushort{
3,112},\ytableausetup{smalltableaux}
\ytableaushort{
2,134}\\
$\begin{pmatrix}3 & 1 & 2 & 1\\ 1 & 2 & 3 & 4\end{pmatrix}$ & \ytableausetup{smalltableaux}
\ytableaushort{
3,2,11},\ytableausetup{smalltableaux}
\ytableaushort{
4,2,13}\\
$\begin{pmatrix}3 & 2 & 1 & 1\\ 1 & 2 & 3 & 4\end{pmatrix}$ & \ytableausetup{smalltableaux}
\ytableaushort{
3,2,11},\ytableausetup{smalltableaux}
\ytableaushort{
3,2,14}\\
$\begin{pmatrix}1 & 3 & 1 & 2\\ 1 & 2 & 3 & 4\end{pmatrix}$ & \ytableausetup{smalltableaux}
\ytableaushort{
3,112},\ytableausetup{smalltableaux}
\ytableaushort{
3,124}\\
$\begin{pmatrix}1 & 3 & 2 & 1\\ 1 & 2 & 3 & 4\end{pmatrix}$ & \ytableausetup{smalltableaux}
\ytableaushort{
3,2,11},\ytableausetup{smalltableaux}
\ytableaushort{
4,3,12}
\end{tabular}

\item[(5)] The $2$-line array that corresponds to the pair of Young Tablau is 
$$\begin{pmatrix}
    2&2&3&4&1&3&4&1&1\\
    1&1&1&1&2&2&2&3&3
\end{pmatrix}$$

\end{itemize}

\end{document}






