\documentclass[12pt]{amsart}
% packages
\usepackage{graphicx}
\usepackage{setspace}
\usepackage{amssymb,amsmath,amsthm,amsfonts,amscd}
\usepackage{hyperref}
\usepackage{color}
\usepackage{booktabs}
\usepackage{tabularx}
\usepackage{enumitem}
\usepackage[retainorgcmds]{IEEEtrantools}
\usepackage[notref,notcite,final]{showkeys}
\usepackage[final]{pdfpages}
\usepackage{fancyhdr}
\usepackage{upgreek}
\usepackage{multicol}
\usepackage{fontawesome}
\usepackage{halloweenmath}
\usepackage{ytableau}
% set margin as 0.75in
\usepackage[margin=0.75in]{geometry}

% tikz-related settings
\usepackage{tikz}
\usepackage{tikz-cd}
\usetikzlibrary{cd}

%% For table
\usepackage{tikz}
\usetikzlibrary{tikzmark}

% theorem environments with italic font
\newtheorem{thm}{Theorem}[section]
\newtheorem*{thm*}{Theorem}
\newtheorem{lemma}[thm]{Lemma}
\newtheorem{prop}[thm]{Proposition}
\newtheorem{claim}[thm]{Claim}
\newtheorem{corollary}[thm]{Corollary}
\newtheorem{conjecture}[thm]{Conjecture}
\newtheorem{question}[thm]{Question}
\newtheorem{procedure}[thm]{Procedure}
\newtheorem{assumption}[thm]{Assumption}

% theorem environments with roman font (use lower-case version in body
% of text, e.g., \begin{example} rather than \begin{Example})
\newtheorem{Definition}[thm]{Definition}
\newenvironment{definition}
{\begin{Definition}\rm}{\end{Definition}}
\newtheorem{Example}[thm]{Example}
\newenvironment{example}
{\begin{Example}\rm}{\end{Example}}

\theoremstyle{definition}
\newtheorem{remark}[thm]{\textbf{Remark}}

% special sets
\newcommand{\A}{\mathbb{A}}
\newcommand{\C}{\mathbb{C}}
\newcommand{\F}{\mathbb{F}}
\newcommand{\N}{\mathbb{N}}
\newcommand{\Q}{\mathbb{Q}}
\newcommand{\R}{\mathbb{R}}
\newcommand{\Z}{\mathbb{Z}}
\newcommand{\cals}{\mathcal{S}}
\newcommand{\ZZ}{\mathbb{Z}_{\ge 0}}
\newcommand{\cala}{\mathcal{A}}
\newcommand{\calb}{\mathcal{B}}
\newcommand{\cald}{\mathcal{D}}
\newcommand{\calh}{\mathcal{H}}
\newcommand{\call}{\mathcal{L}}
\newcommand{\calr}{\mathcal{R}}
\newcommand{\la}{\mathbf{a}}
\newcommand{\lgl}{\mathfrak{gl}}
\newcommand{\lsl}{\mathfrak{sl}}
\newcommand{\lieg}{\mathfrak{g}}

% math operators
\DeclareMathOperator{\kernel}{\mathrm{ker}}
\DeclareMathOperator{\image}{\mathrm{im}}
\DeclareMathOperator{\rad}{\mathrm{rad}}
\DeclareMathOperator{\id}{\mathrm{id}}
\DeclareMathOperator{\hum}{[\mathrm{Hum}]}
\DeclareMathOperator{\eh}{[\mathrm{EH}]}
\DeclareMathOperator{\lcm}{\mathrm{lcm}}
\DeclareMathOperator{\Aut}{\mathrm{Aut}}
\DeclareMathOperator{\Inn}{\mathrm{Inn}}
\DeclareMathOperator{\Out}{\mathrm{Out}}
\DeclareMathOperator{\Gal}{\mathrm{Gal}}


% frequently used shorthands
\newcommand{\ra}{\rightarrow}
\newcommand{\se}{\subseteq}
\newcommand{\ip}[1]{\langle#1\rangle}
\newcommand{\dual}{^*}
\newcommand{\inverse}{^{-1}}
\newcommand{\norm}[2]{\|#1\|_{#2}}
\newcommand{\abs}[1]{\lvert #1 \rvert}
\newcommand{\Abs}[1]{\bigg| #1 \bigg|}
\newcommand\bm[1]{\begin{bmatrix}#1\end{bmatrix}}
\newcommand{\op}{\text{op}}

% nicer looking empty set
\let\oldemptyset\emptyset
\let\emptyset\varnothing

%the var phi gang
\let\oldphi\phi
\let\phi\varphi

%\def\darktheme{} % IAN
\ifx \darktheme\undefined
\else
\pagecolor[rgb]{0.2,0.231,0.302}%{0.23,0.258,0.321}
\color[rgb]{1,1,1}
\fi

\def\multiset#1#2{\ensuremath{\left(\kern-.3em\left(\genfrac{}{}{0pt}{}{#1}{#2}\right)\kern-.3em\right)}}

\setlist[enumerate,1]{topsep=1em,leftmargin=1.8em, itemsep=0.5em, label=\textup{(}\arabic*\textup{)}}
\setlist[enumerate,2]{topsep=0.5em,leftmargin=3em, itemsep=0.3em}

%pagestyle
%\pagestyle{fancy} 

\begin{document}
\begin{center}
    \textsc{Math 502. HW 4\\ Ian Jorquera}
\end{center}
\vspace{1em}
% See http://www.mathematicalgemstones.com/maria/Math501Fall22.php
% for problems

% sage: https://sagecell.sagemath.org/
\begin{itemize}
\item[(1,2)]% (1+) [2 points]
Below we have completed inner slides, with $\times$ indicating the inner corner and $\circ$ the outer corner from the result of the previous slide with the final result boxed


\begin{tikzpicture}[commutative diagrams/every diagram]
\node (P0) at (0     *.8-3,14    *.8-1) {\begin{ytableau}
8 & 10 \\
\none & 5 & 5\\
\none & \none & 4 &6 &6 \\
\none & \none & 1 &2 &2 & 3 \\
\none & \none & \none &\none[\times] &1 & 1 &2 & 7 &9
\end{ytableau}};
\node (P1) at (7.01  *.8-1,11.024*.8) {\begin{ytableau}
8 & 10 \\
\none & 5 & 5\\
\none & \none & 4 &6 &6 \\
\none & \none & 1 &2 &2 & 3 \\
\none & \none & \none[\times] &1 & 1 &2 & 7 &9 & \none[\circ]
\end{ytableau}};
\node (P2) at (11.121*.8-1,4.76  *.8) {\begin{ytableau}
8 & 10 \\
\none & 5 & 5\\
\none & \none[\times] & 4 &6 &6 \\
\none & \none & 2 &2 &3 & \none[\circ] \\
\none & \none & 1 &1 & 1 &2 & 7 &9
\end{ytableau}};
\node (P3) at (10.277*.8-1,-3.7  *.8) {\begin{ytableau}
8 & 10 \\
\none & 5 & \none[\circ]\\
\none & 4 & 5 &6 &6 \\
\none & \none[\times] & 2 &2 &3 \\
\none & \none & 1 &1 & 1 &2 & 7 &9
\end{ytableau}};
\node (P4) at (3.533 *.8,-8.971*.8) {\begin{ytableau}
8 & 10 \\
\none & 5\\
\none & 4 & 5 &6 &\none[\circ] \\
\none  & 2 &2 &3 & 6 \\
\none & \none[\times] & 1 &1 & 1 &2 & 7 &9
\end{ytableau}};
\node (P5) at (-5.082*.8,-6.352*.8) {\begin{ytableau}
8 & 10 \\
\none[\times] & 5\\
\none & 4 & 5 &6 \\
\none  & 2 &2 &3 & 6 \\
\none & 1 &1 & 1 &2 & 7 &9 &\none[\circ]
\end{ytableau}};
\node (P6) at (-6.055*.8,2.238 *.8) {\begin{ytableau}
8 & \none[\circ] \\
5 & 10\\
\none[\times] & 4 & 5 &6 \\
\none  & 2 &2 &3 & 6 \\
\none & 1 &1 & 1 &2 & 7 &9
\end{ytableau}};
\node (P7) at (-0.084*.8,4.969 *.8+2) {\begin{ytableau}
8 \\
5 & 10\\
4 & 5 &6 & \none[\circ]  \\
\none[\times]  & 2 &2 &3 & 6 \\
\none & 1 &1 & 1 &2 & 7 &9
\end{ytableau}};
\node (P8) at (3.369 *.8+1,.997  *.8+1) {\begin{ytableau}
8 \\
5 & 10\\
4 & 5 &6  \\
2 &2 &3 & 6& \none[\circ]  \\
\none[\times] & 1 &1 & 1 &2 & 7 &9
\end{ytableau}};
\node (P9) at (-.23  *.8,-1.617*.8-1) {\boxed{\begin{ytableau}
8 \\
5 & 10\\
4 & 5 &6  \\
2 &2 &3 & 6 \\
1 &1 & 1 &2 & 7 &9& \none[\circ] 
\end{ytableau}}};
\path[commutative diagrams/.cd, every arrow, every label]
(P0) edge node {$ $} (P1)
(P1) edge node {$ $} (P2)
(P2) edge node {$ $} (P3)
(P3) edge node {$ $} (P4)
(P4) edge node {$ $} (P5)
(P5) edge node {$ $} (P6)
(P6) edge node {$ $} (P7)
(P7) edge node {$ $} (P8)
(P8) edge node {$ $} (P9);
\end{tikzpicture}

\item[(3)] % [1] (1 point)
We will show this for fixed number of labels $n$. Each row will have a fixed number of each label, and will have a fixed order. So we only have the count the ways to pick $\alpha_i$ things from $n$ categories and so there are $\multiset{n}{\alpha_i}$ ways to fill the $i$th row. And so for the entire horizontal strip there are $\multiset{n}{\alpha_1}\multiset{n}{\alpha_2}\cdots \multiset{n}{\alpha_k}$ options.\\
%Notice that $\lambda/\mu$ being a horizontal strip means that each row $i$ of length $a_i$ is independent of all other rows, meaning the only restriction on each row is that it is weakly increasing. And so we can count the possible young tableau for each row. 

\item[(4)] % [1+] (2 points) 
Notice that $\lambda/\mu$ being a horizontal strip means that each row $i$ of length $\alpha_i$ is independent of all other rows, meaning the only restriction on each row is that it is weakly increasing. The symmetric polynomial for each row is therefore equivalent to counting the number of ways to fill in a semi-standard young Tableau corresponding to the partition $(\alpha_i)$ for each row $i$. This means the corresponding symmetric function is $s_{(\alpha_i)}=h_{\alpha_i}$. And because each row is independent of each other we know the symmetric function for the skew tableau is just the product of the symmetric function of each row $h_{\alpha_1}h_{\alpha_2}\cdots h_{\alpha_k}=h_\alpha=\sum_\nu k_{\nu \alpha}s_\nu$ where the last equality follows from bases conversion formula from class.\\

\item[(5)] % [1+] (2 points) 
Notice that $\lambda/\mu$ being a vertical strip means that each column $i$ of length $\beta_i$ is independent of all other columns, meaning the only restriction on each column is that it is strictly increasing. The symmetric polynomial for each column is therefore equivalent to counting the number of ways to fill in a semi-standard young Tableau corresponding to the partition $(\alpha_i)^t$ for each column $i$. This means the corresponding symmetric function is $s_{(\beta_i)^t}=e_{\beta}$. And because each column is independent of each other we know the symmetric function for the skew tableau is just the product of the symmetric functions of each column $e_{\beta_1}e_{\beta_2}\cdots e_{\beta_k}=e_\beta=\sum_\nu k_{\nu^t \beta} s_\nu$ where the last equality follows from bases conversion formula from class.\\

\item[(6)]
\begin{enumerate}[label=(\alph*)]
    \item % (1) [1 point]
    We want to show that $X_G(x_1,\dots,x_i,x_{i+1},\dots)=X_G(x_1,\dots,x_{i+1}, x_i,\dots)$. It suffices to show that there is a involution between colorings that switches the number of vertices of color $i$ and $i+1$. Consider a coloring on a fixed graph $G$ where $\psi:G\ra \Z^+$. define $f$ to map $i\mapsto i+1$ and $i+1\mapsto i$ and acts as the identity otherwise. Then the involution between colorings is that of composition with $f$ where $\psi\mapsto f\circ\psi$ which is an involution as $f$ is an involution. Consider any two adjacent vertices which have distinct colors $\ell$ and $k$ in $\psi$. Because $f$ is an involution and so injective we know that $f(\ell)\neq f(k)$, and so $f\circ\psi$ is a valid coloring. Finally notice that because we are switching the colors $i$ and $i+1$ we know that any term $x^C$ is $X_G(x_1,\dots)$ has a corresponding term with the number of $x_i$s and $x_{i+1}$s switched.\\

    \item % (1+) [2 points]
    let $K_n$ be the complete graph on $n$ vertices. And consider the symmetric function
    \begin{align*}
        X_{K_n}(x_1,x_2,\dots)&=\sum_{C}x^C
    \end{align*}
    Notice that for any coloring $C$ there would be $n$ distinct colors as $K_n$ has a chromatic number of $n$ and so
    \begin{align*}
        X_{K_n}(x_1,x_2,\dots)=\sum_{C}x^C&=\sum_{\substack{c_1,c_2,\dots, c_n\\c_i\neq c_j\;\; i\neq j}}x_{c_1}x_{c_2}\cdots x_{c_n}\\
        &=n!\sum_{c_1<c_2<\dots<c_n}x_{c_1}x_{c_2}\cdots x_{c_n}=n!e_n
    \end{align*}
In this case we have the $n!$ when we impose an ordering on the colors as there are $n!$ colorings for any $n$ distinct color.
\end{enumerate}
\end{itemize}

\end{document}






