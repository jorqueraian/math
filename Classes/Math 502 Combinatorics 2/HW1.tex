\documentclass[12pt]{amsart}
% packages
\usepackage{graphicx}
\usepackage{setspace}
\usepackage{amssymb,amsmath,amsthm,amsfonts,amscd}
\usepackage{hyperref}
\usepackage{color}
\usepackage{booktabs}
\usepackage{tabularx}
\usepackage{enumitem}
\usepackage[retainorgcmds]{IEEEtrantools}
\usepackage[notref,notcite,final]{showkeys}
\usepackage[final]{pdfpages}
\usepackage{fancyhdr}
\usepackage{upgreek}
\usepackage{multicol}
\usepackage{fontawesome}
\usepackage{halloweenmath}
\usepackage{ytableau}
% set margin as 0.75in
\usepackage[margin=0.75in]{geometry}

% tikz-related settings
\usepackage{tkz-berge}
\usetikzlibrary{calc,quotes}
\usetikzlibrary{arrows.meta}
\usetikzlibrary{positioning, automata}
\usetikzlibrary{decorations.pathreplacing}

%% For table
\usepackage{tikz}
\usetikzlibrary{tikzmark}

% theorem environments with italic font
\newtheorem{thm}{Theorem}[section]
\newtheorem*{thm*}{Theorem}
\newtheorem{lemma}[thm]{Lemma}
\newtheorem{prop}[thm]{Proposition}
\newtheorem{claim}[thm]{Claim}
\newtheorem{corollary}[thm]{Corollary}
\newtheorem{conjecture}[thm]{Conjecture}
\newtheorem{question}[thm]{Question}
\newtheorem{procedure}[thm]{Procedure}
\newtheorem{assumption}[thm]{Assumption}

% theorem environments with roman font (use lower-case version in body
% of text, e.g., \begin{example} rather than \begin{Example})
\newtheorem{Definition}[thm]{Definition}
\newenvironment{definition}
{\begin{Definition}\rm}{\end{Definition}}
\newtheorem{Example}[thm]{Example}
\newenvironment{example}
{\begin{Example}\rm}{\end{Example}}

\theoremstyle{definition}
\newtheorem{remark}[thm]{\textbf{Remark}}

% special sets
\newcommand{\A}{\mathbb{A}}
\newcommand{\C}{\mathbb{C}}
\newcommand{\F}{\mathbb{F}}
\newcommand{\N}{\mathbb{N}}
\newcommand{\Q}{\mathbb{Q}}
\newcommand{\R}{\mathbb{R}}
\newcommand{\Z}{\mathbb{Z}}
\newcommand{\cals}{\mathcal{S}}
\newcommand{\ZZ}{\mathbb{Z}_{\ge 0}}
\newcommand{\cala}{\mathcal{A}}
\newcommand{\calb}{\mathcal{B}}
\newcommand{\cald}{\mathcal{D}}
\newcommand{\calh}{\mathcal{H}}
\newcommand{\call}{\mathcal{L}}
\newcommand{\calr}{\mathcal{R}}
\newcommand{\la}{\mathbf{a}}
\newcommand{\lgl}{\mathfrak{gl}}
\newcommand{\lsl}{\mathfrak{sl}}
\newcommand{\lieg}{\mathfrak{g}}

% math operators
\DeclareMathOperator{\kernel}{\mathrm{ker}}
\DeclareMathOperator{\image}{\mathrm{im}}
\DeclareMathOperator{\rad}{\mathrm{rad}}
\DeclareMathOperator{\id}{\mathrm{id}}
\DeclareMathOperator{\hum}{[\mathrm{Hum}]}
\DeclareMathOperator{\eh}{[\mathrm{EH}]}
\DeclareMathOperator{\lcm}{\mathrm{lcm}}
\DeclareMathOperator{\Aut}{\mathrm{Aut}}
\DeclareMathOperator{\Inn}{\mathrm{Inn}}
\DeclareMathOperator{\Out}{\mathrm{Out}}
\DeclareMathOperator{\Gal}{\mathrm{Gal}}


% frequently used shorthands
\newcommand{\ra}{\rightarrow}
\newcommand{\se}{\subseteq}
\newcommand{\ip}[1]{\langle#1\rangle}
\newcommand{\dual}{^*}
\newcommand{\inverse}{^{-1}}
\newcommand{\norm}[2]{\|#1\|_{#2}}
\newcommand{\abs}[1]{\lvert #1 \rvert}
\newcommand{\Abs}[1]{\bigg| #1 \bigg|}
\newcommand\bm[1]{\begin{bmatrix}#1\end{bmatrix}}
\newcommand{\op}{\text{op}}

% nicer looking empty set
\let\oldemptyset\emptyset
\let\emptyset\varnothing

%the var phi gang
\let\oldphi\phi
\let\phi\varphi

\setlist[enumerate,1]{topsep=1em,leftmargin=1.8em, itemsep=0.5em, label=\textup{(}\arabic*\textup{)}}
\setlist[enumerate,2]{topsep=0.5em,leftmargin=3em, itemsep=0.3em}

%pagestyle
%\pagestyle{fancy} 

\begin{document}
\begin{center}
    \textsc{Math 502. HW 1\\ Ian Jorquera}
\end{center}
\vspace{1em}
% See http://www.mathematicalgemstones.com/maria/Math501Fall22.php
% for problems

% sage: https://sagecell.sagemath.org/
\begin{itemize}

\item[(1)]  %(2) [3 points]
let $p(x,y)$ be a polynomial over a field $k$ of characteristic not equal to $2$. Notice first that the space of polynomials $k[x,y]$ has as a basis $\{x^ny^m| n,m \in \Z^{\geq 0}\}$. We will show that each element in the basis can be uniquiely written as a sum of a symmetric polynomial and an antisymmetric polynomial. First for $x^ny^m$ where $n>m$ notice that $x^ny^m=\frac{1}{2}(m_{(n,m)}+a_{(n,m)})=\frac{1}{2}((x^ny^m+x^my^n)+(x^ny^m-x^my^n))$. And similarly for $n<m$ we have that $x^ny^m=\frac{1}{2}(m_{(m,n)}-a_{(m,n)})=\frac{1}{2}((x^my^n+x^ny^m)-(x^my^n-x^ny^m))$. And finally for the case that $n=m$ we have that $x^ny^n$ is symmetric. And so we can then write $p(x,y)=sym(x,y)+asym(x,y)$ where $sym(x,y)$ is a symmetric polynomial, the sum of the symmetric monomial basis terms, and $asym(x,y)$ is an anti-symmetric polynomial, the sum of the anti symmetric monomial basis terms. To see that this is unique notice that if $p(x,y)=sym(x,y)+asym(x,y)=sym'(x,y)+asym'(x,y)$ then $sym(x,y)-sym'(x,y)+asym(x,y)-asym'(x,y)=0$ and so $sym(x,y)-sym'(x,y)=asym'(x,y)-asym(x,y)$. And because the only polynomial that is both symmetric and antisymmetric is the zero polynomial we know that $sym'(x,y)-sym(x,y)=0$ and $asym'(x,y)-asym(x,y)=0$ so therefore $sym(x,y)=sym'(x,y)$ and $asym'(x,y)=asym(x,y)$.\\

\item[(2)] % (1+) [2 points]
We can use the relation $p_3-e_2p_1+e_1p_2-3e_3=0$ and so $p_3=e_2e_1-e_1p_2+3e_3$. We can also substitute using the relations $p_2-e_1p_1+2e_2=0$ where $p_2=e_1e_2-2e_2$ and so $p_3= e_2e_1-e_1(e_1e_2-2e_2)+3e_3$.\\

\item[(3)] % (1+) [2 points]
Using the involution discussed in class we have that $\omega(p_3)= \omega(e_2e_1-e_1(e_1e_2-2e_2)+3e_3)=h_2h_1-h_1(h_1h_2-2h_2)+3h_3$. We also know that $\omega(p_3)=(-1)^{3-1}p_3=p_3$. So $p_3=h_2h_1-h_1(h_1h_2-2h_2)+3h_3$.\\

\item[(4)] Using sage we can determine that $e_{(3,2,2)}=s_{(1, 1, 1, 1, 1, 1, 1)} + 2s_{(2, 1, 1, 1, 1, 1)} + 3s_{(2, 2, 1, 1, 1)} + 2s_{(2, 2, 2, 1)} + s_{(3, 1, 1, 1, 1)} + 2s_{(3, 2, 1, 1)} + s_{(3, 2, 2)} + s_{(3, 3, 1)}$ and so is Schur positive.\\\\
$h_{(3,2,2)}=s_{(3, 2, 2)} + s_{(3, 3, 1)} + 2s_{(4, 2, 1)} + 2s_{(4, 3)} + s_{(5, 1, 1)} + 3s{(5, 2)} + 2s_{(6, 1)} + s_{(7)}$ and so is also Schur positive.\\\\
Sage give the following output for $p_{(3,2,2)}=s[1, 1, 1, 1, 1, 1, 1] - s[2, 1, 1, 1, 1, 1] + 2*s[2, 2, 1, 1, 1] - s[2, 2, 2, 1] - s[3, 1, 1, 1, 1] - s[3, 2, 1, 1] + s[3, 2, 2] + s[3, 3, 1] + 2*s[4, 1, 1, 1] - s[4, 2, 1] - s[4, 3] - s[5, 1, 1] + 2*s[5, 2] - s[6, 1] + s[7]$ and so is not Schur positive.\\\\
Finally $m_{(3,2,2)}$ is $3*s[1, 1, 1, 1, 1, 1, 1] - 3*s[2, 1, 1, 1, 1, 1] + 2*s[2, 2, 1, 1, 1] - s[2, 2, 2, 1] + s[3, 1, 1, 1, 1] - s[3, 2, 1, 1] + s[3, 2, 2]$ and so is also not Schur positive.\\


\item[(5)] % (1+) [2 points]
First with the algebraic definition we have that $$s_{(2,1)}(x_1,x_2,x_3)=\frac{a_{(4,2,0)}}{a_{(2,1,0)}}=\frac{\det\begin{pmatrix} x_1^4 & x_1^2 & 1\\ x_2^4 & x_2^2 & 1\\ x_3^4 & x_3^2 & 1\end{pmatrix}}{\det\begin{pmatrix} x_1^2 & x_1 & 1\\ x_2^2 & x_2 & 1\\ x_3^2 & x_3 & 1\end{pmatrix}}$$
and using matlab to compute determinants and to simplify terms we can determine that this is equal to $x_1^2x_2 + x_1^2x_3 + x_1x_2^2 + 2x_1x_2x_3 + x_1x_3^2 + x_2^2x_3 + x_2x_3^2$. Now with the young tableaux we need to consider the semi-standard young tableaux with shape $(2,1)$. These are 
\ytableausetup{smalltableaux}
\ytableaushort{
2,11}
\ytableausetup{nobaseline}
\ytableaushort{
2,12}
\ytableausetup{nobaseline}
\ytableaushort{
2,13}
\ytableausetup{nobaseline}
\ytableaushort{
3,12}
\ytableausetup{nobaseline}
\ytableaushort{
3,13}
\ytableausetup{nobaseline}
\ytableaushort{
3,22}
\ytableausetup{nobaseline}
\ytableaushort{
3,23}\;\;and so the Schur polynomial is $x_1^2x_2 + x_1^2x_3 + x_1x_2^2 + 2x_1x_2x_3 + x_1x_3^2 + x_2^2x_3 + x_2x_3^2$ which coincides with the previous definition.

\end{itemize}

\end{document}






