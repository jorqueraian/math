\subsection{Step 1: A Motivating Example: Matrices with Geometry}
Matrices are cool. Thats it! Thats all the motivation I need.
Groups can be used to talk about the algebra of multiplying 
invertible matrices with common dimensions, with the General Linear group.
Groupoids can be used to work with all square matrices. And a Category can be 
used to study the behavior of matrix multiplication in general. However these 
three algebra leave gaps, categories and groupoids can often be too broad, allowing
for partially defined multiplications while groups and groupoids are too narrow, allowing only 
invertible matrices.

\subsubsection*{A motivating example for Jordan Pairs}
Often one may want to study rectangular matrices. For example Let $V^+$ be the 
$k$-module of $q\times p$ matrices $M_{q,p}(k)$ and $V^-$ be the 
$k$-module of $p\times q$ matrices $M_{p,q}(k)$. These matrices are clearly compatible under 
multiplication but there is a problem! If $A\in V^+$ and $B\in V^-$, then their product is in neither
of the original modules. This requires a new operator, a trinary operation where if $A,C\in V^+$ and $B\in V^-$
then $ABC\in V^+$ and vice-versa. In fact this trinary operator would be tri-linear.
This would give us a nice algebra, but it has a problem, it does not behave nicely with geometry.
Geometry often looks at bilinear forms which can be defined by a transpose. For an multiplication operator to
"behave nicely" often means that that the transpose respects multiplication (that is the transpose and the 
triple multiplication should be commuting operators).
However notice with this simple triple product, we have that $ABC$ is bilinear in $A$ and $C$ but notice that 
under a transpose $(ABC)^t=C^tB^tA^t$ we have a commuting issue. And so instead we want to define an operator
that is symmetric in $A$ and $C$. An easy way to do this is to define an operator $\{A,B,C\}=ABC+CBA$. This gives us
$\{A,B,C\}C^tB^tA^t+=A^tB^tC^t=\{A^t,B^t,C^t\}$ as desired.

\subsubsection*{A motivating example for Jordan Algebras}
\label{sec:symmats}
A simpler example of this commuting issue is with symmetric(self-adjoint) matrices which is our second motivating example, where $A=A^t$ and $B=B^t$. 
If we had a bilinear form: $\ip{x|y}=x^ty$ we could study the self adjoint elements $\ip{Ax|y}=\ip{x|A^ty}$. 
But we run into the exact same problem: that symmetric matrices are generally not closed under multiplication
$(AB)^t=B^tA^t=BA$. And restricting the study of matrices to commutative ones is not a reasonable ask! 
So again the fix is to invent a new matrix multiplication that is symmetric and still does all the same things.
