\subsection{Step 3: Studying The Algebra}
\subsubsection*{Jordan's Web}
Now that we have created the algebra of Jordan Pairs we need to study it. An important notion to study is that of invertibility.
At first glance this may be tricky as Jordan Pairs have no classical binary product. Instead Jordan Pairs have quadratic map $Q_\sigma(x)y$
that acts as a two-sided product: the product of $x$ on both sides of $y$, at least this is the intuition that our example gives us.
This gives us a notion of inverses, that of inverting the two-sided product $Q_\sigma(x)$.
Furthermore an inverse $x^{-1}$ of an element $x$ should satisfy $Q_{-\sigma}(x^{-1})Q_\sigma(x)y=y$.
And Using JP3 we can determine that $Q_\sigma(x)y=Q_\sigma(x)Q_{-\sigma}(x^{-1})Q_\sigma(x)y=Q_\sigma(Q_\sigma(x)x^{-1})y$.
 And so $Q_\sigma(x)x^{-1}=x$ and therefore we have the following definition

\begin{definition}
    Let $V$ be a Jordan Pairs then an element $u\in V^\sigma$ is called invertible if 
    quadratic product $Q_\sigma(u):V^{-\sigma}\ra V^{\sigma}$ is invertible. In this case 
    $u^{-1}=Q_\sigma(u)^{-1}(u)$
\end{definition}

In which case we have that $Q_\sigma(u)^{-1}=Q_{-\sigma}(u^{-1})$ as desired and $(u^{-1})^{-1}$.
Invertible elements are generally rare in Jordan Pairs but have a nice correspondence with unital Jordan Algebras

\begin{prop}
    Let $V$ be a Jordan Pair with invertible element $v\in V^{-}$ and $u=v^{-1}\in V^{+}$. Then the 
    corresponding Jordan Algebras $J=V^{+}_v$ and $J'=V^-_u$ are unital with units $u$ 
    and $v$ respectively. Furthermore $J$ and $J'$ are isomorphism with $Q_-(v):J\ra J'$ and inverse $Q_+(u):J'\ra J$. Additionally
    the map $(\text{Id}_J, Q_{-}(v)):(J,J)\ra (V^+,V^-)$ is an isomorphism of Jordan Pairs.
\end{prop}

This gives a powerful connection between the study of Jordan Pairs with invertible elements
and Unital Jordan Algebras.


\subsubsection{Centering in on the Numbers}
A common technic in the study of algebra is that of extending coefficients.
As an example consider a $\Omega$-module $V$, where $\Omega$ has no required structure, given by the action of 
$\cdot:\Omega\times V\ra V$ that respects the structure of $V$. However $\Omega$ may not
represent all the actions of $V$ and so often it is helpful to extent $\Omega$ to include all the actions of $V$ that respect the structure of $V$ and the action of $\Omega$.
Notice also that actions that preserve the original actions of $\Omega$ commute, and so such an extension of scalar comes down to a collection of commuting requirements.
These new scalars are the endomorphisms $\text{End}(_\Omega V)$ endowing $V$ with a $\text{End}(_\Omega V)$-module with nicer properties.
%The motivation is that often in the study of algebraic structure you may start 
%with an important collection of scalars but 


Let $V=(V^+,V^-)$ be a Jordan pair over $k$, where each $V^\sigma$ is a $k$-module. 
Consider first $\text{End}(V^+)\times\text{End}(V^-)$, the endomorphisms of each $V^\sigma$ preserving the $k$-module structures,
which contains all automorphism of $V$.
However for a map $a=(a_+,a_-)\in\text{End}(V^+)\times\text{End}(V^-)$ to respect the structure of $V$ we would need the following commuting requirements
First we need that the action of $a$ commutes with the two-sided quadratic map in the sense that 
$a_\sigma Q_\sigma(x)y=Q_\sigma(x)a_{-\sigma}(y)$. Similarly we need that 
$a_\sigma(\{x,y,z\})=\{x,y,a_\sigma(z)\}$. These requirements define what is know as the centroid.

\begin{definition}
    Let $V$ be a Jordan Pair over $k$. The Centroid of $V$ denoted as $Z(V)$ is 
    the collection of elements $a=(a_+,a_-)\in\text{End}(V^+)\times\text{End}(V^-)$ such that
    \begin{equation}
        \tag{Z1}
        a_\sigma Q_\sigma(x)y=Q_\sigma(x)a_{-\sigma}(y)
    \end{equation}
    \begin{equation}
        \tag{Z2}
        a_\sigma(\{x,y,z\})=\{x,y,a_\sigma(z)\}
    \end{equation}
    \begin{equation}
        \tag{Z3}
        Q_{\sigma}(a_{\sigma}(x))=a_{\sigma}^2Q_{\sigma}(x)
    \end{equation}
\end{definition}
Notice that requirement Z3 is only needed in the case of characteristic $2$, otherwise Z3 follows from Z2 and the symmetry of the trilinear operator. And likewise the identities
$a_\sigma(\{x,y,z\})=\{a_\sigma(x),y,z\}$ and $a_\sigma(\{x,y,z\})=\{x,a_{-\sigma}(y),z\}$ follow from the above.

The centroid represents the possible actions on $V$ that preserve its structure, 
meaning the centroid may be an extension of the underlying scalars from $k$. To see this
Notice that any scalar $\lambda\in k$ has the corresponding element $(\lambda\text{Id},\lambda\text{Id})\in Z(V)$ 
which is exactly that of scalar multiplication from $k$. This suggests that $Z(V)$ 
knows about the additional scalars that may exist for $V$.

\begin{definition}
    A Jordan Pair $V$ is called central if every $a\in Z(V)$ is of the form $a=(\lambda\text{Id},\lambda\text{Id})$ for some $\lambda\in k$
\end{definition}
Central Jordan Pairs are Jordan Pairs where $k$ already contains all the scalars or actions on $V$ that preserve the structure.


Although $Z(v)$ can bring in new scalars or new actions on $V$ the resulting algebra may no longer be a Jordan Pair. 
$Z(V)$ is often not commutative or even a subalgebra of $\text{End}(V^+)\times\text{End}(V^-)$, meaning $Z(V)$ may not be a scalar extension of $k$.
% recal that Jordan Pair needs k to be commutative unital ring
% why does this matter. Is there something critically important that this breaks when Z(V) isnt commutative 

\begin{prop}
    Let $V$ be a Jordan Pair with $Z(V)$ a commutative subalgebra of 
    $\text{End}(V^+)\times\text{End}(V^-)$. Then $V$ can be a Jordan Pair over the 
    extension $Z(V)$. % im not sure what i am saying.
\end{prop}

Furthermore the structure of $Z(V)$ depends on the structure of $V$, specifically the ideals of $V$.

% other prop. but i dont understand it. it may have a typo but also maybe they are just defining things different and if it 
% is a typo this next prop is enough.

\begin{prop}
    Let $V$ be a Jordan Pair. If $V$ is simple then $Z(V)$ is an extension field of $k$.
\end{prop}

\subsection{Pseudo-Inverses} And application to differential geometry I assume? or some other flavor of geometry