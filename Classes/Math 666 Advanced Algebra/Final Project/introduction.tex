\section{Introduction}
This paper is meant as a way of documenting my understanding of 
algebra with an emphasis on how and why algebra is the way it is, 
and so this document may be obvious or even outright insulting to a trained algebraist.
I've come to the conclusion the conclusion that algebra is a 4+ step process with a heavy amount of over lap between the steps:
\begin{itemize}
\item[]\hspace{-20pt}Step 1) Understanding Examples: Find examples of algebra in the real word. 
\item[]\hspace{-20pt}Step 2) Building Algebra: Using your examples (a) find operators, or a grammar and (b) find identities that your examples follow and make them the laws of your algebra. (c) do Noether's Isomorphism theorem connecting congruences, homomorphisms and quotients.
\item[]\hspace{-20pt}Step 3) Studying Algebra: Study your newly built algebra % by (a) Studying the ideal, homomorphism and other important object, this often involves studying the "simples" of your algebra. (b) Studying the "free" thing. (c) studying the representations.
\item[]\hspace{-20pt}Step 4) Repeat: Study your examples again with new insights or build new algebra from the algebra you created in the previous steps
\end{itemize}

This paper is roughly outlined to follow these steps for Jordan Pairs and Jordan Algebras.

% Ok so i should mention that the reason why Q and D are used ove {} is becasue of associativity.
% {} is very non-asc while Q and D are in a sense as associative as you can get. reap taste of Jordan alg book for more motivation
% but this is why Q and D are used over {}. Event though {} has a more obvious meaning and connection to triple systems.