\section{Introduction}
This paper is meant as a way of documenting my understanding of 
algebra with an emphasis on how and why algebra is the way it is, 
and so may be obvious or even outright insulting to a trained algebraist.
I've come to the conclusion the conclusion that Algebra is a 4 step process:
Step 1) Find example of algebra in the real word. 
Step 2) Using your examples (a) find operators, or a grammar and (b) find laws that your operators seem to follow your examples. This step we create the "free" algebra
Step 3) Study your newly built algebra by (a) studying the "simple" things. (b) study the "free" thing. (c) study the representations.
step 4) idk study your examples again with new insights or don't.

\subsection{Step 1: A Motivating Example}
Matrices are cool. Thats it! Thats all the motivation I need.
Groups can be used to talk about the algebra of multiplying 
invertible matrices with common dimensions, with the General Linear group.
Groupoids can be used to work with all square matrices. And a Category can be 
used to study the behavior of matrix multiplication in general. However these 
three algebra leave gaps, categories and groupoids can often be too broad, allowing
for partially defined multiplications while group are too narrow, allowing only 
invertible matrices.

Often one may want to study rectangular matrices. For example Let $V^+$ be the 
$k$-module of $q\times p$ matrices $M_{q,p}$ and $V^-$ be the 
$k$-module of $p\times q$ matrices $M_{p,q}$. These matrices are clearly compatible under 
multiplication but there is a problem! If $A\in V^+$ and $B\in V^-$, then their product is in neither
of the original modules. This requires a new operator, a trinary operation where if $A,C\in V^+$ and $B\in V^-$
then $ABC\in V^+$ and vice-versa. In fact this trinary operator would be tri-linear.
This would give us a nice algebra, but it has a problem, it does not behave nicely with geometry.
Geometry often looks at bilinear forms which can often be defined by a transpose. 
However notice with this simple triple product we have that $ABC$ is bilinear in $A$ and $C$ but notice that 
under a transpose $(ABC)^t=C^tB^tA^t$ we have a commuting issue. And so instead we want to define an operator
that is symmetric in $A$ and $C$. An easy way to do this is to define an operator $\{A,B,C\}=ABC+CBA$.


A simpler example of this commuting issue is with symmetric(self-adjoint) matrices, where $A=A^t$ and $B=B^t$. 
If we had a bilinear form: $\ip{x|y}=x^ty$ we could study the self adjoint elements $\ip{Ax|y}=\ip{x|A^ty}$. 
But we run into the exact same problem: that symmetric matrices are generally not closed under multiplication
$(AB)^t=B^tA^t=BA$. And restricting you study of matrices to commutative ones is not a reasonable ask! 
So again the fix is to invent a new matrix multiplication that is symmetric and still does all the same things.


\subsection{Step 2: Fumbling to a Definition}
We are not saying Ottmar Loos Fumbled to the definitions we will present below 
(They are simply too nice and also intricate for that to be the case) but at some point to do
algebra you have to just fumble to something that looks good enough, and hopefully in retrospect you can convince yourself
why your algebra is a good one.

As mentioned above we want to create an algebra that looks at how pairs of modules interact, through a symmetric 
trilinear operator that we will think about as being our triple matrix product symmetricified.
So we need to define our two $k$-modules with our trilinear operator that goes between them

\begin{align*}\ip{\ip{V^+}}::&= 0_+ \\
                             &| \ip{V^+}+\ip{V^+} \\
                             &| -\ip{V^+} \\
                             &| \ip{k}\cdot \ip{V^+}\\ 
                             &| \{\ip{V^+}, \ip{V^-}, \ip{V^+}\}\end{align*}
\begin{align*}\ip{\ip{V^-}}::&= 0_- \\
    &| \ip{V^-}+\ip{V^-} \\
    &| -\ip{V^-} \\
    &| \ip{k}\cdot \ip{V^-} \\
    &| \{\ip{V^-}, \ip{V^+}, \ip{V^-}\}\end{align*}
 
And then we can define the grammar for the algebra of the pair of modules. This is the signature for a Jordan pair over $k$

\[\ip{\ip{\text{JP}_k}}::= \ip{V^+} | \ip{V^-}\]

Now that we know what operators we have in our world we need laws to tell us what these operators mean.
First we need the laws that tell us that $V^+$ and $V^-$ are in fact $k$-modules. 
And we need the laws that tell us $\{-,-,-\}:V^\sigma\times V^{-\sigma}\times V^\sigma\ra V^\sigma$ is a trilinear operator and symmetric in the first and third terms. 
From these Laws we can define additional operators: Let $Q_\sigma:V^\sigma\ra \hom(V^{-\sigma},V^{\sigma})$ 
be the Quadratic map defined as $2Q_\sigma(x)y = \{xyx\}$, and its corresponding bilinear map 
$Q_\sigma:V^\sigma\times V^\sigma\ra \hom(V^{-\sigma},V^{\sigma})$ defined as 
$Q_\sigma(x,z)=Q_\sigma(x+z)-Q_\sigma(x)-Q_\sigma(z)$. TODO make sure this actually works. I think i might have an off by 1/2 or 2.
We can also define the bilinear map $D_\sigma(x,y)z=Q_\sigma(x,z)y$
Notice that in the example in section we would have that $Q_\sigma(x)y=xyx$.

Now that we have our algebra with some initial laws we need to build some more laws. Consider the formulas of $\text{JP}_k$
We will build following laws
\begin{equation}
    \{x,y,Q_\sigma(x)z\}=Q_\sigma(x)\{y,x,z\}
\end{equation}
\begin{equation}
    \{Q_\sigma(x)y,y,z\}=\{x,Q_{-\sigma}(y)x,z\}
\end{equation}
\begin{equation}
    Q_{\sigma}(Q_{\sigma}(x)y)z=Q_{\sigma}(x)(Q_{-\sigma}(y)(Q_{\sigma}(x)z))
\end{equation}


Combining these laws we have the following definition of a Jordan Pair

\begin{definition}
    The pair $V=(V^+,V^-)$ is called a Jordan pair if $V\vDash_{\text{JP}_k}\mathcal{L}$ where $\mathcal{L}$ 
    are the laws outlined above. More Concisely we may say

    give a more readable definition
\end{definition}