\section{Introduction}
This paper is meant as a way of documenting my understanding of 
algebra with an emphasis on how and why algebra is the way it is, 
and so this document may be obvious or even outright insulting to a trained algebraist.
I've come to the conclusion the conclusion that Algebra is a 4+ step process:
Step 1) Find example of algebra in the real word. 
Step 2) Using your examples (a) find operators, or a grammar and (b) find laws that your operators seem to follow your examples. This step we create the "free" algebra. In this step will will also homomorphisms and ideals.
Step 3) Study your newly built algebra by (a) Studying the ideal, homomorphism and other important object, this often involves studying the "simples" of your algebra. (b) Studying the "free" thing. (c) studying the representations.
step 4) Study your examples again with new insights or build new algebra from the algebra you created in the previous steps


This paper is roughly outlined to follow these steps for the following two algebraic structure, Jordan pairs and Jordan Algebra.
\subsection{Step 1: A Motivating Example}
Matrices are cool. Thats it! Thats all the motivation I need.
Groups can be used to talk about the algebra of multiplying 
invertible matrices with common dimensions, with the General Linear group.
Groupoids can be used to work with all square matrices. And a Category can be 
used to study the behavior of matrix multiplication in general. However these 
three algebra leave gaps, categories and groupoids can often be too broad, allowing
for partially defined multiplications while group are too narrow, allowing only 
invertible matrices.

\subsubsection*{A motivating example for Jordan Pairs}
Often one may want to study rectangular matrices. For example Let $V^+$ be the 
$k$-module of $q\times p$ matrices $M_{q,p}$ and $V^-$ be the 
$k$-module of $p\times q$ matrices $M_{p,q}$. These matrices are clearly compatible under 
multiplication but there is a problem! If $A\in V^+$ and $B\in V^-$, then their product is in neither
of the original modules. This requires a new operator, a trinary operation where if $A,C\in V^+$ and $B\in V^-$
then $ABC\in V^+$ and vice-versa. In fact this trinary operator would be tri-linear.
This would give us a nice algebra, but it has a problem, it does not behave nicely with geometry.
Geometry often looks at bilinear forms which can often be defined by a transpose. 
However notice with this simple triple product we have that $ABC$ is bilinear in $A$ and $C$ but notice that 
under a transpose $(ABC)^t=C^tB^tA^t$ we have a commuting issue. And so instead we want to define an operator
that is symmetric in $A$ and $C$. An easy way to do this is to define an operator $\{A,B,C\}=ABC+CBA$.

\subsubsection*{A motivating example for Jordan Algebras}
A simpler example of this commuting issue is with symmetric(self-adjoint) matrices which is our second motivating example, where $A=A^t$ and $B=B^t$. 
If we had a bilinear form: $\ip{x|y}=x^ty$ we could study the self adjoint elements $\ip{Ax|y}=\ip{x|A^ty}$. 
But we run into the exact same problem: that symmetric matrices are generally not closed under multiplication
$(AB)^t=B^tA^t=BA$. And restricting the study of matrices to commutative ones is not a reasonable ask! 
So again the fix is to invent a new matrix multiplication that is symmetric and still does all the same things.


\subsection{Step 2: Fumbling to a Definition}
\subsubsection{Jordan Pairs}
We are not saying Ottmar Loos Fumbled to the definitions we will present below 
(They are simply too nice and also intricate for that to be the case) but at some point to do
algebra you have to just fumble to something that looks good enough, and hopefully in retrospect you can convince yourself
why your algebra is a good one.

As mentioned above we want to create an algebra that looks at how pairs of modules interact, through a symmetric 
trilinear operator that we will think about as being our triple matrix product symmetricified.
So we need to define our two $k$-modules with our trilinear operator that goes between them

\begin{align*}\ip{\ip{V^+}}::&= 0_+ \\
                             &| \ip{V^+}+\ip{V^+} \\
                             &| -\ip{V^+} \\
                             &| \ip{k}\cdot \ip{V^+}\\ 
                             &| \{\ip{V^+}, \ip{V^-}, \ip{V^+}\}\end{align*}
\begin{align*}\ip{\ip{V^-}}::&= 0_- \\
    &| \ip{V^-}+\ip{V^-} \\
    &| -\ip{V^-} \\
    &| \ip{k}\cdot \ip{V^-} \\
    &| \{\ip{V^-}, \ip{V^+}, \ip{V^-}\}\end{align*}
 
And then we can define the grammar for the algebra of the pair of modules. This is the signature for a Jordan pair over $k$

\[\ip{\ip{\text{JP}_k}}::= \ip{V^+} | \ip{V^-}\]

Now that we know what operators we have in our world we need laws to tell us what these operators mean and how we can rewrite expressions.
First we need the laws that tell us that $V^+$ and $V^-$ are in fact $k$-modules.
And we also need the laws that tell about our triple product, that $\{-,-,-\}:V^\sigma\times V^{-\sigma}\times V^\sigma\ra V^\sigma$ is a trilinear operator and symmetric in the first and third terms. 

From these laws we can create additional operators: for $\sigma=\pm$ let
\[Q_\sigma:V^\sigma\ra \hom(V^{-\sigma},V^{\sigma})\text{ be quadratic maps given by }2Q_\sigma(x)y = \{xyx\}\]
We can also create their corresponding symmetric bilinear maps 
$Q_\sigma:V^\sigma\times V^\sigma\ra \hom(V^{-\sigma},V^{\sigma})$ which are given by 
$Q_\sigma(x,z)=Q_\sigma(x+z)-Q_\sigma(x)-Q_\sigma(z)$. %TODO make sure this actually works. I think i might have an off by 1/2 or 2.
Finally we will create the bilinear maps $D_\sigma(x,y)z=Q_\sigma(x,z)y$
Notice that in the example in the section above we would have that $Q_\sigma(x)y=xyx$.

Now that we have our algebra with some initial laws we need to build some more laws using the formulas of $\text{JP}_k$.
These laws are created to mirror the behavior of our motivating examples.
We will build the following laws:
\begin{equation}
    \tag{JP1}
    \{x,y,Q_\sigma(x)z\}=Q_\sigma(x)\{y,x,z\}
\end{equation}
\begin{equation}
    \tag{JP2}
    \{Q_\sigma(x)y,y,z\}=\{x,Q_{-\sigma}(y)x,z\}
\end{equation}
\begin{equation}
    \tag{JP3}
    Q_{\sigma}(Q_{\sigma}(x)y)z=Q_{\sigma}(x)(Q_{-\sigma}(y)(Q_{\sigma}(x)z))
\end{equation}


Combining these laws we have the following definition of a Jordan Pair

\begin{definition}
    The pair $V=(V^+,V^-)$ is called a Jordan pair if $V\vDash_{\text{JP}_k}\mathcal{L}$ where $\mathcal{L}$ 
    are the laws outlined above. More Concisely we may say that a pair $V=(V^+,V^-)$ of $k$-modules
    along with a pair of quadratic maps $(Q_+,Q_-)$, which define the trilinear operator $\{-,-,-\}$, 
    is called a Jordan pair if JP1-JP3 hold in all scalar extensions of $V$
    \begin{equation}
        \tag{JP1}
        \{x,y,Q_\sigma(x)z\}=Q_\sigma(x)\{y,x,z\}
    \end{equation}
    \begin{equation}
        \tag{JP2}
        \{Q_\sigma(x)y,y,z\}=\{x,Q_{-\sigma}(y)x,z\}
    \end{equation}
    \begin{equation}
        \tag{JP3}
        Q_{\sigma}(Q_{\sigma}(x)y)z=Q_{\sigma}(x)(Q_{-\sigma}(y)(Q_{\sigma}(x)z))
    \end{equation}
\end{definition}
Note that we will often represent an element of a Jordan Pair as $x\in V$ without directly 
specifying which module $x$ is a member of. And in general a pair of objects signifies that 
objects may come from either of terms as in ...
Now that we have a definition for our algebra we can define what it means to have sub-objects, 
homomorphisms, congruences, quotients and ideals. None of the following are particularly surprising (You see what I did there? Im acting like a trained algebraist)
\begin{definition}
    A homomorphism $h:V\ra W$ of Jordan Pairs is a pair of maps $(h_+,h_-)$ such that $h_\sigma:V^\sigma\ra W^\sigma$ is a 
    $k$-module homomorphism such that the following holds
    \begin{equation}
        h_{\sigma}(\{x,y,z\})=\{h_{\sigma}(x),h_{-\sigma}(y),h_{\sigma}(z)\}
    \end{equation}
    This condition is equivalent to 
    \begin{equation}
        h_{\sigma}(Q_{\sigma}(x)y)=Q_{\sigma}(h_{\sigma}(x))h_{-\sigma}(y)
    \end{equation}
\end{definition}


\begin{definition}
    Let $h=(h_+,h_-)$ be a homomorphism of Jordan Pairs then the kernel of $h$ is the pair 
    $\ker(h)=(\{x\equiv z \pmod h | h_+(x)=h_+(z)\},\{x\equiv z \pmod h | h_-(x)=h_-(z)\})$
    More concisely we will write $\ker(h)=\{x\equiv z \pmod h | h(x)=h(z)\}\se V\times V$
\end{definition}

The kernel of a homomorphism $h:V\ra W$ is the resulting congruence on $V$ that relates elements by their outputs in the
homomorphism $h$. We know from Noether's Isomorphism theorem that the kernels of $V$ represent all the congruences on $V$.


Because of the underlying module structure of $V^\sigma$ we have the following observation for 
congruence of $V$: that $x\equiv z \Leftrightarrow x-z\equiv 0_\sigma$. 
This means that congruences of Jordan Pairs can be identified with the sub-object 
$()\{x\in V | x\equiv 0_\sigma\},\{x\in V | x\equiv 0_\sigma\})\se V$ up to univalence. 
In the context of kernels this means that
$(x,z)\in\ker(h)$ if and only if $h_\sigma(x-z)=0_\sigma$.
This means the Kernel of a homomorphism of Jordan Pairs can be identified with 
$\{x\in V | h_\sigma(x)=0_\sigma\}$.
We will call this type of sub-object an Ideal of $V$.

\begin{definition}
    $I=(I^+,I^-)$ is an ideal of $V$ if there exists a congruence $\equiv$ on $V$ 
    such that $x\equiv z$ if and only if $x-z\in I$. That is $I$ is an ideal if it 
    can be identified with some congruence on $V$. 
    In terms of kernels this means that 
    $I=(I^+,I^-)$ is an ideal of $V$ if there exists a homomorphism $h: V\ra W$ 
    such that $(x,z)\in \ker(h)$ if and only if $x-z\in I$. That is $I$ is an ideal if it 
    can be identified with the kernel of some homomorphism. 
\end{definition}
\begin{definition}
    Let $V$ be a Jordan Pair and $I$ an ideal then then quotient 
    $V/I$ is the partitions of $V$ defined by the congruence 
    corresponding to $I$. That is $V/I=(V^+/I^+,V^-/I^-)=(\{v+I^+|v\in V^+\},\{v+I^-|v\in V^-\})$
\end{definition}
We will henceforth associate congruences with their corresponding ideal, that is $\ker(h)=\{x\in V | h(x)=0_\sigma\}$.
This gives us a special case of Noethers Isomorphism theorem 
relating homomorphism, ideals and Quotients 
\begin{prop}
    Let $I\se V$ then the following are equivalent
    \begin{itemize}
    \item $I$ is an ideal of $V$
    \item $V/I$ is a Jordan Pair
    \item $h:V\ra V/I$ is a epimorphism with $\ker(h)=I$
    \end{itemize}
\end{prop}
This allows us a characterization of Ideals of Jordan pairs using the the equivalence of Ideals and Quotients being Jordan Pairs, and so we need only check the conditions needed on $I$ for $V/I$ to be a Jordan Pair.
\begin{prop}
$I\se V$ is an ideal of $V$ if and only if $Q_\sigma(I^\sigma)V^{-\sigma}+Q_\sigma(V^\sigma)I^{-\sigma}+\{V^\sigma,V^{-\sigma},U^\sigma\}\se U^\sigma$
\end{prop}
An important part of step 3, is the study of simples. As an important observation is that any Jordan Pair can be the image of some homomorphism, 
or in other words any Jordan pair can be the quotient of another Jordan Pair. So the simple Jordan Pairs are those whose quotients do not result any any new Jordan pairs
\begin{definition}
    A non-zero Jordan pair $V$ is called simple if $(0_+,0_-)$ and $V$ are its only ideals.
\end{definition}

\subsubsection*{Jordan Algebras}


\subsection{Step 3: Studying The Algebra}
\subsubsection*{Jordan Pairs}
To study our algebra it is often helpful to identify important objects in our examples, such as automorphisms.
So we define 
$\text{aut}(V)=\{h:V\ra V| h \text{ is a isomorphims}\}\leq \text{GL}(V^+)\times \text{GL}(V^-)$.  
One important example of an automorphism of matrix modules is the transpose. However the standard transpose is not an automorphism of Jordan pairs.
This follows from the fact that $(-)^t:V^+=M_{p,q}(k)\ra V^-=M_{q,p}(k)$ and vice versa.
This however introduces the notion of an anti-homomorphism or twisted homomorphism.
\begin{definition}
    An antihomomorphism between Jordan pairs $V$ and $W$ is a homomorphism $\eta: (V^+,V^-)\ra(W^-,W^+)$, that is a pair of k-module homomorphisms $\eta_\sigma:V^\sigma\ra W^{-\sigma}$ such that
    $\eta_\sigma (Q_\sigma(x)y)=(Q_{-\sigma}(\eta_\sigma(x))\eta_{-\sigma} (y))$. We often denote $W^{\text{op}}$ to be the Jordan Pair $W^{\text{op}}=(W^-,W^+)$ with quadratic maps $(Q_-,Q_+)$.
\end{definition}
Now that we have created the algebra of Jordan Pairs we need to study it.