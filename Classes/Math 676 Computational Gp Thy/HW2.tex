\documentclass[12pt]{amsart}
\usepackage{preamble}
%\DeclareRobustCommand{\flippedSlash}{\text{\reflectbox{$\setminus$}}}

\begin{document}
\begin{center}
   \textsc{Math 676 Computational Group Theory. HW 2\\ Ian Jorquera}
\end{center}
\vspace{1em}

\begin{itemize}
   \item[(7)] 
   \begin{enumerate}[label= (\alph*)]
      \item Notice that $G=\ip{(1,2),(3,4)}$ acts on $\{1,2,3,4\}$ and for either 
      $1$ or $2$ the element $(3,4)$ fixes both $1$ and $2$. Likewise for either 
      $3$ or $4$ the element $(1,2)$ fixes them. So not single element can be a base.

      \item Let $G=\ip{g=(1,2)(3,4,5)(6,7,8,9,10,11)}$ is the cyclic group generated 
      by $g$ and that $6^{(g^k)}=1$ only when $k$ is a multiple of $6$. Meaning 
      $\{6\}$ is a base. Likewise $\{1,3\}$ is also a base but neither $1$ nor $3$ 
      form a base on their own because $1^{(g^2)}=1$ and $3^{(g^3)}=3$.

      \item Let $k<m$ and let $p_1=2,p_2=3,\dots, p_m$ be $m$ distinct primes. For 
      simplicity of writing let $\Omega=\bigcup_{j=1}^m\{1^j,2^j,\dots,p_j^j\}$ 
      which is in bijection with $\{1,\dots, \sum_{j=1}^kp_j\}$. Let 
      $g=(1^1,p_1^1)(1^2,2^2,p_2^2)\cdots (1^m,\dots,p_j^j)$. Notice that $\ip{g}$ 
      has an irreducible base of $\{1^1,1^2,\dots, 1^m\}$ of $m$ elements. Now 
      let $\ell=sum_{j=1}^{m-k+1}p_j$ and let $h=(1^\ell,2^\ell,\dots,\ell^\ell)$
      Notice that $\ip{gh}$ acts on $\Omega\cup\{1^\ell,2^\ell,\dots,\ell^\ell\}$
      And has the irreducible base $\{1^1,1^2,\dots, 1^m\}$ of $m$ elements as before
      but also has an irreducible base of $\{1^\ell,1^{m-k+1},\dots, 1^m\}$ of size $k$.


   \end{enumerate}
   \item[(8)] Let $b$ be the minimum base length and $B=\{\beta_1,\dots,\beta_b\}$ a base such 
   that $|B|=b$ with $n=|\Omega|$ and $m$ the minimum degree. Let $\Delta_g=\{w\in\Omega| w^g=w\}$ for any 
   $g\in G$. Notice that for any $x\in G$ if $B^x\cap \Delta_g=\{\beta_1^x,\dots^x,\beta_b^x\}\cap\Delta_g=\emptyset$
   this means that for all $j$ that $(\beta^x)^g=\beta^x$ and so $\beta^{xgx^{-1}}=\beta$ which must
   imply that $xgx^{-1}=1$ and so $g=1$. Therefore for any $1\neq g\in G$ in must be the case 
   $\Delta_g\cap B^x\neq \empty$. Now consider some $\alpha\in \Omega$ in which case because $G$
   is transitive for each $j$ there is some $x\in G$ such that $\beta_j^x=\alpha$. In fact any 
   $y\in x\text{Stab}_G(\beta_j^x)$ will satisfy $\beta_j^y=\alpha$ and $|x\text{Stab}_G(\beta_j^x)|=|G|/n$.
   So there are $|B||G|/n$ choices for $x\in G$ such that $\alpha\in B^x$. Let $\Delta$ be the smallest 
   $\Delta_g$ with corresponding group element $d$. This means for every $\alpha\in \Delta$ we have that
   there are $|B||G|/n$ choices for $x\in G$ such that $\alpha\in B^x$. Meaning, including repeated elements 
   there are $|\Delta||B||G|/n$ elements for $x\in G$ such that $B^x$ contains an element of $\Delta$. 
   Another way to count this is as $\sum_{x\in G}|B^x\cap \Delta|$.
   
   
   \item[(9)] The group as a permutation group is $G=\ip{g_1=(1,2,4),g_2=(2,3,4)}$ 
   which would have the following Stabilizer chain:\\
   $G=\ip{(1,2,4),(2,3,4)}$ with base $[1,2]$\\
   Orbit: $|1^G|=\{1,2,4,3\}$\\
   Transversal: $[(),(1,2,4), (1,2,4)^2, (1,2,4)(2,3,4)]$\\
   Stabilizer: $\ip{g_2,e,g_1^2g_2g_1^{-1},g_1g_2g_1g_2^{-1}g_1^{-1},g_1g_2^2g_1^{-2}}=\ip{g_2}$\\
      \hspace*{.5in}Orbit: $|2^{G}|=\{2,3,4\}$\\
      \hspace*{.5in}Transversal $[(),g_2,g_2^2]$\\
      \hspace*{.5in}Stabilizer: $\ip{()}$  

      Now consider the base image [4,1]. First consider the element $f_1=g^2_1=(1,2,4)^2=(1,4,2)$ 
      that maps $1\ra 4$.
      Then consider the element $f_2=()$ that maps $2\ra 2=1^{(f_1^{-1})}$ and notice that the element
      $f_2f_1=g_2^2g_1=$ idk this isnt possible.

      Now consider the base image [2,1]. First consider the element $f_1=g^2_1=(1,2,4)$ 
      that maps $1\ra 2$.
      Then consider the element $f_2=g_2^2$ that maps $2\ra 4=1^{(f_1^{-1})}$ and notice that the element
      $f_2f_1=g_2^2g_1=(1,2)(3,4)$.


   \item[(10)]
\end{itemize}

\end{document}