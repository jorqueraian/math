\documentclass[12pt]{amsart}
\usepackage{preamble}
%\DeclareRobustCommand{\flippedSlash}{\text{\reflectbox{$\setminus$}}}

\begin{document}
\begin{center}
   \textsc{Math 676 Computational Group Theory. HW 2\\ Ian Jorquera}
\end{center}
\vspace{1em}

\begin{itemize}
   \item[(7)] 
   \begin{enumerate}[label= (\alph*)]
      \item Notice that $G=\ip{(1,2),(3,4)}$ acts on $\{1,2,3,4\}$ and for either 
      $1$ or $2$ the element $(3,4)$ fixes both $1$ and $2$. Likewise for either 
      $3$ or $4$ the element $(1,2)$ fixes them. So not single element can be a base.

      \item Let $G=\ip{g=(1,2)(3,4,5)(6,7,8,9,10,11)}$ is the cyclic group generated 
      by $g$ and that $6^{(g^k)}=1$ only when $k$ is a multiple of $6$. Meaning 
      $\{6\}$ is a base. Likewise $\{1,3\}$ is also a base but neither $1$ nor $3$ 
      form a base on their own because $1^{(g^2)}=1$ and $3^{(g^3)}=3$.

      \item Let $k<m$ and let $p_1=2,p_2=3,\dots, p_m$ be $m$ distinct primes. For 
      simplicity of writing let $\Omega=\bigcup_{j=1}^m\{1^j,2^j,\dots,p_j^j\}$ 
      which is in bijection with $\{1,\dots, \sum_{j=1}^kp_j\}$. Let 
      $g=(1^1,p_1^1)(1^2,2^2,p_2^2)\cdots (1^m,\dots,p_j^j)$. Notice that $\ip{g}$ 
      has an irreducible base of $\{1^1,1^2,\dots, 1^m\}$ of $m$ elements. Now 
      let $\ell=\sum_{j=1}^{m-k+1}p_j$ and let $h=(1^\ell,2^\ell,\dots,\ell^\ell)$
      Notice that $\ip{gh}$ acts on $\Omega\cup\{1^\ell,2^\ell,\dots,\ell^\ell\}$
      And has the irreducible base $\{1^1,1^2,\dots, 1^m\}$ of $m$ elements as before
      but also has an irreducible base of $\{1^\ell,1^{m-k+1},\dots, 1^m\}$ of size $k$.


   \end{enumerate}
   \item[(8)] Let $b$ be the minimum base length and $B=\{\beta_1,\dots,\beta_b\}$ a base such 
   that $|B|=b$ with $n=|\Omega|$ and $m$ the minimum degree. Let $\Delta_g=\{w\in\Omega| w^g=w\}$ for any 
   $g\in G$. Notice that for any $x\in G$ if $B^x\cap \Delta_g=\{\beta_1^x,\dots,\beta_b^x\}\cap\Delta_g=\emptyset$
   this means that for all $j$ that $(\beta^x)^g=\beta^x$ and so $\beta^{xgx^{-1}}=\beta$ which must
   imply that $xgx^{-1}=1$ and so $g=1$. Therefore for any $1\neq g\in G$ in must be the case 
   $\Delta_g\cap B^x\neq \empty$. Now consider some $\alpha\in \Omega$ in which case because $G$
   is transitive for each $j$ there is some $x\in G$ such that $\beta_j^x=\alpha$. In fact any 
   $y\in x\text{Stab}_G(\beta_j^x)$ will satisfy $\beta_j^y=\alpha$ and $|x\text{Stab}_G(\beta_j^x)|=|G|/n$.
   So there are $|B||G|/n$ choices for $x\in G$ such that $\alpha\in B^x$. Let $\Delta$ be the smallest 
   $\Delta_g$ with corresponding group element $d$. This means for every $\alpha\in \Delta$ we have that
   there are $|B||G|/n$ choices for $x\in G$ such that $\alpha\in B^x$. Meaning, including repeated elements 
   there are $|\Delta||B||G|/n$ elements for $x\in G$ such that $B^x$ contains an element of $\Delta$. 
   Another way to count this is as $\sum_{x\in G}|B^x\cap \Delta|$.
   
   
   \item[(9)] The group as a permutation group is $G=\ip{g_1=(1,2,4),g_2=(2,3,4)}$ 
   which would have the following Stabilizer chain:\\
   $G=\ip{(1,2,4),(2,3,4)}$ with base $[1,2]$\\
   Orbit: $|1^G|=\{1,2,4,3\}$\\
   Transversal: $[(),(1,2,4), (1,2,4)^2, (1,2,4)(2,3,4)]$\\
   Stabilizer: $\ip{g_2,e,g_1^2g_2g_1^{-1},g_1^2g_2g_1^{-1},g_1g_2g_1g_2^{-1}g_1^{-1},g_1g_2^2g_1^{-2}}=\ip{g_2}$\\
      \hspace*{.5in}Orbit: $|2^{G}|=\{2,3,4\}$\\
      \hspace*{.5in}Transversal $[(),g_2,g_2^2]$\\
      \hspace*{.5in}Stabilizer: $\ip{()}$  

      Now consider the base image [4,1]. First consider the element $f_1=g^2_1=(1,2,4)^2=(1,4,2)$ 
      that maps $1\ra 4$.
      Then consider the element $f_2=()$ that maps $2\ra 2=1^{(f_1^{-1})}$ and notice that the resulting element
      $f_2f_1=g_1^2=(1,2,4)$ which does not solve the puzzle meaning that this state is not possible and The
      algorithm would return an error.

      Now consider the base image [2,1]. First consider the element $f_1=g_1=(1,2,4)$ 
      that maps $1\ra 2$.
      Then consider the element $f_2=g_2^2$ that maps $2\ra 4=1^{(f_1^{-1})}$ and notice that the element
      $f_2f_1=g_2^2g_1=(1,2)(3,4)$. This solves the puzzle.


   \item[(10)]
   \begin{enumerate}[label= (\alph*)]
      \item First notice that for any subgroup $S\leq G$ that $G$ acts on the cosets of $S$ 
      transitively by multiplication. This means to compute representatives of the cosets
      we could design an algorithm that follows the same idea as the orbit algorithm but 
      stores only the representatives for the cosets in $\Delta$ and when checking if a new element $\gamma$
      is a representative of pre-existing coset we could use a stabilizer chain for $S$ 
      that allows us to check it if $\gamma \delta^{-1}\in S$ for all $\delta$.

      \item The kernel is the intersection of all the stabilizers $Stab_G(gS)$ for all $g\in G$. This
           means that any element $x$ in the kernel must satisfy $xgS=gS$ for all $g$ which is equivalent
           to the $g^{-1}xg\in S$ which is equivalent to $x\in gSg^{-1}=S^g$ for all $g\in G$. This converse is also true
           meaning the kernel is equal to the core.

      \item This is just the process of completing the stabilizer chain. So the intersection of all the 
      stabilizers is the bottom level in the stabilizer chain.

      \item
   \end{enumerate}

   \item[(11)]
   \begin{enumerate}
      \item First we will show that this is a necessary condition. 
      Let $\phi:\underbar{g}\ra H$ extends to a homomorphism $\phi:G\ra H$. 
      In which case if $g_{i_1}g_{i_2}\cdots g_{i_k}=1$ then it must be 
      the case that $\phi(1)=\phi(g_{i_1}g_{i_2}\cdots g_{i_k})=h_{i_1}h_{i_2}\cdots h_{i_k}=1$.

      To see that this is a sufficient condition notice assume that if 
      $g_{i_1}g_{i_2}\cdots g_{i_k}=1$ then $h_{i_1}h_{i_2}\cdots h_{i_k}=1$. Now let $\phi:\underbar{g}\ra H$
      and extend to a map $\phi: G\ra H$ by group products. Now let $g\in G$ such that 
      $\phi(g_{i_1}g_{i_2}\cdots g_{i_k})=h_{i_1}h_{i_2}\cdots h_{i_k}$ and notice 
      that $\phi(gg^{-1})=\phi(g_{i_1}g_{i_2}\cdots g_{i_k}g^{-1}_{i_{k}}g^{-1}_{i_{k-1}}\cdots g^{-1}_{i_1})=h_{i_1}h_{i_2}\cdots h_{i_k}h^{-1}_{i_{k}}h^{-1}_{i_{k-1}}\cdots h^{-1}_{i_1}$
      and because $g_{i_1}g_{i_2}\cdots g_{i_k}g^{-1}_{i_{k}}g^{-1}_{i_{k-1}}\cdots g^{-1}_{i_1}=1$
      then $h_{i_1}h_{i_2}\cdots h_{i_k}h^{-1}_{i_{k}}h^{-1}_{i_{k-1}}\cdots h^{-1}_{i_1}=1$ so $\phi(g^{-1}=)h^{-1}_{i_{k}}h^{-1}_{i_{k-1}}\cdots h^{-1}_{i_1}=\phi(g)^{-1}$.
      Additionally $\phi$ preserves group multiplication by construction.
      
      \item $\Rightarrow$) Let $\phi:\underbar{g}\ra H$ extends to a homomorphism $\phi:G\ra H$
            and let $(a,b)\in Stab_S(\Omega_G)$ which means that $w^{(a,b)}=w$ for all $w\in\Omega_G$.
            This means that $w^a=w$ for all $w\in\Omega_G$ and if $a=1$.% this may not be true? I think this really means a is in the kernel of the action
             Using part (a) this must mean
            that $\phi(a)=b=1$ so $(a,b)=1$.

            $\Leftarrow$) Let $\phi:\underbar{g}\ra H$ and assume that $Stab_S(\Omega_G)=\ip{1}$ 
            which means that for any $a=g_{i_1}g_{i_2}\cdots g_{i_k}=1$ we know that $w^a=w^1=w$ for all 
            $w\in\Omega_G$. And so $w^{(a,\phi(a))}=w$ for all $w\in \Omega_G$, therefore 
            $(a,\phi(a))\in Stab_S(\Omega_G)$ and so $\phi(a)=h_{i_1}h_{i_2}\cdots h_{i_k}=1$. And by part (a) this proves the claim.
   \end{enumerate}
\end{itemize}

\end{document}