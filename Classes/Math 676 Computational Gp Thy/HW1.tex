\documentclass[12pt]{amsart}
\usepackage{preamble}
\DeclareRobustCommand{\ammaG}{\text{\reflectbox{$\setminus$}}}

\begin{document}
\begin{center}
   \textsc{Math 676 Computational Group Theory. HW 1\\ Ian Jorquera}
\end{center}
\vspace{1em}

\begin{itemize}
   \item[1.]
   Let $S,T$ be subgroups of a group $G$ and consider the double 
   cosets of $S$,

   \item[2.] Let $G=\ip{a=(1,2,\dots,100), b= (1,2)}=S_{100}$. The representatives, 
   using the algorithm from class would be as follows: the representative for $2$ would
    be $(1,2,\dots,100)$ and more generally the representative for $k$ would be
    $(1,2,\dots,100)^{k-1}$. A generating set which shorter representations could be 
    $\ip{a=(1,2,\dots,100), b= (1,33), c=(1,66)}$ in which case no representative would
    have more then $34$ multiplications, while in the previous generating set the 
    number of multiplication would be much longer.
   
    \item[3.] Assume that $G$ is a finite group. And let $M_1$ be a maximal subgroup of $G$
               where we know that $[G:M_1]\geq 2$. Pick an element $m_1\in G$ 
               such that $m_1\not\in M_1$. We can repeat this process inductively on a
               chain of maximum subgroups $M_{k}\leq M_{k-1}$ by
               picking an element $m_k\in M_{k-1}$ where $m_k\not\in M_k$. Each maximal subgroup has
               index $\geq 2$ meaning there are most $\log_2(|G|)$ steps in this process, 
               meaning there are at most $\log_2(|G|)$ elements, all of which generate $G$.

   \item[4.] 
   \begin{itemize}
      \item[a. ] The probability that any single element is in $U$ is $\frac{1}{n}$ and 
      because each element is chosen randomly the probability that all $k$ elements are in $U$
      is $\frac{1}{n^k}$.
      \item[b. ] Because conjugacy classes partition a group $G$ and are all the same size 
                 the probability that a random given element is in a specific conjugacy class
                 is $\frac{1}{n}$. For $k$ random elements, the first element determines which
                 specific conjugacy class the other element need to also be in. And so the 
                 probability of $k$ element falling into the same conjugacy class is $\frac{1}{n^{k-1}}$.
                 %also if k is bigger then size of U and all distinct then prob is zero
      \item[c. ]  id
   \end{itemize}
   \item[5.] Let $G=\ip{g}$ be a cyclic group. Notice that for any $w\in \Omega$  that $w^g$
   
            the orbit of any element $w\in\Omega$ has size $|w^G|=1$ 
            if no group element moves $w$ or $|w^G|=|G|$. This follows from the fact that 
            if $w^{g^k}=w$ for some $k$ then 

\end{itemize}

% note for self: review 10.6 in book and 10.8
\end{document}