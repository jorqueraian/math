\documentclass[12pt]{amsart}
\usepackage{preamble}
%\DeclareRobustCommand{\flippedSlash}{\text{\reflectbox{$\setminus$}}}

\begin{document}
\begin{center}
   \textsc{Math 676 Computational Group Theory. HW 3\\ Ian Jorquera}
\end{center}
\vspace{1em}

\begin{itemize}
   \item[(12)] Let $G$ act transitive on $\Omega$ such that $\mathcal B$ 
   and $\mathcal B'$ are two distinct block systems. Consider the action of 
   $G$ on both of these block systems: $G\times \mathcal B\ra \mathcal B$ and
   $G\times \mathcal B\ra \mathcal B'$. This gives us maps $\alpha:G\ra S_{\mathcal B}$ 
   and $\beta:G\ra S_{\mathcal B'}$ with induced map $\epsilon:G\ra S_{\mathcal B'}\times S_{\mathcal B'}$.
   by $\epsilon(g)=(\alpha(g),\beta(g))$. Notice that $\ker\epsilon=\ker\alpha\cap\ker\beta$ 
   and because both block systems are unrefinable we know that $\ker\epsilon=\ip{1}$ and
   so $\epsilon$ defines the subdirect product $\epsilon(G)\cong G$ of group actions of smaller degree.


   \item[(13)] 
   \begin{enumerate}[label= (\alph*)]
      \item Let $F$ be a finite field of size not equal to $2$.
      Consider the equivalence relation of scaling, that is 
      $x\equiv y \Leftrightarrow x=ty$ for $t\neq 0$. And we can define a 
      block system $\mathcal B$ to be the quotient of the non-zero vectors of 
      dimension $n$ over $F$ with this congruence. That is for a representative $w$ of some block, we
      can define the block as $B_w=\{x\in F^n | x\equiv w\}$. This is a non trivial
      block system as long as the field is not of size $2$.
      
      \item Let $w\in F^n$ and fix a basis $w,v_2,v_3,\dots,v_n$ with $w$ as the 
      first basis element. In this case the stabilizer of $w$ are the invertable matrices 
      of the following form
      \[\begin{bmatrix}[c|ccc]
         1&*&\cdots&*\\
         0&*&\cdots&*\\
         \vdots&\vdots&\ddots&\vdots\\
         0&*&\cdots&*\\
         \end{bmatrix}\]
         The stabilizer of a block is the same but also including scaling by non-zero field elements.

      \item Notice that if $M\in GL_n(F)$ and not in the block stabilizer of the block containing $w$: $B_w$. 
      Then $M$ sends $w$ to some other element and with the elements of the stabilizer 
      we can send the result $Mw$ to any of the basis elements. This shows although lazily that the block stabilizers are maximal.
      %Notice that for any $x,y\in F^n$ that are in a single block we know
       %that $x=ty$ for some $t\neq 0$. This means that for the element $tI\in GL_n(F)$
       %we have that $tIy=x$ meaning $x$ and $y$ must be in the same block or distinct blocks.
       %Becasue this is true for all elements in the block we know that there can only be the trivial
       %block systems. So $GL_n(F)$ is primitive when acting on each block.

   \end{enumerate}
   
   \item[(14)]
   \begin{enumerate}[label= (\alph*)]
      \item First we will show that $T=\text{stab}_G(\Delta)=\{g\in G|\Delta^g=\Delta\}$ 
      contains $S=\text{stab}_G(1)$. This is follows from the construction of $\Delta$. 
      Because of the bijection of subgroups and blocks systems this means that there is a block system With $1^T$ as a block. 
      Notice that by construction $1^T\se \Delta$. And because because $G$ is transitive the 
      reverse inclusion is also true.


      \item Let $g\in\text{stab}_G(\Delta)$. Let $s\in S$ and consider the element $gsg^{-1}$. Notice that 
      $1^{gsg^{-1}}=w^{sg^{-1}}=w^{g^{-1}}=1$ because $w\in \Delta$. Now consider an element $g\in N_G(S)$ meaning for 
      any $s\in S$ we have that $gsg^{-1}\in S$ meaning $1^{gsg^{-1}}=1$ so $gsg^{-1}\in S$.


   \end{enumerate}
   \item[(15)] My notaiton is messy and bad and i leave out a lot of the details but i think eveyrthing is correct. Let $S=\{(), (2,4)\}$ and consider the action of $S_4\times S_4/S\ra S_4/S$ which is imprimitive 
   with non-trivial block system $\mathcal B=\{D_8/S,(1,2)D_8/S,(2,3,4)D_8/S\}\approx\{1,2,3\}$.
   Let $B=D_8/S=\{\{(),(2,4)\},\{(1,2,3,4),(1,4)(2,3)\},\{(1,3)(2,4),(1,3)\},\{(1,4,3,2),(1,2)(3,4)\}\}\approx\{1,2,3,4\}$ be a fixed block. 
   Define the maps $\psi:G\ra S_{\mathcal B}\cong S_3$ as 
   $g\mapsto (A\in\mathcal B\mapsto A^g)$ and $\phi:T\ra S_{B}\cong S_4$ as 
   $t\mapsto (b\in B\mapsto b^t)$. Notice that $G^\psi\cong S_3$ and $T^\phi\cong S_4$.
   So this gives us an embedding of $S_4\hookrightarrow S_4\wr S_3$. To determine the image we need
   only consider the image of the generators $(1,2,3,4)$ and $(1,2)$ of $S_4$.
   First consider $(1,2)$. We know that $(1,2)^\psi=(1,2)$. Now for each representative 
   of the cosets of $T$: $(),(1,2)$, and $(2,3,4)$ we can compute the $\tilde{g_j}$ values 
   for $g=(1,2)$ and each representative. So $\tilde{(1,2)_{()}}=()$, $\tilde{(1,2)_{(1,2)}}=()$ and $\tilde{(1,2)_{()}}=(1,4)$.
   THis means the image of $(1,2)$ is $((1,2),(),(),(1,4))$. The other generator follows the same process.
\end{itemize}

\end{document}