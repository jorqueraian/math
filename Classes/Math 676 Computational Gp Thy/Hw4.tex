\documentclass[12pt]{amsart}
\usepackage{preamble}
%\DeclareRobustCommand{\flippedSlash}{\text{\reflectbox{$\setminus$}}}

\begin{document}
\begin{center}
   \textsc{Math 676 Computational Group Theory. HW 4\\ Ian Jorquera}
\end{center}
\vspace{1em}

\begin{itemize}
   \item[(17)] (a) Let $p$ be a prime and $G\leq S_p$ be generated by two $p$-cycles.
   This means $G$ is transitive on $\{1,2,\dots,p\}$. Becasue $p$ is prime this means 
   $G$ is primitive.

   Well $p$ is prime so $G$ is primitive. I didn't have time to finish this.

   
   \item[(18)] 
   \begin{enumerate}[label= (\alph*)]
      \item let $G$ be finite group with $N\triangleleft G$ and let $F=G/N$.
      Let $\phi:F\ra S_\Omega$ be the regular permutation representation, meaning the action is transitive and every stabilizer is trivial.
      Im not sure, I didn't have time to finish this. But I think its something about thinking about copies of $N$ being the cosets in $F$, being the wreath product.



      \item The 5 subgroups of order $720$ and whose derived subgroup is of index $2$ are\\
      Group([ (7,9)(11,12), (1,3)(4,6), (1,3,5,6)(2,4) ]),\\
      \text{Group([ (1,7,2,8,3,10)(4,9,6,11,5,12), (1,2,4,3)(5,6)(7,8)(9,11,10,12), (1,2)(3,6)(7,12)(9,11) ])},\\
      Group([ (1,7,2,8,3,9,4,10)(5,11,6,12), (1,2)(5,6)(7,12)(9,11) ]),\\
      Group([ (1,7,2,10,5,12,4,8,6,11)(3,9), (1,4)(3,5)(7,12)(9,11) ]),\\
      Group([ (1,7,2,9,6,12,5,10,4,11)(3,8), (2,3)(4,5)(7,12)(9,11) ])\\

      And the number of conjugacy classes of each of the 
      following subgroups is $14, 11, 8, 14, 11$

      This shows that the only subgroups that would be isomprhic from this list are the first and 4th or the 2nd and 5th.
      After running IsomorphismGroups we can determine that the first and fourth are isomorphic.
      Then after running the same function on the subgroups with $S_6$ we can determine that the second subgroup is isomorphic to $S_6$.


   My code:\\

A6:=AlternatingGroup(6);\\
C2:=CyclicGroup(IsPermGroup,2);\\
W:=WreathProduct(A6,C2);\\
u:=List(ConjugacyClassesSubgroups(W),Representative);\\
u2:=Filtered( u, G $->$ Order( G ) = 720 and Order(DerivedSubgroup(G))=360 );\\
for ugrp in u2 do Print(NrConjugacyClasses(ugrp), ``, ''); od;\\
IsomorphismGroups(u2[1], u2[4]);\\
IsomorphismGroups(u2[2], u2[5]);\\
IsomorphismGroups(u2[2], SymmetricGroup(6));\\

   \end{enumerate}


   \item[(19)] Let $G\leq S_n$ be transitive with $\phi\in\Aut(G)$ that is realized by $S_n$, 
   meaning there exists some $h$ in the normalizer of $G$ such that $g^\phi=h^{-1}gh$ for all $g$.
   Notice that this must mean that $\text{Stab}_G(1)^\phi=h^{-1}\text{Stab}_G(1)h=\text{Stab}(j)$ where $j=1^h$.
   Let $h^{-1}sh\in h^{-1}\text{Stab}_G(1)h$ notice that $j^{h^{-1}sh}=1^{sh}=1^{h}=j$.
   Likewise if $s\in \text{Stab}(j)$ then $hsh^{-1}\in\text{Stab}(1)$ meaning $s\in h^{-1}\text{Stab}_G(1)h$.

   For the other direction assume that $\text{Stab}_G(1)^\phi=\text{Stab}(j)$ and consider the cosets of 
   $\text{Stab}(1)\backslash G=\{\text{Stab}(1) g | g\in G\}$ and 
   $\text{Stab}(j)\backslash G=\{\text{Stab}(j) g | g\in G\}$ which are in bijection inducted by $\phi$
   such that $\Phi: \text{Stab}(1)\backslash G\ra \text{Stab}(j)\backslash G$ where 
   $\Phi(\text{Stab}(1) g)=\text{Stab}(j) \phi(g)$. Notice this can be represented by a group element 
   $h$ that maps $g(1)\mapsto \phi(g)(j)$ forall $g$, which is a group element because $G$ is transitive.
   To show that $g^\phi=h^{-1}gh$ we need only show it acts on $\{1,\dots, n\}$ the same, because $G$ is a permutation group.
   Notice that for any $j$ we have that $j^{h^{-1}gh}=1^{gh}=j^{\phi(g)}$.

   \item[(20)] 
   \begin{enumerate}[label= (\alph*)]
      \item Let $G$ be a finite group, and $O_\infty(G)$ its radical, the largest 
      solvable normal subgroup. Let $F=G/O_\infty(G)$. We know that the 
      $Soc(F)=\ip{N | N\triangleleft_{min} F}=\bigoplus_{N\triangleleft_{min} F} N$.
      So we need only show that the minimal normal subgroups are non-abelian.
      Notice that for any $N\triangleleft_{min} F$ we have that 
      $N\cong F/\ip{M | N\neq M\triangleleft_{min} F}$ which we know is not abelian as
      if it were then we could construct a larger solvable radial.
      \item Consider the action of $F$ of $Soc(F)$ by conjugation. Notice that $Z(Soc(G))=1$ and so by lemma II.65 we are done.
      
      \item This follows directly from lemma II.66 and (b) 
   \end{enumerate}

   \item[(21)] Let $A=\begin{bmatrix}2 &-1\\3&-2\end{bmatrix}$ and
    $B=\begin{bmatrix}3 &-4\\2&-3\end{bmatrix}$. This gives us a representation $G\ra \text{GL}_2(\Q)$.
    Where $x\mapsto A$ and $y\mapsto B$. This works because the order of both $A$ and $B$ is $1$.
    Notice that this means $\ip{A,B}$ is isomorphic to a quotient of $G$. And $AB$ has infinite order.
    To see this notice that $AB$ has characteristic polynomial $x^2+2x+1$ meaning any power of $AB$ is not 
    equal to the identity. This means $\ip{A,B}$ is infinite and so $G$ is infinite.

\end{itemize}

\end{document}