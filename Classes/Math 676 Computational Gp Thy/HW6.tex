\documentclass[12pt]{amsart}
\usepackage{preamble}
%\DeclareRobustCommand{\flippedSlash}{\text{\reflectbox{$\setminus$}}}

\begin{document}
\begin{center}
   \textsc{Math 676 Computational Group Theory. HW 6\\ Ian Jorquera}
\end{center}
\vspace{1em}

\begin{itemize}
    \item[(27)] 
    \begin{enumerate}[label=(\alph*)]
        \item we will start with rewriting rules $x^3\ra 1$, $y^3\ra 1$ and 
        $(xy)^3\ra 1$.
        Now we will consider the words that are combinations of left hand 
        sides of rewriting rules. Consider the word $xxxyxyxy$ which reduces 
        in two ways to $yxyxy$ and $xx$ giving the rewriting rule $yxyxy\ra xx$.
        Then consider the word $xyxyxyyy$ which reduces 
        in two ways to $yy$ and $xyxyx$ giving the rewriting rule $xyxyx\ra yy$.
        Then consider the word $yyyxyxy$ which reduces 
        in two ways to $xyxy$ and $yyxx$ giving the rewriting rule $yyxx\ra xyxy$.
        The consider the word $yxyxyyy$ which reduces 
        in two ways to $xxyy$ and $yxyx$ giving the rewriting rule $yxyx\ra xxyy$.
        All other words reduce to the same word. Meaning the confluent rewritting system is
        $x^3\ra 1$, $y^3\ra 1$, $(xy)^3\ra 1$, $yxyxy\ra xx$, $xyxyx\ra yy$, $yyxx\ra xyxy$, $yxyx\ra xxyy$

        \item Notice that the word $(xxyyxy)^k$ is normal form for all $k$, meaning there 
        are infinitly many different words in normal form, so $M$ is infinte.
    \end{enumerate}
    \item[(29)] $\{1,x\}$ where $x^2=x$, or $\ip{x|x^2=x}$
    \item[(30)] Notice that because $N\cong C_2^2$ we can consider it to be the 
    vector space $\F_2^2$ with basis $\{p=(2,6)(3,4),q=(1,5)(3,4)\}$. Now we want to come up with
    elements in $G$ that could act like $G/N$ so consider two possible such element $an$ and $bm$ 
    and we will look at the relations of $G/N$. Where $c^3=1$ tells us that 
    $(an)^3= ananan=a^2n^anan=a^3n^{aa}n^an=a^3n^{a^2+a+1}$ giving us the linear equation
    \[a^3=()=\begin{bmatrix}0& 0\end{bmatrix}=\begin{bmatrix}n_1& n_2\end{bmatrix}\begin{bmatrix}0& 0\\0&0\end{bmatrix}\]
    which gives us nothing. Now the relation $b^2=1$ tells us that 
    $(bm)^3= bmbm=b^2m^bm=b^2m^{b+1}$ giving us the linear equation
    \[-b^2=()=\begin{bmatrix}0& 0\end{bmatrix}=\begin{bmatrix}m_1& m_2\end{bmatrix}\begin{bmatrix}0& 0\\0&0\end{bmatrix}\]
    which gives us nothing
    Finally the relation $cd=dc$ give us that $anbm=bman$ which simplifies to $abn^bm=bam^an$
    which is $a^{-1}b^{-1}abn^bm=m^an$ which gives us the following linear equation
    \[\begin{bmatrix}0& 1\end{bmatrix}+\begin{bmatrix}n_1& n_2\end{bmatrix}I_2+\begin{bmatrix}m_1& m_2\end{bmatrix}=\begin{bmatrix}m_1& m_2\end{bmatrix}\begin{bmatrix}1& 1\\1&0\end{bmatrix}\]
    which tells us that $\begin{bmatrix}m_1& m_2\end{bmatrix}=\begin{bmatrix}1& 1\end{bmatrix}$.
    So $m=pq$ and $n$ can be any element in $N$
     
\end{itemize}


\end{document}