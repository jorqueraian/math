\documentclass[12pt]{amsart}
\usepackage{preamble}
%\DeclareRobustCommand{\flippedSlash}{\text{\reflectbox{$\setminus$}}}

\begin{document}
\begin{center}
   \textsc{Math 676 Computational Group Theory. HW 6\\ Ian Jorquera}
\end{center}
\vspace{1em}

\begin{itemize}
    \item[(27)] 
    \begin{enumerate}[label=(\alph*)]
        \item we will start with rewriting rules $x^3\ra 1$, $y^3\ra 1$ and 
        $(xy)^3\ra 1$.
        Now we will consider the words that are combinations of left hand 
        sides of rewriting rules. Consider the word $xxxyxyxy$ which reduces 
        in two ways to $yxyxy$ and $xx$ giving the rewriting rule $yxyxy\ra xx$.
        Then consider the word $xyxyxyyy$ which reduces 
        in two ways to $yy$ and $xyxyx$ giving the rewriting rule $xyxyx\ra yy$.
        Then consider the word $yyyxyxy$ which reduces 
        in two ways to $xyxy$ and $yyxx$ giving the rewriting rule $yyxx\ra xyxy$.
        The consider the word $yxyxyyy$ which reduces 
        in two ways to $xxyy$ and $yxyx$ giving the rewriting rule $yxyx\ra xxyy$.
        All other words reduce to the same word. Meaning the confluent rewritting system is
        $x^3\ra 1$, $y^3\ra 1$, $(xy)^3\ra 1$, $yxyxy\ra xx$, $xyxyx\ra yy$, $yyxx\ra xyxy$, $yxyx\ra xxyy$

        \item Notice that the word $(xxyyxy)^k$ is normal form for all $k$, meaning there 
        are infinitly many different words in normal form, so $M$ is infinte.
    \end{enumerate}
    \item[(29)] $\{1,x\}$ where $x^2=x$, or $\ip{x|x^2=x}$
    \item[(30)] 
\end{itemize}


\end{document}