\documentclass[12pt]{amsart}
\usepackage{preamble}

\begin{document}
\begin{center}
   \textsc{Math 517. HW 3\\ Ian Jorquera}
\end{center}
\vspace{1em}

\begin{itemize}
   \item[1.] One $\sigma$-algebra containing $\{a,b\}$ is the algebra of all subsets of $\{a,b,c,d,e\}$.
             The smallest $\sigma$-algebra containing $\{a,b\}$ would be 
             $\{\emptyset, \{a,b,c,d,e\}, \{a,b\}, \{c,d,e\}\}$
             As this contains the empty set and the entire as well as $\{a,b\}$ and its compliment.


   \item[2.] Let $X=\{H,T\}^3$ which represents the possible ways to flip a coin $3$ times. 
             Consider the algebra of all subsets of these strings. let $E=\{HHH\}$ representing
             the only possibility of having heads three times.
             Let $\mu(A)=|A|/2^3$ which defines a measure on the algebra of all subsets. In this case
             the probability of event $E$ is $\mu(E)$.


   \item[3.]
   \begin{itemize}
    \item[i.] To show that the open intervals form a sequential covering class of $\R$. 
    We need only show that they cover $\R$ as any subset would be covered by $\R$ 
    and therefore by a sequence of open intervals. Consider the sequence $\{(-n,n)\}_{n=1}^\infty$, 
    and notice that for any element $k\in \R$ we know there exists an integer 
    $n$ such that $|k|<n$ in which case $k\in (-n,n)$. so $\R\se \bigcup_{n=1}^\infty (-n,n)$.

    \item[ii.] Now we will show that 
    $\mu^*(A)=\inf\left\{\lambda(O) | O \text{ is open and } A\se O\right\}$ is an outer 
    measure. Notice that $\lambda$ is non-negative for any open set and so $\mu*(A)$ must also be nonnegative.
    Now consider two set $A\se B$. Given a sequential cover for $B\se \{E_n\}_{n=1}^\infty$, we know
    that $\se$ is transitive so $A\se \{E_n\}_{n=1}^\infty$. Finally let $A_k\se X$ for each $k$.
    For each $k$ consider a sequential cover $A_k\se \{E_{kn}\}_{n=1}^\infty$ Notice that the union
    of all sequential covers $\bigcup_{k=1}^\infty\{E_{kn}\}_{n=1}^\infty$ is its self a sequential cover
    as it is the countable union of countable sequences. And so $\mu^*$ is an outer measure
               
   \end{itemize} 
   \item[4.] Let $f:E\ra \R$ be a measurable function and $B$ a Borel set.
             As per the hint in the book first Consider the set $S=\{E\se \R | f^{-1}(E)\text{ is measurable}\}$.
             First we will show that $S$ is a $\sigma$-algebra. Notice that 
             both $f^{\R}(\R)$ and $f^{-1}(\emptyset)$ are measurable so $\R\in S$ and $\emptyset\in S$.
             Furthermore notice that for any $E\in S$ we have that $f^{-1}(E^c)=f^{-1}(\R)-f^{-1}(E)$ and so is measurable, as the preimage is additive. 
             Finally consider the set $E_k\in S$ for all $k$. Notice that $f^{-1}(\bigcup_{k=1}^\infty E_k)=\bigcup_{k=1}^\infty f^{-1}(E_k)$
             which is measurable, as the preimage is countably additive, meaning that $\bigcup_{k=1}^\infty E_k\in S$.
             
             Next we will show that the generators of the form $[a,b)$ of the Borel $\sigma$-algebra are contained in $S$. 
             Notice that $f^{-1}([a,b))=f^{-1}((-\infty, b))-f^{-1}((-\infty, a))$, as the preimage is additive.
             Because $f$ is measurable this means that both $f^{-1}((-\infty, a))$ and $f^{-1}((-\infty, b))$
             are measurable and so $f^{-1}([a,b))$ is measurable. Finally because 
             $B$ is a Borel set meaning it is generated by sets of the form $[a,b)$ we know that it must be contained in $S$
             as $S$ also contains these sets. So $f^{-1}(B)$ is measurable.


    \item[5.] 
    \begin{itemize}
        \item[i.] Assume that $f\equiv 0$ on $E$ meaning $f$ is a simple function on $E$. Notice also that $E=\bigcup_{k=1}^{\infty}(E\cap(-k,k))$ This means
              \[\int_E f d\mu= \lim_{n\ra \infty}\int_{\bigcup_{k=1}^{n}(E\cap(-k,k))} f d\mu=\lim_{n\ra \infty}0\cdot\mu\left(\bigcup_{k=1}^{n}(E\cap(-n,n))\right)=\lim_{n\ra \infty} 0\]
              which is case even when $\mu(E)=\infty$.
              by definition $0\cdot \mu(E)=0$
        \item[ii.] It suffices to only consider the case of $f\equiv \infty$ on $E$ as in this case for any non-negative 
        measurable function $g(x)$ we have that $0\leq g(x)\leq f(x)$ everywhere on $E$ and so $\int_E g d\mu\leq \int_E f d\mu$

        So let $s$ be a simple function, and therefore real valued, meaning $s=\sum_{i=1}^m a_i \chi_{A_i}(x)$. 
        This also means that $s(x)\leq f(x)$ on $E$. Notice that because $A_i\se E$ and $\mu(E)=0$ we have that $\mu(A_i)=0$ and so
        \[\int_E s d\mu=\sum_{i=1}^m a_i\mu(A_i)=0\]
        This must mean that 
        \[\int_E f d\mu=\sup\left\{\int_E s d\mu\right\} = \sup\left\{\sum_{i=1}^m a_i\mu(A_i)=0\right\}=0\]
        It seems like the inconsistency with the dirac delta function is that the dirac delta function would not be
        an extended real valued function. That is even over the extended real number we would not have $\delta(0)=\infty$. 

    \end{itemize}

    \item[6.] Let $X=(0,\infty)$ and $f_n(x)=-\frac{1}{n}$. Notice that $f_n\ra 0$ pointwise for all $x\in X$. 
    Although it is true that $f_n$ is not non-negative this could be easily fixed by defining a new 
    sequence $g_n=-f_n\ra 0$ however this fails to satisfy the requirements of the monotone convergence theorem
    as this is no longer an increasing sequence. Finally we will show that the conclusion of the theorem is not met.
    Notice first the $\int_X f d\mu = 0$. However notice that for any $n>0$ we have that 
    $\int_X f_n d\mu = -\frac{1}{n}\mu(x)=-\infty$. and so $\lim_{n\ra \infty}\int_X f_n d\mu =-\infty$


    \item[7.] idk
              idk
\end{itemize}

% note for self: review 10.6 in book and 10.8
\end{document}