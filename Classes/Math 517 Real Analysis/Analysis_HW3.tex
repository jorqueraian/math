\documentclass[12pt]{amsart}
\usepackage{preamble}

\begin{document}
\begin{center}
   \textsc{Math 517. HW 3\\ Ian Jorquera}
\end{center}
\vspace{1em}

\begin{itemize}
   \item[1.]
      First we will show that $s(2^kx)=\min_{n\in\Z}|2^kx-n|$ is continuous for all $x\in \R$ for any fixed $k\in\N$.
      To see this let $\epsilon>0$ and $x\in \R$ and assume that the closest integer to $2^kx$ is $n$ and
      notice that for $\delta=\min(\epsilon/2^{k},1/2^{k},1/4)$ we have that for any $y\in\R$
      such that $|x-y|<\delta$ that the closest integer to $2^ky$ is at most $1$ away from $n$.
      This gives us 2 cases WLOG: First assume that $n$ is the closest integer to $2^ky$. In which case
      \begin{align*}
         \abs{s(2^ky)-s(2^kx)} & =\abs{|2^ky-n|-|2^kx-n|}                      \\
                               & \leq \abs{(2^ky-n)-(2^kx-n)}                  \\
                               & =|(2^ky-2^kx)|=2^k|x-y|<2^k\delta\leq\epsilon
      \end{align*}
      Case 2: assume WLOG that $x<y$ and the closest integer to $2^ky$ is $n+1$.
      Notice that from our assumption $\delta<\frac{1}{4}$ so it must be the case that $n<2^kx<2^ky<n+1$.
      Notice however that the distance between $2^ky$ and $n+1$ is bounded by the distance between $2^ky$ and $n$, that is $|2^ky-(n+1)|< 2^ky-n$.
      Likewise $|2^kx-n|< (n+1)-2^kx$.
      Case 2a: If $s(2^ky)<s(2^kx)$ this means that
      \begin{align*}
         \abs{s(2^ky)-s(2^kx)}=s(2^kx)-s(2^ky) & =|2^kx-n|-|2^ky-(n+1)|                                  \\
                                               & <(n+1)-2^kx-|2^ky-(n+1)|                                \\
                                               & =(n+1)-2^kx-(n+1)+2^ky\leq 2^ky-2^kx<2^k\delta<\epsilon
      \end{align*}

      and case 2b: if $s(2^ky)>s(2^kx)$ then this means that
      \begin{align*}
         \abs{s(2^ky)-s(2^kx)}=s(2^ky)-s(2^kx) & =|2^ky-(n+1)|-|2^kx-n|                          \\
                                               & 2^ky-n-|2^kx-n|                                 \\
                                               & =2^ky-n-2^kx-n\leq 2^ky-2^kx<2^k\delta<\epsilon
      \end{align*}

      And so we know that $\abs{s(2^ky)-s(2^kx)}<\epsilon$ when ever $|x-y|<\delta$
      and so $s(2^kx)$ is continuous for all $x\in \R$\\

      Now consider the sequence of functions
      $\displaystyle{\left\{\sum_{k=1}^n\frac{s(2^kx)}{2^k}\right\}_{n=1}^{\infty}}$.
      First we will show that this sequence of functions converges point wise. Which
      follows from the fact that for any $x\in \R$ we have that
      $\frac{s(2^kx)}{2^k}\leq \frac{1}{2^k}$. And so
      $\sum_{k=1}^n\frac{s(2^kx)}{2^k}$ is bounded by a geometric series and so
      converges when $n\ra\infty$ to
      $b(x)=\lim_{n\ra\infty}\sum_{k=1}^n\frac{s(2^kx)}{2^k}=\sum_{k=1}^\infty\frac{s(2^kx)}{2^k}$.
      Now we will show this sequence converges to $b(x)$ uniformly. Let $\epsilon>0$
      and let $N$ be an integer such that $\frac{1}{2^{n-1}}<\epsilon$ we know that
      \[\abs{\sum_{k=1}^\infty\frac{s(2^kx)}{2^k}-\sum_{k=1}^n\frac{s(2^kx)}{2^k}}=\abs{\sum_{k=n}^\infty\frac{s(2^kx)}{2^k}}<\abs{\sum_{k=n}^\infty\frac{1}{2^k}}=\frac{1}{2^n(1-\frac{1}{2})}=\frac{1}{2^{n-1}}<\epsilon\]
      which is the case for all $x\in\R$. Next Notice that for each function in the
      sequence $\sum_{k=1}^n\frac{s(2^kx)}{2^k}$ is continuous as it is the finite
      sun of continuous function, as $s(2^kx)$ is continuous for all $x\in\R$. So we
      know that the sequence of function converges to the function
      $b(x)=\sum_{k=1}^\infty\frac{s(2^kx)}{2^k}$ which is also continuous.\\

      Finally we will show that $b(x)$ is differentiable no where. Fix $x\in \R$ and
      let $m$ be a natural number and consider the integers $n$ and $n+1$ that are
      closest to $2^{m-2}x$. That is $n\leq 2^{m-2}x\leq n+1$ and so
      $\frac{n}{2^{m-2}}\leq x\leq \frac{n+1}{2^{m-2}}$. Now pick
      $h_m=\pm\frac{1}{2^m}$ where the sign is chosen such that
      $\frac{n}{2^{m-2}}\leq x+h_m\leq \frac{n+1}{2^{m-2}}$ and $2^{m-2}x$ and $2^{m-2}(x+h_m)$ have the same closest integer. This is always possible as  $|2^{m-2}h_m|=2^{m-2}2^{-m}=\frac{1}{4}$. % need m-2 and 1/4?
      Now consider the sequence of difference quotients
      $\left\{\frac{b(x+h_m)-b(x)}{h_m}\right\}_{m=2}^\infty$. And notice that
      \begin{align}
         \frac{b(x+h_m)-b(x)}{h_m} & =\frac{1}{h_m}\sum_{k=0}^{\infty}\frac{1}{2^k}\left(s(2^k(x+h_m))-s(2^kx)\right) \\
                                   & =\frac{1}{h_m}\sum_{k=0}^{m-2}\frac{1}{2^k}\left(s(2^k(x+h_m))-s(2^kx)\right) + S_m     \\
                                   & =\frac{1}{h_m}\sum_{k=0}^{m-2}\sigma_k h_m  + S_m                                       \\
                                   & =\sum_{k=0}^{m-2}\sigma_k +S_m
      \end{align}

      Where $S_{m-1}=\frac{1}{h_m}\frac{1}{2^{m-1}}\left(s(2^{m-1}(x+h_m))-s(2^{m-1}x)\right)$ is the 
      $m-1$th term of the sum. Notice that (2) follows from the fact that for $k\geq m$ we have that
      $2^kh_m=2^{k-m}$ is an integer as $k-m\geq 1$, and so from the periodicity of $s$ we know that
      $s(2^k(x+h_m))=s(2^kx)$.  Notice that (3) follows from the fact that for $k\leq
         m-1$ then $2^{k}x$ and $2^{k}(x+h_m)$ have the same closest integer and are
      between the same two closest integers. This has to be true as the line segments of 
      $s(2^kx)$ and nested in the line segments of $s(2^{k-1}x)$, and if two points $x$ and $x+h_m$
      lie on the same line segment of $s(2^kx)$ then $2^{k}x$ and $2^{k}(x+h_m)$ have the same closest 
      integer and are between the same two closest integers.This means that 
      $s(2^k(x+h_m))-s(2^kx)=\sigma 2^kh_m$ where $\sigma=\pm 1$ and the sign is determined by 
      whether the closest integer is less then or greater then both $2^{k}x$ and $2^{k}(x+h_m)$. 
      This follows from the fact that $2^{k}x$ and $2^{k}(x+h_m)$ lie on the same line segment of
      the function $s(x)$ which has slope $1$ or $-1$.

      Notice that this sequence
      $\left\{\frac{b(x+h_m)-b(x)}{h_m}\right\}_{m=1}^\infty=\left\{\sum_{k=0}^{m}\sigma_k+S_{m-1}\right\}_{m=1}^\infty$
      can not converge as $\lim_{m\ra \infty} \sigma_k$ does not exist as each term
      is either $1$ or $-1$. This means that the general limit
      \[\lim_{m\ra \infty} \frac{b(x+h_m)-b(x)}{h_m}\]
      does not exists. And because $\lim_{m\ra \infty}h_m$ =0 we know that
      \[\lim_{h\ra 0} \frac{b(x+h_m)-b(x)}{h_m}\]
      Does not exists. And so $b(x)$ is not differentiable at any $x\in\R$.

      %$|a|-|b|=|a+b-b|-|b|\leq |a+b|+|-b|-|b|=|a+b|$.
      \newpage
   \item[2.] First we will show that $f(x)$ is differentiable at zero. Notice that
      \[f'(0)=\lim_{h\ra 0}\frac{f(0+h)-f(0)}{h}=\lim_{h\ra 0}\frac{f(h)}{h}\]
      And notice that $0$ is a cluster point of the domain of $f(h)/h$, and for any
      $h<0$ we have that $-h\leq \frac{f(h)}{h}<0$ and for any $h>0$ that $0\leq
         \frac{f(h)}{h}<h$ and because $\lim_{h\ra 0^-}-h=0=\lim_{h\ra 0^-}0$ and
      $\lim_{h\ra 0^+}0=0=\lim_{h\ra 0^+}h$ by the squeeze theorem we have that
      \(f'(0)=\lim_{h\ra 0}\frac{f(h)}{h}=0\) To see that $f(x)$ is not
      differentiable anywhere else it is enough to show that it is not continuous
      anywhere else. Let $x\in \Q$ such that $x\neq 0$. Notice that for
      $\epsilon=x^2/2$ and for any $\delta>0$ that there will always exists an
      irrational number $r$ such that $x<r<x+\delta$ which means
      $|f(x)-f(r)|=x^2>x^2/2=\epsilon$. Likewise let $x\in \mathbb{I}$. Notice that
      for $\epsilon=x^2/2$ and for any $\delta>0$ that there will always exists an
      rational number $r$ such that $x<r<x+\delta$ which means
      $|f(x)-f(r)|=r^2>x^2>x^2/2=\epsilon$. So $f(x)$ is discontinuous at all $x\neq
         0$ and therefore not differentiable.

   \item[3.] Let $f(x)=x^3-3x^2+2x$\\ % need some more here. theorems from class only use open intervals and deriv neq 0. but this isnt really needed
      (1) Notice first the $f(x)$ and $f'(x)$ are continuous everywhere and let
      $I=(-\infty, 1-\frac{\sqrt{3}}{3})$. Notice that for all $x<1-\frac{\sqrt{3}}{3}$ that $f'(x)>0$ and so
      by the inverse function theorem $f$ has a local inverse that is differentiable.
      However the largest locally invertible interval is $(-\infty, 1-\frac{\sqrt{3}}{3}]$
      as $f(x)<f(1-\frac{\sqrt{3}}{3})$ for all $x<1-\frac{\sqrt{3}}{3}$. And so there
      exists a function $f^{-1}:(-\infty,f(1-\frac{\sqrt{3}}{3})]\ra(-\infty,1-\frac{\sqrt{3}}{3}]$ 
      which behaves the same as the inverse function on the interval $(-\infty, 1-\frac{\sqrt{3}}{3})$ but also maps
      $f(1-\frac{\sqrt{3}}{3})\mapsto 1-\frac{\sqrt{3}}{3}$.\\
      (2) Recall that from the inverse function theorem that $(f^{-1})'(0)=\frac{1}{f'(f^{-1}(0))}=\frac{1}{f'(0)}=\frac{1}{2}$\\
      (3) The only points where $f'(x)=0$ are $x_1$ and $x_3$. This means that local
      inversion around either $x_1$ or $x_3$ is impossible unless we consider intervals where
      $x_1$ and $x_2$ are not interior points and instead are boundary points.
      Local inversion around $x_2=1$ and $x_4=2$ is possible for any interval that does
      not have $x_1$ or $x_3$ as interior points. % maybe give intervals


   \item[4.] Let $S,T$ be sets such that $s<t$ for all $s\in S$ and $t\in T$, let $s^*:=\sup S$ and $t^*:=\inf T$\\
      (1) Assume that this is not true and that $\sup S=s^*> \inf T=t^*$,
      this means that there must exist some $s\in S$ such that $t^*<s\leq s^*$ as if there were not $t^*$ would be a
      smaller upper bound for the set $S$ than $s^*$. Likewise we know that there must exist
      some $t\in T$ such that $t^*\leq t< s^*$ as if there were not $s^*$ would be a
      larger lower bound for the set $T$ than $t^*$. And by assumption we know that $s<t$ meaning
      $t^*<s<t<s^*$. This must mean that $s$ is a greater lower bound for the set $T$ then $t^*$, as $s<t$
      for all $t\in T$ and $t^*<s$.\\
      (2) Assume that for all $\epsilon>0$ that there exists $s\in S$ and $t\in T$ such that $t-s<\epsilon$. 
      And assume that $s^*\neq t^*$ and that $s^*<t^*$ and so $t^*-s^*=\epsilon^*$ for some $\epsilon^*>0$.
       However by our there must exists $s\in S$ and $t\in T$ such that $t-s<t^*-s^*=\epsilon^*$ but this is a
      contradiction because it must also be the case $s\leq s^*<t^*\leq t$ meaning $t-s\geq t^*-s^*=\epsilon^*$.

\end{itemize}

\end{document}