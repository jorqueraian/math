\documentclass[12pt]{amsart}
\usepackage{preamble}

\begin{document}
\begin{center}
   \textsc{Math 517. HW 3\\ Ian Jorquera}
\end{center}
\vspace{1em}

\begin{itemize}
   \item[1.] Let $f$ be defined on $[0,1]$ such that $f$ is integrable on $[\delta,1]$ for all $0<\delta<1$. % i think this is true
            Becasue 

   \item[2.] First we will show that $\sum_{k=1}^\infty\frac{1}{k}$ diverges. Let $\epsilon=\frac{1}{2}$ 
   and notice that for any $N$ we can pick $n=N$ and $m=2n$ and notice that $|s_m-s_n|=\sum_{k=n}^{2n}\frac{1}{k}\geq \frac{2n-n}{2n}=\frac{1}{2}$
   And so the sequence of partial sums fails to be Cauchy, and so diverges.
   Notice that for any $x<1$ we have that $k^x\leq q k$ for all $k\geq 1$, meaning $\frac{1}{k} \leq \frac{1}{k^x}$ 
   and so by comparison we have that $\sum_{k=1}^\infty\frac{1}{k^x}$ diverges.
   Now we will show that $\sum_{k=1}^\infty\frac{1}{k^x}$ converges when $x>1$.
   Consider the integral comparison test where we have that $\int_1^\infty\frac{1}{k^x}dk=\lim_{t\ra \infty}\frac{1-x}{t^{x-1}}-\frac{1-x}{1^{x-1}}=x-1$.
   And because the integral converges we know the series must as well.

   \item[3.] this seems very easy unless the definition is different then what we had before
   \item[3.] 
   \item[4. ] From a previous HW or Exam or what ever we showed by squeez theorem... Notice that 
   \[DF(0,0)= \begin{pmatrix}\frac{\partial f_1}{\partial x} & \frac{\partial f_1}{\partial y}\\ \frac{\partial f_2}{\partial x} & \frac{\partial f_2}{\partial y}\end{pmatrix}  = \begin{pmatrix}1 & 0\\ 0 & 1\end{pmatrix}\]
    which is a invertible matrix. However the inverse function theorem
\end{itemize}

\end{document}