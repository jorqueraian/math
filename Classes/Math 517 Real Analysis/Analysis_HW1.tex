\documentclass[12pt]{amsart}
\usepackage{preamble}

\begin{document}
\begin{center}
    \textsc{Math 517. HW 1\\ Ian Jorquera}
\end{center}
\vspace{1em}

\begin{enumerate}
    % Let $\N$ be the set of natural numbers. Define the real number intervals $J_n = (1, 1 + 1/n)$ for all $n \in \N$. Find $\bigcup_{n\in\N} J_n$ and 
    % $\bigcap_{n\in\N} J_n$. Then find their respective complementary in \R. [
    \item
          Notice that because $1+\frac{1}{n+1}\leq 1+\frac{1}{n}$ we know that $J_{n+1}
              \se J_n$. And so $\bigcup_{n\in\N}J_n=J_1$. And so its compliment is $(-\infty,1]\cup [2,\infty)$. Now we will show that
          $\bigcap_{n\in\N}J_n$ is empty. To see this consider any real number $r\in
              J_1$. Notice that because $r>1$ we know that there must exist some $k\in \N$
          such that $1<1+\frac{1}{k}<r$ by the Archimedean principle meaning $r\not\in
              J_{k}$, meaning $r\not\in \bigcap_{n\in\N}J_n$, so
          $\bigcap_{n\in\N}J_n=\emptyset$. And likewise this means that the compliment is all of $\R$.



          % Show that the map h : N × N → N given by h(n, m) = 2^n3^m is one-to-one. You may use the Fundamental theorem of arithmetic, but
          % there are alternative routes.
    \item This follows from the fundamental theorem of arithmetic. Consider
          $n_1,n_2,m_1,m_2\in \N$ such that
          $h(n_1,m_1)=2^{n_1}3^{m_1}=2^{n_2}3^{m_2}=h(n_2,m_2)$. From the fundamental
          theorem of arithmetic we know that every positive number factors uniquely into
          the product of prime powers. And because $2^{n_1}3^{m_1}=2^{n_2}3^{m_2}$ we
          have that $n_1=n_2$ and $m_1=m_2$.


          % Suppose that for each k ∈ N, D_k is nonempty and is either a finite or denumerable set. Show that the union D = ∪+{k=2}^∞ D_k is countable
    \item For each $k\in\N$ let $D_k$ be a nonempty denumerable set. Let
          $D=\bigcup_{k=1}^{\infty}D_k=\bigcup_{j\in I_1}D_j\cup \bigcup_{j\in I_2}D_j$
          where the sets $D_j$ for $j\in I_1$ are finite and the sets $D_j$ for $j\in
              I_2$ are denumerable. WLOG we will assume all sets are disjoint. If they were
          not we would skip repeated elements in the processes below. Notice first that
          $\bigcup_{j\in I_1}D_j$ is countable. To see this, first consider the case
          where $I_1$ is finite, in this case the finite union of finite sets would be
          finite and so $\bigcup_{j\in I_1}D_j$ is finite.
          

          Alternatively if $I_1$ is an infinite subset of $\N$ then $\bigcup_{j\in
                  I_1}D_j$ would be denumerable: to see this notice that because every $D_j$ is
          finite there exists some bijection $f_j:J_{|D_j|}\ra D_j$. WLOG we will assume
          that $I_1=\N$. Now we may construct a function $f:\N\ra \bigcup_{j\in I_1}D_j$
          where for $\ell\leq |D_{1}|$ we have that $f(\ell)=f_{1}(\ell)$, and like wise
          when $\sum_{1\leq i <n}|D_{i}|<\ell\leq \sum_{1\leq i \leq n}|D_{i}|$ we have
          that $f(\ell)=f_n(\ell-\sum_{1\leq i <n}|D_{i}|)$. And so $f$ is one-to-one by
          construction and is surjective as for any element in any set there are only
          finitely many sets with finitely many element that come before it.


          Now we will show that $\bigcup_{j\in I_2}D_j$ is denumerable. Because each
          $D_k$ for $k\in I_2$ is denumerable we know that there exists some bijection
          $f_k:\N\ra D_k$. Assume first that $I_2$ is finite, in which case we can
          construct the function $f:\N\ra \bigcup_{j\in I_2}D_j$ such that $n\mapsto
              f_{r}(q+1)$ where $n=q\cdot |I_2|+r$ from the division algorithm. Notice that
          this function is injective from the uniqueness of the division algorithm and
          the fact that $f_k$ is injective for all $k$. Now consider the element
          $f_k(\ell)$ where $1\leq k\leq |I_2|$ and $\ell\geq 1$, then $f((\ell-1)\cdot
              |I_2|+k)=f_k(\ell)$ and so $f$ is surjective.


          Finally consider the case where $I_2$ is an infinite subset of $\N$. WLOG we
          will assume that $I_2=\N$. In which case construct the function $f:\N\ra
              \bigcup_{j\in I_2}D_j$ such that $f(1)=f_1(1)$ and then follow the table below.

          \begin{center}
              \begin{tabular}{c|c|c|c|c|c}
                  $f_1(1)$\tikzmark{a} & $f_1(2)$\tikzmark{b} & $f_1(3)$\tikzmark{d} & $f_1(4)$\tikzmark{g} & \dots\tikzmark{k} & \\ \hline
                  $f_2(1)$\tikzmark{c} & $f_2(2)$\tikzmark{e} & $f_2(3)$\tikzmark{h} & \dots                & \dots             & \\ \hline
                  $f_3(1)$\tikzmark{f} & $f_3(2)$\tikzmark{i} & \dots                & \dots                & \dots             & \\ \hline
                  $f_4(1)$\tikzmark{j} & \dots                & \dots                & \dots                & \dots             & \\ \hline
                  $\vdots$             & $\ddots$             & $\ddots$             & $\ddots$             & $\ddots$          & \\
              \end{tabular}
              \begin{tikzpicture}[overlay, remember picture]
                  \draw [->, blue] ([xshift=0pt, yshift=2pt]{pic cs:a}) to ([xshift=-25pt, yshift=2pt]{pic cs:b});
                  \draw [->, blue] ([xshift=-25pt, yshift=0pt]{pic cs:b}) to ([xshift=0pt, yshift=4pt]{pic cs:c});
                  \draw [->, blue] ([xshift=0pt, yshift=2pt]{pic cs:c}) to ([xshift=-25pt, yshift=2pt]{pic cs:d});

                  \draw [->, blue] ([xshift=-25pt, yshift=0pt]{pic cs:d}) to ([xshift=0pt, yshift=4pt]{pic cs:e});
                  \draw [->, blue] ([xshift=-25pt, yshift=0pt]{pic cs:e}) to ([xshift=0pt, yshift=4pt]{pic cs:f});
                  \draw [->, blue] ([xshift=0pt, yshift=2pt]{pic cs:f}) to ([xshift=-25pt, yshift=2pt]{pic cs:g});

                  \draw [->, blue] ([xshift=-25pt, yshift=0pt]{pic cs:g}) to ([xshift=0pt, yshift=4pt]{pic cs:h});
                  \draw [->, blue] ([xshift=-25pt, yshift=0pt]{pic cs:h}) to ([xshift=0pt, yshift=4pt]{pic cs:i});
                  \draw [->, blue] ([xshift=-25pt, yshift=0pt]{pic cs:i}) to ([xshift=0pt, yshift=4pt]{pic cs:j});
                  \draw [->, blue] ([xshift=0pt, yshift=2pt]{pic cs:j}) to ([xshift=-10pt, yshift=2pt]{pic cs:k});
              \end{tikzpicture}
          \end{center}
          Notice that this function is injective by construction and from the fact that each $f_j$ is injective. To see that this function is
          surjective consider an element $f_k(\ell)$ where
          $k\geq 1$ and $\ell\geq 1$. Notice that this function is achieved in the $k+\ell-1$th diagonal of which each diagonal has finitely many
          elements and so there is a finite number of entries before the function $f$ hits $f_k(\ell)$. And so in all cases $\bigcup_{j\in I_2}D_j$ is denumerable.


          Finally we know that $D=\bigcup_{k=1}^{\infty}D_k=\bigcup_{j\in I_1}D_j\cup
              \bigcup_{j\in I_2}D_j$ is countable as this is union is a special case of the
          previous parts.

        
    % Prove that (a) if x ∈ Q and y ∈ I then x + y ∈ I, and (b) If x ∈ Q, x != 0 and y ∈ I, then xy ∈ I.
    \item (a) Assume otherwise, that for $x\in\Q$ and $y\in\mathbb{I}$ that $x+y\in\Q$. Because $\Q$ is a field we know that $-x\in\Q$
          and so $x+y-x=y\in\Q$. Which is not the case so $x+y\in\mathbb{I}$.

          
          \noindent(b) Assume otherwise that for $x\in\Q$ where $x\neq 0$ and
          $y\in\mathbb{I}$ that $xy\in\Q$. Because $\Q$ is a field and $x\neq 0$ we know that $x^{-1}=1/x \in\Q$
          and so $xy/x=y\in\Q$. Which is not the case so $xy\in\mathbb{I}$.


    % Let (a_n)_n be a Cauchy sequence. Show that there is a subsequence (a_{n_k})_k such that ∣ a_{n_{k+1}} − a_{n_k} < 1/2^k for all k.
    \item Let ${(a_n)}_{n=1}^{\infty}$ be a Cauchy sequence. That is for all
          $\varepsilon>0$ there exists some $N$ such that $|a_n-a_m|<\varepsilon$ for all
          $n,m>N$. Now consider a subsequence constructed as follows: For each $k\in \N$
          Let $\varepsilon_k=\frac{1}{2^k}$ in which case there exists some $N_k$ such
          that $|a_n-a_m|<\frac{1}{2^k}$ for all $n,m>N_k$, so let the $k$th element of
          the sequence be $a_{n_k}=a_n$ where $a_n$ is distinct from all previous element
          and $n>N_k$.


    % Prove that (a) A Cauchy sequence in R is convergent. (b) A convergent sequence in R is Cauchy.
    \item (a) Consider a Cauchy sequence ${(a_k)}_{k=1}^{\infty}$. That is for all $\varepsilon>0$ there exists some $N$
          such that $|a_n-a_m|<\varepsilon$ for all $n,m>N$. First we will show that the sequence $(a_k)_{k=1}^{\infty}$ is bounded.
          Let $\varepsilon=1$ in which case we know that there exists some $N$ such that $|a_N-a_m|<1$ for all $m>N$. This means that
          for all $m>N$ we know that $|a_m|<|a_N|+1$. And so the sequence ${(a_k)}_{k=1}^{\infty}$ is bounded by $\max(|a_1|,\dots,|a_N|,|a_N|+1)$.


          This means the sequence ${(a_k)}_{k=1}^{\infty}$ has a convergent subsequence ${(a_{k_j})}_{j=1}^{\infty}$, and let the limit of this subsequence be $L$.
          Now to see the entire sequence converges to $L$. Let $\varepsilon>0$ in which case there exists some $N$ such that $|a_n-a_m|<\varepsilon/2$
          for all $n,m>N$ and likewise there exists some $M$ such that $|a_{k_j}-L|<\varepsilon/2$ for all $k_j>M$. Now assume that $n,k_j>\max(M,N)$
          In which case we know that $|a_n-L|\leq |a_n-a_{k_j}|+|a_{k_j}-L|\leq \varepsilon/2+\varepsilon/2=\varepsilon$. And so ${(a_k)}_{k=1}^{\infty}$
          converges to $L$.


          \noindent(b) Let ${(a_k)}_{k=1}^{\infty}$ be a convergent sequence with limit $L$. That means for all $\varepsilon>0$ there exists some $M$ such that
          $|a_k-L|<\varepsilon/2$ for all $k>M$. And so for any $n,m>M$ we have that $|a_n-a_m|\leq |a_n-L|+|L-a_m|< \varepsilon/2+\varepsilon/2=\varepsilon$.


    \item Consider the series $\sum_{k=1}^\infty \frac{1}{k}\sin^2(k)=\sum_{k=1}^\infty
              \frac{1}{2k}(1-\cos(2k))$ and assume it converges. Consider also the series
          $\sum_{k=1}^\infty \frac{1}{2k}\cos(2k)$ which converges by theorem 3.11.11 as
          ${\left(\frac{1}{2k}\right)}_k$ is a decreasing sequence with
          $\lim_{k\ra\infty}\frac{1}{2k}=0$ and $2$ is not a multiple of $2\pi$. This
          means that the series
          \[\sum_{k=1}^\infty \frac{1}{k}\sin^2(k)+\sum_{k=1}^\infty \frac{1}{2k}\cos(2k)=\sum_{k=1}^\infty \frac{1}{2k}(1-\cos(2k))+\frac{1}{2k}\cos(2k)
              =\sum_{k=1}^\infty \frac{1}{2k}=\frac{1}{2}\sum_{k=1}^\infty \frac{1}{k}\]
          would converge which is not the case as the harmonic series diverges.
\end{enumerate}

\end{document}