\documentclass[12pt]{amsart}
\usepackage{preamble}

\begin{document}
\begin{center}
    \textsc{Math 517. HW 1\\ Ian Jorquera}
\end{center}
\vspace{1em}

\begin{enumerate}
\item Notice that because $1+\frac{1}{n+1}\leq 1+\frac{1}{n}$ we know that $J_{n+1} \se J_n$. And so $\bigcup_{n\in\N}J_n=J_1$. 
Now we will show that $\bigcap_{n\in\N}J_n$ is empty. To see this consider any real number $r\in J_1$. Notice that because $r>1$ 
we know that there must exist some $k\in \N$ such that $1<1+\frac{1}{k}<r$ by the Archimedean principle meaning $r\not\in J_{k}$, 
meaning $r\not\in \bigcap_{n\in\N}J_n$, so $\bigcap_{n\in\N}J_n=\emptyset$.\\

\item This follows from the fundamental theorem of arithmetic. Consider $n_1,n_2,m_1,m_2\in \N$ such that 
$h(n_1,m_1)=2^{n_1}3^{m_1}=2^{n_2}3^{m_2}=h(n_2,m_2)$. From the fundamental theorem of arithmetic we know that every positive number
factors uniquely into the product of prime powers. And because $2^{n_1}3^{m_1}=2^{n_2}3^{m_2}$ we have that $n_1=n_2$ and $m_1=m_2$.\\


\item For each $k\in\N$ let $D_k$ be a nonempty denumerable set. Consider $D=\bigcup_{k=1}^{\infty}D_k=\bigcup_{j\in I_1}D_j\cup \bigcup_{j\in I_2}D_j$ 
where the sets $D_j$ for $j\in I_1$ are finite and the sets $D_j$ for $j\in I_2$ are denumerable. WLOG we will assume all sets are disjoint. 
Notice first that $\bigcup_{j\in I_1}D_j$ is countable. To this first consider the case where $I_1$ is finite, in this case the finite union 
of finite sets would be finite and so $\bigcup_{j\in I_1}D_j$ finite. Alternatively if $I_1$ is an infinite subset of $\N$ then 
$\bigcup_{j\in I_1}D_j$ would be denumerable. Because every $D_j$ is finite there exists some bijection $f_j:J_{|D_j|}\ra D_j$. 
Meaning we may construct a function $f:\N\ra \bigcup_{j\in I_1}D_j$ where for $\ell\leq |D_{j_1}|$ we have that $f(\ell)=f_{j_i}(\ell)$, 
and like wise when $\sum_{1\leq i <n}|D_{j_i}|<\ell\leq \sum_{1\leq i \leq n}|D_{j_i}|$ we have that $f(\ell)=f_n(\ell-\sum_{1\leq i <n}|D_{j_i}|)$.
$f$ is one-to-one by construction and is surjective as for any element in an set there are only finitely many sets with finitely 
many element that come before it.\\
Now we will show that $\bigcup_{j\in I_2}D_j$ is denumerable. Because each $D_k$ is denumerable we know that there exists some bijection
$f_k:\N\ra D_k$. Assume first that $I_2$ is finite, in which case we can construct the function $f:\N\ra \bigcup_{j\in I_2}D_j$ such that
$n\mapsto f_{r}(q)$ where $n=q\cdot |I_2|+r$ from the division algorithm. Notice that this function is injective from the 
uniqueness of the division algorithm and the fact that $f_k$ in injective for all $k$. Now consider the element $f_k(\ell)$ where 
$1\leq k\leq |I_2|$ and $\ell\geq 0$, then $f(\ell\cdot |I_2|+k)=f_k(\ell)$ and so $f$ is surjective.\\ % should be \ell + 1 i think. fix late
Finally consider the case where $I_2$ is an infinite subset of $\N$. In which case construct the function $f:\N\ra \bigcup_{j\in I_2}D_j$
according to the table below
\begin{center}
\begin{tabular}{c|c|c|c|c|c}
    $f_1(1)$\tikzmark{a} & $f_1(2)$\tikzmark{b} & $f_1(3)$\tikzmark{d} & $f_1(4)$\tikzmark{g} & \dots\tikzmark{k} \\\hline
    $f_2(1)$\tikzmark{c} & $f_2(2)$\tikzmark{e} & $f_2(3)$\tikzmark{h} & \dots & \dots\\ \hline
    $f_3(1)$\tikzmark{f} & $f_3(2)$\tikzmark{i} &\dots & \dots & \dots & \\ \hline
    $f_4(1)$\tikzmark{j} &\dots&\dots&\dots&\dots& \\ \hline
    $\vdots$ & $\ddots$ &$\ddots$ & $\ddots$ & $\ddots$ &\\
\end{tabular}
\begin{tikzpicture}[overlay, remember picture]
        \draw [->, blue] ([xshift=0pt, yshift=2pt]{pic cs:a}) to ([xshift=-25pt, yshift=2pt]{pic cs:b});
        \draw [->, blue] ([xshift=-25pt, yshift=0pt]{pic cs:b}) to ([xshift=0pt, yshift=4pt]{pic cs:c});
        \draw [->, blue] ([xshift=0pt, yshift=2pt]{pic cs:c}) to ([xshift=-25pt, yshift=2pt]{pic cs:d});

        \draw [->, blue] ([xshift=-25pt, yshift=0pt]{pic cs:d}) to ([xshift=0pt, yshift=4pt]{pic cs:e}); 
        \draw [->, blue] ([xshift=-25pt, yshift=0pt]{pic cs:e}) to ([xshift=0pt, yshift=4pt]{pic cs:f});
        \draw [->, blue] ([xshift=0pt, yshift=2pt]{pic cs:f}) to ([xshift=-25pt, yshift=2pt]{pic cs:g});

        \draw [->, blue] ([xshift=-25pt, yshift=0pt]{pic cs:g}) to ([xshift=0pt, yshift=4pt]{pic cs:h}); 
        \draw [->, blue] ([xshift=-25pt, yshift=0pt]{pic cs:h}) to ([xshift=0pt, yshift=4pt]{pic cs:i});
        \draw [->, blue] ([xshift=-25pt, yshift=0pt]{pic cs:i}) to ([xshift=0pt, yshift=4pt]{pic cs:j});
        \draw [->, blue] ([xshift=0pt, yshift=2pt]{pic cs:j}) to ([xshift=-10pt, yshift=2pt]{pic cs:k});
\end{tikzpicture}
\end{center}
Notice that this function is injective by construction and from the fact that each $f_j$ is injective. To see that this function is 
surjective consider an element $f_k(\ell)$ where 
$k\geq 1$ and $\ell\geq 1$. Notice that this function is achieved in the $k+\ell-1$th diagonal of which each diagonal has finitely many 
elements and so there is a finite number of entries before the function $f$ hits $f_k(\ell)$.\\

\item (a) Assume otherwise that for $x\in\Q$ and $y\in\mathbb{I}$ that $x+y\in\Q$. Because $\Q$ is a field we know that $-x\in\Q$ 
and so $x+y-x=y\in\Q$. Which is not the case so $x+y\in\mathbb{I}$. (b) Assume otherwise that for $x\in\Q$ where $x\neq 0$ and 
$y\in\mathbb{I}$ that $xy\in\Q$. Because $\Q$ is a field and $x\neq 0$ we know that $x^{-1}=1/x \in\Q$ 
and so $xy/x=y\in\Q$. Which is not the case so $xy\in\mathbb{I}$.

\item Let $(a_n)_{n=1}^{\infty}$ be a Cauchy sequence. That is for all $\varepsilon>0$ there exists some $N$
such that $|a_n-a_m|<\varepsilon$ for all $n,m>N$. Now consider a subsequence constructed as follows: For each $k\in \N$ Let 
$\varepsilon_k=\frac{1}{2^k}$ in which case there exists some $N_k$ such that $|a_n-a_m|<\frac{1}{2^k}$ for all $n,m>N_k$, 
so let the $k$th element of the sequence be $a_{n_k}=a_n$ where $a_n$ is distinct from all previous element and $n>N_k$.\\

\item (a) Consider a Cauchy sequence $(a_k)_{k=1}^{\infty}$. That is for all $\varepsilon>0$ there exists some $N$
such that $|a_n-a_m|<\varepsilon$ for all $n,m>N$. First we will show that the sequence $(a_k)_{k=1}^{\infty}$ is bounded. 
Let $\varepsilon=1$ in which case we know that there exists some $N$ such that $|a_N-a_m|<1$ for all $m>N$. This means that 
for all $m>N$ we know that $|a_m|<|a_N|+1$. And so the sequence $(a_k)_{k=1}^{\infty}$ is bounded by $\max(a_1,\dots,|a_N|,|a_N|+1)$..\\
This means the sequence $(a_k)_{k=1}^{\infty}$ has a convergent subsequence $(a_{k_j})_{j=1}^{\infty}$, so let the limit of this subsequence be $L$. 
Now to see the entire sequence converges to $L$ let $\varepsilon>0$ in which case there exists some $N$ such that $|a_n-a_m|<\varepsilon/2$ 
for all $n,m>N$ and likewise there exists some $M$ such that $|a_{k_j}-L|<\varepsilon/2$ for all $k_j>M$. Now assume that $n,k_j>\max(M,N)$ 
In which case we know that $|a_n-L|\leq |a_n-a_{k_j}|+|a_{k_j}-L|\leq \varepsilon/2+\varepsilon/2=\varepsilon$. And so $(a_k)_{k=1}^{\infty}$ 
converges to $L$.\\
(b) Let $(a_k)_{k=1}^{\infty}$ be a convergent sequence with limit $L$. That means for all $\varepsilon>0$ there exists some $M$ such that 
$|a_k-L|<\varepsilon/2$ for all $k>M$. And so for any $n,m>M$ we have that $|a_n-a_m|\leq |a_n-L|+|L-a_m|< \varepsilon/2+\varepsilon/2=\varepsilon$\\

\item

\end{enumerate}


\end{document}
