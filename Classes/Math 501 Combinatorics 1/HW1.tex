\documentclass[12pt]{amsart}
% packages
\usepackage{graphicx}
\usepackage{setspace}
\usepackage{amssymb,amsmath,amsthm,amsfonts,amscd}
\usepackage{hyperref}
\usepackage{color}
\usepackage{booktabs}
\usepackage{tabularx}
\usepackage{enumitem}
\usepackage[retainorgcmds]{IEEEtrantools}
\usepackage[notref,notcite,final]{showkeys}
\usepackage[final]{pdfpages}
\usepackage{fancyhdr}
\usepackage{upgreek}
\usepackage{multicol}
% set margin as 0.75in
\usepackage[margin=0.75in]{geometry}

% tikz-related settings
\usepackage{tikz}
\usepackage{tikz-cd}
\usetikzlibrary{cd}

% theorem environments with italic font
\newtheorem{thm}{Theorem}[section]
\newtheorem*{thm*}{Theorem}
\newtheorem{lemma}[thm]{Lemma}
\newtheorem{prop}[thm]{Proposition}
\newtheorem{claim}[thm]{Claim}
\newtheorem{corollary}[thm]{Corollary}
\newtheorem{conjecture}[thm]{Conjecture}
\newtheorem{question}[thm]{Question}
\newtheorem{procedure}[thm]{Procedure}
\newtheorem{assumption}[thm]{Assumption}

% theorem environments with roman font (use lower-case version in body
% of text, e.g., \begin{example} rather than \begin{Example})
\newtheorem{Definition}[thm]{Definition}
\newenvironment{definition}
{\begin{Definition}\rm}{\end{Definition}}
\newtheorem{Example}[thm]{Example}
\newenvironment{example}
{\begin{Example}\rm}{\end{Example}}

\theoremstyle{definition}
\newtheorem{remark}[thm]{\textbf{Remark}}

% special sets
\newcommand{\A}{\mathbb{A}}
\newcommand{\C}{\mathbb{C}}
\newcommand{\F}{\mathbb{F}}
\newcommand{\N}{\mathbb{N}}
\newcommand{\Q}{\mathbb{Q}}
\newcommand{\R}{\mathbb{R}}
\newcommand{\Z}{\mathbb{Z}}
\newcommand{\cals}{\mathcal{S}}
\newcommand{\ZZ}{\mathbb{Z}_{\ge 0}}
\newcommand{\cala}{\mathcal{A}}
\newcommand{\calb}{\mathcal{B}}
\newcommand{\cald}{\mathcal{D}}
\newcommand{\calh}{\mathcal{H}}
\newcommand{\call}{\mathcal{L}}
\newcommand{\calr}{\mathcal{R}}
\newcommand{\la}{\mathbf{a}}
\newcommand{\lgl}{\mathfrak{gl}}
\newcommand{\lsl}{\mathfrak{sl}}
\newcommand{\lieg}{\mathfrak{g}}

% math operators
\DeclareMathOperator{\kernel}{\mathrm{ker}}
\DeclareMathOperator{\image}{\mathrm{im}}
\DeclareMathOperator{\rad}{\mathrm{rad}}
\DeclareMathOperator{\id}{\mathrm{id}}
\DeclareMathOperator{\hum}{[\mathrm{Hum}]}
\DeclareMathOperator{\eh}{[\mathrm{EH}]}
\DeclareMathOperator{\lcm}{\mathrm{lcm}}
\DeclareMathOperator{\Aut}{\mathrm{Aut}}
\DeclareMathOperator{\Inn}{\mathrm{Inn}}
\DeclareMathOperator{\Out}{\mathrm{Out}}
\DeclareMathOperator{\Gal}{\mathrm{Gal}}


% frequently used shorthands
\newcommand{\ra}{\rightarrow}
\newcommand{\se}{\subseteq}
\newcommand{\ip}[1]{\langle#1\rangle}
\newcommand{\dual}{^*}
\newcommand{\inverse}{^{-1}}
\newcommand{\norm}[2]{\|#1\|_{#2}}
\newcommand{\abs}[1]{\lvert #1 \rvert}
\newcommand{\Abs}[1]{\bigg| #1 \bigg|}
\newcommand\bm[1]{\begin{bmatrix}#1\end{bmatrix}}
\newcommand{\op}{\text{op}}

% nicer looking empty set
\let\oldemptyset\emptyset
\let\emptyset\varnothing

\setlist[enumerate,1]{topsep=1em,leftmargin=1.8em, itemsep=0.5em, label=\textup{(}\arabic*\textup{)}}
\setlist[enumerate,2]{topsep=0.5em,leftmargin=3em, itemsep=0.3em}

%pagestyle
%\pagestyle{fancy} 

\begin{document}
\begin{center}
    \textsc{Math 501. HW 1\\ Ian Jorquera\\ Collaborations: Clare Jones}
\end{center}

\vspace{1em}
% See http://www.mathematicalgemstones.com/maria/Math501Fall22.php
% for problems

% sage: https://sagecell.sagemath.org/

\begin{itemize}
\item[(1)] % (2+) [4 points]
For each problem we have the choice to pick or not to pick the problem. This means for any problem with $k$ points we can create the generating function $(1+x^k)$ which represents the option of not picking the problem with $1$ and picking the problem with $x^k$. So the factorization of our generating function is 
$$a_n=(1+x^2)^{16}(1+x^3)^{2}(1+x^4)^{2}(1+x^8)^2$$

At this point to compute $a_{10}$ we can multiple out our generating function and get the coeffecient of $x^{10}$, which represent the exact number of subset of the homework problems whose total number of points equals $10$. We find that $a_{10}=5658$.\\

% sage code
% R.<x> = PolynomialRing(QQ)
% poly = (1+x^2)^16*(1+x^3)^2*(1+x^4)^2*(1+x^8)^2; poly

\item[(2)] % (1+) [2 points]
Let $f:A\ra B$ be a bijection between sets $A$ and $B$. Now let $g:B\ra A$ be a function where an element $b\mapsto a$ where $f(a)=b$. First we know that such an $a$ exists by the surjectivity of $f$. Similarly we know that $a$ is unique by injectivity of $f$ as any elements in $A$ that map to the element $b$ are the same. Now we will show $g$ is the inverse of $f$. Notice first that for any element $a\in A$ where $f$ maps $a\mapsto b$ that $g(f(a))=g(b)=f(a)$. Now consider any element $b\in B$ along with the unique $a\in A$ where $a\mapsto b$ with $f$. Notice that $f(g(b))=f(a)=b$. And so $g$ is the inverse of $f$.\\

Now let $f:A\ra B$ be a function between the sets $A$ and $B$ with the function $g:B\ra A$ such that $f(g(b))=b$ for any $b\in B$ and $g(f(a))=a$ for any $a\in A$. Consider any two elements $a_1,a_2\in A$ such that $f(a_1)=f(a_2)$. Notice that this means $g(f(a_1))=g(f(a_2))$ and so $a_1=a_2$. So $f$ is injective. Now consider an element $b\in B$ and notice that $g(b)\in A$ such that $f(g(b))=b$, meaning $f$ is surjective.\\

\item[(4)] % (1+) [2 points]


Notice that because we can only step north or east and we must pass through $(3,3)$ we can split this problem into two smaller problems, first the problem of going from $(0,0)$ to $(3,3)$ and then the problem of going from $(3,3)$ to $(6,6)$ which are both the same problem. Notice that to get to $(3,3)$ we will have to step north $3$ times and east $3$ times. Meaning there will be a total of $6$ steps, and $3$ will be north, so there are ${6 \choose 3}= 20$ possible paths, or assignments of the three "up"s.
\iffalse Now we can write this problem as a recurrence. So let $R(i,j)$ represents the number of paths from $(i,j)$ to $(3,3)$. For our initial conditions we have that $R(i,j)=1$ if either $i$ or $j$ is $1$ and then for any arbitrary $R(i,j)=R(i+1,j)+R(i,j+1)$. We can this fill this into a $4\times 4$ table

\begin{center}
\begin{tabular}{ |c|c|c|c| } 
 \hline
 1 & 1 & 1 & 1 \\ 
 \hline
 4 & 3 & 2 & 1 \\
 \hline
 10 & 6 & 3 & 1 \\
 \hline
 20 & 10 & 4 & 1 \\
 \hline
\end{tabular}
\end{center}

And then we can extract $R(0,0)=20$.\fi 
And because we need to repeat this process again, between $(3,3)$ and $(6,6)$ we know that there are must be $20\cdot 20=400$ possible paths and bpth problems are independent of each other.\\


\item[(10)] % (2-) [3 points]
Let $R(n)$ represent the number of subsets of $\left[n\right]$ with non consecutive integers, with initial condition $R(0)=1$(the only subset of $\emptyset$ is $\emptyset$) and $R(1)=2$(the subsets of $\{1\}$ are $\emptyset$ and $\{1\}$). In general we can create a recurrence relation for $R(n)$ that counts the number of subsets of $[n]$ with no consecutive integer, by splitting the possible subsets by whether or not they include $n$. If $n$ is included then $n-1$ is not included and so there are $R(n-2)$ possible subsets. And if $n$ is not included then we can still include $n-1$ meaning the number of possible subsets is equal to $R(n-1)$. Therefore in general we can say that $R(n)=R(n-1)+R(n-2)$ for $n>1$. Notice this follow the Fibonacci sequence, where $R(n)=F_{n+2}=F_{n+1}+F_{n}$ with the initial conditions $R(0)=F_2=1$ and $R(1)=F_3=2$.


\end{itemize}

\end{document}


