\documentclass[12pt]{amsart}
% packages
\usepackage{graphicx}
\usepackage{setspace}
\usepackage{amssymb,amsmath,amsthm,amsfonts,amscd}
\usepackage{hyperref}
\usepackage{color}
\usepackage{booktabs}
\usepackage{tabularx}
\usepackage{enumitem}
\usepackage[retainorgcmds]{IEEEtrantools}
\usepackage[notref,notcite,final]{showkeys}
\usepackage[final]{pdfpages}
\usepackage{fancyhdr}
\usepackage{upgreek}
\usepackage{multicol}
\usepackage{fontawesome}
\usepackage{halloweenmath}
% set margin as 0.75in
\usepackage[margin=0.75in]{geometry}

% tikz-related settings
\usepackage{tkz-berge}
\usetikzlibrary{calc,quotes}
\usetikzlibrary{arrows.meta}
\usetikzlibrary{positioning, automata}
\usetikzlibrary{decorations.pathreplacing}

%% For table
\usepackage{tikz}
\usetikzlibrary{tikzmark}

% theorem environments with italic font
\newtheorem{thm}{Theorem}[section]
\newtheorem*{thm*}{Theorem}
\newtheorem{lemma}[thm]{Lemma}
\newtheorem{prop}[thm]{Proposition}
\newtheorem{claim}[thm]{Claim}
\newtheorem{corollary}[thm]{Corollary}
\newtheorem{conjecture}[thm]{Conjecture}
\newtheorem{question}[thm]{Question}
\newtheorem{procedure}[thm]{Procedure}
\newtheorem{assumption}[thm]{Assumption}

% theorem environments with roman font (use lower-case version in body
% of text, e.g., \begin{example} rather than \begin{Example})
\newtheorem{Definition}[thm]{Definition}
\newenvironment{definition}
{\begin{Definition}\rm}{\end{Definition}}
\newtheorem{Example}[thm]{Example}
\newenvironment{example}
{\begin{Example}\rm}{\end{Example}}

\theoremstyle{definition}
\newtheorem{remark}[thm]{\textbf{Remark}}

% special sets
\newcommand{\A}{\mathbb{A}}
\newcommand{\C}{\mathbb{C}}
\newcommand{\F}{\mathbb{F}}
\newcommand{\N}{\mathbb{N}}
\newcommand{\Q}{\mathbb{Q}}
\newcommand{\R}{\mathbb{R}}
\newcommand{\Z}{\mathbb{Z}}
\newcommand{\cals}{\mathcal{S}}
\newcommand{\ZZ}{\mathbb{Z}_{\ge 0}}
\newcommand{\cala}{\mathcal{A}}
\newcommand{\calb}{\mathcal{B}}
\newcommand{\cald}{\mathcal{D}}
\newcommand{\calh}{\mathcal{H}}
\newcommand{\call}{\mathcal{L}}
\newcommand{\calr}{\mathcal{R}}
\newcommand{\la}{\mathbf{a}}
\newcommand{\lgl}{\mathfrak{gl}}
\newcommand{\lsl}{\mathfrak{sl}}
\newcommand{\lieg}{\mathfrak{g}}

% math operators
\DeclareMathOperator{\kernel}{\mathrm{ker}}
\DeclareMathOperator{\image}{\mathrm{im}}
\DeclareMathOperator{\rad}{\mathrm{rad}}
\DeclareMathOperator{\id}{\mathrm{id}}
\DeclareMathOperator{\hum}{[\mathrm{Hum}]}
\DeclareMathOperator{\eh}{[\mathrm{EH}]}
\DeclareMathOperator{\lcm}{\mathrm{lcm}}
\DeclareMathOperator{\Aut}{\mathrm{Aut}}
\DeclareMathOperator{\Inn}{\mathrm{Inn}}
\DeclareMathOperator{\Out}{\mathrm{Out}}
\DeclareMathOperator{\Gal}{\mathrm{Gal}}


% frequently used shorthands
\newcommand{\ra}{\rightarrow}
\newcommand{\se}{\subseteq}
\newcommand{\ip}[1]{\langle#1\rangle}
\newcommand{\dual}{^*}
\newcommand{\inverse}{^{-1}}
\newcommand{\norm}[2]{\|#1\|_{#2}}
\newcommand{\abs}[1]{\lvert #1 \rvert}
\newcommand{\Abs}[1]{\bigg| #1 \bigg|}
\newcommand\bm[1]{\begin{bmatrix}#1\end{bmatrix}}
\newcommand{\op}{\text{op}}

% nicer looking empty set
\let\oldemptyset\emptyset
\let\emptyset\varnothing

%the var phi gang
\let\oldphi\phi
\let\phi\varphi

\setlist[enumerate,1]{topsep=1em,leftmargin=1.8em, itemsep=0.5em, label=\textup{(}\arabic*\textup{)}}
\setlist[enumerate,2]{topsep=0.5em,leftmargin=3em, itemsep=0.3em}

%pagestyle
%\pagestyle{fancy} 

\begin{document}
\begin{center}
    \textsc{Math 501. HW 12\\ Ian Jorquera\\ Collaborators: Kelsey}
\end{center}
\vspace{1em}
% See http://www.mathematicalgemstones.com/maria/Math501Fall22.php
% for problems

% sage: https://sagecell.sagemath.org/
\begin{itemize}

\item[(2)]
\begin{enumerate}[label=(\alph*)]
    \item % (1+) (2 points)
    Consider a locally finite poset $P$ and its corresponding incidence algebra $I(P)$. Consider an element $f:Int(P)\ra \C$ such that $f(x,x)=0$ for some $x\in P$. Notice that in this case for any element $g:Int(P)\ra \C$, we have that $f* g(x,x)=\sum_{y\in[x,x]}f(x,y)g(y,x)=f(x,x)g(x,x)=0$ meaning $f* g\neq \delta$, and so no inverse under convolution exists.\\ 
    Now consider an element $f:Int(P)\ra \C$ such $f(x,x)\neq 0$ for all $x\in P$. And consider the function $g:Int(P)\ra \C$ such that $g(x,x)=\frac{1}{f(x,x)}$ for all $x\in P$ and $g(x,z)=\frac{-1}{f(x,x)}\sum_{y\in(x,z]}f(x,y)g(y,z)$. Notice that in this case $f*g(x,x)=f(x,x)\frac{1}{f(x,x)}=1$ and for any $x<z$ we have that $g(x,z)=\frac{-1}{f(x,x)}\sum_{y\in(x,z]}f(x,y)g(y,z)$ which means that $f(x,x)g(x,z)=-\sum_{y\in(x,z]}f(x,y)g(y,z)$ and so $f*g(x,z)=\sum_{y\in[x,z]}f(x,y)g(y,z)=0$ meaning $g$ is a right inverse of $f$.\\

    The same is true about left inverses that $f(x,x)\neq 0$ for all $x\in P$ if and only if $f$ has a left inverse $g$. Notice that we can construct an inverse $g$ in a similar way such that $g(x,x)=\frac{1}{f(x,x)}$ for all $x\in P$ and $g(x,z)=\frac{-1}{f(x,x)}\sum_{y\in(x,z]}g(x,y)f(y,x)$. Notice that in this case $g*f(x,x)=\frac{1}{f(x,x)}f(x,x)=1$ and for any $x<z$ we have that $g(x,z)=\frac{-1}{f(x,x)}\sum_{y\in(x,z]}g(x,y)f(y,z)$ which means that $g(x,z)f(x,x)=-\sum_{y\in(x,z]}g(x,y)f(y,z)$ and so $g*f(x,z)=\sum_{y\in[x,z]}g(x,y)f(y,z)=0$.\\

    \item %(2-) (3 points)
    Now we have shown that $f(x,x)\neq 0$ for all $x\in P$ if and only if $f$ has a right inverse $g$. Now we will show that a right inverse is also a left inverse. So assume that $f*g=\delta$ which implies that $g*f*g=g$. And from above we know that $g(x,x)\neq 0$ for all $x\in P$ meaning it has a right inverse $h$. This gives us that $g*f*g*h=g*f=\delta=g*h$ meaning $g*f=\delta$. And so $g$ is also a left inverse of $f$.\\
    
    In part (a) we showed that $f(x,x)\neq 0$ for all $x\in P$ if and only if $f$ has a left inverse $g$. Now we will show that left inverses are also right inverse. Assume that we have a left inverse $g$ of $f$, that is $g*f=\delta$. For the same reason as before we know that $g(x,x)\neq 0$ for all $x\in P$, meaning it must have a left inverse $h$ and so $g*f=\delta$ means that $g*f*g=g$ and therefore  $h*g*f*g=f*g=\delta=h*g$, which shows that $f*g=\delta$.\\
\end{enumerate}

\item[(5)] % (2) (3 points)
Here we will construct a bijection $\phi$ between ballot sequences of length $2n$ and linear extension of $[2]\times [n]\ra [2n]$. Here we will consider a ballot sequence of $2n$ to be a rearrangement of the string $1^n2^n$ such that any substring is a rearrangment of $1^\ell2^k$ where $k\leq\ell$ that is there are always at least as many $1$s as there are $2$s. Furthermore for any such substring we will define the notation $n_1(i):=\ell$ and $n_2(i):=k$ to count the number of $1$s and $2$s in the substring of length $i$. And notice that any linear extension $[2]\times [n]\ra [n]$ would preserve the relations $(1,i)<(1,i+1)$ and $(2,i)<(2,i+1)$ for any $1\leq i<n$, and $(1,i)<(2,i)$ for any $1\leq i\leq n$.\\

We will first contract a map $\phi$ from ballot sequences to linear extensions, first consider a fixed ballot sequence, $b_1b_2b_3\dots b_{2n}$. We will construct a bijection such that this string maps to the linear extension of $f:[2]\times [n]\ra [2n]$ where $b_i$ corresponds to the mapping $(1,n_1(i))\ra i$ if $b_i=1$ and $(2,n_2(i))\ra i$ if $b_i=2$. Notice first that this map is a well defined bijection as $(j,k)\mapsto i$ when the $i$th character in the ballot sequence is the $k$th $j\in\{1,2\}$. The $k$th $j$ happens once per ballot sequence and all $i\in[2n]$ have a corresponding input. And because $|[2]\times [n]|=|2n|$ we know this map is a bijection. To see that this map is a linear extension notice that for any elements $(1,i)\leq (1,j)$ (which means $i\leq j$) in the poset $[2]\times[n]$ we know that the position of the $i$th $1$ must be less then or equal to the position of the $j$th $1$ and so $f((1,i))\leq f((1,j))$. For the same reason we know that $(2,i)\leq (2,j)\Rightarrow f((2,i))\leq ((2,j))$. Furthermore consider element $(1,i)< (2,i)$ We know that the position of the $i$th $1$ in the ballot sequence must be before the position of the $i$th $2$ otherwise the original string would not be a ballot sequence meaning $f((1,i)<f (2,i))$. And so we have a linear extension.\\

Now we will construct map $\phi^{-1}$ that maps linear extensions to ballot sequences. consider a linear extension $f:[2]\times [n]\ra [2n]$ and we will construct the string $b_1b_2b_2\dotsb_{2n}$ where $b_i=1$ if $(1,k)\mapsto i$ and $b_i=2$ if $(2,k)\mapsto i$. This effectively tells us that the $k$th $1$ is at position $f((1,k))$, and from the relations of $[2]\times[n]$ we know that the $(1,i)<(2,i)$ which means that the $i$th $1$ occurs before the $i$th 2, meaning any constructed sequence is a ballot sequence.\\

Finally notice that these are inverses, as for any ballot sequence $\phi$ gives us $(j,k)\mapsto i$ when the $i$th character in the ballot sequence is the $k$th $j\in\{1,2\}$, and  $\phi^{-1}$ gives us the $i$th character in the ballot sequence being the $k$th $j\in\{1,2\}$. And alternatively for any linear extension, $f$ where $(j,k)\mapsto i$ we construct a ballot sequence with the $i$th character in the ballot sequence being the $k$th $j\in\{1,2\}$ with  $\phi^{-1}$. And then $\phi$ gives us $(j,k)\mapsto i$ when the $i$th character in the ballot sequence is the $k$th $j\in\{1,2\}$ resulting in the original linear extension.\\ % this is shit

\item[(7)] %(1+)(2 points)

In this case recall that $\zeta^2(x,z)=\sum_{y\in[x,z]}1\cdot 1=|[x,z]|$. And so to find a recursion for the inverse $\mu^2$ we know that $\delta(x,x)=1 = \zeta^2*\mu^2(x,x)=|[x,x]|\mu^2(x,x)$ and so $\mu^2(x,x)=1$. We also have that for any $x<z$ that $\delta(x,z)=0=\zeta^2*\mu^2(x,z)=\sum_{y\in[x,z]}|[x,y]|\mu^2(y,z)$. And we can take out the term with $\mu^2(x,z)$ to find that $\mu^2(x,z)=-\sum_{y\in(x,z]}|[x,y]|\mu^2(y,z)$.\\
  

\end{itemize}

\end{document}


