\documentclass[12pt]{amsart}
\usepackage{preamble}
%\DeclareRobustCommand{\flippedSlash}{\text{\reflectbox{$\setminus$}}}
\DeclareRobustCommand{\rchi}{{\mathpalette\irchi\relax}}
\newcommand{\irchi}[2]{\raisebox{\depth}{$#1\chi$}} % inner command, used by \rchi

\begin{document}
\begin{center}
   \textsc{Math 617. HW 3\\ Ian Jorquera}
\end{center}
\vspace{1em}

\begin{itemize}
   \item[(1)] Let $(X,\mathcal S,\mu)$ be a measure space and $f:X\ra [0,\infty]$ be 
   $\mathcal S$-measurable. We will show that $\nu:\mathcal S\ra [0,\infty]$ where 
   $\nu(A)=\int \rchi_{A}f d\mu$ is a measure.
   To see this notice first that $\nu(\emptyset)=\int \rchi_{\emptyset}f d\mu= 0$
   Now let $E_1,E_2,\dots$ be a sequence of disjoint sets of $X$. Notice that because 
   $\lim\limits_{n\ra\infty}f_n(x)=\rchi_{\bigcup\limits_{k=1}^\infty E_k}f(x)$ where $f_n=\rchi_{\bigcup\limits_{k=1}^n E_k}f(x)$ and
   $f_n(x)\leq f_{n+1}(x)$ we know that with the monotone convergence theorem we have that
   \begin{align*}
      \nu(\bigcup_{k=1}^{\infty} E_k)&=\int \rchi_{\bigcup\limits_{k=1}^\infty E_k}f(x)d\mu=\lim_{n\ra\infty}\int\rchi_{\bigcup\limits_{k=1}^n E_k}f(x)\\
      &=\lim_{n\ra\infty}\sum_{k=1}^n\int\rchi_{E_k}f(x)d\mu=\lim_{n\ra\infty}\sum_{k=1}^n\nu(E_k)=\sum_{k=1}^\infty\nu(E_k)
   \end{align*}
   And so $\nu$ is a measure.


   \item[(2)] Let $f:X\ra[0,\infty]$ be $\mathcal S$-measurable. Meaning there exists a sequence 
   ${f_k}_{k=1}^\infty$ of $\mathcal S$-measurable increasing functions such that $\lim_{k\ra \infty}f_k(x)=f(x)$.
   Notice that this means for any $k$ that $f_k(x)=\sum_{j=1}^{n_k}a_{k}\rchi_{E_j}$ so 
   \[\int f_k d\mu_2=\sum_{j=1}^{n_k}a_{k}\mu_2({E_j})=\sum_{j=1}^{n_k}a_{k}\mu_1({E_j})=\int f_k d\mu_1\]


   \item[(3)]
   \begin{enumerate}[label= (\alph*)]
      \item idk 
      \item idk
   \end{enumerate}

   \item[(4)] Let $\mu(X)=1$ and $h\in\mathcal L^1$. Using 4A.1 and that 
   $h(x)-\int hd\mu\in\mathcal L^1$ by linearity we have that
   \begin{align*}\mu(\{x\in X:|h(x)-\int h d\mu|\geq c\})&\leq\frac{1}{c^2}\int(h(x)-\int hd\mu)^2d\mu\\
   &=\frac{1}{c^2}\int \left(h(x)^2-2h(x)\int hd\mu+\left(\int hd\mu\right)^2\right)d\mu\\
   &=\frac{1}{c^2}\left(\int h(x)^2-2\int \left(h(x)\int hd\mu\right)+\int\left(\int hd\mu\right)^2\right)\\
   &=\frac{1}{c^2}\left(\int h(x)^2-2\left(\int h(x)\right)^2+\mu(X)\left(\int hd\mu\right)^2\right)\\
   &=\frac{1}{c^2}\left(\int h(x)^2-\left(\int h(x)\right)^2\right)\\
   \end{align*}

   \item[(5)] This is true. Let $h:\R\ra[0,\infty)$ be an increasing function meaning 
   that for $x<y$ then $h(x)<h(y)$. We also know that $h$ is Riemann integrable 
   on an closed interval. Notice that for any fixed $t$ and $a<b$ we have that
   \begin{align*}
      \frac{1}{2t}\int_{b-t}^{b-t}h(x)d\lambda(x)-\frac{1}{2t}\int_{a-t}^{a-t}h(x)d\lambda(x)&=
   \frac{1}{2t}\int_{a-t}^{a-t}h(x+(b-a))d\lambda(x)-\frac{1}{2t}\int_{a-t}^{a-t}h(x)d\lambda(x)
   \\&= \frac{1}{2t}\int_{a-t}^{a-t}h(x+(b-a))-h(x)d\lambda(x)>0
   \end{align*}
   which follows from the fact that $h$ is increasing. This means for any $t>0$ and $a<b$
   \[\frac{1}{2t}\int_{a-t}^{a-t}h(x)d\lambda(x)<\frac{1}{2t}\int_{b-t}^{b-t}h(x)d\lambda(x)\]
   and so 
   \[h^*(a)=\sup_{t>0}\frac{1}{2t}\int_{a-t}^{a-t}h(x)d\lambda(x)<\sup_{t>0}\frac{1}{2t}\int_{b-t}^{b-t}h(x)d\lambda(x)=h^*(b)\]

   \item[(6)] This proof is rough, im missing some of the rigorous details.
   Let $G_1=(2/5,3/5)$ the middle 1/5 of the $[0,1]$. Then for $k\geq 1$ let $G_{k+1}=2/5G_k\cup 2/5(G_k+3/2)$ 
   which add the middle 1/5s of the portions not in $G_k$. This construction is very similar to that of the Cantor Set.
   And let $G=\bigcup_{k=1}^{\infty}G_k$ and finally let $H=G\cup(G-1)$.
   First notice that $|G|=\sum_{k=1}^\infty\frac{2^{k-1}}{5^k}=1/3$.
   Notice that the density at $x=0$ is as follows
   \[\lim_{t\downarrow 0}\frac{|H\cap (-t,t)|}{2t}=\lim_{t\downarrow 0}\frac{|G\cap (0,t)|}{t}\]
   
   For simplicity let $D(t)=\frac{|G\cap (0,t)|}{t}$. Notice that $D(t)\leq 2/3$ and $D(3/5)=2/3$. % why because
   Now let $1>t>0$ and let $\ell$ be a positive integer such that $t\in[(2/5)^k,(2/5)^{k-1})$.
   We will show that $D(t)$ is maximized at $(2/5)^k+(1/5)^k$ and minimized at $(2/5)^k$ on the interval $[(2/5)^k,(2/5)^{k-1})$.
   Im really struggling to work these details out, but I believe they are true.


   Notice first that $D((2/5)^k)=\frac{(2/5)^k\frac{1}{3}}{(2/5)^k}=1/3$.
   Notice also that $D((2/5)^k+(1/5)^k)=\frac{(2/5)^k\frac{1}{3}+(1/5)^k}{(2/5)^k+(1/5)^k}$ 
   which goes goes to $\frac{1}{3}$ as $k\ra\infty$.
   This means that $\limsup_{t\downarrow 0}D(t)=1/3$ and $\liminf_{t\downarrow 0}D(t)=1/3$ 
   so the density is $\lim_{t\downarrow 0}D(t)=1/3$.
   
\end{itemize}

\end{document}