\documentclass[12pt]{amsart}
\usepackage{preamble}
%\DeclareRobustCommand{\flippedSlash}{\text{\reflectbox{$\setminus$}}}
\DeclareRobustCommand{\rchi}{{\mathpalette\irchi\relax}}
\newcommand{\irchi}[2]{\raisebox{\depth}{$#1\chi$}} % inner command, used by \rchi

\begin{document}
\begin{center}
   \textsc{Math 617. HW 4\\ Ian Jorquera}
\end{center}
\vspace{1em}

\begin{itemize}
   \item[(1)] % DONE
   Consider the measure space $(X=\{1,\dots,n\},\mathcal S=2^X)$ with the counting measure 
   as $\mu$. Notice that $\mu(X)=n$. Then consider $\mathcal L^1 (X,\mathcal S, \mu)=\{(a_1,\dots,a_n)|a_k\in \F\}$
   equip with the norm $\norm{(a_1,\dots,a_n)}_1=\sum_{k=1}^{n}|a_k|$.
   
   Likewise consider $\mathcal L^5 (X,\mathcal S, \mu)=\{(a_1,\dots,a_n)|a_k\in \F\}$ equip with the 
   norm $\norm{(a_1,\dots,a_n)}_1=\left(\sum_{k=1}^{n}|a_k|^5\right)^{1/5}$.

   From 7.10 we know that $\norm{(a_1,\dots,a_n)}_1\leq \mu(X)^{(5-1)/5}\norm{(a_1,\dots,a_n)}_5$ and so for 
   $(a_1,\dots,a_n)$ a sequence of positive numbers we have that
   \[\left(\sum_{k=1}^{n}a_k\right)^5=\norm{(a_1,\dots,a_n)}_1^5\leq \mu(X)^{4}\norm{(a_1,\dots,a_n)}_5^5=n^4\sum_{k=1}^{n}a_k^5\]


   \item[(2)] % DONE
   Consider the function $f(x)=\log(x)$ when $0<x\leq 1$ and $f(0)=0$. Notice that for any $r>0$ and any $x\in(0,e^{-r})$ that $|\log(x)|>r$. 
   And because $\lambda(0,e^{-r})>0$ we know that $\norm{f}_\infty>r$ for all $r$ and so $f\not\in L^\infty([0,1])$.
   However we can determine with a quick computation on mathematica or integral-calculator that for any fixed $p>0$ that
   \[\int_{0}^1f d\mu=\Gamma(1+p)<\infty\]
   and so $f\in \bigcap_{0<p<\infty}L^p([0,1])$.
   \item[(3)]
   \begin{enumerate}[label=(\alph*)]
    \item % DONE
    Let $(a^n_1,a^n_2,\dots)_{n=1}^{\infty}$ be a convergent sequence of sequences in $c_0$, that converges 
    to a sequence $(a_1,a_2,\dots)$. Meaning for all $\epsilon>0$ there exists some $N>0$ such that for all $n>N$ we have that
    $\norm{(a_k)_k-(a^n_k)_k}_\infty<\epsilon$. Because $\norm{\cdot}_\infty$ is the supremum this means that
    for all $k$ that $|a_k-a_k^n|<\epsilon$.
    Now to show that $(a_1,a_2,\dots)\in c_0$ consider the limit
    \[\lim_{k\ra\infty}|a_k|=\lim_{k\ra\infty}|a_k-a^n_k+a^n_k|\leq\lim_{k\ra\infty}|a_k-a^n_k|+|a^n_k|\]

    Notice that for big enough $n$ we have that $|a_k-a_k^n|<\epsilon_1$ for all $k$. And likewise because $(a^n_1,a^n_2,\dots)\in c_0$ we know that
    for large enough $k$ that $|a^n_k|<\epsilon_2$ so $|a_k-a^n_k|+|a^n_k|<\epsilon_1+\epsilon_2$. And so we have that $\lim_{k\ra\infty}|a_k|=0$.

    So $(a_1,a_2,\dots)\in c_0$, meaning $c_0$ is a closed 
    subspace of $\ell^\infty$ and so is a Banach space.



    \item 
    We can model off the proof of 7.26. That is we will define the map $\Phi:\ell^1\ra c_0'$ that takes an 
    element $b=(b_1,b_2,\dots)\in \ell^1$ to the linear functional $\phi_b:c_0\ra \F$ by 
    $\phi_b(a)=\sum_{k=1}^\infty a_k b_k$. First we will show that $\phi_b$ is well defined linear functional. 
    This follows from a comparison on $\phi_b(a)=\sum_{k=1}^\infty a_k b_k$ with $\sum_{k=1}^\infty b_k$ removing the first so many terms where $a_k\geq 1$.

    Now we want to show that $\Phi$ is a norm preserving isomorphims.

    Let $e_k\in c_0$ be the sequence of all $0$s except the $k$-th term is $1$. 
    If $\phi_b\equiv 0$ and we have that $\phi_b(e_k)=b_k=0$ meaning $b=(0,0,\dots)$ and so the kernel of $\Phi$ is trivial, meaning $\Phi$ is injective.

    Now consider any $\phi\in c_0'$ we can define the sequence $b=(b_1,b_2,\dots)$ such that $b_k=\phi(e_k)$. 
    Notice that for any $a=(a_1,a_2,\dots)\in c_0$ we have
    that $a=\sum a_ke_k$ because which converges in the $\ell^\infty$ norm because $\lim_{k\ra\infty}{a_ke_k}=0$ % idk. but its just true i guess
    This means that 
    \[\phi(a)=\phi(\sum a_ke_k)=\sum \phi(a_ke_k)=\sum a_k\phi(e_k)=\sum a_kb_k\] 
    which follows from $\phi$ being continuous.
    Finally we need to show that $b\in \ell^1$. Consider the counting measure $\mu_n$ on $\{1,2,\dots,n\}$ for a fixed $n\in\Z^+$
    and the space $L^\infty(\mu_n)$ as a subspace of $c_0$ because any sequence $(a_1,\dots,a_n)\in L^\infty(\mu_n)$
    can be identified with the sequence $(a_1,\dots,a_n,0,0,\dots)\in c_0$. Now consider the linear functional $\phi|_{L^\infty(\mu_n)}$ which is such that
    \[\phi|_{L^\infty(\mu_n)}((a_1,\dots,a_n))=\phi((a_1,\dots,a_n,0,0,\dots))=\sum_{k=1}^n a_kb_k\]
    By 7.25 we have a norm preserving map from $L^1(\mu_n)$ into $(L^\infty(\mu_n))'$ by $(c_1,\dots,c_n)\mapsto\sum_{k=1}^n a_kc_k$ that means 
    $\norm{(b_1,\dots,b_n)}_1=\norm{\phi|_{L^\infty(\mu_n)}}\leq \norm{\phi}$ and so 
    $\lim_{n\ra \infty}\norm{(b_1,\dots,b_n)}_1=\norm{(b_1,b_2,\dots)}_1\leq \norm{\phi}<\infty$ and so 
    $(b_k)_k\in \ell^1$. Ans so by constructions $\phi_b=\phi$. Showing the map is surjective.

   \end{enumerate}

   \item[(4)] % help from: https://djalil.chafai.net/blog/2022/10/10/little-ell-p/
   First recall that $(\ell^1)'\cong \ell^\infty$ by 
   $(\phi:\ell^1\ra \F)\mapsto(\phi(e_k))_k$ where $e_k$ is all zero except for the $k$th entry which is $1$.
   Likewise the other direction $(b_k)_k\mapsto \phi_b(a)=\sum_ka_kb_k$. This means that $\Phi:\ell^1\ra (\ell^1)''$ 
   can be reinterpreted as $\Phi:\ell^1\ra (\ell^\infty)'$ where for $f\in\ell^1$ and $(b_k)_k\in\ell^\infty$ we have $\Phi f ((b_k)_k)=\sum b_k f_k$ as these are exactly the linear functionals on $\ell^1$.
   Consider the subspace $S=\{(a_k)_k\in\ell^\infty| \lim_{k\ra\infty}a_k\in\F\}$. 
   And consider the linear functional
   $T:S\ra\F$ where $T((a_k)_k)=\lim_{k\ra \infty}a_k$. Notice that $\norm{T}=1$. This means
   by the Hanh-Banach theorem this extends to a linear functional $T:\ell^\infty\ra \F$.
   Now assume that there exists an element $f\in\ell^1$ such that $T((b_k))=\sum_{k=1}^\infty b_kf_k$
   or in other words $T$ comes from an element of $\ell^1$. Notice that for $r>0$ we may 
   construct an element $(b_k)_k$ where $b_k=1$ for $k\leq r$ and $0$ everywhere else.
   This shows that $\sum_{k=1}^r f_k = 0$ and so $f_k=0$ for all $k$. 
   However this is a contradiction as this would mean $\norm{T}=\norm{0}=0$ but it 
   was assumed to be $1$.


   \item[(5)] 
      \begin{enumerate}[label=(\alph*)]
         \item Notice that $|f-g|^2=(f-g)(\overline{f}-\overline{g})=|f|^2+|g|^2-2|f\overline{g}|$ and so
              $2|f\overline{g}|+|f-g|^2=|f|^2+|g|^2$ and because $|f-g|^2\geq 0$ we have that
              $2|f\overline{g}|\leq|f|^2+|g|^2$
         
         
         Notice that with this result we have that 
         \[\int|f\overline{g}|d\mu\leq\frac{1}{2}\int(|f|^2+|g|^2)d\mu \leq \frac{1}{2}\int|f|^2d\mu +\frac{1}{2}\int|g|^2d\mu <\infty\]
         because $f,g\in L^2(\mu)$ so the last two integrals exists. Futhermore $\ip{f,g}=\int f\overline{g} d\mu$ is an inner product. 
         It is clear that it is linear in the first term and semi-linear in the second. And is positive definite which follows from $\ip{f,f}=\norm{f}_2$ being a norm.

         \item Notice that 
         \[\norm{fg}_1=\int|fg|d\mu =\int |f||g|d\mu = \ip{|f|,|g|}\]
         and with Cauchy-Schwarz we have that
         \[\ip{|f|,|g|}\leq \norm{|f|}_2\norm{|g|}_2=\norm{f}_2\norm{g}_2\]
         and so 
         \[\norm{fg}_1\leq\norm{f}_2\norm{g}_2\]
      \end{enumerate}

   \item[(6)] 
   \begin{enumerate}[label=(\alph*)]
      \item Consider the Banach Space $C([0,1])$ with the norm $\norm{\cdot}=\sup_{[0,1]}|f|$ norm. 
      Consider the normed subspace $B=\{f\in C([0,1])|f(0)=0\}$ with the same norm. 
      We will first show that this is a closed subspace which means it itself is a Banach Space.
      So consider a sequence $f_1,f_2,\dots$ of functions in $B$ that converge to some function $f\in C([0,1])$
      This means that for all $\epsilon>0$ there exists some $N>0$ where for all $n>N$ we have that 
      $\norm{f_n-f}<\epsilon$. This means that $|f(0)-f_n(0)|<\epsilon$ and so $f(0)=0$ meaning $f\in B$. 
      Therefore $B$ is a Banach space.
   
      Now consider the linear functional $\phi(f)=\int_0^1f d\lambda$ where 
      for any $f\in B$ such that $\norm{f}=1$ we have that $|f(x)|\leq 1$ for 
      all $x$ meaning $\left|\int_0^1fd\lambda\right|\leq \int_0^1|f|d\lambda\leq 1$. 
      
      Furthermore fix $r>0$ and consider the function $g_r(x)=\frac{rx}{rx+1}\in B$ 
      and notice that $\lim_{r\ra\infty}\int_0^1g_r(x)d\lambda=\frac{-ln(r+1)+r}{r}=1$.
      Meaning $\norm{\phi}=1$.
   
      Consider any $f\in B$ such that $\norm{f}=1$ and because $f$ is continuous 
      we know that there exists some $\epsilon>0$ where
      $|f(x)|\leq\frac{1}{2}$ on $[0,\epsilon]$. This means 
      that 
      \[\left|\int_0^1fd\lambda\right|\leq \int_0^1|f|d\lambda\leq \frac{\epsilon}{2}+\int_\epsilon^1|f|d\lambda=\frac{\epsilon}{2}+(1-\epsilon)< 1\]
      And so $|\phi(f)|<\norm{\phi}\norm{f}$.
      \item If $V$ is a Hilbert space with $\phi:V\ra\R$ a linear functional 
      then by the Riesz representation theorem 
      there exists a unique $h\in V$ such that $\phi(f)=\ip{f,h}$ with $\norm{\phi}=\norm{h}$.
      And Cauchy-Schwarz tells us that $|\phi(f)|=|\ip{f,h}|\leq\norm{f}\norm{h}$ 
      with equality when $f,h$ are scalar multiples.

   \end{enumerate}

\end{itemize}

\end{document}