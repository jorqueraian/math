\documentclass[12pt]{amsart}
\usepackage{preamble}
%\DeclareRobustCommand{\flippedSlash}{\text{\reflectbox{$\setminus$}}}
\DeclareRobustCommand{\rchi}{{\mathpalette\irchi\relax}}
\newcommand{\irchi}[2]{\raisebox{\depth}{$#1\chi$}} % inner command, used by \rchi

\begin{document}
\begin{center}
   \textsc{Math 617. HW 4\\ Ian Jorquera}
\end{center}
\vspace{1em}

\begin{itemize}
   \item[(1)] Consider the measure space $(X=\{1,\dots,n\},\mathcal S=2^X)$ with the counting measure 
   as $\mu$. Notice that $\mu(X)=n$. Then consider $\mathcal L^1 (X,\mathcal S, \mu)=\{(a_1,\dots,a_n)|a_k\in \F\}$
   equip with the norm $\norm{(a_1,\dots,a_n)}_1=\sum_{k=1}^{n}|a_k|$.
   
   Likewise consider $\mathcal L^5 (X,\mathcal S, \mu)=\{(a_1,\dots,a_n)|a_k\in \F\}$ equip with the 
   norm $\norm{(a_1,\dots,a_n)}_1=\left(\sum_{k=1}^{n}|a_k|^5\right)^{1/5}$.

   From 7.10 we know that $\norm{(a_1,\dots,a_n)}_1\leq \mu(X)^{(5-1)/5}\norm{(a_1,\dots,a_n)}_5$ and so for 
   $(a_1,\dots,a_n)$ a sequence of positive numbers we have that
   \[\left(\sum_{k=1}^{n}a_k\right)^5=\norm{(a_1,\dots,a_n)}_1^5\leq \mu(X)^{4}\norm{(a_1,\dots,a_n)}_5^5=n^4\sum_{k=1}^{n}a_k^5\]


   \item[(2)] Consider the function $f(x)=\log(x)$ when $0<x\leq 1$ and $f(0)=0$. Notice that for ayn $r>0$ we have that for any $x\in(0,e^{-r})$ that $|\log(x)|>r$. 
   And because $\lambda(0,e^{-r})>0$ we know that $\norm{f}_\infty>r$ for all $r$ and so $f\not\int L^\infty([0,1])$.
   However we can determine with a quiz computation on mathematica or integral-calculator that for any $p>0$ that
   \[\int_{0}^1f d\mu=\Gamma(1+p)<\infty\]
   and so $f\in \bigcap_{0<p<\infty}L^p([0,1])$.
   \item[(3)]
   \begin{enumerate}[label=(\alph*)]
    \item Let $(a^n_1,a^n_2,\dots)_{n=1}^{\infty}$ be a convergent sequence in $c_0$, that converges 
    to a sequence $(a_1,a_2,\dots)$. Meaning for all $\epsilon>0$ there exists some $N>0$ such that for all $n>N$ we have that
    $\norm{(a_k)_k-(a^n_k)_k}_\infty<\epsilon$. Because $\norm{\cdot}_\infty$ is the supremum this means that
    for all $k$ that $|a_k-a_k^n|<\epsilon$.
    Now to show that $(a_1,a_2,\dots)\in c_0$ consider the limit
    \[\lim_{k\ra\infty}|a_k|=\lim_{k\ra\infty}|a_k-a^n_k+a^n_k|\leq\lim_{k\ra\infty}|a_k-a^n_k|+|a^n_k|\]

    Notice that for big enough $n$ we have that $|a_k-a_k^n|<\epsilon_1$ for all $k$. And likewise because $(a^n_1,a^n_2,\dots)\in c_0$ we know that
    for large enough $k$ that $|a^n_k|<\epsilon_2$ so $|a_k-a^n_k|+|a^n_k|<\epsilon_1+\epsilon_2$. And so we have that $\lim_{k\ra\infty}|a_k|=0$.

    So $(a_1,a_2,\dots)\in c_0$, meaning $c_0$ is a closed 
    subspace of $\ell^\infty$ and so is a Banach space.



    \item Consider the map $\Phi:\ell^1\ra c_0'$ that takes an element $b=(b_1,b_2,\dots)\in \ell^1$ to the linear functional $\phi_b:c_0\ra \F$ by 
    $\phi_b(a)=\sum_{k=1}^\infty a_k b_k$. First we will show that $\phi_b$ is well defined linear functional. 
    This follows from simple comparison test on $\phi_b(a)=\sum_{k=1}^\infty a_k b_k$ removing the first so many terms where $a_k\geq 1$. IDK write up better late

    Now we want to show that $\Phi$ is an isomorphims. Need to show its linear and bijection

    Let $e_k\in c_0$ be the sequence of all $0$s except the $k$-th term is $1$. 
    If $\phi_b\equiv 0$ and we have that $\phi_b(e_k)=b_k=0$ meaning $b=(0,0,\dots)$ and so the kernel of $\Phi$ is trivial, meaning $\Phi$ is injective.

    Now consider any $\phi\in c_0'$ we can define the sequence $b=(b_1,b_2,\dots)$ such that $b_k=\phi(e_k)$. 
    Notice that for any $a=(a_1,a_2,\dots)\in c_0$ we have
    that

   \end{enumerate}

   \item[4] First recall that $(\ell^1)'\cong \ell^\infty$ meaning the $\Phi:\ell^1\ra (\ell^1)''$ 
   can be reinterpreted as $\Phi:\ell^1\ra (\ell^\infty)'$ where $\Phi f = \phi_f$ where $\phi_f((b_k))=\sum b_k f_k$ as these are exactly the linear functionals on $\ell^1$.
   Consider the subspace $S=\{(a_k)\in\ell^\infty| \lim_{k\ra\infty}a_k\in\F\}$ 
   Notice that this is what is needed for Hahn-Banach. So consider the linear functional
   $T:S\ra\F$ where $T((a_k))=\lim_{k\ra \infty}$ Notice that $\norm{T}=1$. This means
   by the Hanh-Banach theorem this extends to a linear function on $T:\ell^\infty\ra \F$.
   Now assume that there exists an element $f\in\ell^1$ where $T((b_k))=\sum_{k=1}^\infty b_kf_k$
   Notice that for $r>0$ construct an element $(b_k)$ where $b_k=1$ for $k\leq r$ and $0$ everywhere else.
   This shows that $\sum_{k=1}^r f_k = 0$ and so $f_k=0$ for all $k$. However this is a contradiction as this would mean $\norm{T}=0$ but it was assumed to be $1$.
   \item[(5)] 
      \begin{enumerate}
         \item Notice that $|f-g|^2=(f-g)(\overline{f}-\overline{g})=|f|^2+|g|^2-2|f\overline{g}|$ and So
              $2|f\overline{g}|+|f-g|^2=|f|^2+|g|^2$ and because $|f-g|^2\geq 0$ that
              $2|f\overline{g}|\leq|f|^2+|g|^2$
         
         
         Notice that with this result we have that $\int|f\overline{g}|d\mu\leq\frac{1}{2}\int(|f|^2+|g|^2)d\mu \leq \frac{1}{2}\int|f|^2d\mu +\frac{1}{2}\int|g|^2d\mu <\infty$ 
         because $f,g\in L^2(\mu)$ so the integral exists. Futhermore $\ip{f,g}=\int f\overline{g} d\mu$ is an inner product. It is a triviality to show all but positive-definiteness.
         This follows from $\ip{f,f}=\norm{f}_2$ being a norm.

         \item Notice that 
         \[\norm{fg}_1=\int|fg|d\mu =\int |f||g|d\mu = \ip{|f|,|g|}\]
         and with Cauchy-Schwars we have that
         \[\ip{|f|,|g|}\leq \norm{|f|}_2\norm{|g|}_2=\norm{f}_2\norm{g}_2\]
      \end{enumerate}
\end{itemize}

\end{document}