\documentclass[12pt]{amsart}
\usepackage{preamble}
%\DeclareRobustCommand{\flippedSlash}{\text{\reflectbox{$\setminus$}}}
\DeclareRobustCommand{\rchi}{{\mathpalette\irchi\relax}}
\newcommand{\irchi}[2]{\raisebox{\depth}{$#1\chi$}} % inner command, used by \rchi

\begin{document}
\begin{center}
   \textsc{Math 617. HW 4\\ Ian Jorquera}
\end{center}
\vspace{1em}

\begin{itemize}
   \item[(1)] % ahhh
   I more or less followed the proof of 5.41. First notice that $\mathcal B_n$ is closed under translation. % should i prove? explicitly show that t+A\times B = (t_1)
   First we will show that $\lambda_n(t+E)=\lambda(E)$ is true on measurable rectangles. We will do this 
   inductively on $n$. For $n=1$ we already know that $\lambda_1$ is translational invariant.
   Then recall that $\lambda_n=\lambda_1\times\lambda_n$ and so 
   $\lambda_n(a+(A\times B))=\lambda_1(a_1+A)\lambda_{n-1}((a_k)_{k=2}^n + B)=\lambda_1(A)\lambda_{n-1}(B)$
   for $A\times B\in \mathcal B_1\times \mathcal B_{n-1}$. 
   Now consider the set $\mathcal E_m = \{E\in\mathcal B_n : E\se C_m\text{ and }\lambda_n(t+E)=\lambda(E)\}$.  
   Where $C_m$ is the open cube of radius $m$
   Next we need to show that 
   $\mathcal E_m$ is a $\sigma$-algebra on $C_m$, which we will do by the Monotone class theorem.

   Consider an increasing sequence $E_k\in\mathcal E_m$ where $E_k\se E_{k+1}$ and notice that
   by 2.59 \[\lambda_n\left(a+\bigcup_{k=1}^\infty E_k\right)=\lambda_n\left(\bigcup_{k=1}^\infty a+E_k\right)=\lim_{k\ra\infty}\lambda_n(a+E_k)=\lim_{k\ra\infty}\lambda_n(E_k)=\lambda_n\left(\bigcup_{k=1}^\infty E_k\right)\]

   Likewise consider an decreasing sequence $E_k\in\mathcal E_m$ where $E_{k+1}\se E_{k}$ and notice that
   by 2.60 and the fact that $\lambda_n(C_m)=4m^2$ we have
   \[\lambda_n\left(a+\bigcap_{k=1}^\infty E_k\right)=\lambda_n\left(\bigcap_{k=1}^\infty a+E_k\right)=\lim_{k\ra\infty}\lambda_n(a+E_k)=\lim_{k\ra\infty}\lambda_n(E_k)=\lambda_n\left(\bigcap_{k=1}^\infty E_k\right)\]

   This shows that $\mathcal E_m$ is the smallest monotone class containing the measurable rectangles in $C_m$. 
   Next we need to show that it is an algebra which follows from 5.13a.
   And by the monotone class theorem this means that $\mathcal E_m$ is the smallest 
   $\sigma$-algebra containing the measurable rectangles in $C_m$. Finally consider 
   any $E\in \mathcal B_n$ and notice that by 2.59 and because $(E\cap C_m)_{m}$ 
   is an increasing sequence where $\cup_{m=1}^\infty E\cap C_m=E$ we have that

   \begin{align*}
      \lambda_n\left(a+E\right)&=\lambda_n\left(a+\bigcup_{m=1}^\infty (E\cap C_m)\right)
   =\lambda_n\left(\bigcup_{m=1}^\infty a+(E\cap C_m)\right)\\
   &=\lim_{m\ra\infty}\lambda_n(a+(E\cap C_m))=\lim_{m\ra\infty}\lambda_n((E\cap C_m))
   =\lambda_n\left(\bigcup_{m=1}^\infty (E\cap C_m)\right)\\
   &=\lambda_n\left(E\right)
\end{align*}


   \iffalse We can prove this by induction. First recall that the Lebesgue 
   measure $\lambda$ for $\R$ is translation invariant.
   Now fix $n>1$ and recall we know that because $\R^n$ with the Lebesgue 
   measure $\lambda_n$ is $\sigma$-finite we know that $\lambda_n$ is the product measure and
   is uniquely defined to be 
   \[\lambda_n(E)=\int_{\R^{n-1}}\int_{\R^{1}}\rchi_{E}(x,y)d\lambda_1(x)d\lambda_{n-1}(y)\]
   Notice that 
   \[\lambda_n(t+E)=\int_{\R^{n-1}}\int_{\R^{1}}\rchi_{t+E}(x,y)d\lambda_1(x)d\lambda_{n-1}(y)
   = \int_{\R^{n-1}}\lambda_1([t+E]^y) d\lambda_{n-1}(y)\]
   And because $\lambda_1$ is translational invariant we know that for any fixed $y\in\R^{n-1}$ that 
   $[t+E]^y\in \R^n$ is Lebesgue measurable, and so $\lambda_1([t+E]^y)$ is measurable function idk\fi



   \item[(2)] % fucked but done
   Recall that $\ell^1=\{(a_1,a_2,\dots) : \norm{(a_1,a_2,\dots)}_1<\infty \}$. We can prove this 
   statement using $6.41$ from the book. Let $g^1,g^2,\dots $ be a sequence in $\ell^1$ such that 
   $\sum_{k=1}^\infty\norm{g^k}_1<\infty$ where each $g^k=(g^k_n)_{n=1}^\infty$. 
   Let $g=\lim_{m\ra\infty}\sum_{k=1}^m g^k$.
   Notice also that
   \[\lim_{m\ra\infty}\sum_{k=1}^m\norm{g^k}_1=\lim_{m\ra\infty}\sum_{k=1}^m\sum_{n=1}^\infty |g^k_n|
   =\lim_{m\ra\infty}\sum_{n=1}^\infty \sum_{k=1}^m|g^k_n|
   =\sum_{n=1}^\infty \lim_{m\ra\infty}\sum_{k=1}^m|g^k_n|<\infty\]
   where the last equality follows from the monotone convergence theorem. This show us that for any fixed $n$ 
   that $\lim_{m\ra\infty}\sum_{k=1}^m|g^k_n|<\infty$ meaning the limit of elements in 
   $\F$ converges: $\lim_{m\ra\infty}\sum_{k=1}^m g^k_n=\sum_{k=1}^\infty g^k_n$ because $\F$ is a Banach space.
   This means that $g=\lim_{m\ra\infty}\sum_{k=1}^m g^k$ converges component wise. Furthermore we know that $g\in\ell^1$ by construction, more explicitly we can see 
   \[\norm{g}_1=\sum_{n=1}^\infty|\sum_{k=1}^{\infty}g^k_n|\leq\sum_{n=1}^\infty\sum_{k=1}^{\infty}|g^k_n|=\sum_{k=1}^\infty\norm{g^k}_1<\infty\]



   Now notice that for $m>1$ we have that 
   \[\norm{\sum_{k=1}^\infty g^k-\sum_{k=1}^m g^k}=\norm{\sum_{k=m}^\infty g^k}\leq \sum_{k=m}^\infty\norm{g^k}\]
   which can be made arbitrary small for large enough $m$, showing that the limit of partial sums converges to the component wise limit in the norm.
    


   \item[(3)] We can prove this statement using $6.41$ from the book.
   Let $V$ be a finite dimensional vector space with basis $\{e_1,\dots,e_n\}$
   and norm $\norm{\cdot}$. Let $T_\ell:V\ra \F$ be a linear functional that is the projection onto the $\ell$th basis element. 
   Consider a sequence $v_1,v_2,\dots$ in $V$ such that $\sum_{k=1}^\infty \norm{v_k} < \infty$
   Notice that this must mean that 
   \[\lim_{m\ra \infty}\sum_{k=1}^m \norm{T_\ell v_k}\leq\lim_{m\ra \infty}\sum_{k=1}^m \norm{T_\ell}\norm{v_k}
   =\norm{T_\ell}\lim_{m\ra \infty}\sum_{k=1}^m \norm{v_k}<\infty\]
   And because $\F$ is complete we know that $\lim_{m\ra \infty}\sum_{k=1}^m T_\ell v_k =\sum_{k=1}^\infty T_\ell v_k$ converges in $F$
   So let $f=(f_1,\dots,f_n)\in V$ where each $f_\ell=\sum_{k=1}^\infty T_\ell v_k$.

   Now we want to show that $\lim_{m\ra \infty}\norm{\sum_{k=1}^m v_k-f}=0$. By the triangle inequality we 
   have that $\norm{\sum_{k=1}^m v_k-f}\leq \sum_{\ell=1}^n \norm{T_\ell(\sum_{k=1}^m v_k-f)}$ and so it
   suffices to show $\lim_{m\ra \infty}\norm{\sum_{k=1}^m T_\ell v_k-T_\ell f}=0$ for each component or each basis element.
   Notice that 
   \[\lim_{m\ra \infty}\norm{\sum_{k=1}^m T_\ell v_k-T_\ell f}=\lim_{m\ra \infty}\norm{\sum_{k=1}^m T_\ell v_k- \sum_{k=1}^\infty T_\ell v_k}=\norm{T_\ell}\lim_{m\ra \infty}\norm{\sum_{k=m}^\infty v_k}\leq\norm{T_\ell}\lim_{m\ra \infty}\sum_{k=m}^\infty\norm{ v_k}\]
   which can be made arbitrary small for large enough $m$, showing that the limit of partial sums converges to the component wise limit in the norm.



   \item[(4)] % Done
   Let $\Phi:V\ra V''$ be a linear maps such that $(\Phi f)\phi= \phi(f)$. Notice that by definition we know 
   that 
   \[\norm{\Phi f}=\sup\left\{|(\Phi f)\phi| :\phi\in V' \text{ and }\norm{\phi}=1\right\}
   =\sup\left\{|\phi(f)| :\phi\in V' \text{ and }\norm{\phi}=1\right\}\]
   Recall also from 6.72 that for $f\in V$ we have that $\norm{f}=\sup\{|\phi(f)|: \phi\in V'\text{ and } \norm{\phi}=1\}$ and so 
   $\norm{\Phi f}=\norm{f}$



   \item[(5)] % Done
   We will prove the forward direction first. 
   Let $T:V\ra W$ is a bounded linear map between the Banach spaces $V$ and $W$. 
   Now assume that $T$ is bounded below meaning there exists some $0<c<\infty$ 
   such that $\norm{f}\leq c\norm{Tf}$ for all $f\in V$. Notice that for any 
   $f\neq 0$ we have that $0<\norm{f}\leq c\norm{Tf}$ meaning $Tf\neq 0$ by the 
   positive-definiteness of norms and so $\ker T=0$. To see that $\image T$ is closed 
   in $W$ let $w_1=Tv_1,w=2Tv_2,\dots$ be a sequence in $\image T$ where each $v_k\in V$ 
   such that the sequences converges to $w\in W$. Consider the sequences $v_1, v_2,\dots$ 
   which we will show is a Cauchy sequence because 
   $\norm{v_k-v_\ell}\leq\norm{T(v_k-v_\ell)}\leq \norm{Tv_k-Tv_\ell}=\norm{w_k-w_\ell}$ 
   and because the sequence $(w_k)_{k=1}^\infty$ is a Cauchy sequence. And because $V$ 
   is a Banach space we know that the sequence $(v_k)_{k=1}^\infty$ converges to some $v\in V$
   Notice that this means that $\lim_{k\ra\infty} \norm{Tv_k-Tv}=\lim_{k\ra\infty} \norm{T(v_k-v)}=\lim_{k\ra\infty} \norm{T}\norm{v_k-v}=0$
   because $T$ is bounded.



   Now for the other direction. Assume $T:V\ra W$ is a bounded linear map between the Banach spaces $V$ and $W$ such that
   $\ker T=0$ and $\image T$ is closed in $W$.
   Notice that this means that $\image T$ is its self a Banach space meaning that map 
   $T:V\ra\image W$ is invertible and so by the bounded inverse theorem we know that 
   $T^{-1}:\image V \ra V$ is a bounded linear map. This means that $c:=\norm{T^{-1}}<\infty$ 
   and notice that for any $f\in V$ we have that $f=T^{-1}Tf$ and $\norm{f}=\norm{T^{-1}Tf}\leq \norm{T^{-1}}\norm{Tf}$.
\end{itemize}

\end{document}