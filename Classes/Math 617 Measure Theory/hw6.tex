\documentclass[12pt]{amsart}
\usepackage{preamble}
%\DeclareRobustCommand{\flippedSlash}{\text{\reflectbox{$\setminus$}}}
\DeclareRobustCommand{\rchi}{{\mathpalette\irchi\relax}}
\newcommand{\irchi}[2]{\raisebox{\depth}{$#1\chi$}} % inner command, used by \rchi

\begin{document}
\begin{center}
   \textsc{Math 617. HW 6\\ Ian Jorquera}
\end{center}
\vspace{1em}

\begin{itemize}
    \item[(1)] We can use the Riesz representation theorem(revisited). 
    Consider the finite dimensional inner product space $V$ spanned by 
    \[\{e_1=P_0(2x-1),e_2=\sqrt{3}P_1(2x-1),e_3=\sqrt{5}P_2(2x-1),e_4=\sqrt{7}P_3(2x-1),e_5=\sqrt{9}P_4(2x-1)\}\] where $P_n$ is the $n$th 
    Legendre polynomial according to Wikipedia with $L^2([0,1])$ inner product $\ip{f,g}=\int_0^1fg$.
    Now consider the linear functional $T:V\ra\R$ such that $T(f)=f(\frac{1}{2})$.
    With the Riesz representation theorem we have that 
    \[g=\sum_{k=0}^4T(e_k)e_k\]
    \[g=1+\frac{-5}{2}P_2(2x-1)+\frac{27}{8}P_4(2x-1)\]
    \[g=\frac{15120x^4-30240x^3+18480x^2-3360x+120}{64}\]
    such that $T(f)=\ip{f,g}$, meaning $f(\frac{1}{2})=\int_0^1fg$ for all degree $4$ polynomials $f$
    \item[(2)] 
    \begin{enumerate}[label=(\alph*)]
        \item Consider the delta measure for a fixed $x\in \R$ where 
        for $E\in\mathcal B$ the measure $\delta_x$ is 
        defined to be $\delta_x(E)=1$ when $x\in E$ and 
        $\delta_x(E)=0$ when $x\not\in E$. Notice that when $x\neq y$ we have that
         $\norm{\delta_x-\delta_y}=2$. This means that the set $\{B(\delta_x,1)|x\in R\}$ is a 
         set of uncountable many disjoint open balls. So $\mathcal M_\R(\mathcal B)$ is not seperable.
        \item Let $X=\Z^+$ and $\mathcal{S}=2^X$. Notice that for any measure $\mu$
        on $\S$ we need only define $\mu(\{k\})$ for all $k\in \Z^+$ as 
        all other sets in $\mathcal S$ would follow from additivity.
        So let $\mu_k$ me the measure such that $\mu(\{k\})=1$ and 
        $\mu(\{j\})=0$ for $k\neq j\in\Z^+$ in which case the closure of the 
        set $\{r\mu_k | k\in\Z^+, r\in\Q[i]\}$ is $\mathcal M_\C(\mathcal S)$ and
        the closure of the 
        set $\{r\mu_k | k\in\Z^+, r\in\Q\}$ is $\mathcal M_\R(\mathcal S)$ both of which are countable.

    \end{enumerate}
    \item[(3)] Let $(X,\mathcal S)$ be a measurable space with 
    $\mathcal S\neq \{\emptyset, X\}$, meaning there exists some set $A\in \mathcal S$ such 
    that the sets $\{\emptyset, A, X-A, X\}\se \mathcal S$. Consider two positive measures 
    $\mu,\nu$ such that $\mu(A)=3$, $\mu(X-A)=1$,  $\nu(A)=1$ and $\nu(X-A)=4$.
    Notice that this means that $\norm{\mu}=4$ and $\norm{\nu}=5$ and 
    \[\norm{\mu+\nu}=|\mu+\nu|(X)\geq |(\mu+\nu)(A)|+|(\mu+\nu)(X-A)|=9\] 
    and that
    \[\norm{\mu-\nu}=|\mu-\nu|(X)\geq |(\mu-\nu)(A)|+|(\mu-\nu)(X-A)|=5\]
    This means that 
    \[\norm{\mu+\nu}^2+\norm{\mu-\nu}^2\geq 9^2+5^2=106 > 82=2\cdot 4^2+2\cdot 5^2=2\norm{\mu}^2+2\norm{\nu}^2\]
    This shows that the total variation norm fails the parallelogram identity and therefore can not come from a inner product.


    \item[(4)] Let $\mu$ be a positive finite measure on $(X,\mathcal S)$. 
    Let $h$ be a nonnegative function in $\mathcal L^1(\mu)$. Meaning $hd\mu\ll d\mu$.
    We will show that $d\mu\ll hd\mu$ if and only if $h$ is non-zero almost everywhere with respect to $\mu$.
    
    We will show the forward direction by contrapositive. So assume that the set
    $E:=\{x\in X| h(x)=0\}\in\mathcal S$ has non-zero measure, that is $\mu(E)> 0$.
    The notice that $\int_{E}h d\mu=0$ while $\mu(E)\neq 0$ meaning $d\mu\not\ll hd\mu$

    
    First assume that $h$ is non-zero almost everywhere. Meaning for any $E\in\mathcal S$ 
    such that $\mu(E)\neq 0$, let $A=\{x\in E| h(x)=0\}$ and $B=\{x\in E| h(x)\neq 0\}$ 
    and it must be the case that $\mu(B)=\mu(E)$ meaning we
    have that $\int_E h d\mu= \int_A h d\mu+\int_B h d\mu=\int_B h d\mu>0$ and 
    so $h d\mu(E)\neq 0$ when ever $\mu(E)\neq 0$
    This show that $d\mu\ll hd\mu$

    \item[(5)]

\end{itemize}

\end{document}