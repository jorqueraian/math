\documentclass[12pt]{amsart}
\usepackage{preamble}
%\DeclareRobustCommand{\flippedSlash}{\text{\reflectbox{$\setminus$}}}
\DeclareRobustCommand{\rchi}{{\mathpalette\irchi\relax}}
\newcommand{\irchi}[2]{\raisebox{\depth}{$#1\chi$}} % inner command, used by \rchi

\begin{document}
\begin{center}
   \textsc{Math 617. HW 6\\ Ian Jorquera}
\end{center}
\vspace{1em}

\begin{itemize}
    \item[(1)] We can use the Riesz representation theorem(revisited). 
    Consider the finite dimensional inner product space $V$ spanned by 
    \[\{e_1=P_0(2x-1),e_2=\sqrt{3}P_1(2x-1),e_3=\sqrt{5}P_2(2x-1),e_4=\sqrt{7}P_3(2x-1),e_5=\sqrt{9}P_4(2x-1)\}\] where $P_n$ is the $n$th 
    Legendre polynomial according to Wikipedia with $L^2([0,1])$ inner product $\ip{f,g}=\int_0^1fg$.
    Now consider the linear functional $T:V\ra\R$ such that $T(f)=f(\frac{1}{2})$.
    With the Riesz representation theorem we have that 
    \[g=\sum_{k=0}^4T(e_k)e_k\]
    \[g=1+\frac{-5}{2}P_2(2x-1)+\frac{27}{8}P_4(2x-1)\]
    \[g=\frac{15120x^4-30240x^3+18480x^2-3360x+120}{64}\]
    such that $T(f)=\ip{f,g}$, meaning $f(\frac{1}{2})=\int_0^1fg$ for all degree $4$ polynomials $f$
    \item[(2)] 
    \begin{enumerate}[label=(\alph*)]
        \item m
        \item Let $X=\Z^+$ and $\mathcal{S}=2^X$. Notice that for any measure $\mu$
        on $\S$ we need only define $\mu(\{k\})$ for all $k\in \Z^+$ as 
        all other sets in $\mathcal S$ would follow from additivity.
        So let $\mu_k$ me the measure such that $\mu(\{k\})=1$ and 
        $\mu(\{j\})=0$ for $k\neq j\in\Z^+$ in which case the closure of the 
        set $\{\mu_k | k\in\Z^+\}$ is $\mathcal M_\F(\mathcal S)$

    \end{enumerate}
    \item[(3)] consider the sets $\{\emptyset, A, X-A, X\}$ Consider two real measures 
    $\mu,\nu$ such that $\mu(A)=1$ $\mu(X-A)=-2$, $\nu(A)=-3$ and $\nu(X-A)=-4$
    Notice that $\norm{\mu}=3$ and $\norm(\nu)=7$ and $\norm{\mu+\nu}=8$ and $\norm{\mu-\nu}=6$
    Notice that $8^2+6^2\neq 2\cdot 3^2+2\cdot 7^2$. This shows that the total variation norm fails the parallelogram identity.
    This is an example but can be extended to the solution need to just more in depth explain what im doing. I think id need 

    \item[(4)] Let $\mu$ be a positive finite measure on $(X,\mathcal S)$. 
    Let $h\in\mathcal L^1(\mu)$ be a nonnegative function. We know that $hd\mu\ll\mu$

\end{itemize}

\end{document}