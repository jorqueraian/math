\documentclass[12pt]{amsart}
\usepackage{preamble}
%\DeclareRobustCommand{\flippedSlash}{\text{\reflectbox{$\setminus$}}}
\DeclareRobustCommand{\rchi}{{\mathpalette\irchi\relax}}
\newcommand{\irchi}[2]{\raisebox{\depth}{$#1\chi$}} % inner command, used by \rchi

\begin{document}
\begin{center}
   \textsc{Math 617. HW 2\\ Ian Jorquera}
\end{center}
\vspace{1em}

\begin{itemize}
   \item[(1)] %If $\mathcal A$ is uncountable  then there must exists some $0<m\in\R$ such that there exists infinity many 
   Let $n\in\Z^+$ and Let $M_n=\{A\in\mathcal A| \mu(A)>\frac{1}{n}\}$. Notice that for any $n$ we know that $M_n$ 
   must be finite as every set is disjoint and has a positive measure bounded from below and $\mu(X)<\infty$. In fact the $|M_n|\leq n\mu(X)$ 
   We also know that $\bigcup_{n\in\Z^+}M_n=\mathcal A$ and because the countable union of finite sets is countable then $\mathcal A$ is countable.


   \item[(2)] Let $A$ be a Lebesgue Measurable set. That is there is a Borel 
   set $B$ such that $|A-B|=0$. Notice that because $t+(A-B)=(t+A)-(t+B)$ and
   becasue the outer measure is translation invariant we have that 
   $0=|A-B|=|t+(A-B)|=|(t+A)-(t+B)|$. We also know that the Borel sets are 
   translation invariant so the set $t+B$ is Borel. And so $t+A$ is Lebesgue measurable.

   
   \item[(3)] First notice that the cantor function is Riemann integrable as 
   it continuous at all $x\not\in C$ where $C$ is the cantor set. And so the 
   Riemann integral will agree with the Lebesgue integral
   Notice that $\int_{[0,1]}\Lambda d\lambda = \int_{C}\Lambda d\lambda+ \int_{[0,1]-C}\Lambda d\lambda$
   and because $\lambda(C)=0$ we have that $\int_{[0,1]}\Lambda d\lambda = \int_{[0,1]-C}\Lambda d\lambda$
   where $[0,1]-C=\bigcup_{j=1}^{\infty}G_j$ by the definition of $C$ in the book.
   Consider the functions $g_k:[0,1]-C\ra \R$ by $g_k(x)=\Lambda(x)$ 
   if $x\in \bigcup_{j=1}^{k}G_j$ and $g_k(x)=0$ otherwise. Notice that each $g_k$ is a 
   simple function as $\Lambda$ takes on finitely many values on each set $G_j$ as each
   $G_j$ is the finite union of open intervals. Notice also that 
   $g_k(x)\leq g_{k+1}(x)$ and each $g_k$ is a Lebesgue measurable function 
   and $\Lambda=\lim_{k\ra \infty}g_k$ for all $x\in [0,1]-C$.
   So with the monotone convergence theorem we have that
   \[\int_{\bigcup_{j=1}^{\infty}G_j}\Lambda d\lambda=
   \lim_{k\ra \infty}\int_{\bigcup_{j=1}^{\infty}G_j}g_kd\lambda=
   \lim_{k\ra \infty}\int_{\bigcup_{j=1}^{k}G_j}g_k d\lambda= 
   \lim_{k\ra \infty}\sum_{j=1}^{k}|G_j|\frac{1}{2}=\frac{1}{2}\]
   Where the second to last equality follows from the fact that the outer measure 
   $|G_n|=\frac{2^{n-1}}{3^n}$ and the Cantor function on each $G_n$ is symmetric 
   and has an average value of $1/2$.


   \item[(4)] Consider the function $g_k:[0,1]\ra \R$ where $g_k(x)=\frac{x}{x^2+\frac{1}{k}}$
               Notice that $g_k(x)$ is continuous on $[0,1]$ for all $k$ and that
               $\lim_{k\ra \infty}g_k(x)=0$ when $x=0$ and 
               $\lim_{k\ra \infty}g_k(x)=\frac{1}{x}$ when $x\neq 0$. And so $\lim_{k\ra \infty}g_k(x)$ is not bounded.


   
   \item[(5)] This statement is true. We show this by contra-positive. Let $F$ be a set that 
   is not closed. Meaning there exists a converging monotone sequence $x_1,x_2,\dots$ of distinct 
   elements whose limit $c\not\in F$. Let $f:F\ra \R$ be a continuous function such that $f(x_n)=(-1)^n$. 
   This can be done as any two consecutive points can be patched together with a line. However notice 
   that there is no extension $f:\R\ra\R$ as there is no possible choice for $f(c)$ that would make $f$ continuous.


   \item[(6)] This is false. Let $V$ be a Vitali set or any non-Borel measurable set, and 
   let $f_r(x):\R\ra\R$ such that $f_r(r)=1$ if $r\in V$ and $0$ otherwise. Notice that 
   $f_r(x)$ is measurable for all $r\in R$ and 
   $\sup_{r}f_r(x)=\rchi_{V}(x)$ which is not measurable.
\end{itemize}

\end{document}