\documentclass[12pt]{amsart}
\usepackage{preamble}
\DeclareMathOperator{\stab}{\mathrm{stab}}

\begin{document}
\begin{center}
    \textsc{Math 601. HW 1\\ Ian Jorquera}
\end{center}
\vspace{1em}
\begin{itemize}
\item[(1.4)] Let $\C\sqcup\C=\{(1,z) |z\in C\}\cup\{(z,1) |z\in C\}$. Consider the map 
             $\pi_3:\C\sqcup\C\ra \text{Set}(\C\P^1)$ where $(z,1)$ maps to the line spanned by 
             $(z,1)$, $\text{span}{(z,1)}$
             and likewise $(1,z)\mapsto \text{span}{(1,z)}$. 
             
             Notice that this map is surjective as 
             for any line $\ell=\text{span}(a,b)$ 
             we can rescale the vector $(a,b)$ to be either $(a/b, 1)$ or $(1,0)$ which then maps to $\ell$ 
             with $\pi_3$. Finally we can look
             at the quotient space of $\C\sqcup\C$ where we identify inputs that share the same output, 
             which is the equivalence relation 
             $(z,1)\tilde (1,1/z)$.


\item[(2.5)] As shown in Exercise 2.4 we have that $S^2$ is homeomorphic to $\C\P^1$. 
             Likewise by definition of the quotient topology we have a surjective continuous 
             function $\pi_2:S^3\ra\C\P^1\cong S^2$. Let $\ell\in \C\P^1\cong S^2$ and consider an element of the preimage 
             $p\in \pi_2^{-1}(\ell)\se S^4\se \C^2$. 
             We also know from 1.3 that any point $e^{i\theta}p\mapsto \ell$. And so the entire preimage 
             $\pi_2^{-1}(\ell)\cong S^1$ by the map $e^{i \theta}\mapsto e^{i \theta} p$.
\end{itemize}

\end{document}

