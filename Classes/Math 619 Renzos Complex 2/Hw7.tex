% not sure where to put this: https://www.math.colostate.edu/~renzo/teaching/Toric18/Linebundles.pdf
% https://www.math.colostate.edu/~renzo/teaching/Toric18/sheaves.pdf
\documentclass[12pt]{amsart}
\usepackage{preamble}
\DeclareMathOperator{\stab}{\mathrm{stab}}

\begin{document}
\begin{center}
    \textsc{Complex Geometry. HW 7\\ Ian Jorquera}
\end{center}
\vspace{1em}
\begin{itemize}
    \item[(11.3)] Let $\text{Pic}(\C\P^1)$ be the equivalence classes of isomorphic 
    line bundles of $\C\P^1$. We already know that every line bundle is isomorphic to $\mathcal O_{\C\P^1}(d)$
    for some $d\in \Z$, so these can be the representatives of the elements of the Picard Group. 
    We also know that the tensor product is a binary operator on $\text{Pic}(\C\P^1)$.
    Consider the following surjective map $\Phi:\Z\ra \text{Pic}(\C\P^1)$ by 
    $d\mapsto [\mathcal{O}_{\C\P^1}(d)]$ the class of line bundles isomorphic to $\mathcal{O}_{\C\P^1}(d)$
    We show that this is a group isomorphism, showing that the picard group is a group and is isomorphic to $\Z$.

    Notice that $\phi(d_1+d_2)= [\mathcal O_{\C\P^1}(d_1+d_2)]=[\mathcal O_{\C\P^1}(d_1)\otimes \mathcal O_{\C\P^1}(d_2)]=[\mathcal O_{\C\P^1}(d_1)]\otimes [\mathcal O_{\C\P^1}(d_2)]= \phi(d_1)+\phi(d_2)$
    Also notice that $\phi(0)=[\mathcal O_{\C\P^1}(0)]$ which acts as an identity on $\text{Pic}(\C\P^1)$ with tensoring.
    Finally notice that $\phi(-d)=[\mathcal O_{\C\P^1}(-d)]$ which acts as the inverse to $\phi(d)=[\mathcal O_{\C\P^1}(d)]$ under tensoring.
    Finally notice that $\phi$ is injective by previous problems, showing if $\mathcal O_{\C\P^1}(d_1)\cong \mathcal O_{\C\P^1}(d_2)$
    then $d_1=d_2$. \\

    \item[(12.4)]  $H^0(\C\P^1, \mathcal O_{\C\P^1}(d))$ was defined to be the kernel of the map $\delta$. Notice that 
    if $(s_1,s_2)\in \ker\delta$ then $s_1|_{U_0\cap U_1}\equiv s_2|_{U_0\cap U_1}$. And because $s_1$ is holomorphic 
    at $0$, and $s_2$ is holomorphic at $\infty$, it must be the case that $s_1$ has no 
    worse then a removable singularity at $\infty$ and likewise $s_2$ has a removable singularity at $0$. This means with 
    these two points $s_1\equiv s_2$ and so are equal and holomorphic everywhere.
\end{itemize}

\end{document}

