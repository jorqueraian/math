\documentclass[12pt]{amsart}
\usepackage{preamble}
\DeclarePairedDelimiter\bra{\langle}{\rvert}
\DeclarePairedDelimiter\ket{\lvert}{\rangle}



\begin{document}
\begin{center}
    \textsc{Math 999. Thing 0\\ Ian Jorquera}
\end{center}
\vspace{1em}

\begin{enumerate}
    \item \textbf{Context:} Zauner's Conjecture(Originally State in 1999) is 
    an important open problem in quantum information theory and a solution to the problem 
    has a 2024 EUR prize from the National Quantum Information Centre in Poland.
    Zauner's Conjecture is on the existence of SICs in $\C^d$ (Symmetric Informational 
    Complete quantum measurements) which are associated with maximal sets of equiangular 
    lines in $\C^d$ which are equivalent to Equiangular Tight Frames in $\C^d$.\\


    \noindent\textbf{(Weak) Zauner's Conjecture:} For every $d>2$ there exists a maximal system of 
    $d^2$ equiangular lines in $\C^2$. \\


    \noindent These can be thought of as $d^2$ orthogonal projections onto equiangular lines in $\C^2$.
    This also forms a nice basis for self-adjoint $d\times d$ matrices via $\phi\phi^*=\ket{\phi}\bra{\phi}$.\\


    \noindent\textbf{A recent approach for solving Zauner's Conjecture} is looking at 
    ETFs in a finite field analog of the complex numbers: that of $\F_{q^2}$ the finite 
    field or $q^2$ elements for any prime power $q$ equipped with a the field 
    automorphism $(\cdot)^\sigma:\F_{q^2}\ra \F_{q^2}$ where $a^\sigma=a^q$ 
    intended to mimic that of complex conjugation. Little work has been done 
    in studying this analog, but with some theorems ands open problems this analog
    may provide a useful environment for tackling Zauner.\\


    \noindent\textbf{Open Problem:} There are many construction Equiangular Tight Frames in $\C^d$ which
    may be easily extended to the finite field analog of complex numbers.
    Such approaches according to the frames over finite fields paper are 
    Steiner systems [23], hyperovals [22], graph coverings [11, 20, 36], 
    the Tremain construction [19], association schemes [13, 35], or Gelfand pairs [34]. 
    As far as I can tell none of these constructions have been done in the finite field analog.

    \newpage
    \item copied from page 26 of frames over finite field paper: Finally, we remark that Hoggar’s lines are highly symmetric and have a doubly
    transitive automorphism group [60, 36]. We have not investigated the symmetries
    of the ETFs given by Theorem 6.5, and we leave this problem for future study.
    Problem 6.8. For $\Phi$ as in Theorem 6.5, determine the group $\Aut\Phi$ of all permu
   tations $\mu\in S(G \times G)$ having components $\mu(x,y) =: \mu_1(x,y),\mu_2(x,y)$ for which
    there exist scalars $c\mu(x,y) \in \F^\times_{q^2}$
   such that $T\mu_1(x,y)M\mu_2(x,y)\phi = c\mu(x,y)TxMy\phi$ for
    every $x,y \in G$.
\end{enumerate}


\end{document}