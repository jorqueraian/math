\documentclass[12pt]{amsart}
\usepackage{preamble}

\begin{document}
\begin{center}
    \textsc{Math 502. HW 10\\ Ian Jorquera}
\end{center}
\vspace{1em}

\begin{itemize}
\item[(1)] % (1+) [2 points]
This problem involves matroids of rank at most 3, so we will consider all possible matroids of each rank and use the basis construction of matroids.

For rank 3, there is only one matroid up to isomorphism, as there is only one three-element basis $\mathcal{B}={1,2,3}$.

For rank 2, the bases will have size 2. Up to isomorphism, we only need to consider the number of possible basis elements. Because there are 3 two-element sets, there are 3 matroids up to isomorphism.

For rank 1, the bases will have size 1. Up to isomorphism, we only need to consider the number of possible basis elements in the basis set. Because there are 3 one-element sets, there are 3 matroids up to isomorphism.

Finally, for rank 0, the only matroid has basis set ${\emptyset}$. Therefore, there are 8 matroids up to isomorphism.\\

\item[(2)]
Consider the so-called uniform matroid $U_{m,n}=([n],{[n]\choose m})$. Notice first that ${[n]\choose m}$ is non-empty when $m\geq 0$ and $n\geq m$. Now consider any two $k$-subsets $B_1, B_2$ such that $B_1\neq B_2$. This necessarily means that there must exist elements $x\in B_1-B_2$ and $y\in B_2-B_1$ of $[n]$. Notice that for any such element $x\in B_1-B_2$ we can construct a set of size $k$ by adding $y$ that is $B_1-x+y\in {[n]\choose m}$.\\

\item[(3)] % (2) [3 points]
First recall that for a graph $G$ with edges $E$ that $(E,\mathcal{I})$ is a matroid with $\mathcal{I}$ the cycle free subsets of $E$. Here we will show that the set of cycles of the graph $G$ is equal to $\mathcal{C}=\{C\se E | C\not\in \mathcal{I}\text{, and for all }I\subsetneq C, I\in\mathcal{I}\}$. Consider first a cycle in $G$, a set of distinct edges which form a path from a vertex to its self. Notice that by construction of $\mathcal{I}$ we know that no cycle is containing in $\mathcal{I}$. We also know that the removal of any edge say $e_1\ra e_2$ of the cycle will result in either the empty set or a path from $e_2$ to $e_1$. In either case the set of remaining edges would be cycle free. And so all cycles of $G$ are containing in $\mathcal{C}$\\

Now consider a set $C\in\mathcal{C}$ and notice that by $C\not\in\mathcal{I}$ we know there must exist a cycle in $C$. Assume that there exists an edge $e$ not on that cycle. Notice that the removal of that edge $C-e$ still containing a cycle and so $C-e\not\in\mathcal{I}$. This means any $C\in\mathcal{C}$ must be a cycle and contain no additional edges. Showing $\mathcal{C}$ is precisely the set of cycles of $G$. And by problem $(5)$ we know that $(E,\mathcal{C})$ satisfies the axioms C1, C2 and C3.\\

\item[(5)] % (2) [3 points]
Let $M=(E,\mathcal{I})$ be a matroid satisfying the I1, I2,  and I3 axioms. Let $\mathcal{C}=\{C\se E | C\not\in \mathcal{I}\text{, and for all }I\subsetneq C, I\in\mathcal{I}\}$ we will show that $(E,\mathcal{C})$ satisfies the C1, C2, and C3 axioms. First notice that $\emptyset\in \mathcal{C}$ and so by construction we know that $\emptyset\not\in \mathcal{C}$ which satisfies C1.\\

Now let $C_1,C_2\in\mathcal{C}$ and $C_1\se C_2$. Notice that if $C_1\subsetneq C_2$ then by construction we would have that $C_1\in \mathcal{I}$ however this can not be the case as by construction we have that $C_1\in\mathcal{C}$, meaning $C_1\not\in\mathcal{I}$ which show C2. Notice that this axiom shows that if $C_1\neq C_2$ then there must exist some $x\in C_2-C_1$ and likewise there must exist some $y\in C_1-C_2$ as otherwise one would be a subset of the other.\\

Finally let $C_1,C_2\in\mathcal{C}$ such that $C_1\neq C_2$ and assume there exists some $e\in C_1\cap C_2$. We will show that there must exists some circuit containing in $(C_1\cup C_2)-e$. Assume however that there is not then it must be the case that $(C_1\cup C_2)-e\in \mathcal{I}$. we will show this wil induction on $|I|$ where $I\se (C_1\cup C_2)-e$. Notice first that by definition $\emptyset\in\mathcal{I}$. Then fix let $I\se (C_1\cup C_2)-e$. Notice that by induction any proper subset of $I$ is containing in $\mathcal{I}$. This means $I$ is either a circuit or is it self containing in $\mathcal{I}$, and by assume of the there existing no circuits we know that $I\in \mathcal{I}$. This means $(C_1\cup C_2)-e\in \mathcal{I}$.

If $(C_1\cup C_2)-e\in \mathcal{I}$, then consider $C_1-y\in\mathcal{I}$ where $y\in C_1-C_2$. Notice that $((C_1\cup C_2)-e)-(C_1-y)=(C_2-C_1)+y$. Which means by I3 we may add $|C_1-C_2|$ elements of $(C_2-C_1)+y$ to $C_1-y$ such that the resulting set will be in $\mathcal{I}$ and will contain either all of $C_1$, meaning $y$ was added, or all of $C_2$, meaning all of $C_2-C_1$ was added. In either way the result must contain a circuit and so would not be in $\mathcal{I}$. So $(C_1\cup C_2)-e\not\in \mathcal{I}$. And by the previously proved inductive statement we know that this must mean there exists a circuit containing in $(C_1\cup C_2)-e\in \mathcal{I}$ which shows C3.\\


\end{itemize}

\end{document}






