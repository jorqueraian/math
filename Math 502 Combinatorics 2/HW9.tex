\documentclass[12pt]{amsart}
\usepackage{preamble}
\DeclareMathOperator{\stab}{\mathrm{stab}}

\begin{document}
\begin{center}
    \textsc{Math 502. HW 9\\ Ian Jorquera}
\end{center}
\vspace{1em}
% See http://www.mathematicalgemstones.com/maria/Math501Fall22.php
% for problems

% sage: https://sagecell.sagemath.org/
\begin{itemize}
\item[(1)] % (1+) [2 points]
Let $H=(h_{ij})$ and $K=(k_{ij})$ be Hadamard matrices meaning $HH^t=n I_n$ and $KK^t=m I_m$. Notice that 
$$H\otimes K=
\begin{pmatrix}
h_{11}K & h_{12}K&\cdots & h_{1n}K\\
h_{21}K & h_{22}K&\cdots & h_{2n}K\\
\vdots&\vdots&\ddots&\vdots\\
h_{n1}K & h_{n2}K&\cdots & h_{nn}K\\
\end{pmatrix}$$
Notice that each block in this matrix is $\pm K$ meaning the resulting matrix has $\pm 1$ entries. Notice that
    
\begin{align*}
H\otimes K\cdot (H\otimes K)^t &=
\begin{pmatrix}
h_{11}K & h_{12}K&\cdots & h_{1n}K\\
h_{21}K & h_{22}K&\cdots & h_{2n}K\\
\vdots&\vdots&\ddots&\vdots\\
h_{n1}K & h_{n2}K&\cdots & h_{nn}K\\
\end{pmatrix}
\begin{pmatrix}
h_{11}K^t & h_{21}K^t&\cdots & h_{n1}K^t\\
h_{12}K^t & h_{22}K^t&\cdots & h_{n2}K^t\\
\vdots&\vdots&\ddots&\vdots\\
h_{1n}K^t & h_{2n}K^t&\cdots & h_{nn}K^t\\
\end{pmatrix}\\
&=\begin{pmatrix}
\sum_{k=1}^n h_{1k}h_{1k}KK^t & \sum_{k=1}^n h_{1k}h_{2k}KK^t &\cdots & \sum_{k=1}^n h_{1k}h_{nk}KK^t\\
\sum_{k=1}^n h_{2k}h_{1k}KK^t & \sum_{k=1}^n h_{2k}h_{2k}KK^t &\cdots & \sum_{k=1}^n h_{2k}h_{nk}KK^t\\
\vdots&\vdots&\ddots&\vdots\\
\sum_{k=1}^n h_{nk}h_{1k}KK^t & \sum_{k=1}^n h_{nk}h_{2k}KK^t &\cdots & \sum_{k=1}^n h_{nk}h_{nk}KK^t\\
\end{pmatrix}\\
&= \begin{pmatrix}
\ip{\textbf{h}_1,\textbf{h}_1}mI_m & \ip{\textbf{h}_1,\textbf{h}_2}mI_m &\cdots & \ip{\textbf{h}_1,\textbf{h}_n}mI_m\\
\ip{\textbf{h}_2,\textbf{h}_1}mI_m & \ip{\textbf{h}_2,\textbf{h}_2}mI_m &\cdots & \ip{\textbf{h}_2,\textbf{h}_n}mI_m\\
\vdots&\vdots&\ddots&\vdots\\
\ip{\textbf{h}_n,\textbf{h}_1}mI_m & \ip{\textbf{h}_n,\textbf{h}_2}mI_m &\cdots & \ip{\textbf{h}_n,\textbf{h}_n}mI_m\\
\end{pmatrix}\\
&= mn I_{mn}
\end{align*}

where $\textbf{h}_i$ is the $i$th row of $H$.\\

\item[(2)] % (2-) [3 points]
Let $\mathcal{D}_f=([7], \mathcal{B}_f)$ be the Fano plane design. Consider a $2$-$(7,3,1)$-design $\mathcal{D}=(X,\mathcal{B})$ where $X=\{a,b,c,d,e,f,g\}$. We also know that for both designs the number of blocks containing any element is $r=\frac{(v-1)\lambda}{k-1}=3$ and so the number of blocks $b=\frac{vr}{k}=7$. Here we will construct a bijection $\sigma:X\ra X$ that it induced a bijection $\mathbf{B}_f\ra \mathcal{B}$. Consider the map $\sigma(1)=a$ and $\sigma(2)=b$. And notice that there must exist exactly $1$ block $B$ containing both $a$ and $b$ and so WLOG assume that $B=\{a,b,c\}$ meaning $\sigma(3)=c$. We also will have that $a$ is contained in $3$ distinct blocks such that each pair of blocks contains $1$ element in common: the element $a$ (the existence of this follows from Fischer's lemma). WLOG assume these blocks are $\{a,b,c\}$, $\{a,g,d\}$, and $\{a,f,e\}$. We also know that there must exist a unique block $B'$ containing $b$ and $g$ where $a\not\in B'$ as $r=3$. We also know that $c,d\not\in B'$ because this would mean there would be 2 blocks containing $b,c$ or $b,d$ respectively. This means $B'$ must contain either $f$ or $e$. So WLOG assume that $B'=\{b,g,e\}$. And because there exists exactly $3$ distinct blocks containing $b$ such that any pair has only $b$ in common we know that $\{b,d,f\}\in\mathcal{B}$. Likewise because there exists exactly $3$ distinct blocks containing $g$ such that any pair has only $g$ in common we know that $\{e,g,c\}\in\mathcal{B}$. And finally from the same argument on $c$ we know that $\{e,d,c\}\in \mathcal{B}$, which gives us the $7$ blocks of $\mathcal{D}$.\\

Consider the bijection $[7]\ra X$ where $1\mapsto a,2\mapsto b,\dots, 7\mapsto g$. Notice that this induced a bijection on the blocks where $\{1,2,3\}\mapsto \{a,b,c\}$, $\{1,7,4\}\mapsto \{a,g,d\}$, $\{1,6,4\}\mapsto \{a,f,e\}$, $\{2,7,5\}\mapsto \{b,g,e\}$, $\{2,4,6\}\mapsto \{b,d,f\}$, $\{5,7,3\}\mapsto \{e,g,c\}$, and $\{5,4,3\}\mapsto \{e,d,c\}$. which is a bijection of the blocks. So all such designs are isomorphic to the Fano plane.\\

\item[(4)] % (1+) [2 points]
Let $\mathcal{D}=(X,\mathcal{B})$ be a $1$-$(v,k,\lambda)$-design. Then notice that $\mathcal{D}=(X,\mathcal{B}^c)$ is a design. To see this Notice that for $\overline{B}\in \mathcal{B}^c$ there exists some $B\in\mathcal{B}$ such that $\overline{B}=X-B$, meaning $|\overline{B}|=v-k$. 
Notice also that for the design $\mathcal{D}$ we know that $bk=vr=v\lambda$, meaning the number of blocks is $b=v\lambda/k$. We know that for any $x\in X$ there are $\lambda$ blocks of $\mathcal{B}$ that contain $x$, meaning for the design $\mathcal{D}^c$ there are $\lambda$ blocks that don't contain $x$, or $v\lambda/k-\lambda$ blocks containing $x$ in $\mathcal{B}$. And so $\mathcal{D}$ is a $1$-$(v,v-k,v\lambda/k-\lambda)$.\\

\item[(6)] % (2-) [3 points]
Let $G=(V,E)$ be a $(n,k,\lambda,\mu)$-strongly regular graph, and consider the edge compliment $G'=(V,E'={V\choose 2}-E)$. We will show that $G'$ is a $(n,n-k,n-2k-2+\mu,n-2k+\lambda)$-regular graph. First notice that the vertices are unchanged, and for any vertex $v$ in $G$, there are $k$ other vertices with with an edge and $n-(k+1)$ other vertices without edges, meaning the edge compliment will have $n-(k+1)$ vertices connected to $v$. Now for $x,y\in V$ consider first the case that $(x,y)\in E'$ in which case $(x,y)\not\in E$ and there are $\mu$ common neighbors between $x$ and $y$ in $G$ and $n-2k-2+\mu$ with neither a neighbor of $x$ or $y$. To see this notice that there are $2k-\mu$ neighbors of either $x$ or $y$ by inclusion exclusion and so from the vertices not including $x$ and $y$ there are $\lambda'=n-2k-2+\mu$ vertices that share no neighbors with $x$ or $y$ and so in $G$, $x$ and $y$ have $\lambda'$ vertices in common. Now consider the case that $(x,y)\not\in E'$, the edge $xy\not\in E'$ in which case $(x,y)\in E$ and there are $\lambda$ common neighbors between $x$ and $y$ in $G$ and $n-2k+\lambda$ with neither a neighbor of $x$ or $y$. To see this notice that there are $2(k-1)-\lambda$ neighbors of either $x$ or $y$ by inclusion exclusion and so from the vertices not including $x$ and $y$ there are $\mu'=n-2k-2+2+\lambda=n-2k+\lambda$ vertices that share no neighbors with $x$ or $y$ and so in $G$, so $x$ and $y$ have $\mu'$ vertices in common.\\


\end{itemize}

\end{document}






