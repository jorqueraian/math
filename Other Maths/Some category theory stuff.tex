\documentclass[12pt]{amsart}
\usepackage{preamble}

\begin{document}
\begin{center}
    \textsc{Category Theory Notes\\ Ian Jorquera}
\end{center}
\vspace{1em}
\section{What is Category theory?}

\section{First there were morphisms and then there were objects}


\section{Objectless Categories: An even more Abstract definition for an already abstract concept}

\section{Functors as homomorphism of Categories}

\subsection{When is a functor Inner? and what does that even mean}
As far as I can tell this is an idea I invented. And it may not actually be that helpful but what ever. 
For some motivation lets look at the term "Inner"
in the context of groups to get some motivation for why we might want a similar term in 
the world of categories. 

A group automorphism $\phi:G\ra G$ is called an inner automorphism if $\phi(g)=x^{-1}gx$ for 
some fixed element $x\in G$. Inner automorphisms are the automorphisms that are really coming 
from conjugation. A Group $G$ can act on its self by conjugation, and for each element the actions 
by elements give us the inner automorphisms. These are very special automorphisms because they come 
from with\textit{in} the group. In a sense the group already knew about these particular automorphisms.
Furthermore conjugation in groups often looks like relabeling. For example consider the permutation $(1,5)(2,4,7,8)$
conjugated by $(1,2,3,4)$, which would be 
\[(1,2,3,4)^{-1}\cdot(1,5)(2,4,7,8)\cdot(1,2,3,4)=(4,3,2,1)\cdot(1,5)(2,4,7,8)\cdot(1,2,3,4)=(1,7,8,3)(2,5)=(2,5)(3,1,7,8)\]
We can see that the structure hasn't changed drasticly but we have relabeled $1,2,3$ and $4$.

These are really the key idea behind the term "Inner", they refer to things that are coming from within 
the algebraic object you are studying, and dont change the underlying structure is a significant way. 
So how does this apply to functors? We lets look at some diagrams 
and try to deduce what inner could possible mean.

Consider a functor $F:C\ra C$ that maps a morphism $f$ to the morphism $Ff$
\begin{center}
    \begin{tikzcd}
        \cdot \arrow[r, "f"]
        & \cdot \\
        \cdot \arrow[r, "Ff"]
        & \cdot
    \end{tikzcd}
\end{center}

To call this functor inner, it should have more or less the same idea as in the case of groups. Where the 
morphism $f\in C$ is related to morphism $Ff\in C$ by some sort of "conjugation". And this conjugation should 
more or less behave like a relabeling, and should affect the underlying structure in a significant way.
But we have to be a bit more careful here as what ever we choose to do will depend on the sources and 
targets of the morphism $f$, as they determine when we can and cant compose with $f$. So we may want to enforce that the functor
maps morphism according to other morphisms already in the category $C$. Lets add in objects to help contextualize what is going one here.


Out functor $F$ is a mapping a morphism $f:a\ra b$ to the morephims $Ff:Fa\ra Fb$ but now we will enforce that for the object $a$ and $b$, 
the sources and targets of $f$, that there are morphisms $\phi_a:a\ra Fa$ and $\phi_b:b\ra Fb$ that create the following commuting square

 
\begin{center}
    \begin{tikzcd}
        a \arrow[r, "f"]\arrow[d, "\phi_a"]
        & b \arrow[d, "\phi_b"]\\
        Fa \arrow[r, "Ff"]
        & Fb
    \end{tikzcd}
\end{center}

algebraically this is telling us about a commuting $\phi_b \circ f = Ff \circ \phi_a$. In a sense this is telling us 
that the morphism $f$ is equivalent to the morphism $Ff$ up to some morphisms applied to the source and target, and 
we can think about this as being similar to just relabeling the sources and targets. Infact if $\phi_a$ was invertible 
this would give us an explicit conjugation $\phi_b \circ f \circ \phi_a^{-1}= Ff$.

So now we can define what inner will mean in the context of categories. An \textbf{inner endofunctor} is a functor $F: C \ra C$ 
along with a collection of maps $\{\phi_a | a\text{ an object of }C\}$ for each object(each source and target) 
such that for every morphism $f:a\ra b$ we have
\begin{center}
    \begin{tikzcd}
        a \arrow[r, "f"]\arrow[d, "\phi_a"]
        & b \arrow[d, "\phi_b"]\\
        Fa \arrow[r, "Ff"]
        & Fb
    \end{tikzcd}
\end{center}

Again this is really telling us that the morphisms $f$ and $Ff$ agree is some way internal to the category, by other morphism of the category.

May i will give an object free definition.

%If we thought about this in terms of databases, this is basically saying that the functor F is in a sense just 
%relabeling according to the morphisms $\{\phi_a\}$ and the actual information of the morphisms is unchanged.

\subsection{When is a natural transformation? and what makes it natural? A abstract mathematical perspective}
This notion of inner can be used to define an important idea in categories, that of the notion of natural.
What does it mean for a map to be natural? or as we will see soon, what does it mean for a family of maps to be natural?
First it is important to recognize that what it means to be natural should and will depend on the greater context of the maps you're looking at.
If you want a group homomorphism to be called natural you better have a definition that does so in that context, the context of group homomorphisms. This immediately 
points towards categories as a possible way of doing this, and maybe suggests that what ever we end up doing should also be internal (or "Inner") 
to the context or the category. 


It is almost important to note that this notation of natural, that we are about to define, is primarily used in the context of objects. I may 
later comment on how it applies to object free categories but often a different starting point is taken.
That is to say I will unfortunately go against my belief that categories don't need objects and use the objects. But historically that is how natural transformations 
came into existence.

To call a map natural, it should exist broadly. A single morphism between isolated objects is definitely not natural.
This suggests that a map being natural should require that it exists, in some way, for all objects, hence natural maps should really be a collection of maps.
For example we often want to make claims like "Every group is naturally isomorphic to its opposite group" or "the determinant is a natural map".
Lets consider the determinant example to build a bit more intuition. For this to be natural we first need to determine the context in which it exists. 
Often the determinant is thought of as a group homomorphism between the groups of invertible matrices $GL_n(R)$ over some ring $R$ and the ring of units $R^*$.
So our context will be the category of group homomorphisms.
This is a bit tricky as the determinant isnt a map defined for all the objects in the category of group homomorphisms. 
But the object we care about (the groups $GL_n(R)$ and $R^*$) 
are special, they come from the images of Functors, the functors $GL_n(-)$ and $(-)^*$. This suggests that natural transformations need not be a collection of morphisms 
for all objects but should be a collection of morphisms for
all objects coming from some special functors (or equivalently the should be parameterized by the domain of the functors). 
In this example the determinant is going to be a natural transformation, meaning it will be a collection of maps defined on each
matrix group $GL_n(R)$ or equivalently the underlying commutative ring, as each commutative rings gives us the matrix group through the functor
$\det = \{\det_R:GL_n(R)\ra R^* | R\text{ a commutative ring}\}$
This is why many authors will define a natural transformation as a morphism between functors, in this case its its between the functors $GL_n(-)$ and $(-)^*$
as these parameterize the sources and targets of our collection of maps.

So lets get to a definition!
Let $F,G:C\ra D$ be two functors, then a natural transformations $\alpha$ is a collection of maps $\alpha=\{\alpha_a:Fa\ra Gb | a\text{ an object of }C\}$
That defines a functor $A:F(C)\ra G(C)$ on the images of the the functors mapping $Ff$ to $Gf$, that is also inner with respect to $\alpha$, 
meaning we have the following commuting square for every morphism $f\in C$
\begin{center}
    \begin{tikzcd}
        Fa \arrow[r, "Ff"]\arrow[d, "\alpha_a"]
        & Fb \arrow[d, "\alpha_b"]\\
        Ga \arrow[r, "Gf"]
        & Gb
    \end{tikzcd}
\end{center}


All this goes to say that a natural transformation can be though of as a functor on the images of the functors $F$ and $G$, 
that is also "Inner" in the greater category in which the images live.%, ie its just "relabeling" and not affecting the structure of the data.


\subsection*{Natural Transformations as algorithms}
Breath first tree traversal is a natural way the flatten a tree to a list. This is an obvious fact, 
in the sense that any programmer, without a definition for "natural" would agree.
Also this tells you a way of relating two ways of storing the "same" data, as either 
a list or a tree. And a BFS traversal gives a way to translate from one to the other.
But there are multiple ways to do a flattening from a tree to a list, we could also have done a depth first tree tranversal,
which would have also flattened the tree in a "natural" way. So one may ask: are these flattening methods in a sense related, 
can we get from one flattening method to another. This is an important question in computer science because you may have two 
algorithms that do something to the same data structure but you may not know which one you want to use, this higher 
level relation can answer that, or it can tell you that methods are equivalent.

natural is a way of translating data, in such a way that it wont affect the possible functions i may want to work with in the future. 
One key idea here is that "structure" comes from the functions that exists in your 
world and if you program a way of translating data then its should preserve the data in your world, ie the function on that data.

So this tells us we need to answer a few questions, first what are the data structures i care about, and what is the data i am allowing to change.
(one idea maybe ill add later is if the structure you care about is order, then breath first 
search would not preserve that data but depth first would. But to impose this condition you would need a category where the objects 
have an ordering, ie ordered types.)

Ok lets get into it. First we need to ask what is a functor to a programmer? Well from an 
objects point of view they are data structure. You may have a function on your data $A$ that translates it to $B$.
A functor is this a map
\end{document}

