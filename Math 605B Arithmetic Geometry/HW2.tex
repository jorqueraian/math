\documentclass[12pt]{amsart}
% packages
\usepackage{graphicx}
\usepackage{setspace}
\usepackage{amssymb,amsmath,amsthm,amsfonts,amscd}
\usepackage{hyperref}
\usepackage{color}
\usepackage{booktabs}
\usepackage{tabularx}
\usepackage{enumitem}
\usepackage[retainorgcmds]{IEEEtrantools}
\usepackage[notref,notcite,final]{showkeys}
\usepackage[final]{pdfpages}
\usepackage{fancyhdr}
\usepackage{upgreek}
\usepackage{multicol}
\usepackage{fontawesome}
\usepackage{halloweenmath}
% set margin as 0.75in
\usepackage[margin=0.75in]{geometry}

% tikz-related settings
\usepackage{tkz-berge}
\usetikzlibrary{calc,quotes}
\usetikzlibrary{arrows.meta}
\usetikzlibrary{positioning, automata}
\usetikzlibrary{decorations.pathreplacing}

%% For table
\usepackage{tikz}
\usetikzlibrary{tikzmark}

% theorem environments with italic font
\newtheorem{thm}{Theorem}[section]
\newtheorem*{thm*}{Theorem}
\newtheorem{lemma}[thm]{Lemma}
\newtheorem{prop}[thm]{Proposition}
\newtheorem{claim}[thm]{Claim}
\newtheorem{corollary}[thm]{Corollary}
\newtheorem{conjecture}[thm]{Conjecture}
\newtheorem{question}[thm]{Question}
\newtheorem{procedure}[thm]{Procedure}
\newtheorem{assumption}[thm]{Assumption}

% theorem environments with roman font (use lower-case version in body
% of text, e.g., \begin{example} rather than \begin{Example})
\newtheorem{Definition}[thm]{Definition}
\newenvironment{definition}
{\begin{Definition}\rm}{\end{Definition}}
\newtheorem{Example}[thm]{Example}
\newenvironment{example}
{\begin{Example}\rm}{\end{Example}}

\theoremstyle{definition}
\newtheorem{remark}[thm]{\textbf{Remark}}

% special sets
\newcommand{\A}{\mathbb{A}}
\newcommand{\C}{\mathbb{C}}
\newcommand{\F}{\mathbb{F}}
\newcommand{\N}{\mathbb{N}}
\newcommand{\Q}{\mathbb{Q}}
\newcommand{\R}{\mathbb{R}}
\newcommand{\Z}{\mathbb{Z}}
\newcommand{\cals}{\mathcal{S}}
\newcommand{\ZZ}{\mathbb{Z}_{\ge 0}}
\newcommand{\cala}{\mathcal{A}}
\newcommand{\calb}{\mathcal{B}}
\newcommand{\cald}{\mathcal{D}}
\newcommand{\calh}{\mathcal{H}}
\newcommand{\call}{\mathcal{L}}
\newcommand{\calr}{\mathcal{R}}
\newcommand{\la}{\mathbf{a}}
\newcommand{\lgl}{\mathfrak{gl}}
\newcommand{\lsl}{\mathfrak{sl}}
\newcommand{\lieg}{\mathfrak{g}}

% math operators
\DeclareMathOperator{\kernel}{\mathrm{ker}}
\DeclareMathOperator{\image}{\mathrm{im}}
\DeclareMathOperator{\rad}{\mathrm{rad}}
\DeclareMathOperator{\id}{\mathrm{id}}
\DeclareMathOperator{\hum}{[\mathrm{Hum}]}
\DeclareMathOperator{\eh}{[\mathrm{EH}]}
\DeclareMathOperator{\lcm}{\mathrm{lcm}}
\DeclareMathOperator{\Aut}{\mathrm{Aut}}
\DeclareMathOperator{\Inn}{\mathrm{Inn}}
\DeclareMathOperator{\Out}{\mathrm{Out}}
\DeclareMathOperator{\Gal}{\mathrm{Gal}}


% frequently used shorthands
\newcommand{\ra}{\rightarrow}
\newcommand{\se}{\subseteq}
\newcommand{\ip}[1]{\langle#1\rangle}
\newcommand{\dual}{^*}
\newcommand{\inverse}{^{-1}}
\newcommand{\norm}[2]{\|#1\|_{#2}}
\newcommand{\abs}[1]{\lvert #1 \rvert}
\newcommand{\Abs}[1]{\bigg| #1 \bigg|}
\newcommand\bm[1]{\begin{bmatrix}#1\end{bmatrix}}
\newcommand{\op}{\text{op}}

% nicer looking empty set
\let\oldemptyset\emptyset
\let\emptyset\varnothing

%the var phi gang
\let\oldphi\phi
\let\phi\varphi

\setlist[enumerate,1]{topsep=1em,leftmargin=1.8em, itemsep=0.5em, label=\textup{(}\arabic*\textup{)}}
\setlist[enumerate,2]{topsep=0.5em,leftmargin=3em, itemsep=0.3em}

%pagestyle
%\pagestyle{fancy} 

\begin{document}
\begin{center}
    \textsc{Math 605B. HW 2\\ Ian Jorquera}
\end{center}
\vspace{1em}

% https://www.omnicalculator.com/math/chinese-remainder
\section*{The magic of minus 3}
\begin{itemize}
\item[(4)]
\begin{enumerate}[label=(\alph*)]
    \item assume that $p\geq 5$ and let $a$ be a generator for $(\Z/p)^\times$. If $p\equiv 1 \mod 3$ then we know thats $3|p-1$ and so $a^{\frac{p-1}{3}}$ is a primitive $3$-rd root of unity. Meaning from problem 3 we know that $(2a^{\frac{p-1}{3}}+1)^2\equiv-3\mod p$. %other d

    Now assume that there exists a solution $w^2\equiv -3\mod p$. And notice that $2\frac{w-1}{2}+1=w$ as $w\neq 1$. And by the previous problem this means that $\frac{w-1}{2}=:d$ is a primitive third root of unity which must imply that $3|p-1$ as if $d=g^i$ for a generator $g$ we have that $p-1\nmid i$ but $p-1|3i$ and so $3|p-1$.

    \item assume that $1={-3 \choose p}={-1 \choose p}{3 \choose p}$. Notice that there are two cases. Case 1: If $p\equiv 1 \mod 4$ then $1={-3 \choose p}={3 \choose p}={p\choose 3}$ Notice that this means that $p\equiv 1 \mod 3$ as $2$ is a quadratic non-residue, and so $1$ is the only quadratic residue mod $3$. Case 2: If $p\equiv 3 \mod 4$ then $1={-3 \choose p}=-{3 \choose p}={p\choose 3}$ Notice that this means that $p\equiv 1 \mod 3$ for the same reason. And so $p\equiv 1 \mod 3$. The other direct follows exactly from the above.\\ % spell out?
\end{enumerate}

\item[(5)] Notice that $(a)\Leftrightarrow (b)$ by the previous problem. 

We will first prove $c\Rightarrow b$) let $p=x^2+3y^2$ for $x,y\in\Z$ in which case $0\equiv x^2+3y^2 \mod p$ and so $-3y^2\equiv x^2$. Notice that $p\nmid 3y^2$ meaning $-3\equiv \frac{x^2}{y^2}\equiv (\frac{x}{y})^2 \mod p$. 

 %idk fucked shit here
 %Now we will show $d\Rightarrow c$) assume that $p$ factors in $\mathcal{O}_{-3}=\Z[\zeta_3]=\{\frac{a}{2}+\frac{b}{2}\sqrt{-3}\}$, that is $p=z_1z_2$ for non-units in $\mathcal{O}_{-3}$. Consider the norm $N(\frac{a}{2}+\frac{b}{2}\sqrt{-3})=a^2+2b^2$. We know that neither $z_1$ or $z_2$ are units meaning they are not equal to $\pm 1$. Notice that $1=N(z_1)=a^2+2b^2$ which is only the case when $a=\pm 1$. so if $z_1$ and $z_2$ are non-units we know they have norm not equal to $1$. Now notice that $N(p)=p^2=N(z_1)N(z_2)$ and neither norm is $1$ we know that both must be $p$ and so $p=N(z_1)=a^2+2b^2$. 
 
 Finally we will show that $b\Rightarrow d$) first note that $\mathcal{O}_{-3}=\Z[\zeta_3]=\{\frac{a}{2}+\frac{b}{2}\sqrt{-3}\}$ is a UFD and assume we have a solution $w^2\equiv -3 \mod p$ which means that $p|w^2+3=(w-\sqrt{-3})(w+\sqrt{-3})$ and if we assume that $p$ does not factor then by the prime lemma $p$ must divide one of the terms: $p|(w-\sqrt{-3})$ or $p|(w+\sqrt{-3})$ and so we would have that $\frac{w}{p}\pm\frac{1}{p}\sqrt{-3}$. However $\frac{1}{p}$ is not an integer divided by $2$ and so a contradiction.\\

\section*{Quadratic reciprocity}
\item[(3)] We want to find the primes $p$ where $7$ is a square mod $p$. This is that same as asking when ${7 \choose p}=1$. This gives two cases. Case 1: if $p\equiv 1\mod 4$. Then we have that $1={7 \choose p}= {p \choose 7}$. We also know that the squares mod $7$ are $1,2$ and $4$ and so we know that $p\equiv 1,2,4\mod 7$ and $p\equiv 1 \mod 4$. This means that $p$ must be $1,9,25 \mod 28$, by the Chinese remainder theorem. Case 2: $p\equiv 3 \mod 4$. In this case we have that $1={7 \choose p}= -{p \choose 7}$ and non squares mod $7$ are $3,5,6$ and so we know that $p\equiv 3,19,27\mod 28$.\\

\item[(5)]
Notice that the cube map $(\Z/p)^\times\ra (\Z/p)^\times$ is a isomorphism when $p\equiv 2 \mod 3$. To see this first notice that it is a homomorphism, meaning for $(g^ig^j)^3=g^{3(i+j)}=g^{3i}g^{3j}=(g^i)^3(g^j)^3$. To see that this is an isomorphism notice that the cube map is equivalent the multiplying exponents by $3$ in $\Z/p-1$, and $3$ has a multiplicative inverse in $\Z/p-1$ when $p\equiv 2 \mod 3$ which would mean that the cube map is an isomorphism. Now we will consider the solutions to $y^2=x^3+1$ mod $p$ where $p\equiv 2 \mod 3$. In this case the right hand side is a rearrangement of the elements of $(\Z/p)^\times$, and so the solutions are equivalent to the pairs $(z,y)$ where $y^2=z\mod p$. And because there are $\frac{p-1}{2}$ squares there are $\frac{p-1}{2}$ possibilities for $z$, furthermore we know that there are 2 solutions for $y$ in $y^2=z$ for any square mod $p$. Lastly we know there is one point at infinity. So there are a total of $p$ solutions.

\end{itemize}

\end{document}






