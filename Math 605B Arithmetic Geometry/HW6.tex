\documentclass[12pt]{amsart}
\usepackage{preamble}

\begin{document}
\begin{center}
    \textsc{Math 605B. HW 5\\ Ian Jorquera}
\end{center}
\vspace{1em}


\begin{itemize}
\item[(9)] Consider the map $\F_q^\times\ra \F_q^\times$ such that $x\mapsto x^a$. Notice that this is a group homomorphism as $(xy)^a=x^ay^a$. We also know that $\F_q^\times$ is cyclic and so generated by some element $g$ of order $q-1$. Let $d=\gcd(a,q-1)$ and we will show that $g^a$ has order $(q-1)/d$. Let $n$ be the order of $g^a$ meaning $n|q-1$ meaning $q-1=kn$ for some $k$. Notice that it must be the case that $1=g^{an}=g^{a\frac{q-1}{k}}$ which means it must be the case that $k|a$ and $k|q-1$, in which case we know that $k$ is a common divisor. And because the greatest common divisor minimizes $n$, we know that $n=\frac{q-1}{d}$.\\

This means that the image $\ip{g^a}$ has order $\frac{q-1}{d}$ and so by Noether's Isomorphism theorem we know that the kernel has size $d$. \\

\item[(15)] 
% here is a very complete proof of this fact: https://www.docdroid.net/gD4RMlV/proof-of-the-number-of-irr-pol-pdf#page=2
% written by someone smarter then my self.
% TODO: should all be monic poly
First we will show that all monic irreducible divisors of $p(x)=x^{p^n}-x$ in $\F_p=\Z/p$ have degree $d|n$. Let $f(x)$ be a monic irreducible polynomial in $\Z/p$ such that $f(x)|p(x)$. Because $\F_{p^n}$ contains all roots of $p(x)$ we know that the roots of $f(x)$ are also contained in $\F_{p^n}$ ($\F_{p^n}$ is the splitting field of $p(x)$). Let $\alpha$ be a root of $f(x)$, and we know that $\alpha\in \F_{p^n}$. This means that $\Z/p(\alpha)\cong \Z/p[x]/(f(x))$ is a subfield of $\F_{p^n}$ and a degree $\deg(f(x))=d$ extension of $\Z/p$. This means $\Z/p[\alpha]\cong \F_{p^d}$ which is only a subfield when $d|n$.\\

Now we will show that if $f(x)$ is a monic irreducible polynomial such that $\deg(f(x))=d|n$ then $f(x)|p(x)$. In this case we know that $\Z/p[x]/(f(x))\cong \Z/p(\alpha)\cong \F_{p^d}$ for any root $\alpha$ of $f(x)$. And because $\F_{p^d}$ is a subfield of $\F_{p^n}$, which follows from the fact that $d|n$, we know that there is an isomorphic copy of $\alpha$ in $\F_{p^n}$ for any roots $\alpha$ of $f(x)$. And because all roots of $f(x)$ are containing in $\F_{p^n}$ we know that $f(x)|p(x)$.\\

\item[(16)] Let $\ell$ be a prime. We know that the polynomial $p(x)=x^{p^\ell}-x$ containing all monic irreducible polynomial in $\Z/p[x]$ of degree dividing $\ell$ as factors. The only divisors of $\ell$ are $1$ and $\ell$ so all irreducible factors have degree $1$ and $\ell$. The total degree of $p(x)$ is $p^\ell$ and there are $p$ degree $1$ irreducible monic polynomials so $p(x)/(x(x-1)(x-2)\dots(x-(p-1)))$ is a polynomial of degree $p^\ell-p$ whose factors are the irreducible monic polynomials of degree $\ell$. So there are $(p^\ell-p)/\ell$ such polynomials.\\

\item[(17)] Here will will show that the following are equivalent 
\begin{enumerate}
    \item $\zeta_m\in \F_{p^n}$
    \item $x^m-1|x^{p^n-1}-1$
    \item $m|p^n-1$
    \item $p^n\equiv 1\mod m$
\end{enumerate} 

First notice that by definition we have that $p^n\equiv 1 \mod m$ is equivalent to $m|p^n-1$ which shows $(3)\Leftrightarrow (4)$. Now for $(3)\Rightarrow (2)$, assume that $m|p^n-1$ Notice that this means $p^n-1=mk$ for some $k$. We also know that $z^k-1=(z-1)(z^{k-1}+z^{z-2}+\dots+z+1)$ which means for $z=x^m$ we have $(x^m)^k-1=(x^{p^n-1}-1)=(x^m-1)((x^m)^{k-1}+(x^m)^{z-2}+\dots+x^m+1)$ and so $x^m-1|x^{p^n-1}-1$. 
We will then show $(1)\Rightarrow (3)$ Notice that if $\zeta_m\in \F_{p^n}$ then $\zeta_m\in \F_{p^n}^\times$. Meaning $m$ the order of $\zeta_m$ divides the order of the group $p^n-1$

For $(1)\Rightarrow (2)$ Assume that $\zeta_m\in \F_{p^n}$ which means all $m$-th roots of unit are in $\F_{p^n}$ which are the roots of $x^m-1$ so $x^m-1|x^{p^n-1}-1$. Conversely $(2)\Rightarrow (1)$ notice that if $x^m-1|x^{p^n-1}-1$ then there are $m$ elements in $\F_{p^n}$ that act as $m$th roots of unity. Let $n$ be the smallest value such that $a^n=1$ for all such roots of unity. This means all the roots are roots of $x^n-1$ meaning $n=m$. So there must exist an element $\zeta_m\in \F_{p^n}$. 



\end{itemize}
\end{document}