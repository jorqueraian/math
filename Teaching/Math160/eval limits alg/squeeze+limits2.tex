\documentclass[12pt]{report}

\usepackage[left=1in, right=1in, top=1in, bottom=1in]{geometry}

\usepackage{xcolor}
\usepackage{graphicx}
\setlength\parindent{0pt}

\usepackage{graphicx, amsmath,anonchap,tabularx, multicol,verbatim}

\usepackage{enumitem,mdwlist}
\setlist{noitemsep}
\setlist{nolistsep}

\newenvironment{boxe}
    {\begin{center}
    \begin{tabular}{|p{0.9\textwidth}|}
    \hline\\
    }
    { 
    \\\\\hline
    \end{tabular} 
    \end{center}
    }

\begin{document}
\begin{tabular*}{\textwidth}{@{\extracolsep{\fill}}ll}
\textbf{Math 160} More Limits And Asymptotes & \;\;Name: \hrulefill \\
 Hughes-Hallet Chapter 1.8,1.9& Limits and Asymptotes\hspace{1in}  \\
\hline\hline
\end{tabular*} \\

\pagenumbering{gobble}




\begin{enumerate}
\item In this question we will try develop a tool to compute the limit $\displaystyle{\lim_{x\to 0}x^2\cos\left(\frac{1}{x}\right)}$
\begin{enumerate}[label=\alph*.]
    \item Show that $x^2\cos\left(\frac{1}{x}\right)\leq x^2$ for all $x$ (except possible at $x=0$).\\\\\\
    \item Show that $-x^2\leq x^2\cos\left(\frac{1}{x}\right)$ for all $x$ (except possible at $x=0$).\\\\\\
    \item Evaluate $\displaystyle{\lim_{x\to 0}x^2}$ and $\displaystyle{\lim_{x\to 0}-x^2}$\\\\\\\\
    \item Using parts a,b, and c can we determine $\displaystyle{\lim_{x\to 0}x^2\cos\left(\frac{1}{x}\right)}$. (Plotting these functions may help see what is going on around $x=0$)\\\\\\\\
\end{enumerate}

\item Let $\displaystyle{f(x)=\frac{2-\cos(x)}{x+3}}$. In this problem we will try to mimic the method used in the previous problem to solve $\displaystyle{\lim_{x\to \infty}f(x)}$ (read all parts of the problem before starting)
\begin{enumerate}[label=\alph*.]
    \item Find a function $a(x)$ such that $a(x)\leq f(x)$ for all large $x$.\\\\\\\\
    \item Find a function $b(x)$ such that $f(x)\leq b(x)$ for all large $x$.\\\\\\\\
    \item Ensure that $\displaystyle{\lim_{x\to \infty}a(x)}=\displaystyle{\lim_{x\to \infty}b(x)}$. If not find new functions for steps a and b.\\\\\\\\
    \item Using parts a,b, and c what can we conclude about $\displaystyle{\lim_{x\to \infty}f(x)}$. Did we find a horizontal asymptote? discuss with your group.
\end{enumerate}







\newpage
\item Lets try to generalize the process we used in the previous two parts. Fill in the blanks for the theorem below.
\end{enumerate} 

\begin{boxe}  \textbf{Squeeze Theorem:} If $a(x)\leq f(x) \leq b(x)$ for all $x$ near $c$ (except possible $x=c$) and $$\lim_{x\to \infty}a(x)\;\underline{\;\;\;\;\;\;\;\;\;\;\;\;\;\;\;\;\;}\;\lim_{x\to \infty}b(x)$$
then
\[ \lim_{x\to c} f(x) = \underline{\;\;\;\;\;\;\;\;\;\;\;\;\;\;} \]
\end{boxe}
\;\\
\begin{enumerate}[resume]
\item Review Problems: Determine all the horizontal and vertical asymptotes of the following functions. You may use a graphing tool to plot the functions but solve for everything algebraically.\\
\begin{enumerate}[label=\alph*.]
    \item $\displaystyle{\frac{|x+1|}{x+1}}$ (Hint: rewrite as a piece-wise function and simplify as much as possible)\\\\\\\\\\\\\\\\\\\\\\
    \item $\displaystyle{\frac{\sqrt{6+2x^{2}}}{6+4x}}$ (Hint: for horizontal asymptotes multiply by 1 so that the denominator doesn't evaluated to infinity.)\newpage
    \item $\displaystyle{\frac{x^{2}+\left(a-2\right)x-2a}{x^2+(a-1)x-a}}$ for the constant $a$. (Hint: for vertical asymptotes factor the top and bottom)\\\\\\\\\\\\\\\\\\\\\\\\\\\\


    
\end{enumerate}

\item Compute the following limits algebraically or exaplain why they do not exist.\\
\begin{enumerate}[label=\alph*.]
    \item $\displaystyle{\lim_{x\to 4}\frac{x-4}{\sqrt{x}-2}}$ (Hint: Try setting $t=\sqrt{x}$ and rewritting the proble in terms of $t$. if $x\to 4$ then  $t\to \underline{\;\;\;}?$)\\\\\\\\\\\\\\\\\\\\\\
    \item $\displaystyle{\lim_{x\to 0}\frac{4^x-1}{2^x-1}}$ (Hint: recall that $4^x=2^{(2x)}=(2^{x})^2$. Can we cancel out the denominator? )


    
\end{enumerate}


\end{enumerate}


\end{document}
