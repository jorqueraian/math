\documentclass[12pt]{report}

\usepackage[left=1in, right=1in, top=1in, bottom=1in]{geometry}

\usepackage{xcolor}
\usepackage{graphicx}
\setlength\parindent{0pt}

\usepackage{graphicx, amsmath,anonchap,tabularx, multicol,verbatim}

\usepackage{enumitem,mdwlist}
\setlist{noitemsep}
\setlist{nolistsep}

\newenvironment{boxe}
    {\begin{center}
    \begin{tabular}{|p{0.9\textwidth}|}
    \hline\\
    }
    { 
    \\\\\hline
    \end{tabular} 
    \end{center}
    }

\begin{document}
\begin{tabular*}{\textwidth}{@{\extracolsep{\fill}}ll}
\textbf{Math 160} Limits Algebraically & \;\;Name: \hrulefill \\
 Hughes-Hallet Chapter 1.7& Continuity and Limits Algebraically\hspace{1in}  \\
\hline\hline
\end{tabular*} \\

\pagenumbering{gobble}



From the definition, if a function is continuous, then the value of the limit is the same as the value of the function, so when asked to find the limit, we can use the function. We call this ``evaluating limits algebraically,'' or ``using substitution.''\\
\begin{enumerate}
\item Solve the following limits using substitution(make sure the function is continuous before using substitution).\\
\begin{enumerate}[label=\alph*.]
    \item $\displaystyle{\lim_{x\to 0} (\cos(x)+3x^2-9}) =$\\\\\\
    \item $\displaystyle{\lim_{x\to 0} \frac{e^x}{x^4+2}} =$\\\\\\\\
    \item $\displaystyle{\lim_{x\to 0} \frac{x^{2}-6x+8}{x-2}}=$\\\\\\\\
    \item $\displaystyle{\lim_{x\to 2} \frac{x^{2}-6x+8}{x-2}}=$\\\\\\\\
    \item $\displaystyle{\lim_{x\to 9} \frac{\sqrt{x}-3}{x-9}=}$\\
    Hint: Recall the difference of squares: $a^2-b^2=(a-b)(a+b)$. How can you use this on the denominator?
    \\\\\\\\\\\\\\

    Strategy for evaluating limits:
\begin{itemize}
\item Try using substitution. If you get a numeric answer, great, you are done!
\item If something goes wrong, i.e. divide by zero, etc. then we'll need a different technique. We'll build this list of techniques over the next few days.
\item Last resort: graphs or tables (Why are these a last resort?)
\end{itemize} 
    
\end{enumerate}
\newpage
\item Use a counterexample and an explanation to argue that the statement below is false. (Note that ``defined for all $x$" means that a $y$ value exists for each $x$ value.) A counterexample is an example that proves a statement incorrect. \\\\
If $f(x)$ is defined for all $x$ and the limit exists at the point $x=-3$ then $f(x)$ is continuous at $x=-3$.\\\\\\\\\\\\\\\\\\\\


\end{enumerate} 

\begin{boxe}
\textbf{Intermediate Value Theorem}: Suppose $f$ is continuous on a closed interval $[a,b]$. If $u$ is any number between $f(a)$ and $f(b)$, then there is at least one number $c$ in $[a,b]$ such that $f(c)=u$.
\end{boxe}
\begin{enumerate}[resume*]
\item Consider the function $f(x)=x^{6}-2x^{5}+\frac{1}{2}x^{4}-2x^{2}-1$. Explain why each of the following are true - support your answers.\\
\begin{enumerate}[label=\alph*.]
    \item The function $f(x)$ has at least one root between $x=-2$ and $x=0$. (Recall that a root is an $x$-value $c$ where $f(c)=0$)\\\\\\\\\\
    \item The function $f(x)$ achieves the value $-2$ between $x=-1$ and $x=2$. (hint: you may need to use different values for $a$ in $b$)\\\\\\\\\\\\\\\\
\end{enumerate}
\end{enumerate}







\end{document}



