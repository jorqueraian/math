\documentclass[12pt]{report}

\usepackage[left=1in, right=1in, top=1in, bottom=1in]{geometry}

\usepackage{xcolor}
\usepackage{graphicx}
\setlength\parindent{0pt}

\usepackage{graphicx, amsmath,anonchap,tabularx, multicol,verbatim}

\usepackage{enumitem,mdwlist}
\setlist{noitemsep}
\setlist{nolistsep}

\newenvironment{boxe}
    {\begin{center}
    \begin{tabular}{|p{0.9\textwidth}|}
    \hline\\
    }
    { 
    \\\\\hline
    \end{tabular} 
    \end{center}
    }

\begin{document}
\begin{tabular*}{\textwidth}{@{\extracolsep{\fill}}ll}
\textbf{Math 160} Limits Algebraically & \;\;Name: \hrulefill \\
 Hughes-Hallet Chapter 1.7& Continuity and Limits Algebraically\hspace{1in}  \\
\hline\hline
\end{tabular*} \\

\pagenumbering{gobble}





From the definition, if a function is continuous, then the value of the limit is the same as the value of the function, so when asked to find the limit, we can use the function. We call this ``evaluating limits algebraically,'' or ``using substitution.''\\
\begin{enumerate}
\item Solve the following limits using substitution(make sure the function is continuous before using substitution).\\
\begin{enumerate}[label=\alph*.]
    \item $\displaystyle{\lim_{x\to-1} x^2} =$\\\\
    \item $\displaystyle{\lim_{x\to 0} (\cos(x)+3x^2-9}) =$\\\\\\
    \item $\displaystyle{\lim_{x\to 0} \frac{e^x}{x^4+2}} =$\\\\\\\\
    \item $\displaystyle{\lim_{x\to 0} \frac{x^{2}-3x+2}{x-2}}=$\\\\\\\\
    \item $\displaystyle{\lim_{x\to 2} \frac{x^{2}-3x+2}{x-2}}=$\\
    Hint: recall that the definition of a limit(shown of page 2) does not require the value of $f(c)$ to exist. Can we cancel out factors?
    \\\\\\\\\\\\\\\\\\

    Strategy for evaluating limits:
\begin{itemize}
\item Try using substitution. If you get a numeric answer, great, you are done!
\item If something goes wrong, i.e. divide by zero, etc. then we'll need a different technique. We'll build this list of techniques over the next few days.
\item Last resort: graphs or tables (Why are these a last resort?)
\end{itemize} 
    
\end{enumerate}
\newpage
\item Use a counterexample and an explanation to argue that the statement below is false. (Note that ``defined for all $x$" means that a $y$ value exists for each $x$ value.) A counterexample is an example that proves a statement incorrect. \\\\
If $f(x)$ is defined for all $x$ and the limit exists at the point $x=-3$ then $f(x)$ is continuous at $x=-3$.\\\\\\\\\\\\\\\\\\\\

\item Use the limit definition to explain why $\displaystyle{\lim_{x\to 0} \frac{e^x}{x^4}} =$ doesn't exist(Hint: Is there ONE number $L$ that satisfies the definition? What happens to candidates for $L$ as $x$ gets closer to $c$? Does a table or graph help?).\\\\\\\\\\\\\\\\\\\\

\end{enumerate} 



\begin{boxe}  \textbf{Definition of Limit:} Let $f$ be a function defined on an interval around $x=c$ (not necessarily at $x = c.$) We define the limit of the function $f(x)$ as $x$ approaches $c$, written $\displaystyle\lim_{x\to c} f(x)$ to be the number $L$ (if one exists) such that $f(x)$ is as close to $L$ as we want whenever $x$ is sufficiently close to $c$ (but $x \neq c$). If L exists, we write
\[ \lim_{x\to c} f(x) = L. \]
\end{boxe}
\;\\


\begin{boxe}  \textbf{Solutions: 1a: $1$, b: $-8$, c: $\frac{1}{2}$, d: $-1$, e: $1$ } 
\end{boxe}





\end{document}
