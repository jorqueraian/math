\documentclass[12pt]{report}

\usepackage[left=0.75in, right=0.75in, top=0.75in, bottom=0.75in]{geometry}
\setlength\parindent{0pt}

\usepackage{graphicx, amsmath,anonchap,tabularx,multicol}

\usepackage{enumitem}
\setlist{noitemsep}
\setlist{nolistsep}

\newenvironment{boxe}
    {\begin{center}
    \begin{tabular}{|p{0.9\textwidth}|}
    \hline\\
    }
    { 
    \\\\\hline
    \end{tabular} 
    \end{center}
    }

\begin{document}
%%PAGE 1%%
\begin{tabular*}{\textwidth}{@{\extracolsep{\fill}}l l}
\textbf{Derivative Theorems} \\
MATH 160\\
%\textbf{Due Friday, 10/22/18} & MATH 157\\
\hline\hline
\end{tabular*}\\

\section*{Definitions}
\begin{boxe}
A function $f$ is \textbf{Increasing} on $[a,b]$ if for $a\leq x_1< x_2\leq b$ then $f(x_1)<f(x_2)$.\\
A function $f$ is \textbf{Non-decreasing} on $[a,b]$ if for $a\leq x_1< x_2\leq b$ then $f(x_1)\leq f(x_2)$.\\\\

For a function $f$ a ``point of interest'' is a point $x=p$ such that $f'(p)=0$ or is undefined.\\\\
For any function $f$ a point $x=p$ in the domain (that is $f(p)$ exists) is called a \textbf{Critical Point} when $f'(p)=0$ or is undefined. The value $f(p)$ is called the \textbf{Critical Value}. \\\\

Suppose $x=p$ is a point in the domain of $f$.
\begin{itemize}
    \item $f$ has a \textbf{local maximum} at $x=p$ if $f(p)$ is greater than or equal to the values of $f$ for points near $p$.
    \item $f$ has a \textbf{local minimum} at $x=p$ if $f(p)$ is less than or equal to the values of $f$ for points near $p$.
\end{itemize}
We say ``$f$ has a local maxima/minima of $f(p)$ at $x=p$''\\\\

Suppose $x=p$ is a point in the domain of $f$.
\begin{itemize}
    \item $f$ has a \textbf{global maximum} at $p$ if $f(p)$ is greater than or equal to all values of $f$.
    \item $f$ has a \textbf{global minimum} at $p$ if $f(p)$ is less than or equal to all values of $f$.
\end{itemize}\\

A point $x=p$ at which a function $f$ changes concavity is called an \textbf{Inflection Point}.
\end{boxe}

\section*{Theorems}
\begin{boxe}
    \textbf{The Mean Value Theorem:} Suppose $f(x)$ is continuous on $[a,b]$ and differentiable on $(a,b)$. Then there is at least one point $c$ in $(a,b)$ at which $\displaystyle{\frac{f(b)-f(a)}{b-a}=f'(c)}$.\\\\

    \textbf{The Increasing Function Theorem:} Suppose $f(x)$ is continuous on $a\leq x\leq b$ and differentiable on $a< x< b$.
    \begin{itemize}
        \item If $f'(x)> 0$ on $a< x< b$ then $f$ is \textbf{increasing} on $a\leq x\leq b$
        \item If $f'(x)\geq 0$ on $a< x< b$ then $f$ is \textbf{non-decreasing} on $a\leq x\leq b$
    \end{itemize}
    
    \iffalse\textbf{The Racetrack Principle:}
    Suppose $g$ and $h$ are continuous on $a\leq x\leq b$ and differentiable on $a< x< b$ and $g'(x)\leq h'(x)$ on $a< x< b$.
    \begin{itemize}
        \item If $g(a)=h(a)$ then $g(x)\leq h(x)$ on $a\leq x\leq b$
        \item If $g(b)=h(b)$ then $g(x)\geq h(x)$ on $a\leq x\leq b$
    \end{itemize}
    \fi
    \begin{center}
        \begin{tabular}{c|c}
            $f(x)$ & $f'(x)$\\
            \hline
            Increasing & $f'(x)>0$ \\
            \hline
            Decreasing & $f'(x)<0$ \\ \hline

             Point of Interest or& $f'(x)=0$ \\
             Critical Point (if $f(x)$ exists) & $f'(x)$ Does not exist \\
            \hline
        
        \end{tabular}
        \end{center}
\end{boxe}

\section*{Theorems part 2}
\begin{boxe}
    A function $f$ can only change direction (from increasing to decreasing or decreasing to increasing moving from left to right) at a ``point of interest''.\\\\

    \textbf{Local Extrema Theorem:} Suppose $f$ is defined on an interval and has a local maximum or minimum at $x=p$, which is not an endpoint of the interval Then $x=p$ is a critical point. Furthermore if $f$ is differentiable at $x=p$ then $f'(p)=0$.\\\\
    
    \textbf{The First Derivative Test:} Suppose $p$ is a critical point of a continuous function. Moving from Left to Right
    \begin{itemize}
        \item if $f'$ changes from negative to positive at $p$ then $f$ has a \textbf{local minimum} at $x=p$
        \item if $f'$ changes from positive to negative at $p$ then $f$ has a \textbf{local maximum} at $x=p$
    \end{itemize}\\

    \textbf{The Second Derivative Test:} Suppose $f'(p)=0$
    \begin{itemize}
        \item if $f''(p)>0$ then $f$ has a \textbf{local minimum} at $x=p$
        \item if $f''(p)<0$ then $f$ has a \textbf{local maximum} at $x=p$
        \item if $f''(p)=0$ or undefined then the test tells us nothing (This does NOT mean no local extrema exists).
    \end{itemize}\\

    \textbf{The Extreme Value Theorem:} If $f$ is continuous on the closed interval $a\leq x\leq b$ then $f$ has a global maximum and a global minimum on that interval.
\end{boxe}



\end{document}
