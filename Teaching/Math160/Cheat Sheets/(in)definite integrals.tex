\documentclass[12pt]{report}

\usepackage[left=0.75in, right=0.75in, top=0.75in, bottom=0.75in]{geometry}
\setlength\parindent{0pt}

\usepackage{graphicx, amsmath,anonchap,tabularx,multicol}
\usepackage{wrapfig}

\usepackage{enumitem}
\setlist{noitemsep}
\setlist{nolistsep}

\newenvironment{boxe}
    {\begin{center}
    \begin{tabular}{|p{0.9\textwidth}|}
    \hline\\
    }
    { 
    \\\\\hline
    \end{tabular} 
    \end{center}
    }

\begin{document}
\begin{tabular*}{\textwidth}{@{\extracolsep{\fill}}l l}
\textbf{Properties and Applications of Definite Integrals} \\
MATH 160\\
%\textbf{Due Friday, 10/22/18} & MATH 157\\
\hline\hline
\end{tabular*}\\
\begin{boxe}
\textbf{The FUNdamental Theorem of Calculus:} If $f(t)$ is continuous on the interval $[a,b]$ and $F(t)$ is an antiderivative of $f(t)$, meaning $\frac{d}{dt}[F(t)]=f(t)$ then
$$\int_{a}^{b}f(t)\,dt=F(x)\big\vert_a^b=F(b)-F(a)$$
The units of a definite integral can be found by multiplying the units of the integrand (the units of $f(x)$) with the units of $dx$, the variable the integral is with respect to (the variable of integration).\\\\
\textbf{Limits of Integration}:\\ Splitting: $\displaystyle{\int_{a}^{b}f(x)\,dx=\int_{a}^{c}f(x)\,dx+\int_{c}^{b}f(x)\,dx}$\\
Flipping : $\displaystyle{\int_{a}^{b}f(x)\,dx=-\int_{b}^{a}f(x)\,dx}$\\
\textbf{Constant Multiples}: $\displaystyle{\int_{a}^{b}cf(x)\,dx=c\int_{a}^{b}f(x)\,dx}$\\
\textbf{Adding Definite Integrals}: $\displaystyle{\int_{a}^{b}f(x)\pm g(x)\,dx=\int_{a}^{b}f(x)\,dx\pm \int_{a}^{b}g(x)\,dx}$\\
\textbf{Symmetry:}\\ 
If $f(x)$ is an \textbf{even function} (meaning $f(-x)=f(x)$) then $\displaystyle{\int_{-a}^{a}f(x)\,dx=2\int_{0}^{2}f(x)\,dx}$\\
If $f(x)$ is an \textbf{odd function} (meaning $f(-x)=-f(x)$) then $\displaystyle{\int_{-a}^{a}f(x)\,dx=0}$
\end{boxe}

\begin{boxe}
\textbf{Constant Multiples}: $\displaystyle{\int cf(x)\,dx=c\int f(x)\,dx}$\\
\textbf{Adding Definite Integrals}: $\displaystyle{\int f(x)\pm g(x)\,dx=\int f(x)\,dx\pm \int g(x)\,dx}$\\
\textbf{``Power Rule'' for integrals: } For $n\neq -1$ a real number, and $k$ a real number\\
$\displaystyle{\int x^n\,dx=\frac{1}{n+1}x^{n+1}+C}$\;\;\;\;\; $\displaystyle{\int k\,dx=kx+C}$\\
\textbf{Exponent Rules: } $\displaystyle{\int e^{kx}\,dx=\frac{1}{k}e^{kx}+C}$\;\;\;\;\; $\displaystyle{\int a^{kx}\,dx=\frac{1}{k\ln(a)}a^{kx}+C}$\\
\textbf{One over $x$: } $\displaystyle{\int \frac{1}{x}\,dx=\ln |x|+C}$\\
\textbf{Trig: } $\displaystyle{\int \sin(kx)\,dx=-\frac{1}{k}\cos(kx)+C}$\;\;\;\;\; $\displaystyle{\int \cos(kx)\,dx=\frac{1}{k}\sin(kx)+C}$\\ \;\;\;$\displaystyle{\int \sec^2(kx)\,dx=\frac{1}{k}\tan(kx)+C}$
\end{boxe}
\subsection*{Applications of Definite Integrals}
\begin{boxe}
\textbf{Area Between Curves}: If $f(x)\leq g(x)$ (meaning $g(x)$ is above $f(x)$) on the interval $[a,b]$ then the area between the curves $f$ and $g$ on the interval $[a,b]$ is $\displaystyle{\int_{a}^{b}g(x)-f(x)\,dx}$.\\
\begin{quote}
\vspace{-.25in}
    \textbf{CAUTION:} With curves that intersect: find all points of intersection in your interval and compute the area between the intersections separately.
\end{quote}

    \iffalse To the right is a visual example of the area enclosed by the curves $f(x)$ and $g(x)$ and the area \\
    \begin{wrapfigure}{r}{0.20\textwidth}
    \centering
    \includegraphics[width=0.20\textwidth]{areabetweencurve.png}
    \end{wrapfigure}
    \fi

\textbf{Average values}: The average value of a function $f(x)$ on the interval $[a,b]$ can be computed with the definite integral $\displaystyle{\frac{1}{b-a}\int_{a}^{b}f(x)\,dx}$
\end{boxe}



\end{document}
