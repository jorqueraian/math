\documentclass[12pt]{article}

\usepackage[left=.75in, right=.75in, top=1in, bottom=1in]{geometry}
\setlength\parindent{0pt}

\usepackage{graphicx, amsmath,anonchap,tabularx, multicol,array,graphbox,url}
\usepackage{cancel}
\usepackage{enumitem}
\setlist{noitemsep}
\setlist{nolistsep}

%\usepackage{draftwatermark}
%\SetWatermarkText{DRAFT}
%\SetWatermarkScale{5}
%\SetWatermarkColor[rgb]{0.8,0.8,0.8}

\begin{document}

\begin{tabular*}{\textwidth}{@{\extracolsep{\fill}}l l}
\textbf{Reassessment (Module 7)}  & Name: \rule{6cm}{0.5pt} \\
\hline\hline
\end{tabular*} \\



Use this form to unlock standards to reassess in class on Friday October 6th.\\

%\begin{tabular*}{\textwidth}{@{\extracolsep{\fill}}| c | c | c | c |} \hline
\begin{tabular}{| c | c | r | c |} \hline
\textbf{Standard:} & \# met so far & Webwork average &  Tutor or Instructor Signature:\\ \hline
L1 & &min: 80&\\ \hline
L2 & &min:  80 &\\ \hline
L3 & &min:  70 &\\ \hline
L4 & &min:  80 &\\ \hline
L5 & &min:  80 &\\ \hline
L6 & &min:  90&\\ \hline
L7 & &min:  90 &\\ \hline
T1 & &min:  90&  \\ \hline
T2 & &min:  90&  \\ \hline
\hline\hline
\end{tabular} \\
 
 \vspace{.1in}
\small

For each standard that you want to revise:

\begin{itemize}
\item Use the Learning Mastery view on Canvas Grades to see how many times you have met expectations (or exceeded) and enter that number in the table.
\item Use WeBWorK grades to write in your WW average. Make sure your average is greater than or equal to the number in the column.
\item Take the following to the calculus center to discuss with a tutor. After the conversation the tutor can sign in the signature line.
\begin{itemize}
\item An assessment item from class when you missed the standard. This should show the mistake you made.
\item A brief, written statement explaining your mistake or misconception.
\item Full corrections to that assessment item.
\item A brief, written statement about what you learned from this process or how you will be less likely to make a similar kind of mistake in the future.
\end{itemize}
\end{itemize}



\bigskip

The Reassessment Process:
\begin{enumerate}
\item Figure out your mistake. If you don't know, look over course handouts and desmos, and ask your teacher or go to the Calculus Center. Put the time in to make sure you fully understand how to do the problem to ensure this reassessment is successful. WeBWorK can also be a diagnostic tool. Think about how you did or did not use any quiz or exam prep documents.
\item Have a conversation about the mistake and steps you are taking to reduce the likelihood of making this mistake again. Make sure you are ready to reattempt. This should involve working a new problem with the instructor/tutor to uncover \textbf{and fix} any further misconceptions.
\item \textbf{Once you and the tutor feel confident you are ready for a reassessment, the tutor can sign off next to the standard name at the top of the page. Then submit this page to your instructor to be eligible to take a reassessment quiz.}
\item If you cannot get to the Calculus Center for the signature, email the above items to your instructor. This will allow you to take the quiz, but is NOT giving you feedback and helping you know if you are ready to reassess. We do NOT recommend this.
\item How to prioritize: You may take any or all of the standards on the list. If you need to take them all, prioritize standards that are ``locked out'' like L2 and L6. Prioritize standards where you are one away from completion. De-prioritize any standards that you cannot get, or that still have remaining attempts, like T1 and T2.  You can miss 3 standards and still get an A in the class.
\end{enumerate}

\newpage

Where can you get guidance about your mistake?

\begin{itemize}
\item First, go over your class notes and handouts, maybe the Desmos activities.
\item If applicable, read the text book.
\item Look on Canvas for Revision Guides- these point out some common mistakes and provide you with some explanation and extra practice.
\item Calculus Center- most effective if you have done the above steps first, but it's ok if you don't.
\item To get the most out of this process, use the list of questions below.
\end{itemize}


\vskip .4in

What NOT to do:

\begin{itemize}
\item Don't just keep taking new assessments hoping you'll get it right this time. 
\item Don't put this process off until Friday- get to your mistakes as soon as you can!
\item Don't look for shortcuts: there aren't any.
\item Don't do this alone- we are here to support your learning!
\end{itemize}

\vskip .4in
Answering the questions below before initiating the revision process will help you be efficient and get the most out of the process.

\bigskip

It is strongly recommended that you WRITE down answers to the questions below: (adapted from Dr. Akers, Emporia State University) 
\begin{enumerate}
\item Identify where you went wrong in your solution and write up an explanation for WHY it is wrong. (Try writing in complete sentences). Really reflect on this.  Was your incorrect answer due to
\begin{enumerate}
\item not understanding a concept;
\item an error in logical reasoning (e.g., used the correct theorem/test but made the wrong conclusion, used a
theorem/test/technique when it did not apply);
\item being careless (e.g. not reading directions, not answering the question completely, making arithmetic or basic
algebra errors);
\item not knowing how to start or formulate an approach to the problem;
\item others?
\end{enumerate}
\item What helped you recognize your mistake(s). Here are some examples: the course notes, the textbook, homework or conversations from the Calculus Center.  In other words, which strategies for identifying mistakes work well for you and will help you in the future? 
\item Rework the ENTIRE PROBLEM. Rewrite your solution from start to finish, carefully fixing the mistake(s) you diagnosed
above. By doing the entire problem over again, you can make sure you fix your mistake and better understand the point of
the exercise.
\item Describe (in detail) what you have done in order to learn from your mistake(s) and prepare for your next attempt.
Did you read the textbook or class notes? Did you look at examples and/or work problems on your own or with your
tutor/classmate/instructor, and if so, which problems? Did you take a different approach than listed here?
\end{enumerate}



\end{document}
