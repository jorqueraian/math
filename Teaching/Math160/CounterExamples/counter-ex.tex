\documentclass[12pt]{report}

\usepackage[left=1in, right=1in, top=1in, bottom=1in]{geometry}

\usepackage{xcolor}
\usepackage{graphicx}
\setlength\parindent{0pt}

\usepackage{graphicx, amsmath,anonchap,tabularx, multicol,verbatim}

\usepackage{enumitem,mdwlist}
\setlist{noitemsep}
\setlist{nolistsep}

\newenvironment{boxe}
    {\begin{center}
    \begin{tabular}{|p{0.9\textwidth}|}
    \hline\\
    }
    { 
    \\\\\hline
    \end{tabular} 
    \end{center}
    }

\begin{document}
\begin{tabular*}{\textwidth}{@{\extracolsep{\fill}}ll}
\textbf{Math 160} Counter Examples & \;\;Name: \hrulefill \\
\hline\hline
\end{tabular*} \\

\pagenumbering{gobble}

\begin{enumerate} 
\item Use a counterexample and an explanation to argue that the statement 
below is false. (Note that ``defined for all $x$" means that a $y$ value exists for each $x$ value.) A counterexample is an example that proves a statement incorrect. \\\\
If $f(x)$ is defined for all $x$ and the limit exists at the point $x=-3$ then $f(x)$ is continuous at $x=-3$.\\\\\\\\\\\\\\\\\\\\


\end{enumerate} 

\begin{boxe}
\textbf{Intermediate Value Theorem}: Suppose $f$ is continuous on a closed interval $[a,b]$. If $u$ is any number between $f(a)$ and $f(b)$, then there is at least one number $c$ in $[a,b]$ such that $f(c)=u$.
\end{boxe}
\begin{enumerate}[resume*]
\item Consider the function $f(x)=x^{6}-2x^{5}+\frac{1}{2}x^{4}-2x^{2}-1$. Explain why each of the following are true - support your answers.\\
\begin{enumerate}[label=\alph*.]
    \item The function $f(x)$ has at least one root between $x=-2$ and $x=0$. (Recall that a root is an $x$-value $c$ where $f(c)=0$)\\\\\\\\\\
    \item The function $f(x)$ achieves the value $-2$ between $x=-1$ and $x=2$. (hint: you may need to use different values for $a$ in $b$)\\\\\\\\\\\\\\\\
\end{enumerate}
\end{enumerate}

\newpage





\end{document}



