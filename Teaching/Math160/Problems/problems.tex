\documentclass[12pt]{article}

\usepackage[left=.6in, right=.6in, top=1in, bottom=1in]{geometry}
\setlength\parindent{0pt}

\usepackage{graphicx, amsmath,anonchap,tabularx, multicol,array,graphbox,url,tcolorbox}
\usepackage{cancel,mdwlist}
\usepackage{enumitem,verbatim}
\setlist{noitemsep}

%Snell's law diagram stuff
% Author: Jimi Oke
\usepackage{tikz}
\usetikzlibrary{arrows,shapes,positioning}
\usetikzlibrary{decorations.markings}
\tikzstyle arrowstyle=[scale=1]
\tikzstyle directed=[postaction={decorate,decoration={markings,
    mark=at position .65 with {\arrow[arrowstyle]{stealth}}}}]
\tikzstyle reverse directed=[postaction={decorate,decoration={markings,
    mark=at position .65 with {\arrowreversed[arrowstyle]{stealth};}}}]

%\setlist{nolistsep}

%\usepackage{draftwatermark}
%\SetWatermarkText{DRAFT}
%\SetWatermarkScale{5}
%\SetWatermarkColor[rgb]{0.8,0.8,0.8}

\begin{document}

\begin{tabular*}{\textwidth}{@{\extracolsep{\fill}}l l}
\textbf{Module 11 Written Homework Due 11/3}  &Calculus Thinker: \rule{4cm}{0.5pt} \\
\textbf{Standards: T1, L4, L5, D8, D9, D10, D11, I3} & MATH 160 section: \\
\hline\hline
\end{tabular*} \\
 
% \vspace{.1in}
\small

%%%Standards:
{\renewcommand{\arraystretch}{2}%
\begin{tabular}{ |p{5in}| c | c |} \hline
\textbf{Standards Assessed }& \textbf{Question} & \textbf{Grade} \\ \hline
T1: I can apply mathematical definitions. & Q\ref{std:L4} & E M P NG  \\ \hline
L4: I can evaluate limits as $x$ approaches positive or negative infinity and make conclusions about end behavior. & Q\ref{std:L4} & E M P NG  \\ \hline
L5: I can identify limits in indeterminate forms and choose appropriate strategies to evaluate the limit. & Q\ref{std:L4} & E M P NG  \\ \hline
D8: I can compute derivatives using multiple rules in combination. & Q\ref{std:D8} & E M P NG  \\ \hline
D9: I can compute derivatives of implicitly defined functions. & Q\ref{std:D9} & E M P NG  \\ \hline
D10: I can find the equation of the line tangent to a function at a point and use this line as a linear approximation to estimate the value of a function at nearby points. & Q\ref{std:D10} & E M P NG  \\ \hline
D11: I can use derivatives to solve optimization problems. & Q\ref{std:D11} & E M P NG  \\ \hline
I3: I can estimate the value of a definite integral using a table, Riemann sum, or technology.  & Q\ref{std:I3} & E M P NG  \\ \hline

\end{tabular}}

\vskip .1in 

You may print the file to answer questions on paper or use your own paper and write down the questions by hand, or write answers digitally. Take special care to write neatly and in complete sentences. Write your name on every page. If your instructor asks you to submit your homework digitally on Canvas, it must be ONE file in pdf format. If you need help with this, ask your instructor or visit the Calculus Center.

\vspace{.1in}
Resources: You are strongly encouraged to consult your notes, any materials from class and the textbook, and visit the Calculus Center and work with classmates. If you get help in the calculus center, you can discuss concepts, but any specific problems must be different from what is on the assignment. If you work with a classmate, make sure that any writing or drawings you submit are your own work.  You instructor may ask you to explain verbally any work on your written work, so make sure you understand what you've written down well enough to do this. You are not authorized to upload this document or images of this document to any online tutoring or AI service. 
\vspace{.1in}

Honor code:  

By signing below, you acknowledge the academic integrity stipulations around this assignment.
\vskip .2in
I, \rule{6cm}{0.5pt}, certify that I will not give,
receive, or use
any unauthorized assistance.



\normalsize

\subsection*{Hints and Notes on Grading and Definitions}
Please see the companion documents for the homework called {\bf Mod 11 Definitions and Rubrics} and {\bf D11 Thoughts}.

\newpage
\begin{enumerate}


\item  \label{std:D9} \label{std:D10}  {\em  D9, D10}  \vskip .1in
 
    
    Snell's Law or the Law of Refraction describes the behavior of light as it 
    travels between two substances, telling us how light bends.
    \begin{multicols}{2}

    Snell's law tells us about the relationship between the angle of incidence $\theta_1$
    and the angle of refraction $\theta_2$ as a ray of light travels from one medium 
    to another and depends on the refraction indices $n_1$ and $n_2$ of both of 
    the substance.

    \begin{center}
        \textbf{Snell's Law}
    \end{center}
    \[n_1\sin(\theta_1)=n_2\sin(\theta_2)\]
    \vfill \;

        %Snell's law diagram stuff
        % Author: Jimi Oke
        \begin{tikzpicture}

            % define coordinates
            \coordinate (O) at (0,0) ;
            \coordinate (A) at (0,4) ;
            \coordinate (B) at (0,-4) ;
            
            % media
            \fill[blue!25!,opacity=.1] (-4,0) rectangle (4,4);
            \fill[blue!60!,opacity=.2] (-4,0) rectangle (4,-4);
            \node[right] at (1.5,2) {Air ($n_1$)};
            \node[left] at (-1.5,-2) {Glass ($n_2$)};
        
            % axis
            \draw[dash pattern=on5pt off3pt] (A) -- (B) ;
        
            % rays
            \draw[black,ultra thick,reverse directed] (O) -- (130:5.2);
            \draw[black,directed,ultra thick] (O) -- (-70:4.24);
        
            % angles
            \draw (0,1) arc (90:130:1);
            \draw (0,-1.4) arc (270:290:1.4) ;
            \node[] at (280:1.8)  {$\theta_{2}$};
            \node[] at (110:1.4)  {$\theta_{1}$};
        \end{tikzpicture}
    \end{multicols}

\begin{enumerate}
        \item (D9) Use Implicit differentiation to compute $\displaystyle{\frac{d\theta_2}{d\theta_1}}$
        \item The refraction index of air is $n_1=1$ and the refraction index of glass is $n_2=1.52$. 
        If the angle of incidence is $\theta_1=20^\circ$ verify that the angle 
        of refraction is approximately $\theta_2\approx 13^\circ$. 
        (Make sure you calculator is in degrees mode)
        \item (D10) Find an equation for the tangent line when $\theta_1=20^\circ$ and $\theta_2= 13^\circ$
        \item (D10) Use your tangent line to approximate the value of $\theta_2$ when $\theta_1=21^\circ$
\end{enumerate}



 \end{enumerate}

\newpage
\end{document}
