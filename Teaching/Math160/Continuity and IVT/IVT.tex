\documentclass[12pt]{article}

\usepackage[left=0.75in, right=0.75in, top=0.75in, bottom=0.75in]{geometry}
\setlength\parindent{0pt}

\usepackage{graphicx, amsmath,anonchap,tabularx,multicol}

\usepackage{enumitem,pifont}
\setlist{noitemsep}
\setlist{nolistsep}

\newenvironment{boxe}
    {\begin{center}
    \begin{tabular}{|p{0.9\textwidth}|}
    \hline\\
    }
    { 
    \\\\\hline
    \end{tabular} 
    \end{center}
    }

\pagestyle{empty}

\begin{document}

\begin{tabular*}{\textwidth}{@{\extracolsep{\fill}}l l}
\textbf{Counter Examples and the Intermediate Value Theorem}  & Middle Name: \rule{6cm}{0.5pt} \\
MATH 160 & Section:\\
\hline\hline
\end{tabular*} \\

\sf
Learning Targets:
\begin{itemize}
\item[\ding{112}]  Identify the hypothesis/antecedent and conclusion/consequent parts of a theorem.
\item[\ding{112}]  Construct a counterexample to a false statement and explain how the example satisfies the hypothesis but does not satisfy the conclusion.
\item[\ding{112}]  Construct an example that does not satisfy the hypothesis but still satisfies the conclusion.
\item[\ding{112}]  Quote the Intermediate Value Theorem.
\item[\ding{112}]  Apply the Intermediate Value Theorem to explain why a given function must take on certain values and why certain equations must have solutions.
\end{itemize}

Aligns with Standard S1 (Counter Examples), adjacent to Standard L3 (Continuity).
\rm

\hrulefill

\subsubsection*{Parts of a Theorem}

If $x$ is a water-type pokemon then $x$ is weak to grass

\textbf{Hypothesis}: $x$ is a water-type pokemon

\textbf{Conclusion}: $x$ is weak to grass.\\

\begin{itemize}
\item[(1)] Identify the hypothesis and conclusion of the following equivalent theorem statements:
\begin{itemize}
\item[(a)] Suppose $x$ is a water-type pokemon then $x$ is weak to grass.\\\\

\item[(b)] $x$ is weak to grass if $x$ is a water-type pokemon.\\\\
\end{itemize}
\item[(2)] Which of the following pokemon does the theorem apply to: Squirtle, Pikachu, Charmander, a tree.

\vskip .6in


\item[(3)] Theorems can have multiple parts in the hypothesis:
Here is a theorem we will be working with in Math 160 called the Intermediate Value Theorem:

\begin{boxe} Suppose f is continuous on a closed interval $[a,b]$. If $u$ is any number between $f(a)$ and $f(b)$, then there is at least one number $c$ in $[a,b]$ such that $f(c)=u$. \end{boxe}

Which of the following belongs to the hypothesis? What is the subject of the theorem?

\vskip .5in

What is the conclusion of the Intermediate Value Theorem? \vskip .5in

\end{itemize}


\newpage
\subsubsection*{Counterexamples}

Take the statement: Fruit which are apples, are always red.

Translate this into an If/Then statement: \vskip .3in


What would it take to prove this statement is true? \vskip .3in

What would it take to prove this statement is false?\vskip .3in


Which of the following examples would prove the above statement false.
\begin{multicols}{2}
\begin{itemize}
\item red apple
\item green apple
\item orange
\item banana
\item strawberry
\item raspberry
\end{itemize}
\end{multicols}

\subsubsection*{Now with 100\% more math}

\begin{itemize}
\item[(1)] Statement: If $f$ is defined at $x=c$ and $\displaystyle{\lim_{x\rightarrow c}f(x)}$ exists, then $\displaystyle{\lim_{x\rightarrow c}f(x)}=f(c)$.

Use the following four examples to answer the next few parts
\begin{itemize}[parsep=.2in]
\item[i.] $f(x)=x$ at $x=2$
\item[ii.] $f(x)=\dfrac{1}{x}$ at $x=0$
\item[iii.] $f(x)=\sin(x)$ at $x=\pi$
\item[iv.] $f(x)= \begin{cases} x & x \neq0 \\ 3 & x=0\\ \end{cases}$ at $x=0$.
\end{itemize}
\vskip.1in
\begin{enumerate}
\item Which functions satisfy the {\em hypothesis}?\\

\vskip.3in

\item What would need to be true for the conclusion to be false? Which of the functions satisfy the conclusion and which don't?

\vskip 1in


\item Which of the above functions, if any, satisfy the hypothesis but do NOT satisfy the conclusion?

\end{enumerate}
\end{itemize}
\newpage

\subsubsection*{Practice}

\begin{itemize}
\item[(2)] Use a counterexample to explain why the following is false: If $h(x)$ is a polynomial then it has at least one root (Recall: a root is an $x$-value where the function has an output of $0$. In informal words: a place where the function touches the $x$-axis).
\vskip .2in

Create a counterexample:

\vskip 1.5 in

Justify your counter example!  Clearly outline and explain how your example satisfies the hypothesis.  Then clearly outline how your example does NOT meet the conclusion. 

\vskip 1.5in
\item[(3)] True or False? If $\lim\limits_{x \to 8^+} f(x)$ exists, then $\lim\limits_{x \to 8} f(x)$ exists.

\vfill Check list for grading: Does the work given

\begin{itemize}
\item[\ding{112}] Show/describe an adequate counterexample?
\item[\ding{112}] Explain how the example satisfies all parts of the hypothesis? (Not just restating, but explaining!)
\item[\ding{112}] Explain how the example satisfies the NEGATION of the conclusion? (Again, not just stating but explaining!)
\end{itemize}
\newpage

\item[(4)] Use a counterexample to explain why the following is false: If $f(x)$ is defined for all $x$ on the interval $[1,5]$ such that $f(-1)=4$ and $f(5)=-3$ then $f$ has a root on the interval $[1,5]$.

\vfill Check list for grading: Does the work given

\begin{itemize}
\item[\ding{112}] Show/describe an adequate counterexample?
\item[\ding{112}] Explain how the example satisfies all parts of the hypothesis? (Not just restating, but explaining!)
\item[\ding{112}] Explain how the example satisfies the NEGATION of the conclusion? (Again, not just stating but explaining!)
\end{itemize}
\end{itemize}
\end{document}
