\documentclass[12pt]{article}

\usepackage[left=0.75in, right=0.75in, top=0.75in, bottom=0.75in]{geometry}
\setlength\parindent{0pt}

\usepackage{graphicx, amsmath,anonchap,tabularx,multicol}

\usepackage{enumitem,pifont}
\setlist{noitemsep}
\setlist{nolistsep}

%\pagestyle{empty}

\begin{document}

\begin{tabular*}{\textwidth}{@{\extracolsep{\fill}}l l}
\textbf{Tolerance}  & Identifier: \rule{6cm}{0.5pt} \\
MATH 160 & Section:\\
\hline\hline
\end{tabular*} \\

\sf
Aligns with Standard L1 and forms the basis of the definition of {\bf limit}.
\rm

\hrulefill

\begin{itemize}

\item[(1)]
Let $f(x)=\begin{cases} -x+4 &   -2\leq x \leq 2  \\ -2\left(x-3\right)^{2}+4 & x>2 \end{cases}$ which is graphed below
\begin{multicols}{2}
\includegraphics[width=0.8\linewidth]{graphTol.png}

Compute the following
\begin{itemize}
    \item[(a)] $f(0)=$\\
    \item[(b)] $f(3)=$\\
    \item[(c)] $f(2)=$\\
    \item[(d)] $f(100)=$\\
    \item[(e)] $f(-3)=$\\
\end{itemize}
\end{multicols}

\item[(2)] This problem we will answer the following question: Find the largest interval of $x$-values around $x=2$ that guarantees that the outputs of $f(x)$ are no more than $1$ away from $2$.
\begin{itemize}
    \item[(a)] What is the interval of allowed output values? This is called the \textbf{output tolerance}. What is the maximum allowed $y$-value? what is the minimum? On the graph above sketch horizontal lines for the minimum and maximum $y$-values.\\\\\\\\
    \item[(b)] Does $x=2$ give an output value that falls within the allowed output tolerance? Estimate the maximum $x$-value that gives an output within the output tolerance. Sketch a vertical line for this maximum.\\\\\\\\
    \item[(c)] Explain why the following statement is correct or incorrect: All $x$-values on the interval $-1.4\leq x\leq 2$ have output values which stay within the output tolerance.\\\\\\\\
    \item[(d)] Use the graph to estimate what the largest interval of $x$-values around $x=2$ that guarantees that the outputs of $f(x)$ are within the output tolerance.
\end{itemize}
\newpage

\item[(3)] Now we will repeat what we did in (2), but using algebra (round to 3 decimal places, and make sure you round inward!)
\begin{itemize}
    \item[(a)] Set up an equation to solve for the maximum $x$-value.\\\\\\\\\\\\\\\\
    \item[(b)] Set up an equation to solve for the minimum $x$-value.\\\\\\\\\\\\\\\\\\\\\\
\end{itemize}

\item[(4)] Consider the function $g(x)=\begin{cases} x^2-2 &  -2\leq x < 2 \\ -x+5 & x>2 \end{cases}$. Find the largest interval of $x$-values around $x=2$ that guarantees that the outputs of $g(x)$ are no more than $1$ away from $3$\\

%\vspace{0.25in}

\includegraphics[scale=0.7, trim=0 70 0 0,clip]{Axes.jpeg}\\
\end{itemize}

\end{document}
