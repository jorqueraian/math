\documentclass[12pt]{amsart}
% packages
\usepackage{graphicx}
\usepackage{setspace}
\usepackage{amssymb,amsmath,amsthm,amsfonts,amscd}
\usepackage{hyperref}
\usepackage{color}
\usepackage{booktabs}
\usepackage{tabularx}
\usepackage{enumitem}
\usepackage[retainorgcmds]{IEEEtrantools}
\usepackage[notref,notcite,final]{showkeys}
\usepackage[final]{pdfpages}
\usepackage{fancyhdr}
\usepackage{upgreek}
\usepackage{multicol}
\usepackage{fontawesome}
\usepackage{halloweenmath}
% set margin as 0.75in
\usepackage[margin=0.75in]{geometry}

% tikz-related settings
\usepackage{tkz-berge}
\usetikzlibrary{calc,quotes}
\usetikzlibrary{arrows.meta}
\usetikzlibrary{positioning, automata}
\usetikzlibrary{decorations.pathreplacing}

%% For table
\usepackage{tikz}
\usetikzlibrary{tikzmark}

% theorem environments with italic font
\newtheorem{thm}{Theorem}[section]
\newtheorem*{thm*}{Theorem}
\newtheorem{lemma}[thm]{Lemma}
\newtheorem{prop}[thm]{Proposition}
\newtheorem{claim}[thm]{Claim}
\newtheorem{corollary}[thm]{Corollary}
\newtheorem{conjecture}[thm]{Conjecture}
\newtheorem{question}[thm]{Question}
\newtheorem{procedure}[thm]{Procedure}
\newtheorem{assumption}[thm]{Assumption}

% theorem environments with roman font (use lower-case version in body
% of text, e.g., \begin{example} rather than \begin{Example})
\newtheorem{Definition}[thm]{Definition}
\newenvironment{definition}
{\begin{Definition}\rm}{\end{Definition}}
\newtheorem{Example}[thm]{Example}
\newenvironment{example}
{\begin{Example}\rm}{\end{Example}}

\theoremstyle{definition}
\newtheorem{remark}[thm]{\textbf{Remark}}

% special sets
\newcommand{\A}{\mathbb{A}}
\newcommand{\C}{\mathbb{C}}
\newcommand{\F}{\mathbb{F}}
\newcommand{\N}{\mathbb{N}}
\newcommand{\Q}{\mathbb{Q}}
\newcommand{\R}{\mathbb{R}}
\newcommand{\Z}{\mathbb{Z}}
\newcommand{\cals}{\mathcal{S}}
\newcommand{\ZZ}{\mathbb{Z}_{\ge 0}}
\newcommand{\cala}{\mathcal{A}}
\newcommand{\calb}{\mathcal{B}}
\newcommand{\cald}{\mathcal{D}}
\newcommand{\calh}{\mathcal{H}}
\newcommand{\call}{\mathcal{L}}
\newcommand{\calr}{\mathcal{R}}
\newcommand{\la}{\mathbf{a}}
\newcommand{\lgl}{\mathfrak{gl}}
\newcommand{\lsl}{\mathfrak{sl}}
\newcommand{\lieg}{\mathfrak{g}}

% math operators
\DeclareMathOperator{\kernel}{\mathrm{ker}}
\DeclareMathOperator{\image}{\mathrm{im}}
\DeclareMathOperator{\rad}{\mathrm{rad}}
\DeclareMathOperator{\id}{\mathrm{id}}
\DeclareMathOperator{\hum}{[\mathrm{Hum}]}
\DeclareMathOperator{\eh}{[\mathrm{EH}]}
\DeclareMathOperator{\lcm}{\mathrm{lcm}}
\DeclareMathOperator{\Aut}{\mathrm{Aut}}
\DeclareMathOperator{\Inn}{\mathrm{Inn}}
\DeclareMathOperator{\Out}{\mathrm{Out}}
\DeclareMathOperator{\Gal}{\mathrm{Gal}}
\DeclareMathOperator{\Real}{\mathrm{Re}}
\DeclareMathOperator{\Imag}{\mathrm{Im}}


% frequently used shorthands
\newcommand{\ra}{\rightarrow}
\newcommand{\se}{\subseteq}
\newcommand{\ip}[1]{\langle#1\rangle}
\newcommand{\dual}{^*}
\newcommand{\inverse}{^{-1}}
\newcommand{\norm}[2]{\|#1\|_{#2}}
\newcommand{\abs}[1]{\lvert #1 \rvert}
\newcommand{\Abs}[1]{\bigg| #1 \bigg|}
\newcommand\bm[1]{\begin{bmatrix}#1\end{bmatrix}}
\newcommand{\op}{\text{op}}

% nicer looking empty set
\let\oldemptyset\emptyset
\let\emptyset\varnothing

%the var phi gang
\let\oldphi\phi
\let\phi\varphi

\setlist[enumerate,1]{topsep=1em,leftmargin=1.8em, itemsep=0.5em, label=\textup{(}\arabic*\textup{)}}
\setlist[enumerate,2]{topsep=0.5em,leftmargin=3em, itemsep=0.3em}

%pagestyle
%\pagestyle{fancy} 

\begin{document}
\begin{center}
    \textsc{Math 519. HW 1\\ Ian Jorquera}
\end{center}
\vspace{1em}
% See http://www.mathematicalgemstones.com/maria/Math501Fall22.php
% for problems

% sage: https://sagecell.sagemath.org/
\begin{enumerate}

\item 
\begin{enumerate}[label=(\alph*)]
    \item $|z-z_1|=|z-z_2|$ represents the line of points $z\in \C$ such that $z$ is equidistant from the points $z_1$ and $z_2$.
    \item in this case $\overline{z}$ is the multiplicative inverse meaning $|z|^2=z\overline{z}=1$ or that $|z|=1$. This means this represent the unit circle.
    \item This represents the vertical line in the complex plane where $z=3+ai$ 
    \item This represents a half plan, or all points above the line $z=c+ai$ for a fixed $c$. Including the line when $\geq$ and not including the line when $>$.
    \item Notice that for $z=x+yi$, $a=a_1+ia_2$ and $b=b_1+ib_2$ we have $\Real(az+b)=\Real(az)+b_1$ and because $az=(a_1x- a_2y) + (a_2x+a_1y)i$ and so $\Real(az+b) = a_1x- a_2y + b_i> 0$ is represents a half plane shifted by $b_i$ and skewed according to $a$.
    \item Let $z=x+iy$ and we have that $|z|=\sqrt{x^{2}+y^{2}}=x+1=\Real(z)+1$ and so this represents a parabola opening to the right with vertex at $-1+0\cdot i=-1$ or $(x,y)=(-1,0)$.\\
    \item This represents a horizontal line where $z=a+ci$ for some fixed $c$.\\
\end{enumerate}

\item
\begin{enumerate}[label=(\alph*)]
    \item First we will show that complex conjugation is a ring endomorphism $\C\ra\C$. Let $a+bi,c+di\in \C$ and notice that $\overline{(a+bi)+(c+di)}=\overline{(a+c)+(b+d)i}=(a+c)-(b+d)i=(a-bi)+(c-di)=\overline{(a+bi)}+\overline{(c+di)}$. We also have that $\overline{(a+bi)(c+di)}=\overline{(ac-bd)+(ad+bc)i}=(ac-bd)-(ad+bc)i=(a-bi)(c-di)=\overline{(a+bi)}\cdot\overline{(c+di)}$. Becasue $\C$ is a field this is a field endomorphism and therefore a ring homomorphism. Finally to see that this is an involution not that for $a+bi\in \C$ we have that $\overline{\overline{a+bi}}=\overline{a-bi}=a+bi$.\\

    \item Now consider an element $z=c+di\in \C$ and notice that for any $a\in \R$ we have that $\Real(az)=\Real(ac+adi)=ac=a \Real(a+di)=a\Real(z)$. And similarly we have that $\Imag(az)=\Imag(ac+adi)=ad=a \Imag(a+di)=a\Imag(z)$.\\

    \end{enumerate}

    \item
    \begin{enumerate}[label=(\alph*)]
    \item Notice that for $z=a+bi,w=c+di\in \C$ we have that $|z+w|^2=\sqrt{(z+w)\overline{(z+w)}}^2= (z+w)\overline{(z+w)}= |z|^2+|w|^2+z\overline{w}+\overline{z}w$. And notice that $z\overline{w}+\overline{z}w=(a+bi)(c-di)+(a-bi)(c+di)=2ac+2bd$ and that $\Real(z\overline{w})=ac+bd$. So $|z+w|^2=|z|^2+|w|^2+2\Real(z\overline{w})$.\\

    \item Using the previous identity we have that for any $z,w\in \C$, that $|z+w|^2+|z-w|^2= |z|^2+|w|^2+|z|^2+|-w|^2+2\Real(z\overline{w})+2\Real(-z\overline{w})$. And from problem 2b we know that $|z|^2+|w|^2+|z|^2+|-w|^2+2\Real(z\overline{w})+2\Real(-z\overline{w})=|z|^2+|w|^2+|z|^2+|-w|^2+2\Real(z\overline{w})-2\Real(z\overline{w})=|z|^2+|w|^2+|z|^2+|w|^2=2(|z|^2+|w|^2)$.
\end{enumerate}

\item 
\begin{enumerate}[label=(\alph*)]
    \item First notice that $z(t^*)\not\in \Omega_1$. So see this notice that if it were there would exists some open neighborhood $N_r(z(t^*))$ for some $r>0$. And becasue $z$ is continuous this would mean there would exist some $\delta$ such that for $|t-t^*|<\delta$ we would have $|z(t)-z(t^*)|<r$ and so there would exist some $t^*<t_0<t^*+\delta$ such that $z(t_0)\in\Omega_1$ a contradiction as $t^*$ was assumed to be the supremum. Because $z(t^*)\not\in\Omega_1$ we know that $z(t^*)\in \Omega_2$. However this meaning there exist some $\epsilon>0$ such that there exists a nieborhood $N_\epsilon(z(t^*))\se \Omega_2$. And by the continuity of $z$ we know that there would exist some $\delta$ such that for $|t-t^*|<\delta$ where we would have $|z(t)-z(t^*)|<\epsilon$ and so there would exist some $t^*-\delta<t_0<t^*$ such that $z(t_0)\in\Omega_1$ buy definition of $t^*$ and $z(t_0)\in\Omega_1$ as it is contained in the neighborhood a contradiction as $\Omega_1$ and $\Omega_2$ are assumed to be disjoint.
    \item Consider any point $w_1\in\Omega_1$. There exists some $\epsilon>0$ with a neighborhood $N_\epsilon(w_1)\se \Omega$. Let $\Tilde{w_1}\in N_\epsilon(w_1)$. In which case $w_1$ and $\Tilde{w_1}$ are connected by the curve that is the line between them %explicit construction?
    And because the composition of piecewise continuous curves is piece wise continuous we know that this defines a curve from $w$ to $\Tilde{w_1}$ and so $\Tilde{w_1}\in \Omega_1$ and so $N_\epsilon(w_1)\se \Omega_1$ meaning $\Omega_1$ is open. Like wise consider any point $w_2\in\Omega_2$. There exists some $\epsilon>0$ with a neighborhood $N_\epsilon(w_2)\se \Omega$. Let $\Tilde{w_2}\in N_\epsilon(w_2)$. In which case $w_2$ and $\Tilde{w_2}$ are connected by the curve that is the line between them %explicit construction?
    And because the composition of piecewise continuous curves is piece wise continuous we know that if there existed a piece wise continuous curve between $w$ and $\Tilde{w_2}$ it would define a curve from $w$ to $w_2$ and so $\Tilde{w_2}\in \Omega_2$ and so $N_\epsilon(w_2)\se \Omega_2$ meaning $\Omega_2$ is open. Notice that these two sets must be disconnected as otherwise there would exist a point that both has a path from $w$ and doesn't, a contradiction. We also know that these sets form the entire set $\Omega$ as $\Omega_2$ contains all points in $\Omega$ that are not in $\Omega_1$. Finally notice that $\Omega_1$ is not empty as it contains the point $w$, as there is trivial continuous map from $w$ to its self $w$. From the assumption that $\Omega$ is connected we know that there can not exist any set $\Omega_2$ and so $\Omega=\Omega_1$.\\
\end{enumerate}
\item 
\begin{enumerate}
    \item Consider the inequality $\left|\frac{w-z}{1-\overline{w}z}\right|<1$ in which case we have that $\left|{w-z}\right|<\left|{1-\overline{w}z}\right|$ where we know that the square function is an increasing function, and both sides are positive so we know that 
    $$\left|{w-z}\right|<\left|{1-\overline{w}z}\right|$$
    $$\left({w-z}\right)\left({\overline{w-z}}\right)<\left({1-\overline{w}z}\right)\left(\overline{1-\overline{w}z}\right)$$
    $$\left({w-z}\right)\left({\overline{w}-\overline{z}}\right)<\left({1-\overline{w}z}\right)\left({1-w\overline{z}}\right)$$
    $$|w|+|z|<1+|w||z|$$
    $$0<1-|w|-|z|+|w||z|$$
    $$0<(1-|w|)(1-|z|)$$
And from the assumption we know that $|w|<1$ and $|z|<1$ and so $(1-|w|)(1-|z|)$ is the product of two positive numbers and so is positive.\\

Notice by the same argument we can see that $\left|\frac{w-z}{1-\overline{w}z}\right|=1$ is equivalent to $0<(1-|w|)(1-|z|)$ and with the assumption that $|w|=1$ and $|z|=1$. \\

\item Fix $w\in \mathbb{D}$ meaning $|w|<1$. (i) Consider any point $z\in\mathbb{D}$ which is the case when $|z|<1$ we have that $\left|\frac{w-z}{1-\overline{w}z}\right|<1$ meaning $\frac{w-z}{1-\overline{w}z}\in \mathbb{D}$.
(ii) Notice that $F(0)=\frac{w-0}{1-\overline{w}0}=w$ and $F(w)=\frac{w-0}{1-\overline{w}w}=0$, as $|w|<1$ so $1-\overline{w}w\neq 0$. 
(iii) Consider any point $z$ where $|z|=1$. We have that $\left|\frac{w-z}{1-\overline{w}z}\right|=1$.
(iv) To see that $F$ is a bijection we will show it is an involution. Notice that $F(F(z))= \frac{w-\frac{w-z}{1-\overline{w}z}}{1-\overline{w}\frac{w-z}{1-\overline{w}z}}=\frac{\frac{w-w\overline{w}z}{1-\overline{w}z}-\frac{w-z}{1-\overline{w}z}}{\frac{1-\overline{w}z}{1-\overline{w}z}-\frac{\overline{w}w-\overline{w}z}{1-\overline{w}z}}=\frac{-w\overline{w}z+z}{1-\overline{w}w}=z\frac{1-w\overline{w}}{1-\overline{w}w}=z$ snd so 
$F$ is its own inverse.
    
\end{enumerate}

\end{enumerate}

\end{document}






