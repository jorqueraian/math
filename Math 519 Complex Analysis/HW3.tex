\documentclass[12pt]{amsart}
% packages
\usepackage{graphicx}
\usepackage{setspace}
\usepackage{amssymb,amsmath,amsthm,amsfonts,amscd}
\usepackage{hyperref}
\usepackage{color}
\usepackage{booktabs}
\usepackage{tabularx}
\usepackage{enumitem}
\usepackage[retainorgcmds]{IEEEtrantools}
\usepackage[notref,notcite,final]{showkeys}
\usepackage[final]{pdfpages}
\usepackage{fancyhdr}
\usepackage{upgreek}
\usepackage{multicol}
\usepackage{fontawesome}
\usepackage{halloweenmath}
% set margin as 0.75in
\usepackage[margin=0.75in]{geometry}

% tikz-related settings
\usepackage{tkz-berge}
\usetikzlibrary{calc,quotes}
\usetikzlibrary{arrows.meta}
\usetikzlibrary{positioning, automata}
\usetikzlibrary{decorations.pathreplacing}

%% For table
\usepackage{tikz}
\usetikzlibrary{tikzmark}

% theorem environments with italic font
\newtheorem{thm}{Theorem}[section]
\newtheorem*{thm*}{Theorem}
\newtheorem{lemma}[thm]{Lemma}
\newtheorem{prop}[thm]{Proposition}
\newtheorem{claim}[thm]{Claim}
\newtheorem{corollary}[thm]{Corollary}
\newtheorem{conjecture}[thm]{Conjecture}
\newtheorem{question}[thm]{Question}
\newtheorem{procedure}[thm]{Procedure}
\newtheorem{assumption}[thm]{Assumption}

% theorem environments with roman font (use lower-case version in body
% of text, e.g., \begin{example} rather than \begin{Example})
\newtheorem{Definition}[thm]{Definition}
\newenvironment{definition}
{\begin{Definition}\rm}{\end{Definition}}
\newtheorem{Example}[thm]{Example}
\newenvironment{example}
{\begin{Example}\rm}{\end{Example}}

\theoremstyle{definition}
\newtheorem{remark}[thm]{\textbf{Remark}}

% special sets
\newcommand{\A}{\mathbb{A}}
\newcommand{\C}{\mathbb{C}}
\newcommand{\F}{\mathbb{F}}
\newcommand{\N}{\mathbb{N}}
\newcommand{\Q}{\mathbb{Q}}
\newcommand{\R}{\mathbb{R}}
\newcommand{\Z}{\mathbb{Z}}
\newcommand{\cals}{\mathcal{S}}
\newcommand{\ZZ}{\mathbb{Z}_{\ge 0}}
\newcommand{\cala}{\mathcal{A}}
\newcommand{\calb}{\mathcal{B}}
\newcommand{\cald}{\mathcal{D}}
\newcommand{\calh}{\mathcal{H}}
\newcommand{\call}{\mathcal{L}}
\newcommand{\calr}{\mathcal{R}}
\newcommand{\la}{\mathbf{a}}
\newcommand{\lgl}{\mathfrak{gl}}
\newcommand{\lsl}{\mathfrak{sl}}
\newcommand{\lieg}{\mathfrak{g}}

% math operators
\DeclareMathOperator{\kernel}{\mathrm{ker}}
\DeclareMathOperator{\image}{\mathrm{im}}
\DeclareMathOperator{\rad}{\mathrm{rad}}
\DeclareMathOperator{\id}{\mathrm{id}}
\DeclareMathOperator{\hum}{[\mathrm{Hum}]}
\DeclareMathOperator{\eh}{[\mathrm{EH}]}
\DeclareMathOperator{\lcm}{\mathrm{lcm}}
\DeclareMathOperator{\Aut}{\mathrm{Aut}}
\DeclareMathOperator{\Inn}{\mathrm{Inn}}
\DeclareMathOperator{\Out}{\mathrm{Out}}
\DeclareMathOperator{\Gal}{\mathrm{Gal}}
\DeclareMathOperator{\Real}{\mathrm{Re}}
\DeclareMathOperator{\Imag}{\mathrm{Im}}


% frequently used shorthands
\newcommand{\ra}{\rightarrow}
\newcommand{\se}{\subseteq}
\newcommand{\ip}[1]{\langle#1\rangle}
\newcommand{\dual}{^*}
\newcommand{\inverse}{^{-1}}
\newcommand{\norm}[2]{\|#1\|_{#2}}
\newcommand{\abs}[1]{\lvert #1 \rvert}
\newcommand{\Abs}[1]{\bigg| #1 \bigg|}
\newcommand\bm[1]{\begin{bmatrix}#1\end{bmatrix}}
\newcommand{\op}{\text{op}}

%\def\darktheme{} % IAN
\ifx \darktheme\undefined
\else
\pagecolor[rgb]{0.2,0.231,0.302}%{0.23,0.258,0.321}
\color[rgb]{1,1,1}
\fi

% nicer looking empty set
\let\oldemptyset\emptyset
\let\emptyset\varnothing

%the var phi gang
\let\oldphi\phi
\let\phi\varphi

\setlist[enumerate,1]{topsep=1em,leftmargin=1.8em, itemsep=0.5em, label=\textup{(}\arabic*\textup{)}}
\setlist[enumerate,2]{topsep=0.5em,leftmargin=3em, itemsep=0.3em}

%pagestyle
%\pagestyle{fancy} 

\begin{document}
\begin{center}
    \textsc{Math 519. HW 3\\ Ian Jorquera}
\end{center}
\vspace{1em}
% See http://www.mathematicalgemstones.com/maria/Math501Fall22.php
% for problems

% sage: https://sagecell.sagemath.org/
\begin{enumerate}

\item 
Let $z:[0,2\pi]\ra \C$ where $t\mapsto e^{it}$ be a parameterization of the contour $\gamma_1(0)$. This gives us that 
\begin{align*}
    \int_{\gamma_1(0)}z^ndz&=\int_{0}^{2\pi} (z(t))^nz'(t)dt\\
    &= \int_{0}^{2\pi} e^{int}ie^{it}dt\\
    &= i\int_{0}^{2\pi} e^{i(n+1)t}dt\\
    &= i \frac{1}{n+1}e^{(n+1)ti}\Big|_{0}^{2\pi}\\
    &= 0
\end{align*}

\item 
\begin{enumerate}
    \item First we will use the definition where we have that 

    \begin{align*}
        \int_{\gamma_1(0)}\Real(z)dz&=\int_{0}^{2\pi} \Real(z(t))z'(t)dt\\
        &= \int_{0}^{2\pi} \Real(e^{it})ie^{it}dt\\
        &= \int_{0}^{2\pi} \cos(t)(-\sin(t)+i\cos(t))dt\\
        &= \int_{0}^{2\pi} -\cos(t)\sin(t)dt+i\int_{0}^{2\pi}\cos^2(t)dt\\
        &= \frac{1}{2}\cos^2(x)\Big|_{0}^{2\pi}+\frac{i}{2}(\cos(x)\sin(x)+x)\Big|_{0}^{2\pi}\\
        &=0+\pi i=\pi i
    \end{align*}

    \item Now we will use the fact that $\Real(z)=\frac{z+\overline{z}}{2}$ which gives us that 
    \begin{align*}
        \int_{\gamma_1(0)}\Real(z)dz&=\int_{\gamma_1(0)}\frac{z+\overline{z}}{2}dz\\
        &=\frac{1}{2}\int_{\gamma_1(0)}zdz+\frac{1}{2}\int_{\gamma_1(0)}\overline{z}dz\\
        &=0+\pi i=\pi i
    \end{align*}
    Which follows from $z$ being holomorphic.\\

\end{enumerate}

\item Here we will bound the norm of the integral for $\epsilon>0$
\begin{align*}
    \Big|\int_\gamma f(z)dz\Big|&= \Big|\int_\gamma f(z)-P_\epsilon(z)+P_\epsilon(z)dz\Big| = \Big|\int_\gamma f(z)-P_\epsilon(z)dz+\int_\gamma P_\epsilon(z)dz\Big|\\
    &\leq \Big|\int_\gamma f(z)-P_\epsilon(z)dz\Big|+\Big|\int_\gamma P_\epsilon(z)dz\Big|\\
    &\leq \text{Len}(\gamma)\sup_{z\in\gamma} |f(z)-P_\epsilon(z)|+0\\
    &\leq \text{Len}(\gamma)\epsilon
\end{align*}

And so because $\text{Len}(\gamma)$ is a constant we know that $|\int_\gamma f(z)dz|=0$. This gives us the bounds $-|\int_\gamma f(z)dz|\leq \int_\gamma f(z)dz\leq |\int_\gamma f(z)dz|$ which tells us that $\int_\gamma f(z)dz=0$\\

\item We we will consider the toy contour $C$ that first goes from $0$ to $R$ on the real line, then  follows the path $\gamma_1: z_1:[0,\frac{\pi}{4}]\ra \C$ where $t\mapsto Re^{it}$ and then $\gamma_2: z_2:[-R,0]\ra \C$ where $t\mapsto -te^{i\frac{\pi}{4}}$. We will integrate over the function $f(z)=e^{iz^2}$. First Notice that $\int_{C}f(z)=\int_0^Rf(x)dx+\int_{\gamma_1}f(z)dz+\int_{\gamma_2}f(z)dz=0$ and the function we are integrating is holomorphic on $\C$ and we are integrating on a toy contour.
And we will look at each integral individually. First notice that 
\begin{align*}
    \int_0^Re^{ix^2}dx=\int_0^R\cos(x^2)dx+i\int_{0}^R\sin(x^2)dx\\
\end{align*}
Also notice that
\begin{align*}
    \int_{\gamma_2}e^{iz^2}dz&= \int_{-R}^0\exp(i(-te^{i\frac{\pi}{4}})^2)(-e^{i\frac{\pi}{4}})dt\\
    &=-e^{i\frac{\pi}{4}}\int_{-R}^0\exp(it^2e^{i\frac{\pi}{2}})dt\\
    &=-e^{i\frac{\pi}{4}}\int_{-R}^0\exp(i^2t^2)dt\\
    &=-e^{i\frac{\pi}{4}}\int_{-R}^0\exp(-t^2)dt\\
    &=-(\frac{\sqrt{2}}{2}+i\frac{\sqrt{2}}{2})\frac{\sqrt{\pi}}{2}dt\\
    &=-\frac{\sqrt{2\pi}}{4}-i\frac{\sqrt{2\pi}}{4}dt
\end{align*}

Finally notice that

\begin{align*}
    \int_{\gamma_2}e^{iz^2}dz&=\int_{0}^\frac{\pi}{4}\exp(i(Re^{it})^2)Rie^{it}dt\\
    &=\int_{0}^\frac{\pi}{4}\exp(iR^2e^{i2t})Rie^{it}dt\\
    &=\int_{0}^\frac{\pi}{4}\exp(iR^2(\cos(2t)+i\sin(2t)))Rie^{it}dt\\
    &=\int_{0}^\frac{\pi}{4}\exp(iR^2\cos(2t)-R^2\sin(2t))Rie^{it}dt\\
    &=\int_{0}^\frac{\pi}{4}\exp(iR^2\cos(2t))\exp(-R^2\sin(2t))Rie^{it}dt
\end{align*}
Now we will look at the magnitude of this
\begin{align*}
    \Big|\int_{0}^\frac{\pi}{4}\exp(iR^2\cos(2t))\exp(-R^2\sin(2t))Rie^{it}dt\Big|&\leq \int_{0}^\frac{\pi}{4}|\exp(iR^2\cos(2t))\exp(-R^2\sin(2t))Rie^{it}|dt \\ % can do?
    &\leq \int_{0}^\frac{\pi}{4}|\cos(R^2\cos(2t))\exp(-R^2\sin(2t))R|dt\\
    &\leq \int_{0}^\frac{\pi}{4}\exp(-R^2\sin(2t))Rdt
\end{align*}
Now notice that as $R\ra \infty$ we have that this integral goes to $0$. This follows from the fact that for any fixed $0<t<\frac{\pi}{4}$ we have that $\sin(2t)$ is a constant and so for big enough $R$ the integrand approaches zero. Meaning $\lim_{R\ra \infty}\int_{\gamma_2}e^{iz^2}dz=0$.\\

This gives us
\begin{align*}
0=\int_{C}f(z)&=\int_0^Rf(x)dx+\int_{\gamma_1}f(z)dz+\int_{\gamma_2}f(z)dz\\
\int_0^Rf(x)dx&=-\int_{\gamma_2}f(z)dz\\
\int_0^R\cos(x^2)dx+i\int_{0}^R\sin(x^2)dx&=\frac{\sqrt{2\pi}}{4}+i\frac{\sqrt{2\pi}}{4}
\end{align*}
giving us the desired result.\\
\end{enumerate}

\end{document}






