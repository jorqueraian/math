\documentclass[12pt]{amsart}
% packages
\usepackage{graphicx}
\usepackage{setspace}
\usepackage{amssymb,amsmath,amsthm,amsfonts,amscd}
\usepackage{hyperref}
\usepackage{color}
\usepackage{booktabs}
\usepackage{tabularx}
\usepackage{enumitem}
\usepackage[retainorgcmds]{IEEEtrantools}
\usepackage[notref,notcite,final]{showkeys}
\usepackage[final]{pdfpages}
\usepackage{fancyhdr}
\usepackage{upgreek}
\usepackage{multicol}
\usepackage{fontawesome}
\usepackage{halloweenmath}
% set margin as 0.75in
\usepackage[margin=0.75in]{geometry}

% tikz-related settings
\usepackage{tkz-berge}
\usetikzlibrary{calc,quotes}
\usetikzlibrary{arrows.meta}
\usetikzlibrary{positioning, automata}
\usetikzlibrary{decorations.pathreplacing}

%% For table
\usepackage{tikz}
\usetikzlibrary{tikzmark}

% theorem environments with italic font
\newtheorem{thm}{Theorem}[section]
\newtheorem*{thm*}{Theorem}
\newtheorem{lemma}[thm]{Lemma}
\newtheorem{prop}[thm]{Proposition}
\newtheorem{claim}[thm]{Claim}
\newtheorem{corollary}[thm]{Corollary}
\newtheorem{conjecture}[thm]{Conjecture}
\newtheorem{question}[thm]{Question}
\newtheorem{procedure}[thm]{Procedure}
\newtheorem{assumption}[thm]{Assumption}

% theorem environments with roman font (use lower-case version in body
% of text, e.g., \begin{example} rather than \begin{Example})
\newtheorem{Definition}[thm]{Definition}
\newenvironment{definition}
{\begin{Definition}\rm}{\end{Definition}}
\newtheorem{Example}[thm]{Example}
\newenvironment{example}
{\begin{Example}\rm}{\end{Example}}

\theoremstyle{definition}
\newtheorem{remark}[thm]{\textbf{Remark}}

% special sets
\newcommand{\A}{\mathbb{A}}
\newcommand{\C}{\mathbb{C}}
\newcommand{\F}{\mathbb{F}}
\newcommand{\N}{\mathbb{N}}
\newcommand{\Q}{\mathbb{Q}}
\newcommand{\R}{\mathbb{R}}
\newcommand{\Z}{\mathbb{Z}}
\newcommand{\cals}{\mathcal{S}}
\newcommand{\ZZ}{\mathbb{Z}_{\ge 0}}
\newcommand{\cala}{\mathcal{A}}
\newcommand{\calb}{\mathcal{B}}
\newcommand{\cald}{\mathcal{D}}
\newcommand{\calh}{\mathcal{H}}
\newcommand{\call}{\mathcal{L}}
\newcommand{\calr}{\mathcal{R}}
\newcommand{\la}{\mathbf{a}}
\newcommand{\lgl}{\mathfrak{gl}}
\newcommand{\lsl}{\mathfrak{sl}}
\newcommand{\lieg}{\mathfrak{g}}

% math operators
\DeclareMathOperator{\kernel}{\mathrm{ker}}
\DeclareMathOperator{\image}{\mathrm{im}}
\DeclareMathOperator{\rad}{\mathrm{rad}}
\DeclareMathOperator{\id}{\mathrm{id}}
\DeclareMathOperator{\hum}{[\mathrm{Hum}]}
\DeclareMathOperator{\eh}{[\mathrm{EH}]}
\DeclareMathOperator{\lcm}{\mathrm{lcm}}
\DeclareMathOperator{\Aut}{\mathrm{Aut}}
\DeclareMathOperator{\Inn}{\mathrm{Inn}}
\DeclareMathOperator{\Out}{\mathrm{Out}}
\DeclareMathOperator{\Gal}{\mathrm{Gal}}
\DeclareMathOperator{\Real}{\mathrm{Re}}
\DeclareMathOperator{\Imag}{\mathrm{Im}}


% frequently used shorthands
\newcommand{\ra}{\rightarrow}
\newcommand{\se}{\subseteq}
\newcommand{\ip}[1]{\langle#1\rangle}
\newcommand{\dual}{^*}
\newcommand{\inverse}{^{-1}}
\newcommand{\norm}[2]{\|#1\|_{#2}}
\newcommand{\abs}[1]{\lvert #1 \rvert}
\newcommand{\Abs}[1]{\bigg| #1 \bigg|}
\newcommand\bm[1]{\begin{bmatrix}#1\end{bmatrix}}
\newcommand{\op}{\text{op}}

\def\darktheme{} % IAN
\ifx \darktheme\undefined
\else
\pagecolor[rgb]{0.2,0.231,0.302}%{0.23,0.258,0.321}
\color[rgb]{1,1,1}
\fi

% nicer looking empty set
\let\oldemptyset\emptyset
\let\emptyset\varnothing

%the var phi gang
\let\oldphi\phi
\let\phi\varphi

\setlist[enumerate,1]{topsep=1em,leftmargin=1.8em, itemsep=0.5em, label=\textup{(}\arabic*\textup{)}}
\setlist[enumerate,2]{topsep=0.5em,leftmargin=3em, itemsep=0.3em}

%pagestyle
%\pagestyle{fancy} 

\begin{document}
\begin{center}
    \textsc{Math 519. HW 2\\ Ian Jorquera}
\end{center}
\vspace{1em}
% See http://www.mathematicalgemstones.com/maria/Math501Fall22.php
% for problems

% sage: https://sagecell.sagemath.org/
\begin{enumerate}

\item 
\begin{enumerate}[label=(\alph*)]
    \item Here we will compute the limit along the trajectory $x_0+\Delta x+iy_0$. That is if the limit exist we know that 
    \begin{align*}
        \lim_{\Delta x\ra 0}\frac{f(x_0+\Delta x+iy_0)-f(x_0+iy_0)}{x_0+\Delta x+iy_0-x_0-iy_0}&=\lim_{\Delta x\ra 0}\frac{u(x_0+\Delta x,y_0)+iv(x_0+\Delta x,y_0)-u(x_0,y_0)-iv(x_0,y_0)}{\Delta x}\\
        &=\lim_{\Delta x\ra 0}\frac{u(x_0+\Delta x,y_0)-u(x_0,y_0)}{\Delta x}+\lim_{\Delta x\ra 0}\frac{iv(x_0+\Delta x,y_0)-iv(x_0,y_0)}{\Delta x}\\
        &=u_x(x_0,y_0)+iv_x(x_0,y_0)\\
        &=\left(\frac{\partial u}{\partial x}+i\frac{\partial v}{\partial x}\right)|_{z_0}
    \end{align*}
    \item  Here we will compute the derivative along the trajectory $x_0+i(y_0+\Delta y)$. That is if the limit exist we know that 
    \begin{align*}
        \lim_{\Delta y\ra 0}\frac{f(x_0+i(y_0+\Delta y))-f(x_0+iy_0)}{x_0+i(y_0+\Delta y)-x_0-iy_0}&=\lim_{\Delta y\ra 0}\frac{u(x_0,y_0+\Delta y)+iv(x_0,y_0+\Delta y)-u(x_0,y_0)-iv(x_0,y_0)}{i\Delta y}\\
        &=\lim_{\Delta y\ra 0}\frac{u(x_0,y_0+\Delta y)-u(x_0,y_0)}{i\Delta y}+\lim_{\Delta y\ra 0}\frac{iv(x_0,y_0+\Delta y)-iv(x_0,y_0)}{i\Delta y}\\
        &=\frac{1}{i}(u_y(x_0,y_0)+iv_x(x_0,y_0))\\
        &=\frac{1}{i}\left(\frac{\partial u}{\partial y}+i\frac{\partial v}{\partial y}\right)|_{z_0}
    \end{align*}
    \item for the derivative of the function $f$ to exist we would need at a minimum that the limits computed along both trajectories are the same. That is $\left(\frac{\partial u}{\partial x}+i\frac{\partial v}{\partial x}\right)|_{z_0}=\frac{1}{i}\left(\frac{\partial u}{\partial y}+i\frac{\partial v}{\partial y}\right)|_{z_0}=\left(-i\frac{\partial u}{\partial y}+\frac{\partial v}{\partial y}\right)|_{z_0}$. And because both $u$ and $v$ are $\R$-valued functions with $\R$-valued derivatives we know that this must imply that $\frac{\partial u}{\partial x}|_{z_0}=\frac{\partial v}{\partial y}|_{z_0}$ and $\frac{\partial v}{\partial x}|_{z_0}=-\frac{\partial u}{\partial y}|_{z_0}$ which is exactly the Cauchy-Riemann equations.\\
\end{enumerate}

\item  Let $f(z)=u(x,y)+iv(x,y)$ be a holomorphic function meaning $u_x=v_y$ and $u_y=-v_x$ on the set $\Omega$.\\ 
\begin{enumerate}
    \item Now assume that $\Real(f)=c$ for some constant $c\in \R$. This means that $u_x(x,y)=0=v_y(x,y)$ Similarly we know that $u_y(x,y)=0=-v_x(x,y)$ this means that $u(x,y)$ and $v(x,y)$ must be constants as they have partial derivatives zero.\\
    \item Now assume that $\Imag(f)=c$ for some constant $c\in \R$. This means that $v_y(x,y)=0=u_x(x,y)$ Similarly we know that $v_x(x,y)=0=-u_y(x,y)$ and this means that $u(x,y)$ and $v(x,y)$ must be constants as they have partial derivatives zero.\\
    \item Now assume that $|f|$ is constant which means that $|f|=u(x,y)^2+v(x,y)^2$ is constant. This means that taking the derivative with respect to $x$ and $y$ gives us that $2uu_x+2vv_x=0$ and $2uu_y+2vv_y=0$ and so using Cauchy Riemann equations we have that $uv_y-vu_y=0$ and $-uv_x+vu_x=0$. Notice that if $u=0$ then $0^2+v^2$ would be constant and so $v$ would be constant. So we can assume that $u\neq 0$ in which case we find that
    \begin{equation*}
      \begin{split}
      uv_y-vu_y=0\\
      uv_y=vu_y\\
      v_y=\frac{v}{u}u_y
      \end{split}
    \quad\quad\quad
      \begin{split}
        -uv_x-vu_x=0\\
      uv_x=vu_x\\
      v_x=\frac{v}{u}u_x
      \end{split}
    \end{equation*}
    And so plugging back into the original equations
    \begin{equation*}
      \begin{split}
      uu_y+v(\frac{v}{u}u_y)=0\\
      \frac{u^2+v^2}{u}u_y=0
      \end{split}
    \quad\quad\quad
      \begin{split}
       uu_x+v(\frac{v}{u}u_x)=0\\
      \frac{u^2+v^2}{u}u_x=0
      \end{split}
    \end{equation*}
And in both cases we have that $u^2+v^2\neq 0$ meaning $u_x=0$ and $u_y=0$ and so by Cauchy Riemann equation we also have that $v_x=0$ and $v_y=0$ and so $u$ and $v$ are constant, and therefore $f$ is constant.\\ 
\end{enumerate}

\item
\begin{enumerate}
    \item We will show that the sequence of partial sums does not converge by showing it is not Cauchy. Consider the sequence of partial sums $\{s_n=\sum_{k=0}^{n}kz^k\}$ and notice that for $m=n-1$ we have that $|s_n-s_{n-1}|=|nz^n| = n$ this follows from the fact that $z$ is on the unit circle and the norm is multiplicative. Notice also that $\lim_{n\ra \infty} |s_n-s_{n-1}|=\lim_{n\ra\infty} n$ which diverges. And so this sequence is not Cauchy.\\

    \item Consider the sequence of partial sums $S_n=\sum_{k=1}^{n}z^k/k^2$ Let $\epsilon>0$ and pick $N>\frac{1}{\epsilon}$. Notice that for any values $N\leq m\leq n$ we have that $|S_n-S_m|=|\sum_{k=m+1}^{n}z^k/k^2|\leq \sum_{k=m+1}^{n}|z^k/k^2|=\sum_{k=m+1}^{n}k^2$ by the triangle inequality and because $|z|=1$ and the norm is multiplicative. Also notice that each term $\frac{1}{k^2}\leq \frac{1}{k(k-1)}=\frac{1}{k-1}-\frac{1}{k}$. Meaning $\sum_{k=m+1}^{n}\frac{1}{k^2}\leq\sum_{k=m+1}^{n}\frac{1}{k-1}-\sum_{k=m+1}^{n}\frac{1}{k}=\sum_{k=m}^{n-1}\frac{1}{k}-\sum_{k=m+1}^{n}\frac{1}{k}=\frac{1}{m}-\frac{1}{n}< \frac{1}{m}< \frac{1}{N}<\epsilon$.\\
\end{enumerate}

\item First notice that when $z=1$ we have that $\sum_{n=1}^\infty z^n/n=\sum_{n=1}^\infty 1/n$ is the harmonic series and so diverges.

Now assume that $z\neq 1$ and is on the unit circle and consider the difference of partial sums 
\begin{align*}
|S_N-S_{M-1}|&=|\sum_{n=M}^N\frac{z^n}{n}|\\
&=|\frac{1}{N}\sum_{n=1}^Nz^n-\frac{1}{M}\sum_{n=1}^{M-1}z^n-\sum_{n=M}^{N-1}\left[(\frac{1}{n+1}-\frac{1}{n})\sum_{j=1}^nz^j\right]|\\
&\leq |\frac{1}{N}\sum_{n=1}^Nz^n|+|\frac{1}{M}\sum_{n=1}^{M-1}z^n|+\sum_{n=M}^{N-1}|\left[(\frac{1}{n+1}-\frac{1}{n})\sum_{j=1}^nz^j\right]|\\
&=\frac{1}{N}|\frac{z^{N+1}-z}{z-1}|+\frac{1}{M}|\frac{z^{M}-z}{z-1}|+\sum_{n=M}^{N-1}|\left[\frac{1}{n^2+n}\frac{z^{j+1}-z}{z-1}\right]|
\end{align*}
And from the triangle inequality and from the fact that $|z|=1$ we know that we can bound the numerators of each fraction by 2 giving 
$$|S_N-S_{M-1}|\leq \frac{2}{N}|\frac{1}{z-1}|+\frac{2}{M}|\frac{1}{z-1}|+\sum_{n=M}^{N-1}\frac{2}{n^2+n}|\frac{1}{z-1}|$$
Notice that $|\frac{1}{z-1}|$ is constant for any fixed $z$. So for big enough $N$ and $M$ the first two terms go to zero. Furthermore because $\sum_{n=M}^{N-1}\frac{2}{n^2+n}$ is the difference of two partial sums of the converging sequence $\sum \frac{1}{n^2+n}$ we know that for big enough $N$ and $M$ the sum goes to zero. And so this entire expression goes to zero for big enough $N$ and $M$. So by analysis this sequence is Cauchy and so converges.\\  

\item Let $z=x+iy=re^{i\gamma}$ and $w=u+iv=se^{iu\beta}$ and notice that $\overline{zw}=\overline{(x+iy)(u+iv)}=\overline{(xu-yw)+(xw+yu)i}=(xu-yw)-(xw+yu)i$ and likewise $\overline{z}\;\overline{w}=(x-iy)(u-iv)=(xu-yw)-(xw+yu)i$ and so $\overline{zw}=\overline{z}\;\overline{w}$. This gives us that $|zw|=\sqrt{zw\overline{zw}}=\sqrt{wz\overline{z}\;\overline{w}}=\sqrt{z\overline{z}}\sqrt{w\overline{w}}=|z||w|$.\\ Notice also that $\text{arg}(re^{i\gamma}se^{i\beta})=\text{arg}(rse^{i\gamma+\beta})=\gamma+\beta=\text{arg}(re^{i\gamma})+\text{arg}(se^{i\beta})$. This shows that $|-|$ and $\text{arg}(-)$ are both group homomoprhism and so $\alpha$ is a group homomorphism.\\

Now we will show there exists an inverse map $\alpha^{-1}:\R_{>0}\oplus \R/2\pi\Z\ra C^\times$ that maps $(r,\theta)\mapsto re^{i\theta}$. Notice that for an element  $\alpha(\alpha^{-1}((r,\theta)))=\alpha(re^{i\theta})=(r,\theta)$. And similarly $\alpha^{-1}(\alpha(re^{i\theta}))=\alpha^{-1}((r,\theta))=re^{i\theta}$. This shows that $\alpha$ is injective and surjective.\\

\end{enumerate}

\end{document}






