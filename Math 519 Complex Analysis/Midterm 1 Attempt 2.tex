\documentclass[12pt]{amsart}
% packages
\usepackage{graphicx}
\usepackage{setspace}
\usepackage{amssymb,amsmath,amsthm,amsfonts,amscd}
\usepackage{hyperref}
\usepackage{color}
\usepackage{booktabs}
\usepackage{tabularx}
\usepackage{enumitem}
\usepackage[retainorgcmds]{IEEEtrantools}
\usepackage[notref,notcite,final]{showkeys}
\usepackage[final]{pdfpages}
\usepackage{fancyhdr}
\usepackage{upgreek}
\usepackage{multicol}
\usepackage{fontawesome}
\usepackage{halloweenmath}
% set margin as 0.75in
\usepackage[margin=0.75in]{geometry}

% tikz-related settings
\usepackage{tkz-berge}
\usetikzlibrary{calc,quotes}
\usetikzlibrary{arrows.meta}
\usetikzlibrary{positioning, automata}
\usetikzlibrary{decorations.pathreplacing}

%% For table
\usepackage{tikz}
\usetikzlibrary{tikzmark}

% theorem environments with italic font
\newtheorem{thm}{Theorem}[section]
\newtheorem*{thm*}{Theorem}
\newtheorem{lemma}[thm]{Lemma}
\newtheorem{prop}[thm]{Proposition}
\newtheorem{claim}[thm]{Claim}
\newtheorem{corollary}[thm]{Corollary}
\newtheorem{conjecture}[thm]{Conjecture}
\newtheorem{question}[thm]{Question}
\newtheorem{procedure}[thm]{Procedure}
\newtheorem{assumption}[thm]{Assumption}

% theorem environments with roman font (use lower-case version in body
% of text, e.g., \begin{example} rather than \begin{Example})
\newtheorem{Definition}[thm]{Definition}
\newenvironment{definition}
{\begin{Definition}\rm}{\end{Definition}}
\newtheorem{Example}[thm]{Example}
\newenvironment{example}
{\begin{Example}\rm}{\end{Example}}

\theoremstyle{definition}
\newtheorem{remark}[thm]{\textbf{Remark}}

% special sets
\newcommand{\A}{\mathbb{A}}
\newcommand{\C}{\mathbb{C}}
\newcommand{\F}{\mathbb{F}}
\newcommand{\N}{\mathbb{N}}
\newcommand{\Q}{\mathbb{Q}}
\newcommand{\R}{\mathbb{R}}
\newcommand{\Z}{\mathbb{Z}}
\newcommand{\cals}{\mathcal{S}}
\newcommand{\ZZ}{\mathbb{Z}_{\ge 0}}
\newcommand{\cala}{\mathcal{A}}
\newcommand{\calb}{\mathcal{B}}
\newcommand{\cald}{\mathcal{D}}
\newcommand{\calh}{\mathcal{H}}
\newcommand{\call}{\mathcal{L}}
\newcommand{\calr}{\mathcal{R}}
\newcommand{\la}{\mathbf{a}}
\newcommand{\lgl}{\mathfrak{gl}}
\newcommand{\lsl}{\mathfrak{sl}}
\newcommand{\lieg}{\mathfrak{g}}

% math operators
\DeclareMathOperator{\kernel}{\mathrm{ker}}
\DeclareMathOperator{\image}{\mathrm{im}}
\DeclareMathOperator{\rad}{\mathrm{rad}}
\DeclareMathOperator{\id}{\mathrm{id}}
\DeclareMathOperator{\hum}{[\mathrm{Hum}]}
\DeclareMathOperator{\eh}{[\mathrm{EH}]}
\DeclareMathOperator{\lcm}{\mathrm{lcm}}
\DeclareMathOperator{\Aut}{\mathrm{Aut}}
\DeclareMathOperator{\Inn}{\mathrm{Inn}}
\DeclareMathOperator{\Out}{\mathrm{Out}}
\DeclareMathOperator{\Gal}{\mathrm{Gal}}
\DeclareMathOperator{\Real}{\mathrm{Re}}
\DeclareMathOperator{\Imag}{\mathrm{Im}}
\DeclareMathOperator{\res}{\mathrm{res}}


% frequently used shorthands
\newcommand{\ra}{\rightarrow}
\newcommand{\se}{\subseteq}
\newcommand{\ip}[1]{\langle#1\rangle}
\newcommand{\dual}{^*}
\newcommand{\inverse}{^{-1}}
\newcommand{\norm}[2]{\|#1\|_{#2}}
\newcommand{\abs}[1]{\left| #1 \right|}
\newcommand{\Abs}[1]{\bigg| #1 \bigg|}
\newcommand\bm[1]{\begin{bmatrix}#1\end{bmatrix}}
\newcommand{\op}{\text{op}}

%my stupid commands
\newcommand{\overbar}[1]{\overline{#1}}

%\def\darktheme{} % IAN
\ifx \darktheme\undefined
\else
\pagecolor[rgb]{0.2,0.231,0.302}%{0.23,0.258,0.321}
\color[rgb]{1,1,1}
\fi

% nicer looking empty set
\let\oldemptyset\emptyset
\let\emptyset\varnothing

%the var phi gang
\let\oldphi\phi
\let\phi\varphi

\setlist[enumerate,1]{topsep=1em,leftmargin=1.8em, itemsep=0.5em, label=\textup{(}\arabic*\textup{)}}
\setlist[enumerate,2]{topsep=0.5em,leftmargin=3em, itemsep=0.3em}

%pagestyle
%\pagestyle{fancy} 

\begin{document}
\begin{center}
    \textsc{Math 519. Exam 1 Attempt 2\\ Ian Jorquera}
\end{center}
\vspace{1em}
% See http://www.mathematicalgemstones.com/maria/Math501Fall22.php
% for problems

% sage: https://sagecell.sagemath.org/
\begin{enumerate}
\item
\begin{enumerate}
    \item Let $f$ be a holomorphic function on $\C-\{0\}$ such that if $|z|<1$ then $f(z)=\Real(z)$. This means that for $z=x+yi$ where $|z|<1$ then $f(x+iy)=x$. Notice that this does not satisfy the Cauchy Riemann equations as in this case $f(x+iy)=u(x,y)+iv(x,y)$ where $u(x,y)=x$ and $v(x,y)=0$, this means that $u_x(x,y)=1$ while $v_y(x,y)=0$ and so they do not agree on $|z|< 1$, meaning no such example exists.\\
    
    \item  Recall that on the unit circle we have that $1/z=\overbar{z}$. Meaning that on the unit circle $\Real(z)=\frac{1}{2}(z+\overbar{z})=\frac{1}{2}(z+\frac{1}{z})$. So $f(z)=\frac{1}{2}(z+\frac{1}{z})$ is an example of a function that is holomorphic on $\C-\{0\}$ and has $f(z)=\Real(z)$ on $|z|=1$.\\ %As an additional comment the Cauchy Riemann Equations are not applicable in this situation as $|z|=1$ is not an open set.\\

    \item Consider the function $f$ which is holomorphic on the domain $N_2(0)$. Notice that the toy contour $\gamma_1(0)$ and its interior are contained in $N_2(0)$ which means the integral $$\int_{\gamma_1(0)}f(z)dz=0$$
    So no such example exist where the integral is non-zero.\\

    \item Consider the function $f(z)=z^{-2}$ which is holomorphic on the deleted neighborhood $N_2(0)^\times$ but has a pole at $z=0$ and so is not holomorphic on $N_2(0)$. Notice also that $\res(f,0)=0$ as $f(z)=\frac{1}{z^2}+\frac{0}{z}+0$ which is in the form $\frac{a_{-2}}{z^2}+\frac{a_{-1}}{z}+g(z)$. meaning by the residue formula
    $$\int_{\gamma_1(0)}f(z)dz=2\pi i\cdot 0=0$$
\end{enumerate}

\item 
\begin{enumerate}
    \item Notice that the function 
    $$f(z)=\frac{1}{z^3+1}=\frac{1}{(z+1)(z-e^{\pi i /3})(z-e^{5 \pi i /3})}$$
    which means the function $f$ has $3$ simple poles at $z=-1$, $z=e^{\pi i /3}$, and $z=e^{5\pi i /3}$. Notice how ever that the only pole contained in the contour $\gamma_R$ is $z=e^{\pi i /3}$. The residue of this pole can then be computed by $\res(f,e^{\pi i /3})=\frac{1}{3}(e^{\pi i /3})^{-2}=\frac{1}{3}(e^{4\pi i /3})$. This means that from the Residue integral formula 
    $$\int_{\gamma_R}f(z)dz=\frac{2}{3}\pi i e^{4\pi i /3}=\frac{2}{3}\pi e^{-\pi i /6}$$

    \item Consider the segment of the contour $B_R$ where we know that values on this contour have $|z|=R$, meaning \begin{align*}
        \left|\int_{B_R}f(z)dz\right|&\leq \text{len}(B_R)\sup_{z\in B_R}|\frac{1}{z^3+1}|\\
        &\leq \frac{2R\pi}{3}\sup_{z\in B_R}\frac{1}{|z^3+1|}\\
        &\leq \frac{2R\pi}{3}\frac{1}{R^3-1}
    \end{align*}
    And so this integral is bounded by a constant multiple of a function that behaves like $R^{-2}$ so $\lim_{R\ra \infty}\left|\int_{B_R}f(z)dz\right|=0$ and so
    $$\lim_{R\ra \infty}\int_{B_R}f(z)dz=0$$

    \item Consider the parameterization of the curve $C_R$ such that $z:[-R,0]\ra \C$ where $t\mapsto -te^{2\pi i /3}$ and $z'(t)=-e^{2\pi i /3}$. This gives us that
    \begin{align*}
        \int_{C_R}f(z)dz&=\int_{-R}^0 f(-te^{2\pi i /3})(-e^{2\pi i /3})dt\\
        &=e^{2\pi i /3}\int_{0}^{-R} \frac{1}{(-te^{2\pi i /3})^3+1}dt\\
        &=e^{2\pi i /3}\int_{0}^{-R} \frac{1}{-t^3+1}dt\\
    \end{align*}
    Now let $s=-t$ and $ds=-dt$ and we have that
    \begin{align*}
        e^{2\pi i /3}\int_{0}^{-R} \frac{1}{-t^3+1}dt&=-e^{2\pi i /3}\int_{0}^{R} \frac{1}{s^3+1}ds\\
        &=-e^{2\pi i /3}\int_{A_R} f(z)dz\\
    \end{align*}

    and so $\alpha=-e^{2\pi i /3}=e^{5\pi i /3}$.

    \item Finally from part (a) we know that
    $$\int_{\gamma_R}f(z)dz=\int_{A_R}f(z)dz+\int_{B_R}f(z)dz+\int_{C_R}f(z)dz=\frac{2}{3}\pi e^{-\pi i /6}$$
    And from part (b) and (c) we know that on the limit as $R\ra \infty$ we have that 
    \begin{align*}
    \lim_{R\ra \infty}\int_{A_R}f(z)dz+\int_{C_R}f(z)dz&=\frac{2}{3}\pi e^{-\pi i /6}-\lim_{R\ra \infty}\int_{B_R}f(z)dz\\
    \lim_{R\ra \infty}(e^{5\pi i /3}+1)\int_{A_R}f(z)dz&=\frac{2}{3}\pi e^{-\pi i /6}\\
    \lim_{R\ra \infty}(e^{5\pi i /3}e^{\pi i /6}+e^{\pi i /6})\int_{A_R}f(z)dz&=\frac{2}{3}\pi\\
    \lim_{R\ra \infty}(e^{11\pi i /6}+e^{\pi i /6})\int_{A_R}f(z)dz&=\frac{2}{3}\pi \\
    \lim_{R\ra \infty}\sqrt{3}\int_{A_R}f(z)dz&=\frac{2}{3}\pi\\
    \lim_{R\ra \infty}\int_{0}^R f(z)dz=\lim_{R\ra \infty}\int_{A_R}f(z)dz&=\frac{2}{3^{3/2}}\pi
    \end{align*}
\end{enumerate}

\item 
\begin{enumerate}
    \item Let $f$ be an entire function and suppose for each $z\in \C$ that $|f(z)|\leq A|z|$ for some constant $A$. Now consider some value $z_0\in \C$ and fix $j\geq 2$. From the Cauchy inequality we know that for any $\overbar{N_R(z_0)}$ that
    \begin{align*}
        \abs{f^{(j)}(z_0)}&\leq \frac{j!}{R^j}||f||_{\gamma_R(z_0)} \\
        &\leq \frac{j!}{R^j}A||z||_{\gamma_R(z_0)}\\
        &\leq \frac{j!}{R^j}A(|z_0|+R)\\
        &\leq \frac{j!A|z_0|}{R^j}+\frac{j!A}{R^{j-1}}\\
    \end{align*}
    Notice that because $j$, $A$ and $z_0$ are fixed that $$\abs{f^{(j)}(z_0)}\leq \lim_{R\ra \infty}\frac{j!A|z_0|}{R^j}+\lim_{R\ra \infty}\frac{j!A}{R^{j-1}}=0$$
    and so we know that $f^{(j)}(z_0)=0$ for all $z_0\in\C$ where $j\geq 2$.\\
    
    \item Because $f$ is entire we know $f'$ is entire. From the Cauchy inequality we know that for any $\overbar{N_R(z_0)}$ that
    \begin{align*}
        \abs{f'(z_0)}&\leq \frac{1}{R}||f||_{\gamma_R(z_0)} \\
        &\leq \frac{A|z_0|}{R}+\frac{AR}{R}\\
    \end{align*}
    Notice that because $A$ and $z_0$ are fixed that 
    $$\abs{f'(z_0)}\leq \lim_{R\ra \infty}\frac{A|z_0|}{R}+\frac{AR}{R}=A$$
    And so  $f'(z)$ is bounded by $A$ and because $f$ is entire it is therefore constant by Liouville's theorem. This must mean that $f(z)=c+\alpha z$. And because $|f(0)|\leq A|0|=0$ we know that $f(z)=\alpha z$ for some complex number $\alpha$.
\end{enumerate}

\end{enumerate}

\end{document}






