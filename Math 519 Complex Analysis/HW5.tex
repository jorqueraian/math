\documentclass[12pt]{amsart}
% packages
\usepackage{graphicx}
\usepackage{setspace}
\usepackage{amssymb,amsmath,amsthm,amsfonts,amscd}
\usepackage{hyperref}
\usepackage{color}
\usepackage{booktabs}
\usepackage{tabularx}
\usepackage{enumitem}
\usepackage[retainorgcmds]{IEEEtrantools}
\usepackage[notref,notcite,final]{showkeys}
\usepackage[final]{pdfpages}
\usepackage{fancyhdr}
\usepackage{upgreek}
\usepackage{multicol}
\usepackage{fontawesome}
\usepackage{halloweenmath}
% set margin as 0.75in
\usepackage[margin=0.75in]{geometry}

% tikz-related settings
\usepackage{tkz-berge}
\usetikzlibrary{calc,quotes}
\usetikzlibrary{arrows.meta}
\usetikzlibrary{positioning, automata}
\usetikzlibrary{decorations.pathreplacing}

%% For table
\usepackage{tikz}
\usetikzlibrary{tikzmark}

% theorem environments with italic font
\newtheorem{thm}{Theorem}[section]
\newtheorem*{thm*}{Theorem}
\newtheorem{lemma}[thm]{Lemma}
\newtheorem{prop}[thm]{Proposition}
\newtheorem{claim}[thm]{Claim}
\newtheorem{corollary}[thm]{Corollary}
\newtheorem{conjecture}[thm]{Conjecture}
\newtheorem{question}[thm]{Question}
\newtheorem{procedure}[thm]{Procedure}
\newtheorem{assumption}[thm]{Assumption}

% theorem environments with roman font (use lower-case version in body
% of text, e.g., \begin{example} rather than \begin{Example})
\newtheorem{Definition}[thm]{Definition}
\newenvironment{definition}
{\begin{Definition}\rm}{\end{Definition}}
\newtheorem{Example}[thm]{Example}
\newenvironment{example}
{\begin{Example}\rm}{\end{Example}}

\theoremstyle{definition}
\newtheorem{remark}[thm]{\textbf{Remark}}

% special sets
\newcommand{\A}{\mathbb{A}}
\newcommand{\C}{\mathbb{C}}
\newcommand{\F}{\mathbb{F}}
\newcommand{\N}{\mathbb{N}}
\newcommand{\Q}{\mathbb{Q}}
\newcommand{\R}{\mathbb{R}}
\newcommand{\Z}{\mathbb{Z}}
\newcommand{\cals}{\mathcal{S}}
\newcommand{\ZZ}{\mathbb{Z}_{\ge 0}}
\newcommand{\cala}{\mathcal{A}}
\newcommand{\calb}{\mathcal{B}}
\newcommand{\cald}{\mathcal{D}}
\newcommand{\calh}{\mathcal{H}}
\newcommand{\call}{\mathcal{L}}
\newcommand{\calr}{\mathcal{R}}
\newcommand{\la}{\mathbf{a}}
\newcommand{\lgl}{\mathfrak{gl}}
\newcommand{\lsl}{\mathfrak{sl}}
\newcommand{\lieg}{\mathfrak{g}}

% math operators
\DeclareMathOperator{\kernel}{\mathrm{ker}}
\DeclareMathOperator{\image}{\mathrm{im}}
\DeclareMathOperator{\rad}{\mathrm{rad}}
\DeclareMathOperator{\id}{\mathrm{id}}
\DeclareMathOperator{\hum}{[\mathrm{Hum}]}
\DeclareMathOperator{\eh}{[\mathrm{EH}]}
\DeclareMathOperator{\lcm}{\mathrm{lcm}}
\DeclareMathOperator{\Aut}{\mathrm{Aut}}
\DeclareMathOperator{\Inn}{\mathrm{Inn}}
\DeclareMathOperator{\Out}{\mathrm{Out}}
\DeclareMathOperator{\Gal}{\mathrm{Gal}}
\DeclareMathOperator{\Real}{\mathrm{Re}}
\DeclareMathOperator{\Imag}{\mathrm{Im}}
\DeclareMathOperator{\res}{\mathrm{res}}


% frequently used shorthands
\newcommand{\ra}{\rightarrow}
\newcommand{\se}{\subseteq}
\newcommand{\ip}[1]{\langle#1\rangle}
\newcommand{\dual}{^*}
\newcommand{\inverse}{^{-1}}
\newcommand{\norm}[2]{\|#1\|_{#2}}
\newcommand{\abs}[1]{\lvert #1 \rvert}
\newcommand{\Abs}[1]{\bigg| #1 \bigg|}
\newcommand\bm[1]{\begin{bmatrix}#1\end{bmatrix}}
\newcommand{\op}{\text{op}}

%\def\darktheme{} % IAN
\ifx \darktheme\undefined
\else
\pagecolor[rgb]{0.2,0.231,0.302}%{0.23,0.258,0.321}
\color[rgb]{1,1,1}
\fi

% nicer looking empty set
\let\oldemptyset\emptyset
\let\emptyset\varnothing

%the var phi gang
\let\oldphi\phi
\let\phi\varphi

\setlist[enumerate,1]{topsep=1em,leftmargin=1.8em, itemsep=0.5em, label=\textup{(}\arabic*\textup{)}}
\setlist[enumerate,2]{topsep=0.5em,leftmargin=3em, itemsep=0.3em}

%pagestyle
%\pagestyle{fancy} 

\begin{document}
\begin{center}
    \textsc{Math 519. HW 5\\ Ian Jorquera}
\end{center}
\vspace{1em}
% See http://www.mathematicalgemstones.com/maria/Math501Fall22.php
% for problems

% sage: https://sagecell.sagemath.org/
\begin{enumerate}
\item Consider first the function $\frac{1}{f(1/z)}$ which for $|z|>1$ we know is holomorphic as $1/z$ is holomorphic away from $0$ and because $|1/z|<1$ we know that $f(1/z)$ is holomorphic and non-vanishing, so $1/f(1/z)$ is holomorphic by composition of holomorphic functions. Now we will show that $\frac{1}{\overline{f(1/\overline{z})}}$ is holomorphic for $|z|> 1$. Consider a point $z_0\in \C$ such that $|z_0|> 1$. Notice that $|\overline{z_0}|> 1$. And so $1/f(1/z)$ admits a power series near $\overline{z_0}$, that is $$1/f(1/z)=\sum_{n\geq 0}a_n(z-\overline{z_0})$$ Notice that this means that $$1/f(1/\overline{z})=\sum_{n\geq 0}a_n(\overline{z}-\overline{z_0})$$
This then gives us that $$1/\overline{f(1/\overline{z})}=\overline{\sum_{n\geq 0}a_n(\overline{z}-\overline{z_0})}=\sum_{n\geq 0}\overline{a_n}({z}-{z_0})$$ This shows that $1/\overline{f(1/\overline{z})}$ has a power series representation for any $|z_0|\geq 1$ and so is holomorphic

Now we will show that $1/\overline{f(1/\overline{z})}$ and $f(z)$ agree on their boundary. Notice that for $|z|=1$ we know that $\frac{1}{z}=\frac{\overline{z}}{z\overline{z}}=\overline{z}$, meaning $1/\overline{f(1/\overline{z})}=1/\overline{f(z)}=f(z)$. This means we have the extension 
$$F(z)=\begin{cases}f(z) & |z|\leq 1\\
\frac{1}{\overline{f(1/\overline{z})}}&\text{otherwise}\end{cases}$$
We know this extension is entire because both functions agree at the boundary so any closed triangular contour can be split up to be the sum of contours on both sides.
Notice that because $f$ is continuous on the unit disk we know that $f$ is bounded on the unit disk. We also know that because it is non-vanishing there is some $A$ such that $0<A\leq |f(z)|$ for all $z$ on the unit disk by the extreme value theorem. This means $\frac{1}{\overline{f(1/\overline{z})}}$ is bounded by $\frac{1}{A}$. Because $F(z)$ is entire and bounded we know that $F(z)$ is constant, by Liouville's theorem, meaning $f(z)$ is constant.\\

\item 
    \begin{enumerate}
    \item Notice that $$f(z)=\frac{2z-3}{z^2-4}=\frac{2z-3}{z^2-4}=\frac{2z-3}{(z-2)(z+2)}$$ 
    which means there is a simple pole at $z=2$ and $z=-2$. First to compute the residue at $z=2$ we know that 
    \begin{align*}
        \res(f;2)&=\lim_{z\ra 2}(z-2)f(z)\\
        &=\lim_{z\ra 2}(z-2)\frac{2z-3}{(z-2)(z+2)}\\
        &=\lim_{z\ra 2}\frac{2z-3}{(z+2)}=\frac{1}{4}
    \end{align*}
    \item Again using the same factorization as above we know there is a simple pole at $z=-2$. To compute the residue at $z=-2$ we know that 
    \begin{align*}
        \res(f;-2)&=\lim_{z\ra -2}(z+2)f(z)\\
        &=\lim_{z\ra -2}(z+2)\frac{2z-3}{(z-2)(z+2)}\\
        &=\lim_{z\ra -2}\frac{2z-3}{(z-2)}=\frac{7}{4}
    \end{align*}

    \item In this case $C$ is a toy contour where the function is holographic everywhere on an in the contour except for the points $z=2$ and $z=-2$. In this case we know that by the Cauchy residue formula that
    \begin{align*}
        \int_C f(z)dz=2\pi i (\res(f,2)+\res(f,-2))=4\pi i
    \end{align*}

    \item The partial fractions decomposition is 
    $$f(z)=\frac{2z-3}{z^2-4}=\frac{2z-3}{(z-2)(z+2)}=\frac{7}{4(z+2)}+\frac{1}{4(z-2)}$$ 
    \item In this case 
    \begin{align*}
        \int_C f(z)dz&=\int_C \frac{7}{4(z+2)}+\frac{1}{4(z-2)}dz\\
        &=\int_C \frac{7}{4(z+2)}dz+\int_C\frac{1}{4(z-2)}dz\\
        &=2\pi i(\frac{4}{7}+\frac{1}{4}) = 4\pi i
    \end{align*}
    Where the last line follows from the Cauchy integration formula (or Cauchy residue formula).\\
    \end{enumerate}

    \item
    \begin{enumerate}
        \item Consider the function $\frac{g(z)}{f(z)}$ which is holomorphic around $z_0$ with a zero at $z_0$. Notice that the derivative of this function $\frac{g'(z)f(z)-g(z)f'(z)}{f(z)^2}$ when evaluated at $z_0$ is $\frac{g'(z_0)f(z_0)}{f(z_0)^2}\neq 0$ and so the order of the hole, and therefore the order of the pole, at $z_0$ on the function $\frac{f(z)}{g(z)}$ is $1$. So $z_0$ is a simple pole.\\

        \item By the assumption we know that $g(z)$ is holomorphic with a hole of order $1$, meaning $g(z)=(z-z_0)h(z)$ where $h(z)$ is holomorphic and non-vanishing on points near $z_0$. In this case we can see that
        \begin{align*}
        \res\left(\frac{f(z)}{g(z)};z_0\right)&=\lim_{z\ra z_0}(z-z_0)\frac{f(z)}{g(z)}\\
        &= \lim_{z\ra z_0}(z-z_0)\frac{f(z)}{(z-z_0)h(z)}\\
        &=\frac{f(z_0)}{h(z_0)}=\frac{f(z_0)}{g'(z_0)}
    \end{align*}
        where the last line follows from $g'(z)=(z-z_0)h'(z)+h(z)$ and so $g'(z_0)=h(z_0)$.\\
    \end{enumerate}

    \item
    \begin{enumerate}
        \item Let $f,h$ be holomorphic functions defined on a deleted neighborhood $N_R(z_0)^\times$ where $R$ is as small as needed which means we can write $f(z)=(z-z_0)^n g(z)$ and $h(z)=(z-z_0)^mk(z)$ where $n=\text{ord}_{z_0}f(z)$, $m=\text{ord}_{z_0}h(z)$ and $g,k$ are holomorphic non-vanishing functions on $N_R(z_0)$. Notice that because we are working on a deleted neighborhood we are not concerned with an singularities at $z=z_0$, meaning $$(fh)(z)=(z-z_0)^n g(z)(z-z_0)^m k(z)=(z-z_0)^{n+m} g(z) k(z)$$
        and so $\text{ord}_{z_0}fh=n+m$ as the product of two non-vanishing functions in non-vanishing.\\

        \item let $f,h$ be holomorphic functions defined on a deleted neighborhood $N_R(z_0)^\times$ where $R$ is as small as needed which means we can write $f(z)=(z-z_0)^n g(z)$ and $h(z)=(z-z_0)^mk(z)$ where $n=\text{ord}_{z_0}f(z)$, $m=\text{ord}_{z_0}h(z)$ and $g,k$ are holomorphic non-vanishing functions on the neighborhood $N_R(z_0)$. Consider first the case where $f(z)=-h(z)$ which would result in $f+h\equiv 0$ and so the order is not well defined. Now consider the case that $n=m$ and $g(z_0)+k(z_0)=0$ but $f+h\not\equiv 0$. In this case we know that $g(z)+k(z)=(z-z_0)^pr(z)$ on $N_R(z_0)$ where $p>0$. In this case we have that $f(z)+g(z)=(z-z_0)^{n+p}r(z)$. If these two situations are not the case then we may assume WLOG that $n\leq m$. Notice that
        \begin{align*}
            (f+h)(z)&=(z-z_0)^n g(z)+(z-z_0)^m k(z)\\
            &=(z-z_0)^n( g(z)+(z-z_0)^{m-n} k(z))
        \end{align*}
        Notice that in this case there must be an open neighborhood in which $g(z)+(z-z_0)^{m-n} k(z)$ is non vanishing, if there wasn't there would exist distinct points $z_j\in N_{R/j}(z_0)^\times$ such that $z_j\neq z_0$ and $g(z_j)+(z_j-z_0)^{m-n} k(z_j)=0$ for all $j\in \N$. Notice that the limit as $j\ra \infty$ we have that $z_j\ra z_0$. If this is the case we know that $g(z)+(z-z_0)^{m-n} k(z)\equiv 0$ on $N_R$ which would imply $f(z)=-h(z)$. And so we have that $\text{ord}_{z_0}f+h=\min(n,m)$
    \end{enumerate}

\end{enumerate}

\end{document}






