\documentclass[12pt]{amsart}
\usepackage{preamble}
\DeclarePairedDelimiter\bra{\langle}{\rvert}
\DeclarePairedDelimiter\ket{\lvert}{\rangle}



\begin{document}
\begin{center}
    \textsc{Reading Group on Lie Algebras and Lie Groups\\(Never did this, instead I took Math 160)\\ Ian Jorquera}
\end{center}
\vspace{1em}

\section*{Main Resources}
\begin{itemize}
    \item[] (WilErd) Introduction to Lie Algebras, Erdmann and Wildon
    \item[] (Hump) Introduction to Lie Algebras and Repn Thy, Humphreys
    \item[] (Hall) Lie Groups, Lie Algebras and Representations, Hall 
\end{itemize}

\section*{Topics}
\subsection*{General Topics}
\begin{enumerate}
    \item The Basics of Lie Groups, with many motivating examples
    \begin{itemize}
        \item Matrix Lie Groups Theory (Hall Chap 1: 1.3 1.4 1.5) \textbf{Covered in MATH 601}
        \item Symplectic Groups (Hall ch 1.2.4) The Heisenberg Group (Hall 1.2.6)
        \item The Matrix Exponential (Hall Ch 2)
        \item The Lie Algebra of a Matrix Lie Group (Hall Ch 3.3) \textbf{Covered in MATH 601}
    \end{itemize}
    \item The Basics of Lie Algebras, with many motivating examples
    \begin{itemize}
        \item Chapters 1, 2, 4 of WilErd, and Chpters 1,2,3 of Hump. \textbf{Covered in MATH 601}
        \item Semisimple lie algebras (WilErd Def 4.6 Ch 4, Ch 3 Hump)
        \item Engel's theorem (WilErd Ch 6, Hump Ch 3)
        \item Lie Theorem (WilErd Ch 6, Hump Ch 4)
    \end{itemize}
    \item Representations of $\mathfrak{sl}_2$ (WilErd CH 8, Hump Ch 7) \textbf{Covered in MATH 601}
    \begin{itemize}
        \item Probably need WilErd Ch 7
    \end{itemize}
    \item Killing form (WilErd Ch 9.3, Hump Ch 5)
    \item Root Space Decompositions (WilErd Ch 10, Hump Ch 8)
    \begin{itemize}
        \item Cartan subalgs (WilErd Ch 10.1) or Maximal Toral 
        Subalgebras (Hump 8.1). These notations I guess are 
        equivalent (WilErd Appendix C)
    \end{itemize}
    \item Dynkin diagrams  (WilErd Ch 11.4, 13.1, Hump Ch 11.2)
    \begin{itemize}
        \item Classification of all simple lie algebras
    \end{itemize}
    \item Universal Enveloping Algebra and Highest weight modules (WilErd 15.2, Hump Ch 17)
\end{enumerate}

\subsection*{Special Topics}
Topics primarily found from 
\href{https://math.stackexchange.com/questions/1322206/what-are-applications-of-lie-groups-algebras-in-mathematics}{What are applications of Lie groups/algebras in mathematics?} 
And \href{https://mathoverflow.net/questions/58696/why-study-lie-algebras?_gl=1*1ggsex7*_ga*ODQwNjU2OTkxLjE2ODk5NjEzNDc.*_ga_S812YQPLT2*MTcxNzgzOTc5OC4xNy4xLjE3MTc4NDAwMzQuMC4wLjA.}{Why study Lie algebras?}
\begin{itemize}
    \item Lie Groups and Applications to ``Geometry": ``groups of symmetries of geometric objects"
    \begin{itemize}
        \item Groups of Lie Type (WilErd Ch 15.3)
        \item Some \href{http://www.math.sunysb.edu/~kirillov/mat552/liegroups.pdf}{Notes}
        \item \href{http://ocw.mit.edu/courses/mathematics/18-755-introduction-to-lie-groups-fall-2004/}{Helgason's notes}
        \item ``Lie groups provide a way to express the concept of a continuous family of symmetries for geometric objects"
        \item Some references and Books: John Lee's ``Introduction to Smooth Manifolds", Spivak's ``comprehensive introduction to differential geometry" and Sharpe's Differential Geometry Text.
        \item \href{https://en.wikipedia.org/wiki/Chern%E2%80%93Weil_homomorphism}{Chern-Weil theory}
    \end{itemize}
    \item Applications to Harmonic Analysis
    \begin{itemize}
        \item \href{https://en.wikipedia.org/wiki/Peter%E2%80%93Weyl_theorem}{Peter-Weyl} theorem
        \item \href{https://en.wikipedia.org/wiki/Automorphic_form}{Automorphic Forms} and number theory
        \item Zauner's Conjecture
    \end{itemize}
    \item Applications to Differential Equations
    \begin{itemize}
        \item ``Lie algebras arise as the infinitesimal symmetries of differential equations"
        \item A reference book: Olver, Peter J., Applications of Lie groups to differential equations
    \end{itemize}
    \item \href{https://web.archive.org/web/20160909004917/http://www.technicoder.com/blog/Lie_Group_in_Computer_Vision.html}{Computer Vision}
\end{itemize}

\section*{An Outline (10-12 weeks)}

\end{document}