\section{Introduction}
\label{sec:intro}
An orthonormal basis is a fundamental concept in linear algebra that plays a crucial role in many computational tasks, as these bases span the entire underlying space and their vectors are pairwise orthogonal. For example, orthonormal bases have many applications in information theory especially with compression algorithms that use nonstandard orthonormal basis to better align with the structure of the data. Frames aim to generalize orthonormal bases by adding redundancy which can act as a method of error and noise mitigation and often align with the structure of the data even more so than an orthonormal basis, which is a motivation behind dictionary learning algorithms.

This paper aims to explore the theory and applications of frames, building on concepts from linear algebra and combinatorics. In Section \ref{sec:orth_frames}, we define orthonormal bases and discuss some of the properties that motivate the use of frames. We then introduce frames in Section \ref{ssec:frames}, exploring their linear algebraic properties and providing proofs for many of the fundamental theorems in frame theory. In Section \ref{sec:matroids}, we study the structure of frames through their linear independence, using matroids. Finally, in Section \ref{sec:example}, we provide an example of an equiangular tight frame and its associated Naimark complement, using matroids to analyze their structure and properties.

% idk why i put this here but there you go
% https://math.berkeley.edu/~hutching/teach/54-2017/svd-notes.pdf